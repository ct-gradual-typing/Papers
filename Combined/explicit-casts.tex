In this section we introduce something that is seemingly small, but
very useful when programming in gradual type systems.  First, recall
the untyped Y combinator defined in Core Grady:
\begin{lstlisting}[language=Haskell]
  omega : (? -> ?) -> ?
  omega = \(x : ? -> ?) -> (x (box (? -> ?) x));
  
  fix : (? -> ?) -> ?
  fix = \(f : ? -> ?) -> omega (\(x:?) -> f ((unbox (? -> ?) x) x));
\end{lstlisting}
In Surface Grady we obtain a much cleaner definition without explicit
casts:
\begin{lstlisting}[language=Haskell]
  omega : (? -> ?) -> ?
  omega = \(x : ? -> ?) -> (x x);
  
  fix : (? -> ?) -> ?
  fix = \(f : ? -> ?) -> omega (\(x:?) -> f (x x));
\end{lstlisting}
The previous definition is in the style of programming in the untyped
$\lambda$-calculus.

Now suppose we added polymorphism -- in Haskell style -- to Grady,
then we might want to define a typed version of $\lstinline{fix}$ like
the following:
\begin{lstlisting}[language=Haskell]
  fixT : (X -> X) -> X
  fixT = \(f:X -> X) -> fix (\(y:?)-> (f y)));
\end{lstlisting}
However, $\lstinline{fixT}$ does not type check.  Notice that we must
produce something of type $\lstinline{X}$, but
$\lstinline{fix (\(y:?)-> (f y))}$ has type $\lstinline{?}$ and it will not be
implicitly cast.

We can better understand the issue by examining the function
application typing rule:
\begin{center}
  \begin{math}
    \inferrule* [flushleft,right=$\to_e$] {
      [[G |- t1 : C]] \\ \,\,\,[[A2 ~ A1]]\\\\    
      [[G |-t2 : A2]] \\ [[fun(C) = A1 -> B1]]
    }{[[G |- t1 t2 : B1]]}
  \end{math}
\end{center}
As we can see the only implicit casting that occurs is in the case of
the function, $[[t1]]$, and the argument, $[[t2]]$, but not the actual
result of the application.  Thus, in order to fix $\lstinline{fixT}$
we must insert an explicit cast, but we have removed the explicit
casts from Surface Grady.

All is not lost, because as it turns out, explicit casts can be
defined in Surface Grady:
\begin{center}
  \begin{math}
    \begin{array}{lll}
      [[box A t]]   & = & [[(\x:A.x) t]]\\
      [[unbox A t]] & = & [[(\x:?.x) t]]\\
    \end{array}
  \end{math}
\end{center}
Their typing rules are also derivable (proof omitted).
\begin{lemma}[Box and Unbox in Surface Grady]
  \label{lemma:box_and_unbox_in_surface_grady}
  The following typing rules are derivable:
  \begin{mathpar}
    \SGradydruleTXXbox{} \and
    \SGradydruleTXXunbox{}
  \end{mathpar}
\end{lemma}
Lastly, if we run the cast insertion algorithm
(Figure~\ref{fig:cast-insert}) on $[[box A t]]$, then it will produce
the Core Grady program $[[(\x:A.x) (box A t)]]$ and if we run the
algorithm on $[[unbox A t]]$, then it will produce $[[(\x:?.x) (unbox
    A t)]]$. This shows that the Surface Grady explicit casts
correspond to the Core Grady explicit casts.  We can now use Surface
Grady's explicit casts to properly define $\lstinline{fixT}$:
\begin{lstlisting}[language=Haskell]
  fixT : (X -> X) -> X
  fixT = \(f:X -> X) -> unbox X (fix (\(y:?)-> (f y))));
\end{lstlisting}

The problem we outline here is not specific to Surface Grady, but to
all gradual type systems.  If we program only using implicit casts,
then we are not taking advantage of the full power of our type system,
but if we combine them with explicit casts, then more programs become
typable. 
