Just as the simply typed $\lambda$-calculus corresponds to cartesian
closed categories our categorical model has a corresponding type
theory we call Core Grady.  It consists of all of the structure found
in the model.  To move from the model to Core Grady we apply the
Curry-Howard-Lambek
correspondence~\cite{Wadler:2015:PT:2847579.2699407,Lambek:1980}.
Objects become types, and morphisms, $t : [[G]] \mto A$, become
programs in context usually denoted by $[[G |- t : A]]$ which
corresponds to Core Grady's type checking judgment.  We will discuss
this correspondence in detail in
Section~\ref{sec:interpreting_surface_grady_in_the_model}.

The syntax for Core Grady is defined in
Figure~\ref{fig:syntax-core-grady}.
\begin{figure}
  \scriptsize
  \begin{mdframed}
    \[
      \setlength{\arraycolsep}{0.6pt}
      \begin{array}{cl}      
        \begin{array}{l}
          \text{(types)}\\
        \end{array}     &
        \begin{array}{lcl}
          [[A]],[[B]],[[C]] & ::= & [[Unit]] \mid [[Nat]] \mid [[?]] \mid [[A x B]] \mid [[A -> B]] \\
        \end{array}\\\\
        
        \begin{array}{lll}
          \text{(skeletons)}\\
        \end{array} &
        \begin{array}{lcl}
          [[S]],[[K]],[[U]] & ::= & [[?]] \mid [[S1 x S2]] \mid [[S1 -> S2]]\\
        \end{array}\\\\
        
        \begin{array}{l}
          \text{(terms)}\\\\\\
        \end{array}     &
        \begin{array}{lcl}
          [[t]] & ::= & [[x]] \mid [[triv]] \mid [[0]] \mid [[succ t]] \mid [[(t1 , t2)]] \mid [[fst t]] \mid [[snd t]] \\ & \mid & [[\x : A.t]] \mid [[t1 t2]] \mid [[case t : Nat of 0 -> t1, (succ x) -> t2]]\\ & \mid & [[box A]] \mid [[unbox A]] \mid [[error A]]\\
        \end{array}
        \\\\
        \begin{array}{l}
          \text{(values)}\\
        \end{array}     &
        \begin{array}{lcl}
          [[v]] & ::= & [[\x : A.t]] \mid [[unbox A]] \\
        \end{array}
        \\\\
        \begin{array}{lll}
          \text{(evaluation contexts)}\\\\
        \end{array}  &
        \begin{array}{lcl}
          [[EC]] & ::= & [[v HL]] \mid [[succ HL]] \mid [[fst HL]] \mid [[snd HL]] \mid [[(HL, t)]] \\ & \mid & [[(t, HL)]] \mid [[case HL : Nat of 0 -> t1, (succ x) -> t2]]\\
        \end{array}\\\\
        
        \begin{array}{lll}
          \text{(contexts)}\\
        \end{array}  &
        \begin{array}{lcl}
          [[G]] & ::= & [[.]] \mid [[x : A]] \mid [[G1,G2]]\\
        \end{array}\\
      \end{array}
      \]    
    \end{mdframed}
    \caption{Syntax for Core Grady}
    \label{fig:syntax-core-grady}
\end{figure}
The syntax is a straightforward extension of the simply typed
$\lambda$-calculus.  Arbitrary programs or terms are denoted by
$[[t]]$ and values by $[[v]]$. The later are used to influence the
evaluation strategy used by Core Grady.  Natural numbers are denoted
by $[[0]]$ and $[[succ t]]$ where the latter is the successor of
$[[t]]$.  The non-recursive natural number eliminator is denoted by
$[[case t : Nat of 0 -> t1, (succ x) -> t2]]$.  The most interesting
aspect of the syntax is that $[[box A]]$ and $[[unbox A]]$ are not
restricted to atomic types, but actually correspond to $[[Box A]]$ and
$[[Unbox A]]$ from Lemma~\ref{lemma:casting_morphisms}.  That result
shows that these can actually be defined in terms of $[[lbox A]]$,
$[[lunbox A]]$, $[[lsplit S]]$, and $[[lsquash S]]$ when $[[A]]$ is
any type and $[[S]]$ is a skeleton, but we take the general versions
as primitive, because they are the most useful from a programming
perspective.  In addition, as we mentioned above $[[Box A]]$ and
$[[Unbox A]]$ divert to $[[lsquash A]]$ and $[[lsplit A]]$
respectively when $[[A]]$ is a skeleton.  This implies that we no
longer need two retracts, and hence, simplifies the language.

Multisets of pairs of variables and types, denoted by $[[x : A]]$,
called a typing context or just a context is denoted by $[[G]]$.  The
empty context is denoted by $[[.]]$, and the union of contexts
$[[G1]]$ and $[[G2]]$ is denoted by $[[G1,G2]]$.  Typing contexts are
used to keep track of the types of free variables during type
checking.

The typing judgment is denoted by $[[G |- t : A]]$, and is read ``the
term $[[t]]$ has type $[[A]]$ in context $[[G]]$.''  The typing
judgment is defined by the type checking rules in
Figure~\ref{fig:typing-core-grady}.
\begin{figure} \scriptsize
  \begin{mdframed}
    \begin{mathpar}
      \CGradydruleTXXvar{} \and
      \CGradydruleTXXBox{} \and
      \CGradydruleTXXUnbox{} \and
      \CGradydruleTXXunitP{} \and
      \CGradydruleTXXzeroP{} \and
      \CGradydruleTXXsucc{} \and
      \CGradydruleTXXncase{} \and
      \CGradydruleTXXpair{} \and
      \CGradydruleTXXfst{} \and
      \CGradydruleTXXsnd{} \and
      \CGradydruleTXXlam{} \and
      \CGradydruleTXXapp{} \and
      \CGradydruleTXXerror{} 
    \end{mathpar}
  \end{mdframed}
  \caption{Typing rules for Core Grady}
  \label{fig:typing-core-grady}
\end{figure}
\begin{figure} 
  \begin{mdframed} \scriptsize
    \begin{mathpar}
      \CGradydrulerdXXretracT{} \and
      \CGradydrulerdXXretracTE{} \and
      \CGradydrulerdXXerror{} \and
      \CGradydrulerdXXncaseZero{} \and
      \CGradydrulerdXXncaseSucc{} \and
      \CGradydrulerdXXbeta{} \and
      \CGradydrulerdXXprojOne{} \and
      \CGradydrulerdXXprojTwo{} \and
      \CGradydrulerdXXCong{}
    \end{mathpar}
  \end{mdframed}
  \caption{Reduction rules for Core Grady}
  \label{fig:reduction-core-grady}
\end{figure}
The type checking rules are an extension of the typing rules for the
simply typed $\lambda$-calculus.  The casting terms are all typed as
axioms with their expected types.  This implies that applying either
$[[box A]]$ or $[[unbox A]]$ to some other term corresponds to
function application as opposed to $[[succ t]]$ which cannot be used
without its argument. This fact is used in the definition of the
evaluation strategy.

Computing with terms is achieved by defining a reduction relation
denoted by $[[t1 ~> t2]]$ and is read as ``the term $[[t1]]$ reduces
(or evaluates) in one step to the term $[[t2]]$.''  The reduction
relation is defined in Figure~\ref{fig:reduction-core-grady}.  Core
Grady's reduction strategy is an extended version of call-by-name.  It
is specified using evaluation contexts that are denoted by $[[EC]]$
and are defined in Figure~\ref{fig:syntax-core-grady}.

An evaluation contexts is essentially a term with a hole, denoted by
$[[HL]]$, in it.  The hole can be filled (or plugged) with a term and
is denoted by $[[EC[t] ]]$.  Note that plugging the hole of an
evaluation context results in a term.  Evaluation contexts are used to
give a compact definition of an evaluation strategy by first specifying
the reduction axioms (Figure~\ref{fig:reduction-core-grady}), then
defining the evaluation contexts by placing a hole within the syntax
of terms that specifies where evaluation is allowed to take place
(Figure~\ref{fig:syntax-core-grady}), finally, the following reduction
rule is then added:
\begin{center}\small
  \begin{math}
    \CGradydrulerdXXCong{}
  \end{math}
\end{center}
This rule states that evaluation can take place in the locations of
the holes given in the definition of evaluation contexts
(Figure~\ref{fig:syntax-core-grady}).

How we define the syntax of values and evaluation contexts, and the
evaluation rules determines the evaluation strategy.  We consider as
values $\lambda$-abstractions and $[[unbox A]]$. In addition, the
expression $\lambda (x:A).[[HL]]$ is not an evaluation context, and
hence, there is no evaluation under $\lambda$-abstractions.  The
evaluation context $[[v HL]]$ stipulates that the only time an argument
to a function can be evaluated is if the term in the function position
is a value.  Thus, $[[v]]$ is allowed to be a $\lambda$-abstraction or
$[[unbox A]]$.  We do not allow reduction under $[[box A]]$.
Similarly, we have no evaluation contexts which allow evaluation under
the branches of a case-expression.  Both of these restrictions are
used to prevent infinite reduction from occurring in those positions.
We want evaluation to make as much overall progress as possible.

Perhaps the most interesting reduction rules from
Figure~\ref{fig:reduction-core-grady} are the first three: retract,
raise, and error.  The first two handle dynamic type casts and the
third preserves dynamic type errors that have been raised in an
evaluation position.  The error reduction rule depends on typing which
is necessary to insure that the type annotation is correct.  This
insures that type preservation will hold.  Practically speaking, this
dependency on typing is not significant, because we only evaluate
closed well-typed programs anyway.

Just as Abadi et al.~\cite{Abadi:1989} argue it is quite useful to
have access to the untyped $\lambda$-calculus.  We give some example
Core Grady programs utilizing this powerful feature.  We have a full
implementation of every language in this paper
available\footnote{Please see
  \url{https://ct-gradual-typing.github.io/Grady/} for access to the
  implementation as well as full documentation on how to install and
  use it.}.  All examples in this section can be typed and ran in the
implementation, and thus, we make use of Core Grady's concrete syntax
which is very similar to Haskell's and does not venture too far from
the mathematical syntax given above.

Core Grady does not have a primitive notion of recursion, but it is
well-known that we can define the Y combinator in the untyped
$\lambda$-calculus.  Its definition in Core Grady is as follows:
\begin{lstlisting}[language=Haskell]
  omega : (? -> ?) -> ?
  omega = \(x : ? -> ?) -> (x (box (? -> ?) x));
  
  fix : (? -> ?) -> ?
  fix = \(f : ? -> ?) -> omega (\(x:?) -> f ((unbox (? -> ?) x) x));
\end{lstlisting}
Using $\lstinline{fix}$ we can define the usual arithmetic operations
in Core Grady, but we use a typed version of $\lstinline{fix}$.
\begin{lstlisting}[language=Haskell]
  fixNat : ((Nat -> Nat) -> (Nat -> Nat)) -> (Nat -> Nat)
  fixNat = \(f:(Nat -> Nat) -> (Nat -> Nat)) ->
     unbox{Nat -> Nat} (fix (\(y:?)->box{Nat -> Nat} (f (unbox{Nat -> Nat} y))));
  
  pred : Nat -> Nat
  pred = \(n:Nat) -> case n of 0 -> 0, (succ n') -> n';

  add : Nat -> Nat -> Nat
  add = \(m:Nat) -> fixNat
       (\(r:Nat -> Nat) ->
        \(n:Nat) -> case n of 0 -> m, (succ n') -> succ (r n'));

  sub : Nat -> Nat -> Nat
  sub = \(m:Nat) -> fixNat
       (\(r:Nat -> Nat) ->
        \(n:Nat) -> case n of 0 -> m, (succ n') -> pred (r n'));        
        
  mult : Nat -> Nat -> Nat
  mult = \(m:Nat) -> fixNat
       (\(r:Nat -> Nat) ->
        \(n:Nat) -> case n of 0 -> 0, (succ n') -> add m (r n'));
\end{lstlisting}
The function $\lstinline{fixNat}$ is defined so that it does recursion
on the type $[[Nat -> Nat]]$, thus, it must take in an argument,
$f : [[(Nat -> Nat) -> (Nat -> Nat)]]$, and
produce a function of type $[[Nat -> Nat]]$.  However, we
already have $\lstinline{fix}$ defined in the untyped fragment, and
so, we can define $\lstinline{fixNat}$ using $\lstinline{fix}$ by
boxing up the typed data.  This means we must cast $[[f]] : [[(Nat -> Nat) -> (Nat -> Nat)]]$ into a function of type
$[[(? -> ?) -> ?]]$ and we do this by $\eta$-expanding
$[[f]]$ and casting the input and output using $[[box]]$ and
$[[unbox]]$ to arrive at the function
$[[\y:?.H(box (Nat -> Nat) (f (unbox (Nat -> Nat) y)))]] : [[? -> ?]]$.  Finally, we can apply $\lstinline{fix}$, and then unbox its output to the type $[[Nat -> Nat]]$.

Extending Grady with polymorphism would allow for the definition of
$\lstinline{fixNat}$ to be abstracted, and then we could do statically
typed recursion at any type.  We extend Core Grady with bounded
polymorphism in Section~\ref{sec:bounded_quantification}.

From a programming perspective Core Grady has a lot going for it, but
it is unfortunate the programmer is required to insert explicit casts
when wanting to program dynamically.  This implies that it is not
possible to program in dynamic style when using Core Grady.  In the
next section we fix this problem by developing a gradually typed
surface language for Core Grady in the spirit of Siek and Taha's
gradually typed $\lambda$-calculus \cite{Siek:2006,Siek:2015}.

%%% Local Variables: ***
%%% mode:latex ***
%%% TeX-master: "main.tex"  ***
%%% End: ***
