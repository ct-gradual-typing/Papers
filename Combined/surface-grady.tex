Programming in dynamic style requires the ability to implicitly cast
data between types during eliminations. For example, $[[(\x:?.(succ
    (succ x))) 3]]$ should type check with type $[[Nat]]$ even though
$[[3]]$ has type $[[Nat]]$ and $[[x]]$ has type $[[?]]$. Now not every
cast should work, for example, $[[(\x:Bool.t) 3]]$ should not type check,
because it is inconsistent to allow different atomic types to be cast
between each other.  Therefore, we must be able to decide when casts
are consistent and which are not.

Gradual type systems are the combination of static and dynamic typing
with the ability to implicitly cast data between consistent types in
such a way that the gradual guarantee is satisfied.  A gradually typed
programming language consists of two languages: a core language and a
surface language.  The core language contains explicit casts and an
operational semantics, while the surface language is the gradual type
system that allows casts to be left implicit.  Programs are written
and then translated into the core language by inserting the explicit
casts.  The gradual guarantee states the following:
\begin{itemize}
\item Adding or removing casts to/from any well-typed surface language
  program remains typeable at a potentially less precise type.
\item Adding explicit casts to a terminating well-typed core language
  program also terminates at a related value, adding explicit casts to
  a diverging well-typed core language program diverges, and removing
  explicit casts from a terminating or diverging well-type core
  language programs either terminates, diverges, or raises a dynamic
  type error.
\end{itemize}
Briefly, the gradual guarantee states that any well-typed program can
slide between more statically typed and more dynamically typed by
inserting or removing casts without changing the meaning or the
behavior of the program. The formal statement of the gradual guarantee
is given in Section~\ref{sec:the_gradual_guarantee}.  This property
was first proposed by Siek et al.~\cite{Siek:2015} to set apart
systems that simply combine dynamic and static typing and gradual type
systems.

Now we introduce the gradually typed surface language Surface Grady.  
\begin{figure}
  \scriptsize
  \begin{mdframed}
    \textbf{Syntax:}\\
      \[
    \setlength{\arraycolsep}{1pt}
    \begin{array}{cl}      
        \begin{array}{l}
          \text{(terms)}\\\\
        \end{array}     &
        \begin{array}{lcl}
          [[t]] & ::= & [[x]] \mid [[triv]] \mid [[0]] \mid [[succ t]] \mid [[(t1 , t2)]] \mid [[fst t]] \mid [[snd t]] \\ & \mid & [[\x : A.t]] \mid [[t1 t2]] \mid [[case t of 0 -> t1, (succ x) -> t2]]\\
        \end{array}\\\\
        
      \end{array}
      \]    
    \textbf{Metafunctions:}\\
    \begin{mathpar}
      \begin{array}{lll}
        [[nat(?) = Nat]]\\
        [[nat(Nat) = Nat]]\\
      \end{array}
      \and
      \begin{array}{lll}
        [[prod(?) = ? x ?]]\\
        [[prod(A x B) = A x B]]\\
      \end{array}\\
      \and
      \begin{array}{lll}
        [[fun(?) = ? -> ?]]\\
        [[fun(A -> B) = A -> B]]\\
      \end{array}
    \end{mathpar}
  \end{mdframed}
  \caption{Syntax and Metafunctions for Surface Grady}
  \label{fig:syntax-surface-grady}
\end{figure}
Surface Grady is a small extension of the surface language given by
Siek et al.~\cite{Siek:2015}.  We have added natural numbers with their
eliminator as well as cartesian products.  The Surface Grady syntax is
defined in Figure~\ref{fig:syntax-surface-grady}, and it corresponds
to Core Grady's syntax (Figure~\ref{fig:syntax-core-grady}), but
without the explicit casts.  The syntax for types and typing contexts
do not change, and so we do not repeat them here.

The metafunctions $[[nat(A)]]$, $[[prod(A)]]$, and $[[fun(A)]]$ are
partial functions that will be used to determine when to use either
$[[box]]$ or $[[split]]$ in the elimination type checking rules for
natural numbers, cartesian products, and function applications
respectively.  For example, if $[[nat(A) = Nat]]$, then the type
$[[A]]$ must have been either $[[?]]$ or $[[Nat]]$, and if it were the
former then we know we can cast $[[A]]$ to $[[Nat]]$ via $[[box
    Nat]]$.  If $[[prod(A) = B x C]]$, then either $[[A]] = [[?]]$ and
$[[B x C]] = [[? x ?]]$ or $[[A]] = [[B x C]]$ for some other types
$[[B]]$ and $[[C]]$.  This implies that if the former is true, then we
can cast $[[A]]$ to $[[B x C]]$ via $[[split (? x ?)]]$.  The case is
similar for $[[fun(A)]]$.

The type checking and type consistency rules are given in
Figure~\ref{fig:typing-surface-grady}.
\begin{figure}
  \scriptsize
  \begin{mdframed}
    \textbf{Typing Rules:}\\
    \begin{mathpar}
      \SGradydruleTXXvarP{} \and
      \SGradydruleTXXunitP{} \and
      \SGradydruleTXXzeroP{} \and
      \SGradydruleTXXsucc{} \and
      \SGradydruleTXXncase{} \and
      \SGradydruleTXXpair{} \and
      \SGradydruleTXXfst{} \and
      \SGradydruleTXXsnd{} \and      
      \SGradydruleTXXlam{} \and
      \SGradydruleTXXapp{}
    \end{mathpar}

    \textbf{Type Consistency:}\\
    \begin{mathpar}
      \SGradydruleCXXRefl{} \and
      \SGradydruleCXXBox{} \and
      \SGradydruleCXXUnbox{} \and
      \SGradydruleCXXArrow{} \and
      \SGradydruleCXXProd{}
    \end{mathpar}
  \end{mdframed}
  \caption{Typing rules for Surface Grady}
  \label{fig:typing-surface-grady}
\end{figure}
Similarly to Core Grady the typing judgment is denoted by $[[G |- t :
    A]]$.  Type checking depends on the notion of type consistency;
first proposed by Siek and Taha~\cite{Siek:2006}.  This is a reflexive
and symmetric, but non-transitive, relation on types denoted by $[[G
    |- A ~ B]]$ which can be read as ``the type $[[A]]$ is consistent
with the type $[[B]]$ in context $[[G]]$.''  Note that the typing
context $[[G]]$ does not yet play a role in the definition of type
consistency, but it will when we add bounded quantification in
Section~\ref{sec:bounded_quantification}.  Non-transitivity is
important, because if type consistency were transitive, then all types
would be consistent, but this is too general.

Type consistency states when two types are safely castable between
each other when inserting explicit casts.  From semantical perspective
if $[[G |- A ~ B]]$ holds, then there are casting morphisms
(Definition~\ref{def:casting-mor}) $[[c1]] : [[A --> B]]$ and $[[c2]]
: [[B --> A]]$.  This follows from
Corollary~\ref{corollary:arbitrary_casting_morphisms}.  We make use of
this fact in the next section. Consider an example, type consistency
is responsible for the function application $[[(\x:?.(succ x)) 3]]$
being typable in the surface language, because type $[[Nat]]$ is
consistent with the type $[[3]]$.  This implies that the elimination
rule for function types must be extended with type consistency.

The typing rules for Surface Grady are a conservative extension of the
typing rules for Core Grady (Figure~\ref{fig:typing-core-grady}). The
extension is the removal of explicit casts and the addition of type
consistency and the metafunctions from
Figure~\ref{fig:syntax-surface-grady}.  Each rule is modified in
positions where casting is likely to occur which is all of the
elimination rules as well as the typing rule for successor, because it
is a type of application.  Consider the elimination rule for function
applications:
\[
\SGradydruleTXXapp{}
\]
This rule has been extended with type consistency.  The type of
$[[t1]]$ is allowed to be either $[[?]]$ or a function type $[[A1 ->
    B1]]$, by the definition of $[[fun(C)]]$, if the former is true,
then $[[A1 -> B1]] = [[? -> ?]]$ and $[[A2]]$ can be any type at all,
but if $[[C]] = [[A1 -> B1]]$, then $[[A2]]$ must be consistent with
$[[A1]]$.  Notice that if $[[C]] = [[A1 -> B1]]$ and $[[A2]] =
[[A1]]$, then this rule is equivalent to the usual rule for function
application. We can now see that our example program $[[(\x:?.(succ
    x)) 3]]$ is typable in Surface Grady.  Similar reasoning can be
used to understand the other typing rules as well.

\begin{figure}
  \scriptsize
  \begin{mdframed}
    \begin{mathpar}
      \SGradydruleciXXvar{}    \and
      \SGradydruleciXXzero{}   \and
      \SGradydruleciXXtriv{}   \and
      \SGradydruleciXXsuccU{}  \and
      \SGradydruleciXXsucc{}   \and      
      \inferrule* [flushleft] {
        { 
          \begin{array}{llllllllllllll}
                                         & [[G |- A1 ~ A]]        &                    \\
        [[G |- t => t' : ?]]             & [[G |- A2 ~ A]]        & [[t''1 = (c1 t'1)]]\\[1px]
        [[G |- t1 => t'1 : A1]]          & [[caster(A1,A) = c1]]  & [[t''2 = (c2 t'2)]]\\[1px]
        [[G, x : Nat |- t2 => t'2 : A2]] & [[caster(A2,A) = c2]]  & [[t'' = (unbox Nat t')]]
          \end{array}
        }
      }{[[G |- (case t of 0 -> t1, (succ x) -> t2) => (case t'' of 0 -> t''1, (succ x) -> t''2) : A]]}
      \and
      \inferrule* [right=] {
        {
          \begin{array}{llllllllll}
            [[G |- t => t' : Nat]]                               \\
            [[G |- t1 => t'1 : A1]]          & [[G |- A1 ~ A ]]  \\
            [[G, x : Nat |- t2 => t'2 : A2]] & [[G |- A2 ~ A]]   \\
          \end{array}
        }
      }{[[G |- (case t of 0 -> t1, (succ x) -> t2) => (case t' of 0 -> t'1, (succ x) -> t'2) : A]]}
      \and
      \SGradydruleciXXpair{}   \and
      \SGradydruleciXXfstU{}   \and
      \SGradydruleciXXfst{}    \and
      \SGradydruleciXXsndU{}   \and
      \SGradydruleciXXsnd{}    \and
      \SGradydruleciXXlam{}    \and
      \SGradydruleciXXappU{}   \and
      \SGradydruleciXXapp{} 
    \end{mathpar}
  \end{mdframed}
  \caption{Cast Insertion Algorithm}
  \label{fig:cast-insert}
\end{figure}




\subsection{Interpreting Surface Grady in the Model}
\label{subsec:interpreting_surface_grady_in_the_model}
%% Interpreting a programming language into a categorical model requires
%% three steps.  First, the types are interpreted as objects.  Then
%% programs are interpreted as morphisms in the category, but this is a
%% simplification.  Every morphism, $f$, in a category has a source
%% object and a target object, we usually denote this by $f : A \mto B$.
%% Thus, in order to interpret programs as morphisms the program must
%% have a source and target.  So instead of interpreting raw terms as
%% morphisms we interpret terms in their typing context.  That is, we
%% must show how to interpret every $[[G |- t : A]]$ as a morphism $[[t]]
%% : \interp{[[G]]} \mto \interp{[[A]]}$.  The third step is to show that
%% whenever one program reduces to another their interpretations are
%% isomorphic in the model. This means that whenever $[[G |- t1 ~> t2 : A]]$, then $[[ [|t1|] ]] = [[ [|t2|] ]] : [[ [| G |] --> [| A |] ]]$.
%% This is the reason why we defined our reduction in a
%% typed fashion to aid us in understanding how it relates to the model.
% subsection interpreting_surface_grady_in_the_model (end)
