Programming in dynamic style requires the ability to implicitly cast
data between types during eliminations. For example, $[[(\x:?.(succ
    (succ x))) 3]]$ should type check with type $[[Nat]]$ even though
$[[3]]$ has type $[[Nat]]$ and $[[x]]$ has type $[[?]]$. Now not every
cast should work, for example, $[[(\x:Bool.t) 3]]$ should not type check,
because it is inconsistent to allow different atomic types to be cast
between each other.  Therefore, we must be able to decide when casts
are consistent and which are not.

Gradual type systems are the combination of static and dynamic typing
with the ability to implicitly cast data between consistent types in
such a way that the gradual guarantee is satisfied.  A gradually typed
programming language consists of two languages: a core language and a
surface language.  The core language contains explicit casts and an
operational semantics, while the surface language is the gradual type
system that allows casts to be left implicit.  Programs are written
and then translated into the core language by inserting the explicit
casts.  The gradual guarantee states the following:
\begin{itemize}
\item Adding or removing casts to/from any well-typed surface language
  program remains typeable at a potentially less precise type.
\item Adding explicit casts to a terminating well-typed core language
  program also terminates at a related value, adding explicit casts to
  a diverging well-typed core language program diverges, and removing
  explicit casts from a terminating or diverging well-type core
  language programs either terminates, diverges, or raises a dynamic
  type error.
\end{itemize}
Briefly, the gradual guarantee states that any well-typed program can
slide between being more statically typed or being more dynamically
typed by inserting or removing casts without changing the meaning or
the behavior of the program. The formal statement of the gradual
guarantee is given in Section~\ref{sec:the_gradual_guarantee}.  This
property was first proposed by Siek et al.~\cite{Siek:2015} to set
apart systems that simply combine dynamic and static typing and
gradual type systems.

Now we introduce the gradually typed surface language Surface Grady.  
\begin{figure}
  \scriptsize
  \begin{mdframed}
    \textbf{Syntax:}\\
      \[
    \setlength{\arraycolsep}{1pt}
    \begin{array}{cl}      
        \begin{array}{l}
          \text{(terms)}\\\\
        \end{array}     &
        \begin{array}{lcl}
          [[t]] & ::= & [[x]] \mid [[triv]] \mid [[0]] \mid [[succ t]] \mid [[(t1 , t2)]] \mid [[fst t]] \mid [[snd t]] \\ & \mid & [[\x : A.t]] \mid [[t1 t2]] \mid [[case t of 0 -> t1, (succ x) -> t2]]\\
        \end{array}\\\\
        
      \end{array}
      \]    
    \textbf{Metafunctions:}\\
    \begin{mathpar}
      \begin{array}{lll}
        [[nat(?) = Nat]]\\
        [[nat(Nat) = Nat]]\\
      \end{array}
      \and
      \begin{array}{lll}
        [[prod(?) = ? x ?]]\\
        [[prod(A x B) = A x B]]\\
      \end{array}\\
      \and
      \begin{array}{lll}
        [[fun(?) = ? -> ?]]\\
        [[fun(A -> B) = A -> B]]\\
      \end{array}
    \end{mathpar}
  \end{mdframed}
  \caption{Syntax and Metafunctions for Surface Grady}
  \label{fig:syntax-surface-grady}
\end{figure}
Surface Grady is a small extension of the surface language given by
Siek et al.~\cite{Siek:2015}.  We have added natural numbers with their
eliminator as well as cartesian products.  The Surface Grady syntax is
defined in Figure~\ref{fig:syntax-surface-grady}, and it corresponds
to Core Grady's syntax (Figure~\ref{fig:syntax-core-grady}), but
without the explicit casts.  The syntax for types and typing contexts
do not change, and so we do not repeat them here.

The metafunctions $[[nat(A)]]$, $[[prod(A)]]$, and $[[fun(A)]]$ are
partial functions that will be used to determine when to use either
$[[box]]$ or $[[split]]$ in the elimination type checking rules for
natural numbers, cartesian products, and function applications
respectively.  For example, if $[[nat(A) = Nat]]$, then the type
$[[A]]$ must have been either $[[?]]$ or $[[Nat]]$, and if it were the
former then we know we can cast $[[A]]$ to $[[Nat]]$ via $[[box
    Nat]]$.  If $[[prod(A) = B x C]]$, then either $[[A]] = [[?]]$ and
$[[B x C]] = [[? x ?]]$ or $[[A]] = [[B x C]]$ for some other types
$[[B]]$ and $[[C]]$.  This implies that if the former is true, then we
can cast $[[A]]$ to $[[B x C]]$ via $[[split (? x ?)]]$.  The case is
similar for $[[fun(A)]]$.

The type checking and type consistency rules are given in
Figure~\ref{fig:typing-surface-grady}.
\begin{figure}
  \scriptsize
  \begin{mdframed}
    \textbf{Typing Rules:}\\
    \begin{mathpar}
      \SGradydruleTXXvarP{} \and
      \SGradydruleTXXunitP{} \and
      \SGradydruleTXXzeroP{} \and
      \SGradydruleTXXsucc{} \and
      \inferrule* [flushleft,right=$\mathsf{Nat}_e$] {
      {
        \begin{array}{lll}
          [[G |- t : C]]            & [[nat(C) = Nat]]\\
          [[G |- t1 : A1]]          & [[G |- A1 ~ A]]\\
          [[G, x : Nat |- t2 : A2]] & [[G |- A2 ~ A]]
        \end{array}
      }
      }{[[G |- case t of 0 -> t1, (succ x) -> t2 : A]]}
      \and
      \SGradydruleTXXpair{} \and
      \SGradydruleTXXfst{} \and
      \SGradydruleTXXsnd{} \and      
      \SGradydruleTXXlam{} \and
      \inferrule* [flushleft,right=$\to_e$] {
      [[G |- t1 : C]] \\ \,\,\,[[G |- A2 ~ A1]]\\\\    
      [[G |-t2 : A2]] \\ [[fun(C) = A1 -> B1]]
    }{[[G |- t1 t2 : B1]]}
    \end{mathpar}

    \textbf{Type Consistency:}\\
    \begin{mathpar}
      \SGradydruleCXXRefl{} \and
      \SGradydruleCXXBox{} \and
      \SGradydruleCXXUnbox{} \and
      \SGradydruleCXXArrow{} \and
      \SGradydruleCXXProd{}
    \end{mathpar}
  \end{mdframed}
  \caption{Typing rules for Surface Grady}
  \label{fig:typing-surface-grady}
\end{figure}
Similarly to Core Grady the typing judgment is denoted by $[[G |- t :
    A]]$.  Type checking depends on the notion of type consistency;
first proposed by Siek and Taha~\cite{Siek:2006}.  This is a reflexive
and symmetric, but non-transitive, relation on types denoted by $[[G
    |- A ~ B]]$ which can be read as ``the type $[[A]]$ is consistent
with the type $[[B]]$ in context $[[G]]$.''  Note that the typing
context $[[G]]$ does not yet play a role in the definition of type
consistency, but it will when we add bounded quantification in
Section~\ref{sec:bounded_quantification}.  Non-transitivity is
important, because if type consistency were transitive, then all types
would be consistent, but this is too general. Consider
an example, type consistency is responsible for the function
application $[[(\x:?.(succ x)) 3]]$ being typable in the surface
language, because type $[[Nat]]$ is consistent with the type $[[3]]$.
This implies that the elimination rule for function types must be
extended with type consistency.

Type consistency states when two types are safely castable between
each other when inserting explicit casts, and so, from a semantical
perspective if $[[G |- A ~ B]]$ holds, then there are casting
morphisms (Definition~\ref{def:casting-mor}) $[[c1]] : [[A --> B]]$
and $[[c2]] : [[B --> A]]$; see
Lemma~\ref{lemma:type_consistency_in_the_model} in
Section~\ref{sec:interpreting_surface_grady_in_the_model}.

The typing rules for Surface Grady are a conservative extension of the
typing rules for Core Grady (Figure~\ref{fig:typing-core-grady}). The
extension is the removal of explicit casts and the addition of type
consistency and the metafunctions from
Figure~\ref{fig:syntax-surface-grady}.  Each rule is modified in
positions where casting is likely to occur which would be in all of
the elimination rules as well as the typing rule for successor,
because it is a type of application.  Consider the elimination rule
for function applications:
\[
\SGradydruleTXXapp{}
\]
This rule has been extended with type consistency.  The type of
$[[t1]]$ is allowed to be either $[[?]]$ or a function type $[[A1 ->
    B1]]$, by the definition of $[[fun(C)]]$, if the former is true,
then $[[A1 -> B1]] = [[? -> ?]]$ and $[[A2]]$ can be any type at all,
but if $[[C]] = [[A1 -> B1]]$, then $[[A2]]$ must be consistent with
$[[A1]]$.  Notice that if $[[C]] = [[A1 -> B1]]$ and $[[A2]] =
[[A1]]$, then this rule is equivalent to the usual rule for function
application. We can now see that our example program $[[(\x:?.(succ
    x)) 3]]$ is typable in Surface Grady.  Similar reasoning can be
used to understand the other typing rules as well.

Surface Grady is translated into Core Grady using the cast insertion
algorithm detailed in Figure~\ref{fig:cast-insert}.
\begin{figure}
  \scriptsize
  \begin{mdframed}
    \begin{mathpar}
      \SGradydruleciXXvar{}    \and
      \SGradydruleciXXzero{}   \and
      \SGradydruleciXXtriv{}   \and
      \SGradydruleciXXsuccU{}  \and
      \SGradydruleciXXsucc{}   \and      
      \inferrule* [flushleft] {
        { 
          \begin{array}{llllllllllllll}
                                         & [[G |- A1 ~ A]]        &                    \\
        [[G |- t => t' : ?]]             & [[G |- A2 ~ A]]        & [[t''1 = (c1 t'1)]]\\[1px]
        [[G |- t1 => t'1 : A1]]          & [[caster(A1,A) = c1]]  & [[t''2 = (c2 t'2)]]\\[1px]
        [[G, x : Nat |- t2 => t'2 : A2]] & [[caster(A2,A) = c2]]  & [[t'' = (unbox Nat t')]]
          \end{array}
        }
      }{[[G |- (case t of 0 -> t1, (succ x) -> t2) => (case t'' of 0 -> t''1, (succ x) -> t''2) : A]]}
      \and
      \inferrule* [right=] {
        {
          \begin{array}{llllllllll}
                                             & [[G |- A1 ~ A ]]     \\
            [[G |- t => t' : Nat]]           & [[G |- A2 ~ A]]      \\
            [[G |- t1 => t'1 : A1]]          & [[caster(A1,A) = c1]] & [[t''1 = (c1 t'1)]]\\[1px]
            [[G, x : Nat |- t2 => t'2 : A2]] & [[caster(A2,A) = c2]] & [[t''2 = (c2 t'2)]]\\[1px]
          \end{array}
        }
      }{[[G |- (case t of 0 -> t1, (succ x) -> t2) => (case t' of 0 -> t''1, (succ x) -> t''2) : A]]}
      \and
      \SGradydruleciXXpair{}   \and
      \SGradydruleciXXfstU{}   \and
      \SGradydruleciXXfst{}    \and
      \SGradydruleciXXsndU{}   \and
      \SGradydruleciXXsnd{}    \and
      \SGradydruleciXXlam{}    \and
      \SGradydruleciXXappU{}   \and
      \SGradydruleciXXapp{} 
    \end{mathpar}
  \end{mdframed}
  \caption{Cast Insertion Algorithm}
  \label{fig:cast-insert}
\end{figure}
The cast insertion judgment is denoted by $[[G |- t1 => t2 : A]]$
which is read as ``the Surface Grady program $[[t1]]$ of type $[[A]]$
is translated into the Core Grady program $[[t2]]$ of type $[[A]]$ in
context $[[G]]$.''  This algorithm is type directed, and is dependent
on the partial metafunction $<<caster(A,B)>>$ that constructs the
casting morphism -- in Core Grady -- of type $[[A -> B]]$:
\begin{center}
  \begin{math}\small
    \begin{array}{lll}
      <<caster(A, A)>> = <<\x:A.x>>\\
      <<caster(A,?)>> = <<box A>>\\
      <<caster(?,B)>> = <<unbox B>>\\
      <<caster((A1 x B1),(A2 x B2))>> = <<caster(A1, A2) XX caster(B1, B2) >>\\
      <<caster((A1 -> B1),(A2 -> B2))>> = <<caster(A2, A1) -> caster(B1, B2) >>\\
    \end{array}    
  \end{math}
\end{center}
The previous definition uses the following derivable functor rules:
\begin{center}
  \begin{math}\small
    \begin{array}{lll}
      $$\mprset{flushleft}
      \inferrule* [right=] {
        <<G |- t1 : A1 -> A2>> \\ <<G |- t2 : B1 -> B2>>
      }{<<G |- t1 XX t2 : (A1 x B1) -> (A2 x B2)>>}
      \\[10px]
      $$\mprset{flushleft}
      \inferrule* [right=] {
        <<G |- t1 : A2 -> A1>> \\ <<G |- t2 : B1 -> B2>>        
      }{<<G |- t1 -> t2 : (A1 -> B1) -> (A2 -> B2)>>}
    \end{array}
  \end{math}
\end{center}
They are defined as follows:
\begin{center}
  \begin{math}
    \small
    \begin{array}{lll}
      <<t1 XX t2>> = <<\x:A1 x B1.(t1 (fst x), t2 (snd x))>>\\
      <<t1 -> t2>> = <<\x:A1 -> B1.\y:A2.H(t2 (x (t1 y)))>>
    \end{array}
  \end{math}
\end{center}
\noindent
The definition of $<<caster(A,B)>>$ is based on the definition of type
consistency.
\begin{lemma}[Type Consistency and Caster]
  \label{lemma:type_consistency_and_caster}
  If $[[G |- A ~ B]]$, then $<<G |- caster(A,B) : A -> B>>$.
\end{lemma}
\begin{proof}
  This proof holds by induction on $[[G |- A ~ B]]$, but is vary
  routine, and so we omit its proof.
\end{proof}
\noindent
Notice that for each typing rule that uses one of the metafunctions
from Figure~\ref{fig:syntax-surface-grady} there are two cast
insertion rules corresponding to the typing rule.

The cast insertion algorithm is designed around where explict casts
need to be inserted.  This is accomplished by case splitting on the
input to each metafunction from Figure~\ref{fig:syntax-surface-grady}
resulting in two rules per elimination rule.  The first is the case
where the input to the metafunction is $[[?]]$, and the second is the
case where the input to the metafunction is a type of the appropriate
structure.  For example, in the case of the elimination rule for
function application, $\SGradydruleTXXappName{}$, from
Figure~\ref{fig:typing-surface-grady}, there are two cast insertion
rules, the last two in Figure~\ref{fig:cast-insert}, where the first
is when the type of the function, $[[t1]]$, is $[[?]]$, and the second
when the type of $[[t1]]$ is an arrow type.  The former requires the
type $[[?]]$ to be split into $[[? -> ?]]$ using $[[unbox (? -> ?)]]$,
and a casting morphism to cast the argument to the appropriate input
type.  The second cast insertion rule only needs to cast the argument
type, because $[[t1]]$ already has a function type.

The cast insertion algorithm preserves the type of the program.
\begin{lemma}[Cast Insertion Preserves the Type]
  \label{lemma:cast_insertion_preserves_the_type}
  If $\,[[G |- t1 : A]]$ and $[[G |- t1 => t2 : A]]$, then $<<G |- t2 : A>>$.
\end{lemma}
\begin{proof}
  This proof holds by induction on $[[G |- t1 : A]]$ which will
  determine which of the cast insertion rules need to be considered.
  At that point, a case split over the input to any metafunctions from
  Figure~\ref{fig:syntax-surface-grady} used in the typing rule may be
  necessary.  We omit the proof in the interest of brevity.
\end{proof}
\noindent
Finally, cast insertion also plays a role when interpreting Surface
Grady into the categorical model.  The next section gives the details.


\section{Interpreting Surface and Core Grady in the Model}
\label{sec:interpreting_surface_grady_in_the_model}

Interpreting a programming language into a categorical model requires
three steps.  First, the types are interpreted as objects.  Then
programs are interpreted as morphisms in the category, but this is a
simplification.  Every morphism, $f$, in a category has a source
object and a target object, we usually denote this by $f : A \mto B$.
Thus, in order to interpret programs as morphisms the program must
have a source and target.  So instead of interpreting raw terms as
morphisms we interpret terms in their typing context.  That is, we
must show how to interpret every $[[G |- t : A]]$ as a morphism $[[t]]
: \interp{[[G]]} \mto \interp{[[A]]}$.  The third step is to show that
whenever one program reduces to another their interpretations are
isomorphic in the model. This means that whenever $<<G |- t1 : A>>$,
$<<G |-t2 : A>>$, and $<<t1 ~> t2>>$, then $<< [|t1|] >> = << [|t2|] >>
: << [| G |] --> [| A |] >>$.  The goal of this section is to prove
these two facts for Surface Grady and Core Grady.  This section
heavily depends on Section~\ref{subsec:the_categorical_model},
Section~\ref{sec:core_grady}, and Section~\ref{sec:surface_grady}.

First, we must give the interpretation of types and contexts, but this
interpretation is obvious, because we have been making sure to match
the names of types and objects throughout this paper.
\begin{definition}
  \label{def:interpretation-of-gradual-types}
  Suppose $(\cat{T}, \cat{C}, ?, \T, \split, \squash, \bx, $\\ $ \unbox,
  \error)$ is a gradual $\lambda$-model.  Then we define the
  interpretation of types into $\cat{C}$ as follows:
  \[
  \begin{array}{cccccccc}
    \begin{array}{cccccc}
      \begin{array}{lll}
      [[ [| ? |] ]] = [[?]]\\
      [[ [| Unit |] ]] = [[Unit]]\\
      [[ [| Nat |] ]] = [[Nat]]
      \end{array}      
    \end{array}
    & \quad & 
    \begin{array}{lll}
      [[ [| A1 -> A2 |] ]] = [[ [| A1 |] -> [| A2 |] ]]\\
      [[ [| A1 x A2 |] ]] = [[ [| A1 |] x [| A2 |] ]]\\
      \\
    \end{array}
  \end{array}
  \]
  We extend this interpretation to typing contexts as follows:
  \[
  \begin{array}{lll}
    \begin{array}{lll}
      [[ [| . |] ]] = 1      
    \end{array}
    & \quad &
    \begin{array}{lll}
      [[ [| G , x : A |] ]] = [[ [| G |] ]] \times [[ [| A |] ]]
    \end{array}
  \end{array}
  \]
\end{definition}
\noindent Throughout the remainder of this paper we will drop the
interpretation symbols around types.

Before we can interpret the typing rules of Surface and Core Grady we
must show how to interpret the consistency relation from
Figure~\ref{fig:syntax-surface-grady}.  These will correspond to
casting morphisms (Definition~\ref{def:casting-mor}).
\begin{lemma}[Type Consistency in the Model]
  \label{lemma:type_consistency_in_the_model}
  Suppose $(\cat{T}, $ $ \cat{C}, ?, \T, \split, \squash, \bx,
  \unbox, \error)$ is a gradual $\lambda$-model, and $[[G |- A ~ B]]$ for
  some types $[[A]]$ and $[[B]]$.  Then there are two casting
  morphisms $c_1 : [[ A --> B ]]$ and $c_2 : [[ B --> A ]]$.
\end{lemma}
\begin{proof}
This proof holds by induction on the form $[[G |- A ~ B]]$ using the
morphisms $<<Box A>> : [[A --> ?]]$ and $<<Unbox A>> : [[? --> A]]$
from Lemma~\ref{lemma:casting_to_?}.  Please see
Appendix~\ref{subsec:proof_of_type_consistency_in_the_model} for the
complete proof.
\end{proof}
\noindent
Showing that both $c_1$ and $c_2$ exist corresponds to the fact that
$[[G |- A ~ B]]$ is symmetric.  

At this point we have everything we need to show our main result which
is that typing in both Surface and Core Grady, and evaluation in Core
Grady can be interpreted into the categorical model.  The
interpretation of terms used in the following proofs is summarized in
Figure~\ref{fig:interp-terms}.
\begin{figure}
  \small  
    \begin{center}
      \begin{math}
          \begin{array}{|l|}
            \hline
            [[ [| G1 , x : Ai , G2 |- x : Ai |] ]] = \pi_i\\
            \hline
            [[ [| G |- triv : Unit |] ]] = [[triv]]_{[[ [| G |] ]]}\\
            \hline
            [[ [| G |- 0 : Nat |] ]] = [[triv]]_{[[ [| G |] ]]};\z\\
            \hline
            [[ [| G |- succ t : Nat |] ]] = [[ [| t |] ]];c;[[succ]]\\
            \,\,\,\,\,\,\,\,\,\,\,\text{ where } [[G |- t : A]]\text{ and }c : [[A --> Nat]]\\
            \hline
            [[ [| G |- case t of 0 -> t1, (succ x) -> t2 : A |] ]]\\
            \,\,\,\,\,\,\,\,\,\,\,\,\,\,\,\,\,\,\,\,\,\,\,\,\,\,\,\,\,\,\,\,\,\,\,\,\,\,\,\,           
     = \langle \id_{[[ [| G |] ]]}, [[ [| t |] ]];[[c1]] \rangle;\Case_{[[ [| G |] ]], [[ [| A |] ]]}\langle [[ [|t1|] ]];[[c2]], [[ [| t2 |] ]];[[c3]] \rangle\\
     \,\,\,\,\,\,\,\,\,\,\,\text{ where } [[G |- t : C]], [[G |- t1 : A1]], [[G |- t2 : A2]],\\
     \,\,\,\,\,\,\,\,\,\,\,\,[[c1]] : [[C --> Nat]], [[c2]] : [[A1 --> A]], \text{ and } [[c3]] : [[A2 --> A]]\\
            \hline
            [[ [| G |- (t1,t2) : A1 x A2 |] ]] = \langle [[ [| t1 |] ]] , [[ [| t2 |] ]] \rangle\\
            \,\,\,\,\,\,\,\,\,\,\,\text{ where } [[G |- t1 : A1]]\text{ and } [[G |- t2 : A2]]\\
            \hline
            [[ [| G |- fst t : A1 |] ]] = [[ [| t |] ]];c;\pi_1\\
            \,\,\,\,\,\,\,\,\,\,\,\text{ where } [[G |- t : B]]\text{ and }c : [[B --> (A1 x A2)]]\\
            \hline
            [[ [| G |- snd t : A2 |] ]] = [[ [| t |] ]];c;\pi_2\\           
            \,\,\,\,\,\,\,\,\,\,\,\text{ where } [[G |- t : B]]\text{ and }c : [[B --> (A1 x A2)]]\\
            \hline
            [[ [| G |- \x:A.t : A -> B |] ]] = \curry{[[ [| t |] ]]}\\
            \,\,\,\,\,\,\,\,\,\,\,\text{ where } [[G, x : A |- t : B]]\\
            \hline
            [[ [| G |- t1 t2 : B1 |] ]] = \langle [[ [| t1 |] ]];c_1, [[ [| t2 |] ]];c_2\rangle;\app_{[[A1]],[[B1]]}\\
            \,\,\,\,\,\,\,\,\,\,\,\text{ where } [[G |- t1 : C]]\text{, }[[G |- t2 : A2]], \\
            \,\,\,\,\,\,\,\,\,\,\,\,c_1 : [[C --> (A1 -> B1)]],\text{ and } c_2 : [[A2 --> A1]]\\
            \hline
            << [| G |- box A : A -> ? |] >> = [[triv]]_{[[ [| G |] ]]};\curry{<<Box A>>}\\
            \hline
            << [| G |- unbox A : ? -> A |] >> = [[triv]]_{[[ [| G |] ]]};\curry{<<Unbox A>>}\\            
            \hline
          \end{array}
      \end{math}
    \end{center}
  \caption{Interpretation of Terms}
  \label{fig:interp-terms}
\end{figure}
We only summarize the interpretation of the Surface Grady programs and
just $[[box]]$ and $[[unbox]]$ in Core Grady, because the
interpretation of Core Grady is equivalent to the interpretation of
Surface Grady where all casting morphisms have been replaced with the
identity morphism.

\begin{theorem}[Interpretation of Typing]
  \label{thm:interpretation_of_typing}
  Suppose $(\cat{T}, \cat{C}, ?, \T,$ $\split, \squash, \bx, \unbox, \error)$
  is a gradual $\lambda$-model. If $[[G |- t : A]]$ or $<<G |- t : A>>$, then
  there is a morphism $[[ [| t |] ]] : [[ [| G |] --> A ]]$ in $\cat{C}$.  
\end{theorem}
\begin{proof}
  Both parts of the proof hold by induction on the form of the assumed
  typing derivation, and uses most of the results we have developed up
  to this point.  Please see
  Appendix~\ref{subsec:proof_of_interpretation_of_types} for the
  complete proof.
\end{proof}

\begin{theorem}[Interpretation of Evaluation]
  \label{thm:interpretation_of_evaluation}
  Suppose $(\cat{T}, \cat{C}, ?,$\\$\T, \split, \squash, \bx, \unbox, \error)$
  is a gradual $\lambda$-model.  If\\$<<G |- t1 : A>>$, $<<G |- t2 : A>>$, and $<<t1 ~> t2>>$, then
  $[[ [| t1 |] ]] = [[ [| t2 |] ]] : [[ [| G |] --> A]]$.
\end{theorem}
\begin{proof}
  This proof holds by induction on the form of $<<t1 ~> t2>>$, and
  uses Theorem~\ref{thm:interpretation_of_typing},
  Lemma~\ref{lemma:type_consistency_in_the_model},
  Corollary~\ref{corollary:type_consist_coro}, and
  Lemma~\ref{lemma:inversion_for_typing}.  Please see
  Appendix~\ref{subsec:proof_of_interpretation_of_evaluation} for the
  complete proof.
\end{proof}

One can see a direct connection between the proof of interpretation of
typing (Theorem~\ref{thm:interpretation_of_typing}) and the cast
insertion algorithm (Figure~\ref{fig:cast-insert}).  During the proof
-- summarized in Figure~\ref{fig:interp-terms} -- we construct casting
morphisms from type consistency which is essentially the semantic
equivalent to $<<caster(A, B)>>$.
% section interpreting_surface_grady_in_the_model (end)
