\section{Proofs}
\label{sec:proofs}

\subsection{Proof of Lifted Retract (Lemma~\ref{lemma:lifted_retract})}
\label{subsec:proof_of_lifted_retract}
We only show that $<<lbox A>>;<<lunbox A>> = \id_A$, because the case
when a dynamic type error is raised is similar using the fact that
$<<A>>$ and $<<B>>$ must have the same skeleton or one could not
compose $<<lbox A>>$ and $<<lunbox B>>$. This implies that $<<A>>$ and
$<<B>>$ only differ at an atomic type.

\ \\
\noindent
This is a proof by induction on the form of $<<A>>$.

\begin{itemize}
\item[] Case. Suppose $<<A>>$ is atomic.  Then:
  \[
    <<lbox A>>;<<lunbox A>> = <<box A>>;<<unbox A>> = \id_{<<A>>}
    \]
    
  \item[] Case. Suppose $<<A>>$ is $<<?>>$.  Then:
    \[
    \begin{array}{lll}
      <<lbox A>>;<<lunbox A>> & = & <<lbox ?>>;<<lunbox ?>>\\
      & = & \id_{<<?>>};\id_{<<?>>}\\
      & = & \id_{<<?>>}\\
      & = & \id_{<<A>>}
    \end{array}
    \]

  \item[] Case. Suppose $<<A>> = <<A1 -> A2>>$.  Then:
    \[\small
    \begin{array}{lll}
      <<lbox A>>;<<lunbox A>>
      & = & <<lbox (A1 -> A2)>>;<<lunbox (A1 -> A2)>>\\
      & = & (<<lunbox A1>> \to <<lbox A2>>);(<<lbox A1>> \to <<lbox A2>>)\\
      & = & (<<lbox A1>>;<<lunbox A1>>) \to (<<lbox A2>>;<<lunbox A2>>)\\
    \end{array}
    \]
    By two applications of the induction hypothesis we know the
    following:
    \[
    \begin{array}{lll}
      <<lbox A1>>;<<lunbox A1>> = \id_{<<A1>>} & \text{ and } & <<lbox A2>>;<<lunbox A2>> = \id_{<<A2>>}
    \end{array}
    \]
    Thus, we know the following:
    \[
    \begin{array}{lll}
      (<<lbox A1>>;<<lunbox A1>>) \to (<<lbox A2>>;<<lunbox A2>>)
      & = & \id_{<<A1>>} \to \id_{<<A2>>}\\
      & = & \id_{<<A1 -> A2>>}\\
      & = & \id_{<<A>>}
    \end{array}
    \]

  \item[] Case. Suppose $<<A>> = <<A1 x A2>>$.  Then:
    \[\small
    \begin{array}{lll}
      <<lbox A>>;<<lunbox A>>
      & = & <<lbox (A1 x A2)>>;<<lunbox (A1 x A2)>>\\
      & = & (<<lbox A1>> \times <<lbox A2>>);(<<lunbox A1>> \times <<lbox A2>>)\\
      & = & (<<lbox A1>>;<<lunbox A1>>) \times (<<lbox A2>>;<<lunbox A2>>)\\
    \end{array}
    \]
    By two applications of the induction hypothesis we know the
    following:
    \[
    \begin{array}{lll}
      <<lbox A1>>;<<lunbox A1>> = \id_{<<A1>>} & \text{ and } & <<lbox A2>>;<<lunbox A2>> = \id_{<<A2>>}
    \end{array}
    \]
    Thus, we know the following:
    \[
    \begin{array}{lll}
      (<<lbox A1>>;<<lunbox A1>>) \times (<<lbox A2>>;<<lunbox A2>>)
      & = & \id_{<<A1>>} \times \id_{<<A2>>}\\
      & = & \id_{<<A1 x A2>>}\\
      & = & \id_{<<A>>}
    \end{array}
    \]
\end{itemize}
% subsection proof_of_lifted_retract (end)

\subsection{Proof of Lemma~\ref{lemma:S_is_faithful}}
\label{subsec:proof_of_S_is_faithful}
We must show that the function
\[ \S_{A,B} : \Hom{C}{A}{B} \mto \Hom{S}{\S A}{\S B} \]
is injective.

So suppose $f \in \Hom{C}{A}{B}$ and $g \in \Hom{C}{A}{B}$ such that
$\S f = \S g : \S A \mto \S B$.  Then we can easily see that:
\[
\begin{array}{lll}
  \S f & = & <<lunbox A>>;f;<<lbox B>> \\
  & = & <<lunbox A>>;g;<<lbox B>>\\
  & = & \S g\\
\end{array}
\]
But, we have the following equalities:
\[\small
\setlength{\arraycolsep}{1px}
\begin{array}{rll}
  <<lunbox A>>;f;<<lbox B>> & = & <<lunbox A>>;g;<<lbox B>>\\
  <<lbox A>>;<<lunbox A>>;f;<<lbox B>>;<<lunbox B>> & = & <<lbox A>>;<<lunbox A>>;g;<<lbox B>>;<<lunbox B>>\\
  \id_A;f;<<lbox B>>;<<lunbox B>> & = & \id_A;g;<<lbox B>>;<<lunbox B>>\\
  \id_A;f;\id_B & = & \id_A;g;\id_B\\
  f & = & g\\
\end{array}
\]
The previous equalities hold due to
Lemma~\ref{lemma:lifted_retract}.
% subsection proof_of_S_is_faithful (end)

\subsection{Proof of Type Consistency in the Model (Lemma~\ref{lemma:type_consistency_in_the_model})}
\label{subsec:proof_of_type_consistency_in_the_model}
This is a proof by induction on the form of $[[A ~ B]]$.
\begin{description}
\item Case: $\quad \SGradydruleCXXRefl{}$\\
  \ \\
  \noindent  
  Choose $c_1 = c_2 = \id_A : <<A --> A>>$.
  \ \\
\item Case: $\quad \SGradydruleCXXBox{}$\\
  \ \\
  \noindent
  Choose $c_1 = <<Box A>> : <<A --> ?>>$ and $c_2 = <<Unbox A>> : <<? -> A>>$.
  \ \\
\item Case: $\quad \SGradydruleCXXUnbox{}$\\
  \ \\
  \noindent
  Choose $c_1 = <<Unbox A>> : <<? --> A>>$ and $c_2 = <<Box A>> : <<A -> ?>>$.
  \ \\
\item Case: $\quad \SGradydruleCXXArrow{}$\\
  \ \\
  \noindent  
  By the induction hypothesis there exists four casting morphisms
  $c'_1 : <<A1 --> A2>>$, $c'_2 : <<A2 --> A1>>$, $c'_3 : <<B1 --> B2>>$,
  and $c'_4 : <<B2 --> B1>>$.  Choose
  $c_1 = c'_2 \to c'_3 : <<(A1 -> B1) --> (A2 -> B2)>>$
  and
  $c_2 = c'_1 \to c'_4 : <<(A2 -> B2) --> (A1 -> B1)>>$.
  \ \\
\item Case: $\quad \SGradydruleCXXProd{}$\\
  \ \\
  \noindent  
  By the induction hypothesis there exists four casting morphisms
  $c'_1 : <<A1 --> A2>>$, $c'_2 : <<A2 --> A1>>$, $c'_3 : <<B1 --> B2>>$,
  and $c'_4 : <<B2 --> B1>>$.
  Choose
  $c_1 = c'_1 \times c'_3 : <<H(A1 x B1) --> H(A2 x B2)>>$
  and
  $c_2 = c'_2 \times c'_4 : <<H(A2 x B2) --> H(A1 x B1)>>$.
\end{description}
% subsection proof_of_type_consistency_in_the_model (end)

\subsection{Proof of Interpretation of Types Theorem~\ref{thm:interpretation_of_typing}}
\label{subsec:proof_of_interpretation_of_types}
First, we show how to interpret the rules of Surface Grady and then Core Grady.
This is a proof by induction on $[[G |- t : A]]$.
\begin{description}
\item Case: $\quad \SGradydruleTXXvarP{}$\\
    \\
    \noindent
    Suppose without loss of generality that $[[ [| G |] ]] = [[A1]]
    \times \cdots \times [[Ai]] \times \cdots \times [[Aj]]$ where
    $[[Ai]] = [[A]]$.  We know that $j > 0$ or the assumed typing
    derivation would not hold.  Then take
    $[[ [| x |] ]] = \pi_i : [[ [| G |] --> A]]$.

    \ \\
  \item Case: $\quad \SGradydruleTXXunitP{}$\\
    \\
    \noindent     
    Take $[[ [| triv |] ]] = [[triv]]_{[[ [| G |] ]]} : [[ [| G |] --> Unit ]]$ where $[[triv]]_{[[ [| G |] ]]}$
    is the unique terminal arrow that exists because $\cat{C}$ is cartesian closed.

    \ \\
  \item Case: $\quad \SGradydruleTXXzeroP{}$\\
    \\
    \noindent
    Take $[[ [| 0 |] ]] = [[triv]]_{[[ [| G |] ]]};\z : [[ [| G |] --> Nat]]$
    where $\z : [[Unit --> Nat]]$ exists because $\cat{C}$
    has an NRNO.

    \ \\
  \item Case: $\quad \SGradydruleTXXsucc{}$\\
    \\
    \noindent
    First, by the induction hypothesis there is a morphism $[[ [| t |] ]] : [[ [| G |] --> A]]$.
    Now we have two cases to consider, one when $<<A>> = <<?>>$ and one when $<<A>> = <<Nat>>$.
    Consider the former.  Then interpret
    $<< [| succ t |] >> = << [| t |] >>;[[ unbox Nat ]];[[succ]] : [[ [| G |] --> Nat]]$ where
    $[[succ]] : [[Nat --> Nat]]$ exists because $\cat{C}$ has an NRNO.  Similarly,
    when $<<A>> = <<Nat>>$, 
    $<< [| succ t |] >> = << [| t |] >>;[[succ]] : [[ [| G |] --> Nat]]$.

    \ \\
  \item Case: $\quad {    
    \inferrule* [flushleft,right=$\mathsf{Nat}_e$] {
      {
        \begin{array}{lll}
          [[G |- t : C]]            & [[nat(C) = Nat]]\\
          [[G |- t1 : A1]]          & [[A1 ~ A]]\\
          [[G, x : Nat |- t2 : A2]] & [[A2 ~ A]]
        \end{array}
      }
    }{[[G |- case t of 0 -> t1, (succ x) -> t2 : A]]}
    }$\\
    \\
    \noindent
    By three applications of the induction hypothesis we have the following morphisms:
    \begin{center}
      \begin{math}
        \begin{array}{lll}
          [[ [| t |] ]] : [[ [|G|] --> C]]\\
          [[ [| t1 |] ]] : [[ [|G|] --> A1]]\\
          [[ [| t2 |] ]] : [[ H([|G|] x Nat) --> A2]]
        \end{array}
      \end{math}
    \end{center}
    In addition, we know $[[A1 ~ A]]$ and $[[A2 ~ A]]$ by
    assumption, and hence, by type consistency in the model
    (Lemma~\ref{lemma:type_consistency_in_the_model}) we know there are
    casting morphisms $[[c1]] : [[A1 --> A]]$ and $[[c2 : A2 --> A]]$.
    Now every gradual $\lambda$-model has an NRNO
    (Definition~\ref{def:SNNO},
    Definition~\ref{def:gradual-lambda-model}), and so, there is a
    unique morphism:
    \begin{center}
      \begin{math}
        \Case_{[[ [| G |] ]],A}\langle [[ [| t1 |];c1 ]],[[ [| t2 |];c2 ]] \rangle : [[ H([|G|] x Nat) --> A]]
      \end{math}
    \end{center}
    
    At this point we have two cases to consider: one when $[[C]] = [[?]]$ and one when $[[C]] = [[Nat]]$.  Consider the former.
    Then we have the following:
    \begin{center}
      \begin{math}\small
        \begin{array}{lll}
          [[ [| case t of 0 -> t1, (succ x) -> t2 |] ]] \\
          = \langle \id_{[[ [| G |] ]]}, [[ [| t |];unbox Nat ]] \rangle;\Case_{[[ [| G |] ]],A}\langle [[ [| t1 |];c1 ]],[[ [| t2 |];c2 ]] \rangle\\
          : [[ [| G |] --> A ]]
        \end{array}
      \end{math}
    \end{center}
    In the second case we have the following:
    \begin{center}
      \begin{math}\small
        \begin{array}{lll}
          [[ [| case t of 0 -> t1, (succ x) -> t2 |] ]] \\
          = \langle \id_{[[ [| G |] ]]}, [[ [| t |] ]] \rangle;\Case_{[[ [| G |] ]],A}\langle [[ [| t1 |];c1 ]],[[ [| t2 |];c2 ]] \rangle \\
          : [[ [| G |] --> A ]]
        \end{array}
      \end{math}
    \end{center}

    \ \\
  \item Case: $\quad \SGradydruleTXXpair{}$\\
    \\
    \noindent
    By two applications of the induction hypothesis there are two morphisms
    $[[ [| t1 |] ]] : [[ [| G |] --> A]]$ and $[[ [| t2 |] ]] : [[ [| G |] --> B]]$.
    Then using the fact that $\cat{C}$ is cartesian we take
    $[[ [| (t1 , t2) |] ]] = \langle [[ [| t1 |] ]] , [[ [| t2 |] ]] \rangle : [[ [| G |] --> H(A x B)]]$.

    \ \\
  \item Case: $\quad \SGradydruleTXXfst{}$\\
    \\
    \noindent
    First, by the induction hypothesis there is a morphism $[[ [| t |] ]] : [[ [| G |] --> B]]$.
    Now we have two cases to consider, one when $<<B>> = <<?>>$ and one when $<<B>> = <<A1 x A2>>$
    for some types $<<A1>>$ and $<<A2>>$.  Consider the former.  We then know that it must
    be the case that $<<A1 x A2>> = <<? x ?>>$.  Thus, we obtain the following interpretation
    $[[ [| fst t |] ]] = [[ [| t |] ]];[[unbox (? x ?)]];\pi_1 : [[ [| G |] --> ?]]$.  Similarly,
    when $<<B>> = <<A1 x A2>>$, then
    $[[ [| fst t |] ]] = [[ [| t |] ]];\pi_1 : [[ [| G |] --> A1]]$.

    \ \\
  \item Case: $\quad \SGradydruleTXXsnd{}$\\
    \\
    \noindent
    First, by the induction hypothesis there is a morphism $[[ [| t |] ]] : [[ [| G |] --> B]]$.
    Now we have two cases to consider, one when $<<B>> = <<?>>$ and one when $<<B>> = <<A1 x A2>>$
    for some types $<<A1>>$ and $<<A2>>$.  Consider the former.  We then know that it must
    be the case that $<<A1 x A2>> = <<? x ?>>$.  Thus, we obtain the following interpretation
    $[[ [| snd t |] ]] = [[ [| t |] ]];[[unbox (? x ?)]];\pi_2 : [[ [| G |] --> ?]]$.  Similarly,
    when $<<B>> = <<A1 x A2>>$, then
    $[[ [| snd t |] ]] = [[ [| t |] ]];\pi_2 : [[ [| G |] --> A2]]$.

    \ \\
  \item Case: $\quad \SGradydruleTXXlam{}$\\
    \\
    \noindent    
    By the induction hypothesis there is a morphism $[[ [| t |] ]] :
    [[ H([| G |] x A) --> B]]$.  Then take $[[ [| \x : A.t |] ]] =
    \curry {[[ [| t |] ]]} : [[ [| G |] --> (A -> B)]]$, where
    \begin{center}
      \begin{math}
        \mathsf{curry} : \Hom{C}{X \times Y}{Z} \mto \Hom{C}{X}{Y \to Z}
      \end{math}
    \end{center}
    exists because $\cat{C}$ is closed.    
   
    \ \\
  \item Case: $\quad {
    \inferrule* [flushleft,right=$\to_e$] {
      [[G |- t1 : C]] \\ \,\,\,[[A2 ~ A1]]\\\\    
      [[G |-t2 : A2]] \\ [[fun(C) = A1 -> B1]]
    }{[[G |- t1 t2 : B1]]}
  }$\\
    \\
    \noindent
    By the induction hypothesis there are two morphisms
    $[[ [| t1 |] ]] : [[ [| G |] --> C ]]$ and
    $[[ [| t2 |] ]] : [[ [| G |] --> A2 ]]$.  In addition, by assumption we know that
    $[[A2 ~ A1]]$, and hence, by type consistency in the model (Lemma~\ref{lemma:type_consistency_in_the_model})
    there are casting morphisms $c_1 : [[A2 --> A1]]$ and $c_2 : [[A1 --> A2]]$.  We have two cases to consider,
    one when $[[C]] = [[?]]$ and one when $[[C]] = [[A1 -> B1]]$.  Consider
    the former. Then we have the interpretation:
    \begin{center}
      \begin{math}
        [[ [| t1 t2 |] ]] = \langle [[ [| t1 |] ]];[[unbox (? -> ?)]], [[ [| t2 |] ]];c_1 \rangle;\app_{[[A1]],[[B1]]} : [[ [| G |] --> B1 ]]
      \end{math}
    \end{center}
    Similarly, for the case when $[[C]] = [[A1 -> B1]]$ we have the interpretation:
    \begin{center}
      \begin{math}
        [[ [| t1 t2 |] ]] = \langle [[ [| t1 |] ]], [[ [| t2 |] ]];c_1 \rangle;\app_{[[A1]],[[B1]]} : [[ [| G |] --> B1 ]]
      \end{math}
    \end{center}
    Note that $\app_{[[A1]],[[B1]]} : [[ H((A1 -> B1) x A1) --> B1]]$ exists because the
    model is cartesian closed.
  \end{description}

  Next we turn to Core Grady, but we do not show every rule, because
  it is similar to what we have already shown above except without
  casting morphism, and so we only show the case for the $[[box]]$ and
  $[[unbox]]$ rules, and the error rule.

  The first two cases use the well-known bijection:  
  \begin{center}
    \begin{math}
      \begin{array}{lll}
        \Hom{C}{A}{B} & \cong & \Hom{C}{[[Unit x A]]}{B}\\
                      & \cong & \Hom{C}{[[Unit]]}{[[A -> B]]}
      \end{array}
    \end{math}
  \end{center}
  When $f \in \Hom{C}{A}{B}$, then we denote by $\curry{f}$, the
  morphism $\curry{f} \in \Hom{C}{[[Unit]]}{[[A -> B]]}$.
\begin{description}
\item Case: $\quad \CGradydruleTXXBox{}$\\
  \\
  \noindent
  We have the following interpretation:
  \begin{center}
    \begin{math}
      << [| box A |] >> = [[triv]]_{[[ [| G |] ]]};\curry{<<Box A>>} : [[ [| G |] --> (A -> ?) ]]
    \end{math}
  \end{center}
  \ \\
\item Case: $\quad \CGradydruleTXXUnbox{}$\\
  \\
  \noindent
  We have the following interpretation:
  \begin{center}
    \begin{math}
      \begin{array}{lll}
        << [| unbox A |] >> \\
        = [[triv]]_{[[ [| G |] ]]};\curry{<<Unbox A>>}\\
        : [[ [| G |] --> (? -> A)]]
      \end{array}
    \end{math}
  \end{center}

\item Case: $\quad \CGradydruleTXXerror{}$\\
  \\
  \noindent
  We have the following interpretation:
  \begin{center}
    \begin{math}
      << [| error A |] >> = <<error>>_{[[ [| G |] ]],[[A]]} : [[ [| G |] --> A]]
    \end{math}
  \end{center}
\end{description}
% subsection proof_of_interpretation_of_types (end)

\subsection{Proof of Interpretation of Evaluation (Theorem~\ref{thm:interpretation_of_evaluation})}
\label{subsec:proof_of_interpretation_of_evaluation}
This proof requires the following corollary to
Lemma~\ref{lemma:type_consistency_in_the_model}, and the following
lemma called inversion for typing.
\begin{corollary}
  \label{corollary:type_consist_coro}
  Suppose $(\cat{T}, \cat{C}, ?, \T,$ $\split, \squash, \bx, \unbox)$ is
  a gradual $\lambda$-model.  Then we know the following:
  \begin{enumR}
  \item If $[[A ~ A]]$, then $c_1 = c_2 = \id_{[[A]]} : [[A --> A]]$.

  \item If $[[A ~ ?]]$, then there are casting morphisms:
    \[
    \begin{array}{lll}
      c_1 & = & [[Box A]] : [[A --> ?]]  \\    
      c_2 & = & [[Unbox A]] : [[? --> A]]
    \end{array}
    \]

    \item If $[[? ~ A]]$, then there are casting morphisms:
    \[
    \begin{array}{lll}
      c_1 & = & [[Unbox A]] : [[? --> A]]\\
      c_2 & = & [[Box A]] : [[A --> ?]]
      \end{array}
    \]
    
  \item If $[[A1 -> B1 ~ A2 -> B2]]$, then there are casting morphisms:
    \[
    \begin{array}{lllll}
      c & = & c_1 \to c_2 : [[(A1 -> B1) --> (A2 -> B2)]]\\
      c' & = & c_3 \to c_4 : [[(A2 -> B2) --> (A1 -> B1)]]
    \end{array}
    \]
    where $c_1 : [[A2 --> A1]]$ and $c_2 : [[B1 --> B2]]$, and $c_3 :
    [[A1 --> A2]]$ and $c_4 : [[B2 --> B1]]$.
    
  \item If $[[A1 x B1 ~ A2 x B2]]$, then there are casting
    morphisms:
    \[
    \begin{array}{lll}
       c & = & c_1 \times c_2 : [[(A1 x B1) --> (A2 x B2)]]\\
      c' & = & c_3 \times c_4 : [[(A2 x B2) --> (A1 x B1)]]
    \end{array}
    \]
    where $c_1 : [[A1 --> A2]]$ and $c_2 : [[B1 --> B2]]$, and $c_3 :
    [[A2 --> A1]]$ and $c_4 : [[B2 --> B1]]$.
  \end{enumR}
\end{corollary}
\begin{proof}
  This proof holds by the construction of the casting morphisms from
  the proof of the previous result, and the fact that the type
  consistency rules are unique for each type.
\end{proof}

\begin{lemma}[Inversion for Typing]
  \label{lemma:inversion_for_typing}
  \begin{enumerate}[label=\roman*.]
  \item[]
  \item If $<<G |- succ t : A>>$, then $<<A>> = <<Nat>>$ and $<<G |- t : Nat>>$.
  \item If $<<G |- case t : Nat of x -> t1, (succ x) -> t2 : A>>$, then $<<G |- t : Nat>>$, $<<G |- t1 : A>>$, and $<<G, x : Nat |- t2 : A>>$.
  \item If $<<G |- (t1, t2) : A>>$, then there are types $<<B>>$ and $<<C>>$, such that, $<<A>> = <<B x C>>$, $<<G |- t1 : B>>$, and $<<G |- t2 : C>>$.
  \item If $<<G |- fst t : A>>$, then there is a type $<<B>>$, such that, $<<G |- t : A x B>>$.
  \item If $<<G |- snd t : A>>$, then there is a type $<<B>>$, such that, $<<G |- t : B x A>>$.
  \item If $<<G |- \x:A.t : A>>$, then there are types $<<B>>$ and $<<C>>$, such that, $<<A>> = <<B -> C>>$ and $<<G, x : B |- t : C>>$.
  \item If $<<G |- t1 t2 : A>>$, then there is a type $<<B>>$, such that, $<<G |- t1 : B -> A>>$ and $<<G |- t2 : B>>$.
  \end{enumerate}
\end{lemma}
\begin{proof}
  Each case of this proof holds trivially by induction on the assumed
  typing derivation, because there is only one typing rule per term
  constructor.
\end{proof}

This proof holds by induction on the form of $<<t1 ~> t2>>$ with an
appeal to inversion for typing on $[[G |- t1 : A]]$ and $[[G |- t2 :
    A]]$.  We only show the cases for the retract rules, and the error
rule, because the others are well-known to hold within any cartesian
closed category; see \cite{Lambek:1980} or \cite{Crole:1994}.  We will
routinely use the interpretation given in the proof of
Theorem~\ref{thm:interpretation_of_typing} and summarized in
Figure~\ref{fig:interp-terms} throughout this proof without mention.

The cases to follow will make use of the following result, essentially
the semantic equivalent to an instance of the $\beta$-rule, that holds
in any cartesian closed category:
\begin{center}
  \begin{math} \footnotesize
    \begin{array}{lll}        
      \langle [[triv]]_{[[C]]};\curry{g}, f\rangle;\app_{[[A]],[[B]]}\\
      = \langle [[triv]]_{[[C]]}, f\rangle;(\curry{g} \times \id_A);\app_{[[A]],[[B]]}\\
      = \langle [[triv]]_{[[C]]}, f\rangle;[[snd]];g\\
      = f;g\\
    \end{array}
  \end{math}
\end{center}
where $g : [[A --> B]]$ and $f : [[C --> A]]$.  Note that
$\app_{[[A]],[[B]]} : [[H((A -> B) x A) --> B]]$ exists, because
$\cat{C}$ is a cartesian closed category.
\begin{description}
\item Case: $\quad \CGradydrulerdXXretracT{}$\\

  \ \\
  \noindent
  We know by assumption that $<<G |- unbox A (box A t) : A>>$ and $<<G |- t : A>>$.
  By interpretation for typing (Theorem~\ref{thm:interpretation_of_typing}) and using the above
  equation we obtain the following morphisms:
  \begin{center}
    \begin{math}\footnotesize
      \begin{array}{lll}
        << [| box A t |] >>\\
        = \langle [[triv]]_{[[ [| G |] ]]};\curry{<<Box A>>}, [[ [| t |] ]]\rangle;\app_{[[A]],[[?]]}\\
        = [[ [| t |] ]];<<Box A>>\\
        : [[ [| G |] --> ? ]]\\
        \\
        << [| unbox A (box A t) |] >> \\
        = \langle [[triv]]_{[[ [| G |] ]]};\curry{<<Unbox A>>}, << [| box A t |] >>\rangle;\app_{[[?]],[[A]]} \\
        = << [| box A t |] >>;<<Unbox A>> \\
        = [[ [| t |] ]];<<Box A>>;<<Unbox A>> \\
        : [[ [| G |] --> A ]]\\
      \end{array}
    \end{math}
  \end{center}
  where $[[ [| t |] ]] : [[ [| G |] --> A]]$.  At this point it is
  easy to see that $[[ [| t |] ]];<<Box A>>;<<Unbox A>> = [[ [| t |]
  ]];\id_A = [[ [| t |] ]]$. Thus, we obtain our result.
  

\item Case: $\quad \CGradydrulerdXXretracTE{}$\\

  \ \\
  \noindent
  This case follows similarly to the previous case.  Using the semantic $\beta$-equation given above,
  then we will obtain $[[ [| t |] ]];<<Box B>>;<<Unbox A>> = <<error>>_{[[ [| G |] ]], [[ A ]]}$ using
  the error axioms from the definition of the gradual $\lambda$-model (Definition~\ref{def:gradual-lambda-model}).
  \ \\

\item Case: $\quad \CGradydrulerdXXerror{}$\\

  \ \\
  \noindent
  This case follows from a case analysis over the structure of $<<EC>>$, and then using the error axioms
  from the definition of the gradual $\lambda$-model (Definition~\ref{def:gradual-lambda-model}).
  \ \\  

\end{description}
%% % subsection proof_of_interpretation_of_evaluation (end)

\subsection{Proof of Congruence of Type Consistency Along Type Precision (Lemma~\ref{lemma:congruence_of_type_consistency_along_type_precision})}
\label{subsec:proof_of_congruence_of_type_consistency_along_type_precision_lemma:congruence_of_type_consistency_along_type_precision}

\begin{proof}
  
The proofs of both parts are similar, and so we only show a few
cases of the first part, but the omitted cases follow similarly.

\noindent
\textbf{Proof of part one.} This is a proof by induction on the form
of $[[A1 <= A1']]$.
\begin{description}
\item
  \begin{typeProofCase}
    $$\mprset{flushleft}
      \inferrule* [right=$\SGradydrulePXXUName{}$] {
        \,
      }{[[A1 <= ?]]}
  \end{typeProofCase}

  \noindent
  In this case $[[A1']] = [[?]]$.  Suppose $[[A1 ~ A2]]$.  Then
  it suffices to show that $[[? ~ A2]]$, but this easily follows.
  
\item[]
\item
  \begin{typeProofCase}
    $$\mprset{flushleft}
      \inferrule* [right=$\SGradydrulePXXarrowName{}$] {
        [[A <= C && B <= D]]
      }{[[(A -> B) <= (C -> D)]]}
  \end{typeProofCase}

  \noindent
  In this case $[[A1]] = [[A -> B]]$ and $[[A1']] = [[C -> D]]$.  Suppose
  $[[A1 ~ A2]]$.  Then by inversion for type consistency it must
  be the case that either $[[A2]] = [[?]]$, or
  $[[A2]] = [[A' -> B']]$, $[[A ~ A']]$, and $[[B ~ B']]$.
  
  Consider the former.  Then it suffices to show that $[[A1' ~ ?]]$,
  but this easily follows.

  Consider the case when $[[A2]] = [[A' -> B']]$, $[[A ~ A']]$, and $[[B ~ B']]$.
  It suffices to show that $[[(C -> D) ~ (A' -> B')]]$ which follows from
  $[[A' ~ C]]$ and $[[D ~ B']]$.  Thus, it suffices to show that latter.
  By assumption we know the following:
  \begin{center}
    \begin{tabular}{lll}
      $[[A <= C]]$ and $[[A ~ A']]$\\
      $[[B <= D]]$ and $[[B ~ B']]$
    \end{tabular}
  \end{center}
  Now by two applications of the induction hypothesis we obtain $[[C ~ A']]$
  and $[[D ~ B']]$. By symmetry the former implies $[[A' ~ C]]$ and
  we obtain our result.
\end{description}  
\end{proof}
% subsection proof_of_congruence_of_type_consistency_along_type_precision_lemma:congruence_of_type_consistency_along_type_precision (end)

\subsection{Proof of Gradual Guarantee Part One (Lemma~\ref{lemma:gradual_guarantee_part_one})}
\label{subsec:proof_of_gradual_guarantee_part_one_lemma:gradual_guarantee_part_one}

  This is a proof by induction on $[[G |- t : A]]$.  We only show the
most interesting cases, because the others follow similarly.

\begin{description}
\item[]
\item
  \begin{typeProofCase}
    $$\mprset{flushleft}
    \inferrule* [right=$\SGradydruleTXXvarPName{}$] {
      [[x : A elem G]]
    }{[[G |- x : A]]}
  \end{typeProofCase}

\noindent
In this case $[[t]] = [[x]]$.  Suppose $[[t <= t']]$.  Then
it must be the case that $[[t']] = [[x]]$.  If $[[x : A elem G]]$,
then there is a type $[[A']]$ such that $[[x : A' elem G']]$ and
$[[A <= A']]$.  Thus, choose $[[B]] = [[A']]$ and the result follows.

\item[]
\item 
  \begin{typeProofCase}
    $$\mprset{flushleft}
      \inferrule* [right=$\SGradydruleTXXsuccName{}$] {
        [[G |- t1 : A' && nat(A') = Nat]]
      }{[[G |- succ t1 : Nat]]}
  \end{typeProofCase}

  \noindent
  In this case $[[A]] = [[Nat]]$ and $[[t]] = [[succ t1]]$.  Suppose $[[t <= t']]$ and $[[G <= G']]$.
  Then by definition it must be the case that $[[t']] = [[succ t2]]$ where $[[t1 <= t2]]$.
  By the induction hypothesis $[[G' |- t2 : B']]$ where $[[A' <= B']]$.  Since $[[nat(A') = Nat]]$
  and $[[A' <= B']]$, then it must be the case that $[[nat(B') = Nat]]$ by Lemma~\ref{lemma:fun_type_pre}.
  At this point we obtain our result by choosing $[[B]] = [[Nat]]$, and reapplying the rule above.

\item[]
\item
  \begin{typeProofCase}
    \inferrule* [flushleft,right=$\mathsf{Nat}_e$] {
      {
        \begin{array}{lll}
          [[G |- t1 : C]]            & [[nat(C) = Nat]]\\
          [[G |- t2 : A1]]          & [[A1 ~ A]]\\
          [[G, x : Nat |- t3 : A2]] & [[A2 ~ A]]
        \end{array}
      }
      }{[[G |- case t1 of 0 -> t2, (succ x) -> t3 : A]]}
  \end{typeProofCase}
      
  \noindent
  In this case $[[t]] = [[case t1 of 0 -> t2, (succ x) -> t3]]$.  Suppose $[[t <= t']]$ and $[[G <= G']]$.  This
  implies that \\ $[[t']] = [[case t1' of 0 -> t2', (succ x) -> t3']]$ such that
  $[[t1 <= t1']]$, $[[t2 <= t2']]$, and $[[t3 <= t3']]$.  Since $[[G <= G']]$ then $[[(G,x:Nat) <= (G',x:Nat)]]$.
  By the induction hypothesis we know the following:
  \begin{center}
    \begin{tabular}{lll}
      $[[G' |- t1' : C']]$ for $[[C <= C']]$\\
      $[[G' |- t2 : A1']]$ for $[[A1 <= A1']]$\\
      $[[G', x : Nat |- t3 : A2']]$ for $[[A2 <= A2']]$
    \end{tabular}
  \end{center}
  By assumption we know that $[[A1 ~ A]]$, $[[A2 ~ A]]$, and $[[G <= G']]$.
  By the induction hypothesis we know that $[[A1 <= A1']]$ and $[[A2 <= A2']]$, so
  by using Lemma~\ref{lemma:congruence_of_type_consistency_along_type_precision} we may obtain that
  $[[A1' ~ A]]$ and $[[A2' ~ A]]$.  At this point choose $[[B]] = [[A]]$
  and we obtain our result by reapplying the rule.
  
\item[]
\item
  \begin{typeProofCase}
    $$\mprset{flushleft}
    \inferrule* [right=$\SGradydruleTXXpairName{}$] {
      [[G |- t1 : A1 && G |- t2 : A2]]
    }{[[G |- (t1,t2) : A1 x A2]]}
  \end{typeProofCase}

  \noindent
  In this case $[[A]] = [[A1 x A2]]$ and $[[t]] = [[(t1,t2)]]$. Suppose
  $[[t <= t']]$ and $[[G <= G']]$.  This implies that $[[t']] = [[(t1',t2')]]$ where
  $[[t1 <= t1']]$ and $[[t2 <= t2']]$.
  
  By the induction hypothesis we know:
  \begin{center}
    \begin{tabular}{lll}
      $[[G' |- t1' : A1']]$ and $[[A1 <= A1']]$\\
      $[[G' |- t2' : A2']]$ and $[[A2 <= A2']]$\\
    \end{tabular}
  \end{center}
  Then choose $[[B]] = [[A1' x A2']]$ and the result follows by reapplying
  the rule above and the fact that $[[(A1 x A2) <= (A1' x A2')]]$.  

\item[]
\item 
  \begin{typeProofCase}
    $$\mprset{flushleft}
    \inferrule* [right=$\SGradydruleTXXlamName{}$] {
      [[G, x : A1 |- t1 : B1]]
    }{[[G |- \x:A1.t1 : A1 -> B1]]}
  \end{typeProofCase}
  \noindent
  In this case $[[A1 -> B2]]$ and $[[t]] = [[\x:A1.t1]]$.  Suppose $[[t <= t']]$ and $[[G <= G']]$.
  Then it must be the case that $[[t']] = [[\x:A2.t2]]$, $[[t1 <= t2]]$, and $[[A1 <= A2]]$.
  Since $[[G <= G']]$ and $[[A1 <= A2]]$, then $[[(G, x : A1) <= (G', x : A2)]]$ by definition.
  Thus, by the induction hypothesis we know the following:
  \begin{center}
    \begin{tabular}{lll}
      $[[G', x : A2 |- t1' : B2]]$ and $[[B1 <= B2]]$
    \end{tabular}
  \end{center} 
  Choose $[[B]] = [[A2 -> B2]]$ and the result follows by reapplying the rule above
  and the fact that $[[(A1 -> B1) <= (A2 -> B2)]]$.

\item[]
\item 
  \begin{typeProofCase}
    $$\mprset{flushleft}
    \inferrule* [right=$\SGradydruleTXXappName{}$] {
      [[G |- t1 : C && fun(C) = A1 -> B1]]
      \\\\
          [[G |-t2 : A2 && A2 ~ A1]]
    }{[[G |- t1 t2 : B1]]}
  \end{typeProofCase}
  
  \noindent
  In this case $[[A]] = [[B1]]$ and $[[t]] = [[t1 t2]]$.  Suppose $[[t <= t']]$
  and $[[G <= G']]$.  The former implies that $[[t']] = [[t1' t2']]$ such that
  $[[t1 <= t1']]$ and $[[t2 <= t2']]$.  By the induction hypothesis we know the
  following:
  \begin{center}
    \begin{tabular}{lll}
      $[[G' |- t1' : C']]$ for $[[C <= C']]$\\
      $[[G' |- t2' : A2']]$ for $[[A2 <= A2']]$\\
    \end{tabular}
  \end{center}
  We know by assumption that $[[A2 ~ A1]]$.  
  Since $[[C <= C']]$ and $[[fun(C) = A1 -> B1]]$, then $[[fun(C') = A1' -> B1']]$
  where $[[A1 <= A1']]$ and $[[B1 <= B1']]$ by Lemma~\ref{lemma:fun_type_pre}.
  Furthermore, we know $[[A2 ~ A1]]$ and $[[A2 <= A2']]$ and $[[A1 <= A1']]$, then
  we know $[[A2' ~ A1']]$ by Corollary~\ref{corollary:congruence_of_type_consistency_along_type_precision}.
  So choose $[[B]] = [[B1']]$. Then reapply the rule above and the result follows, because
  $[[B1 <= B1']]$.
\end{description}
% subsection proof_of_gradual_guarantee_part_one_lemma:gradual_guarantee_part_one (end)

\subsection{Proof of Simulation of More Precise Programs (Lemma~\ref{lemma:simulation_of_more_precise_programs})}
\label{subsec:proof_of_simulation_of_more_precise_programs_lemma:simulation_of_more_precise_programs}

  This is a proof by induction on $<<G |- t1 : A1>>$.  We only give the
most interesting cases.  All others follow similarly.  Throughout the
proof we implicitly make use of typability inversion
(Lemma~\ref{lemma:typeability_inversion}) when applying the induction
hypothesis.

\begin{description}
\item
  \begin{typeProofCase}
    $$\mprset{flushleft}
    \inferrule* [right=$\CGradydruleTXXsuccName{}$] {
      <<G |- t : Nat>>
    }{<<G |- succ t : Nat>>}
  \end{typeProofCase}
  
  \noindent
  In this case $<<t1>> = <<succ t>>$ and $<<A>> = <<Nat>>$.  Suppose $<<G |- t1' : A'>>$.
  By inversion for term precision we must consider the following cases:
  \begin{enumR}
  \item $<<t1'>> = <<succ t'>>$ and $<<G |- t <= t'>>$
  \item $<<t1'>> = <<box Nat t1>>$ and $<<G |- t1 : Nat>>$    
  \end{enumR}

  \ \\
  \noindent
  \textbf{Proof of part i.}  Suppose $<<t1'>> = <<succ t'>>$, $<<G |- t <= t'>>$, and $<<t1 ~> t2>>$.
  Then $<<t2>> = <<succ t''>>$ and $<<t ~> t''>>$.  Then by the induction hypothesis
  we know that there is some $<<t'''>>$ such that $<<t' ~>* t'''>>$ and $<<G |- t'' <= t'''>>$.  Choose
  $<<t2'>> = <<succ t'''>>$ and the result follows.

  \ \\
  \noindent
  \textbf{Proof of part ii.} Suppose $<<t1'>> = <<box Nat t1>>$, $<<G |- t1 : Nat>>$, and $<<t1 ~> t2>>$.
  Then choose $<<t'2>> = <<box Nat t2>>$, and the result follows, because we know by type preservation
  that $<<G |- t2 : Nat>>$, and hence, $<<G |- t2 <= t2'>>$.

\item[]
\item
  \begin{typeProofCase}
    $$\mprset{flushleft}
    \inferrule* [right=$\CGradydruleTXXncaseName{}$] {
      <<G |- t : Nat>>
      \\\\
      <<G |- t3 : A && G, x : Nat |- t4 : A>>
    }{<<G |- case t : Nat of 0 -> t3, (succ x) -> t4 : A>>}  
  \end{typeProofCase}
  
  \noindent
  In this case $<<t1>> = <<case t : Nat of 0 -> t3, (succ x) -> t4>>$.  Suppose
  $<<G |- t1' : A'>>$. 
  Then inversion of term precision implies that one of the following must hold:
  \begin{enumR}
    \item $<<t1'>> = <<case t' : Nat of 0 -> t3', (succ x) -> t4'>>$, $<<G |- t <= t'>>$, $<<G |- t3 <= t3'>>$,
    and $<<G, x : Nat |- t4 <= t4'>>$
  \item $<<t1'>> = <<box A t1>>$ and $<<G |- t1 : A>>$
  \end{enumR}

  \ \\
  \noindent
  \textbf{Proof of part i.}  Suppose $<<t1'>> = <<case t' : Nat of 0 -> t3', (succ x) -> t4'>>$,
  $<<G |- t <= t'>>$, $<<G |- t3 <= t3'>>$, and $<<G, x : Nat |- t4 <= t4'>>$.

  \ \\
  \noindent
  We case split over $<<t1 ~> t2>>$.
  \begin{itemize}
  \item[] Case.  Suppose $<<t>> = <<0>>$ and $<<t2>> = <<t3>>$.  Since $<<G |- t1 <= t1'>>$ we know that
    it must be the case that $<<t'>> = <<0>>$ and $<<t1' ~> t3'>>$ by inversion for term precision
    or $<<t1'>>$ would not be typable which is a contradiction.  Thus, choose $<<t2'>> = <<t3'>>$ and the result follows.
    
  \item[] Case.  Suppose $<<t>> = <<succ t''>>$ and $<<t2>> = << [t''/x]t4>>$.  Since $<<G |- t1 <= t1'>>$
    we know that $<<t'>> = <<succ t'''>>$, or $<<t1'>>$ would not be typable,
    and $<<G |- t'' <= t'''>>$ by inversion for term precision. In addition,
    $<<t'1 ~> [t'''/x]t'4>>$. Choose $<<t2>> = << [t'''/x]t'4>>$.  Then it suffices to show that
    $<<G |- [t''/x]t4 <= [t'''/x]t'4>>$ by substitution for term precision (Lemma~\ref{lemma:substitution_for_term_precision}).    
    
  \item[] Case.  Suppose a congruence rule was used.  Then $<<t2>> = <<case t'' : Nat of 0 -> t3'', (succ x) -> t4''>>$.
    This case will follow straightforwardly by induction and a case split over which congruence rule was used.    
  \end{itemize}

  \ \\
  \noindent
  \textbf{Proof of part ii.}  Suppose $<<t1'>> = <<box A t1>>$, $<<G |- t1 : A>>$, and $<<t1 ~> t2>>$.
  Then choose $<<t'2>> = <<box A t2>>$, and the result follows, because we know by type preservation
  that $<<G |- t2 : A>>$, and hence, $<<G |- t2 <= t2'>>$.

\item[] 
\item
  \begin{typeProofCase}
     $$\mprset{flushleft}
      \inferrule* [right=$\CGradydruleTXXfstName{}$] {
        <<G |- t : A x B>>
      }{<<G |- fst t : A>>}
  \end{typeProofCase}

  \noindent
  In this case $<<t1>> = <<fst t>>$.  Suppose $<<G |- t1 <= t1'>>$ and $<<G |- t1' : A'>>$.
  Then inversion for term precision implies that one of the following must hold:
  \begin{enumR}
  \item $<<t'1>> = <<fst t'>>$ and $<<G |- t <= t'>>$
  \item $<<t1'>> = <<box A t1>>$ and $<<G |- t1 : A>>$
  \end{enumR}

  We only consider the proof of part i, because the other follows similarly to
  the previous case. Case split over $<<t1 ~> t2>>$.
  \begin{itemize}
  \item[] Case. Suppose $<<t>> = <<(t'3,t''3)>>$ and $<<t2>> = <<t'3>>$.  By inversion for term precision it must be the case
    that $<<t'>> = <<(t'4,t''4)>>$ because $<<G |- t1 <= t'1>>$ or else $<<t1'>>$ would not be typable.  In addition,
    this implies that $<<G |- t'3 <= t'4>>$ and $<<G |- t''3 <= t''4>>$.
    Thus, $<<t'1 ~> t'4>>$. Thus, choose $<<t2'>> = <<t'4>>$ and the result follows.

  \item[] Case. Suppose a congruence rule was used.  Then $<<t2>> = <<fst t''>>$.
    This case will follow straightforwardly by induction and a case split over which congruence rule was used.
  \end{itemize}

\item[]
\item
  \begin{typeProofCase}
    $$\mprset{flushleft}
    \inferrule* [right=$\CGradydruleTXXlamName{}$] {
      <<G, x : A1 |- t : A2>>
    }{<<G |- \x:A1.t : A1 -> A2>>}
  \end{typeProofCase}

  \noindent
  In this case $<<t1>> = <<\x:A1.t>>$ and $<<A>> = <<A1 -> A2>>$.
  Suppose $<<G |- t1 <= t1'>>$ and $<<G |- t1' : A'>>$.
  Then inversion of term precision implies that one of the following must hold:
  \begin{enumR}
  \item $<<t'1>> = <<\x:A'1.t'>>$
  \item $<<t1'>> = <<box A t1>>$ and $<<G |- t1 : A>>$
  \end{enumR}
  
  \noindent
  We only consider the proof of part i. The reduction relation does not reduce under
  $\lambda$-expressions.  Hence, $<<t2>> = <<t1>>$, and thus, choose $<<t'2>> = <<t'1>>$, and
  the case trivially follows.  
  
\item[]
\item 
  \begin{typeProofCase}
    $$\mprset{flushleft}
    \inferrule* [right=$\CGradydruleTXXappName{}$] {
      <<G |- t3 : A1 -> A2 && G |- t4 : A1>>
    }{<<G |- t3 t4 : A2>>}
  \end{typeProofCase}

  \noindent
  In this case $<<t1>> = <<t3 t4>>$.  Suppose $<<G |- t1 <= t1'>>$ and $<<G |- t1' : A'>>$.
  Then by inversion for term prevision we know one of the following is true:
  \begin{enumR}
  \item $<<t1'>> = <<t'3 t'4>>$, $<<G |- t3 <= t'3>>$, and $<<G |- t4 <= t'4>>$
  \item $<<t'1>> = <<box A2 t1>>$ and $<<G |- t1 : A>>$
  \item $<<t3>> = <<unbox A2>>$, $<<t'1>> = <<t4>>$, and $<<G |- t4 : ?>>$
  \end{enumR}

  \ \\
  \noindent
  \textbf{Proof of part i.}  Suppose $<<t1'>> = <<t'3 t'4>>$, $<<G |- t3 <= t'3>>$, and $<<G |- t4 <= t'4>>$.

  We case split on the from of $<<t1 ~> t2>>$.
  \begin{itemize}
  \item[] Case.  Suppose $<<t3>> = <<\x:A1.t5>>$ and $<<t2>> = << [t4/x]t5>>$.
    Then by inversion for term precision we know that
    $<<t'3>> = <<\x:A1'.t'5>>$ and $<<G, x : A2' |- t5 <= t'5>>$,
    because $<<G |- t3 <= t'3>>$ and the requirement that $<<t1'>>$ is typable. Choose $<<t'2>> = << [t'4/x]t'5>>$
    and it is easy to see that $<<t'1 ~> [t'4/x]t'4>>$.
    We know that $<<G, x : A2' |- t5 <= t'5>>$ and $<<G |- t4 <= t'4>>$, and hence,
    by Lemma~\ref{lemma:substitution_for_term_precision} we know that
    $<<G |- [t4/x]t5 <= [t'4/x]t'5>>$, and we obtain our result.
    
  \item[] Case.  Suppose $<<t3>> = <<unbox A>>$, $<<t4>> = <<box A t5>>$, and $<<t2>> = <<t5>>$.
    Then by inversion for term prevision $<<t'3>> = <<unbox A>>$, $<<t'4>> = <<box A t'5>>$, and $<<G |- t5 <= t'5>>$.
    Note that $<<t'4>> = <<box A t'5>>$ and $<<G |- t5 <= t'5>>$ hold even though there are two potential rules
    that could have been used to construct $<<G |- t4 <= t4'>>$. 
    Choose $<<t'2>> = <<t'5>>$ and it is easy to see that $<<t'1 ~> t'5>>$.  Thus, we obtain our result.    

  \item[] Case.  Suppose $<<t3>> = <<unbox A>>$, $<<t4>> = <<box B t5>>$, $<<A != B>>$,
    and $<<t2>> = <<error B>>$.  Then $<<t'3>> = <<unbox A>>$ and $<<t'4>> = <<box B t'5>>$.
    Choose $<<t'2>> = <<error B>>$ and it is easy to see that $<<t'1 ~> t'5>>$.  Finally,
    we can see that $<<G |- t2 <= t'2>>$ by reflexivity.
    
  \item[] Case. Suppose a congruence rule was used.  Then $<<t2>> = <<t'5 t'6>>$.
    This case will follow straightforwardly by induction and a case split over which congruence rule was used.
  \end{itemize}

  \ \\
  \noindent
  \textbf{Proof of part ii.} We know that $<<t1>> = <<t3 t4>>$.  Suppose $<<t'1>> = <<box A2 t1>>$ and $<<G |- t1 : A>>$.
  If $<<t1 ~> t2>>$, then $<<t'1>> = <<(box A2 t1) ~> (box A2 t2)>>$.  Thus, choose $<<t2'>> = <<box A2 t2>>$.

  \ \\
  \noindent
  \textbf{Proof of part iii.} We know that $<<t1>> = <<t3 t4>>$. Suppose
  $<<t3>> = <<unbox A2>>$, $<<t'1>> = <<t4>>$, and $<<G |- t4 : ?>>$.  Then $<<t1>> = <<unbox A2 t4>>$.  We
  case split over $<<t1 ~> t2>>$.  We have three cases to consider.

  \ \\
  \noindent
  Suppose $<<t4>> = <<box A2 t5>>$ and $<<t2>> = <<t5>>$.  Then choose $<<t2'>> = <<t4>> = <<t'1>>$, and we
  obtain our result.

  \ \\
  \noindent
  Suppose $<<t4>> = <<box A3 t5>>$, $<<A2>> \neq <<A3>>$, and $<<t2>> = <<error A2>>$. Then choose $<<t2'>> = <<t4>> = <<t'1>>$,
  and we obtain our result.

  \ \\
  \noindent
  Suppose a congruence rule was used.  Then $<<t2>> = <<t3 t'4>>$. This case will follow straightforwardly by induction.
\end{description}
% subsection proof_of_simulation_of_more_precise_programs_lemma:simulation_of_more_precise_programs (end)

% section proofs (end)

%%% Local Variables: ***
%%% mode:latex ***
%%% TeX-master: "main.tex"  ***
%%% End: ***
