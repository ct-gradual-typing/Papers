\section{Proofs}
\label{sec:proofs}

\subsection{Proof of Lifted Retract (Lemma~\ref{lemma:lifted_retract})}
\label{subsec:proof_of_lifted_retract}
We only show that $<<lbox A>>;<<lunbox A>> = \id_A$, because the case
when a dynamic type error is raised is similar using the fact that
$<<A>>$ and $<<B>>$ must have the same skeleton or one could not
compose $<<lbox A>>$ and $<<lunbox B>>$. This implies that $<<A>>$ and
$<<B>>$ only differ at an atomic type.

\ \\
\noindent
This is a proof by induction on the form of $<<A>>$.

\begin{itemize}
\item[] Case. Suppose $<<A>>$ is atomic.  Then:
  \[
    <<lbox A>>;<<lunbox A>> = <<box A>>;<<unbox A>> = \id_{<<A>>}
    \]
    
  \item[] Case. Suppose $<<A>>$ is $<<?>>$.  Then:
    \[
    \begin{array}{lll}
      <<lbox A>>;<<lunbox A>> & = & <<lbox ?>>;<<lunbox ?>>\\
      & = & \id_{<<?>>};\id_{<<?>>}\\
      & = & \id_{<<?>>}\\
      & = & \id_{<<A>>}
    \end{array}
    \]

  \item[] Case. Suppose $<<A>> = <<A1 -> A2>>$.  Then:
    \[\small
    \begin{array}{lll}
      <<lbox A>>;<<lunbox A>>
      & = & <<lbox (A1 -> A2)>>;<<lunbox (A1 -> A2)>>\\
      & = & (<<lunbox A1>> \to <<lbox A2>>);(<<lbox A1>> \to <<lbox A2>>)\\
      & = & (<<lbox A1>>;<<lunbox A1>>) \to (<<lbox A2>>;<<lunbox A2>>)\\
    \end{array}
    \]
    By two applications of the induction hypothesis we know the
    following:
    \[
    \begin{array}{lll}
      <<lbox A1>>;<<lunbox A1>> = \id_{<<A1>>} & \text{ and } & <<lbox A2>>;<<lunbox A2>> = \id_{<<A2>>}
    \end{array}
    \]
    Thus, we know the following:
    \[
    \begin{array}{lll}
      (<<lbox A1>>;<<lunbox A1>>) \to (<<lbox A2>>;<<lunbox A2>>)
      & = & \id_{<<A1>>} \to \id_{<<A2>>}\\
      & = & \id_{<<A1 -> A2>>}\\
      & = & \id_{<<A>>}
    \end{array}
    \]

  \item[] Case. Suppose $<<A>> = <<A1 x A2>>$.  Then:
    \[\small
    \begin{array}{lll}
      <<lbox A>>;<<lunbox A>>
      & = & <<lbox (A1 x A2)>>;<<lunbox (A1 x A2)>>\\
      & = & (<<lbox A1>> \times <<lbox A2>>);(<<lunbox A1>> \times <<lbox A2>>)\\
      & = & (<<lbox A1>>;<<lunbox A1>>) \times (<<lbox A2>>;<<lunbox A2>>)\\
    \end{array}
    \]
    By two applications of the induction hypothesis we know the
    following:
    \[
    \begin{array}{lll}
      <<lbox A1>>;<<lunbox A1>> = \id_{<<A1>>} & \text{ and } & <<lbox A2>>;<<lunbox A2>> = \id_{<<A2>>}
    \end{array}
    \]
    Thus, we know the following:
    \[
    \begin{array}{lll}
      (<<lbox A1>>;<<lunbox A1>>) \times (<<lbox A2>>;<<lunbox A2>>)
      & = & \id_{<<A1>>} \times \id_{<<A2>>}\\
      & = & \id_{<<A1 x A2>>}\\
      & = & \id_{<<A>>}
    \end{array}
    \]
\end{itemize}
% subsection proof_of_lifted_retract (end)

\subsection{Proof of Lemma~\ref{lemma:S_is_faithful}}
\label{subsec:proof_of_S_is_faithful}
We must show that the function
\[ \S_{A,B} : \Hom{C}{A}{B} \mto \Hom{S}{\S A}{\S B} \]
is injective.

So suppose $f \in \Hom{C}{A}{B}$ and $g \in \Hom{C}{A}{B}$ such that
$\S f = \S g : \S A \mto \S B$.  Then we can easily see that:
\[
\begin{array}{lll}
  \S f & = & <<lunbox A>>;f;<<lbox B>> \\
  & = & <<lunbox A>>;g;<<lbox B>>\\
  & = & \S g\\
\end{array}
\]
But, we have the following equalities:
\[\small
\setlength{\arraycolsep}{1px}
\begin{array}{rll}
  <<lunbox A>>;f;<<lbox B>> & = & <<lunbox A>>;g;<<lbox B>>\\
  <<lbox A>>;<<lunbox A>>;f;<<lbox B>>;<<lunbox B>> & = & <<lbox A>>;<<lunbox A>>;g;<<lbox B>>;<<lunbox B>>\\
  \id_A;f;<<lbox B>>;<<lunbox B>> & = & \id_A;g;<<lbox B>>;<<lunbox B>>\\
  \id_A;f;\id_B & = & \id_A;g;\id_B\\
  f & = & g\\
\end{array}
\]
The previous equalities hold due to
Lemma~\ref{lemma:lifted_retract}.
% subsection proof_of_S_is_faithful (end)

\subsection{Proof of Type Consistency in the Model (Lemma~\ref{lemma:type_consistency_in_the_model})}
\label{subsec:proof_of_type_consistency_in_the_model}
This is a proof by induction on the form of $[[G |- A ~ B]]$.
\begin{description}
\item Case: $\quad \SGradydruleCXXRefl{}$\\
  \ \\
  \noindent  
  Choose $c_1 = c_2 = \id_A : <<A --> A>>$.
  \ \\
\item Case: $\quad \SGradydruleCXXBox{}$\\
  \ \\
  \noindent
  Choose $c_1 = <<Box A>> : <<A --> ?>>$ and $c_2 = <<Unbox A>> : <<? -> A>>$.
  \ \\
\item Case: $\quad \SGradydruleCXXUnbox{}$\\
  \ \\
  \noindent
  Choose $c_1 = <<Unbox A>> : <<? --> A>>$ and $c_2 = <<Box A>> : <<A -> ?>>$.
  \ \\
\item Case: $\quad \SGradydruleCXXArrow{}$\\
  \ \\
  \noindent  
  By the induction hypothesis there exists four casting morphisms
  $c'_1 : <<A1 --> A2>>$, $c'_2 : <<A2 --> A1>>$, $c'_3 : <<B1 --> B2>>$,
  and $c'_4 : <<B2 --> B1>>$.  Choose
  $c_1 = c'_2 \to c'_3 : <<(A1 -> B1) --> (A2 -> B2)>>$
  and
  $c_2 = c'_1 \to c'_4 : <<(A2 -> B2) --> (A1 -> B1)>>$.
  \ \\
\item Case: $\quad \SGradydruleCXXProd{}$\\
  \ \\
  \noindent  
  By the induction hypothesis there exists four casting morphisms
  $c'_1 : <<A1 --> A2>>$, $c'_2 : <<A2 --> A1>>$, $c'_3 : <<B1 --> B2>>$,
  and $c'_4 : <<B2 --> B1>>$.
  Choose
  $c_1 = c'_1 \times c'_3 : <<H(A1 x B1) --> H(A2 x B2)>>$
  and
  $c_2 = c'_2 \times c'_4 : <<H(A2 x B2) --> H(A1 x B1)>>$.
\end{description}
% subsection proof_of_type_consistency_in_the_model (end)

\subsection{Proof of Interpretation of Types Theorem~\ref{thm:interpretation_of_typing}}
\label{subsec:proof_of_interpretation_of_types}
First, we show how to interpret the rules of Surface Grady and then Core Grady.
This is a proof by induction on $[[G |- t : A]]$.
\begin{description}
\item Case: $\quad \SGradydruleTXXvarP{}$\\
    \\
    \noindent
    Suppose without loss of generality that $[[ [| G |] ]] = [[A1]]
    \times \cdots \times [[Ai]] \times \cdots \times [[Aj]]$ where
    $[[Ai]] = [[A]]$.  We know that $j > 0$ or the assumed typing
    derivation would not hold.  Then take
    $[[ [| x |] ]] = \pi_i : [[ [| G |] --> A]]$.

    \ \\
  \item Case: $\quad \SGradydruleTXXunitP{}$\\
    \\
    \noindent     
    Take $[[ [| triv |] ]] = [[triv]]_{[[ [| G |] ]]} : [[ [| G |] --> Unit ]]$ where $[[triv]]_{[[ [| G |] ]]}$
    is the unique terminal arrow that exists because $\cat{C}$ is cartesian closed.

    \ \\
  \item Case: $\quad \SGradydruleTXXzeroP{}$\\
    \\
    \noindent
    Take $[[ [| 0 |] ]] = [[triv]]_{[[ [| G |] ]]};\z : [[ [| G |] --> Nat]]$
    where $\z : [[Unit --> Nat]]$ exists because $\cat{C}$
    has an NRNO.

    \ \\
  \item Case: $\quad \SGradydruleTXXsucc{}$\\
    \\
    \noindent
    First, by the induction hypothesis there is a morphism $[[ [| t |] ]] : [[ [| G |] --> A]]$.
    Now we have two cases to consider, one when $<<A>> = <<?>>$ and one when $<<A>> = <<Nat>>$.
    Consider the former.  Then interpret
    $<< [| succ t |] >> = << [| t |] >>;[[ unbox Nat ]];[[succ]] : [[ [| G |] --> Nat]]$ where
    $[[succ]] : [[Nat --> Nat]]$ exists because $\cat{C}$ has an NRNO.  Similarly,
    when $<<A>> = <<Nat>>$, 
    $<< [| succ t |] >> = << [| t |] >>;[[succ]] : [[ [| G |] --> Nat]]$.

    \ \\
  \item Case: $\quad {    
    \inferrule* [flushleft,right=$\mathsf{Nat}_e$] {
      {
        \begin{array}{lll}
          [[G |- t : C]]            & [[nat(C) = Nat]]\\
          [[G |- t1 : A1]]          & [[G |- A1 ~ A]]\\
          [[G, x : Nat |- t2 : A2]] & [[G |- A2 ~ A]]
        \end{array}
      }
    }{[[G |- case t of 0 -> t1, (succ x) -> t2 : A]]}
    }$\\
    \\
    \noindent
    By three applications of the induction hypothesis we have the following morphisms:
    \begin{center}
      \begin{math}
        \begin{array}{lll}
          [[ [| t |] ]] : [[ [|G|] --> C]]\\
          [[ [| t1 |] ]] : [[ [|G|] --> A1]]\\
          [[ [| t2 |] ]] : [[ H([|G|] x Nat) --> A2]]
        \end{array}
      \end{math}
    \end{center}
    In addition, we know $[[G |- A1 ~ A]]$ and $[[G |- A2 ~ A]]$ by
    assumption, and hence, by type consistency in the model
    (Lemma~\ref{lemma:type_consistency_in_the_model}) we know there are
    casting morphisms $[[c1]] : [[A1 --> A]]$ and $[[c2 : A2 --> A]]$.
    Now every gradual $\lambda$-model has an NRNO
    (Definition~\ref{def:SNNO},
    Definition~\ref{def:gradual-lambda-model}), and so, there is a
    unique morphism:
    \begin{center}
      \begin{math}
        \Case_{[[ [| G |] ]],A}\langle [[ [| t1 |];c1 ]],[[ [| t2 |];c2 ]] \rangle : [[ H([|G|] x Nat) --> A]]
      \end{math}
    \end{center}
    
    At this point we have two cases to consider: one when $[[C]] = [[?]]$ and one when $[[C]] = [[Nat]]$.  Consider the former.
    Then we have the following:
    \begin{center}
      \begin{math}\small
        \begin{array}{lll}
          [[ [| case t of 0 -> t1, (succ x) -> t2 |] ]] \\
          = \langle \id_{[[ [| G |] ]]}, [[ [| t |];unbox Nat ]] \rangle;\Case_{[[ [| G |] ]],A}\langle [[ [| t1 |];c1 ]],[[ [| t2 |];c2 ]] \rangle\\
          : [[ [| G |] --> A ]]
        \end{array}
      \end{math}
    \end{center}
    In the second case we have the following:
    \begin{center}
      \begin{math}\small
        \begin{array}{lll}
          [[ [| case t of 0 -> t1, (succ x) -> t2 |] ]] \\
          = \langle \id_{[[ [| G |] ]]}, [[ [| t |] ]] \rangle;\Case_{[[ [| G |] ]],A}\langle [[ [| t1 |];c1 ]],[[ [| t2 |];c2 ]] \rangle \\
          : [[ [| G |] --> A ]]
        \end{array}
      \end{math}
    \end{center}

    \ \\
  \item Case: $\quad \SGradydruleTXXpair{}$\\
    \\
    \noindent
    By two applications of the induction hypothesis there are two morphisms
    $[[ [| t1 |] ]] : [[ [| G |] --> A]]$ and $[[ [| t2 |] ]] : [[ [| G |] --> B]]$.
    Then using the fact that $\cat{C}$ is cartesian we take
    $[[ [| (t1 , t2) |] ]] = \langle [[ [| t1 |] ]] , [[ [| t2 |] ]] \rangle : [[ [| G |] --> H(A x B)]]$.

    \ \\
  \item Case: $\quad \SGradydruleTXXfst{}$\\
    \\
    \noindent
    First, by the induction hypothesis there is a morphism $[[ [| t |] ]] : [[ [| G |] --> B]]$.
    Now we have two cases to consider, one when $<<B>> = <<?>>$ and one when $<<B>> = <<A1 x A2>>$
    for some types $<<A1>>$ and $<<A2>>$.  Consider the former.  We then know that it must
    be the case that $<<A1 x A2>> = <<? x ?>>$.  Thus, we obtain the following interpretation
    $[[ [| fst t |] ]] = [[ [| t |] ]];[[unbox (? x ?)]];\pi_1 : [[ [| G |] --> ?]]$.  Similarly,
    when $<<B>> = <<A1 x A2>>$, then
    $[[ [| fst t |] ]] = [[ [| t |] ]];\pi_1 : [[ [| G |] --> A1]]$.

    \ \\
  \item Case: $\quad \SGradydruleTXXsnd{}$\\
    \\
    \noindent
    First, by the induction hypothesis there is a morphism $[[ [| t |] ]] : [[ [| G |] --> B]]$.
    Now we have two cases to consider, one when $<<B>> = <<?>>$ and one when $<<B>> = <<A1 x A2>>$
    for some types $<<A1>>$ and $<<A2>>$.  Consider the former.  We then know that it must
    be the case that $<<A1 x A2>> = <<? x ?>>$.  Thus, we obtain the following interpretation
    $[[ [| snd t |] ]] = [[ [| t |] ]];[[unbox (? x ?)]];\pi_2 : [[ [| G |] --> ?]]$.  Similarly,
    when $<<B>> = <<A1 x A2>>$, then
    $[[ [| snd t |] ]] = [[ [| t |] ]];\pi_2 : [[ [| G |] --> A2]]$.

    \ \\
  \item Case: $\quad \SGradydruleTXXlam{}$\\
    \\
    \noindent    
    By the induction hypothesis there is a morphism $[[ [| t |] ]] :
    [[ H([| G |] x A) --> B]]$.  Then take $[[ [| \x : A.t |] ]] =
    \curry {[[ [| t |] ]]} : [[ [| G |] --> (A -> B)]]$, where
    \begin{center}
      \begin{math}
        \mathsf{curry} : \Hom{C}{X \times Y}{Z} \mto \Hom{C}{X}{Y \to Z}
      \end{math}
    \end{center}
    exists because $\cat{C}$ is closed.    
   
    \ \\
  \item Case: $\quad {
    \inferrule* [flushleft,right=$\to_e$] {
      [[G |- t1 : C]] \\ \,\,\,[[G |- A2 ~ A1]]\\\\    
      [[G |-t2 : A2]] \\ [[fun(C) = A1 -> B1]]
    }{[[G |- t1 t2 : B1]]}
  }$\\
    \\
    \noindent
    By the induction hypothesis there are two morphisms
    $[[ [| t1 |] ]] : [[ [| G |] --> C ]]$ and
    $[[ [| t2 |] ]] : [[ [| G |] --> A2 ]]$.  In addition, by assumption we know that
    $[[G |- A2 ~ A1]]$, and hence, by type consistency in the model (Lemma~\ref{lemma:type_consistency_in_the_model})
    there are casting morphisms $c_1 : [[A2 --> A1]]$ and $c_2 : [[A1 --> A2]]$.  We have two cases to consider,
    one when $[[C]] = [[?]]$ and one when $[[C]] = [[A1 -> B1]]$.  Consider
    the former. Then we have the interpretation:
    \begin{center}
      \begin{math}
        [[ [| t1 t2 |] ]] = \langle [[ [| t1 |] ]];[[unbox (? -> ?)]], [[ [| t2 |] ]];c_1 \rangle;\app_{[[A1]],[[B1]]} : [[ [| G |] --> B1 ]]
      \end{math}
    \end{center}
    Similarly, for the case when $[[C]] = [[A1 -> B1]]$ we have the interpretation:
    \begin{center}
      \begin{math}
        [[ [| t1 t2 |] ]] = \langle [[ [| t1 |] ]], [[ [| t2 |] ]];c_1 \rangle;\app_{[[A1]],[[B1]]} : [[ [| G |] --> B1 ]]
      \end{math}
    \end{center}
    Note that $\app_{[[A1]],[[B1]]} : [[ H((A1 -> B1) x A1) --> B1]]$ exists because the
    model is cartesian closed.
  \end{description}

  Next we turn to Core Grady, but we do not show every rule, because
  it is similar to what we have already shown above except without
  casting morphism, and so we only show the case for the $[[box]]$ and
  $[[unbox]]$ rules, and the error rule.

  The first two cases use the well-known bijection:  
  \begin{center}
    \begin{math}
      \begin{array}{lll}
        \Hom{C}{A}{B} & \cong & \Hom{C}{[[Unit x A]]}{B}\\
                      & \cong & \Hom{C}{[[Unit]]}{[[A -> B]]}
      \end{array}
    \end{math}
  \end{center}
  When $f \in \Hom{C}{A}{B}$, then we denote by $\curry{f}$, the
  morphism $\curry{f} \in \Hom{C}{[[Unit]]}{[[A -> B]]}$.
\begin{description}
\item Case: $\quad \CGradydruleTXXBox{}$\\
  \\
  \noindent
  We have the following interpretation:
  \begin{center}
    \begin{math}
      << [| box A |] >> = [[triv]]_{[[ [| G |] ]]};\curry{<<Box A>>} : [[ [| G |] --> (A -> ?) ]]
    \end{math}
  \end{center}
  \ \\
\item Case: $\quad \CGradydruleTXXUnbox{}$\\
  \\
  \noindent
  We have the following interpretation:
  \begin{center}
    \begin{math}
      \begin{array}{lll}
        << [| unbox A |] >> \\
        = [[triv]]_{[[ [| G |] ]]};\curry{<<Unbox A>>}\\
        : [[ [| G |] --> (? -> A)]]
      \end{array}
    \end{math}
  \end{center}

\item Case: $\quad \CGradydruleTXXerror{}$\\
  \\
  \noindent
  We have the following interpretation:
  \begin{center}
    \begin{math}
      << [| error A |] >> = <<error>>_{[[ [| G |] ]],[[A]]} : [[ [| G |] --> A]]
    \end{math}
  \end{center}
\end{description}
% subsection proof_of_interpretation_of_types (end)

\subsection{Proof of Interpretation of Evaluation (Theorem~\ref{thm:interpretation_of_evaluation})}
\label{subsec:proof_of_interpretation_of_evaluation}
This proof requires the following corollary to
Lemma~\ref{lemma:type_consistency_in_the_model}.
\begin{corollary}
  \label{corollary:type_consist_coro}
  Suppose $(\cat{T}, \cat{C}, ?, \T,$ $\split, \squash, \bx, \unbox)$ is
  a gradual $\lambda$-model.  Then we know the following:
  \begin{enumR}
  \item If $[[G |- A ~ A]]$, then $c_1 = c_2 = \id_{[[A]]} : [[A --> A]]$.

  \item If $[[G |- A ~ ?]]$, then there are casting morphisms:
    \[
    \begin{array}{lll}
      c_1 & = & [[Box A]] : [[A --> ?]]  \\    
      c_2 & = & [[Unbox A]] : [[? --> A]]
    \end{array}
    \]

    \item If $[[G |- ? ~ A]]$, then there are casting morphisms:
    \[
    \begin{array}{lll}
      c_1 & = & [[Unbox A]] : [[? --> A]]\\
      c_2 & = & [[Box A]] : [[A --> ?]]
      \end{array}
    \]
    
  \item If $[[G |- A1 -> B1 ~ A2 -> B2]]$, then there are casting morphisms:
    \[
    \begin{array}{lllll}
      c & = & c_1 \to c_2 : [[(A1 -> B1) --> (A2 -> B2)]]\\
      c' & = & c_3 \to c_4 : [[(A2 -> B2) --> (A1 -> B1)]]
    \end{array}
    \]
    where $c_1 : [[A2 --> A1]]$ and $c_2 : [[B1 --> B2]]$, and $c_3 :
    [[A1 --> A2]]$ and $c_4 : [[B2 --> B1]]$.
    
  \item If $[[G |- A1 x B1 ~ A2 x B2]]$, then there are casting
    morphisms:
    \[
    \begin{array}{lll}
       c & = & c_1 \times c_2 : [[(A1 x B1) --> (A2 x B2)]]\\
      c' & = & c_3 \times c_4 : [[(A2 x B2) --> (A1 x B1)]]
    \end{array}
    \]
    where $c_1 : [[A1 --> A2]]$ and $c_2 : [[B1 --> B2]]$, and $c_3 :
    [[A2 --> A1]]$ and $c_4 : [[B2 --> B1]]$.
  \end{enumR}
\end{corollary}
\begin{proof}
  This proof holds by the construction of the casting morphisms from
  the proof of the previous result, and the fact that the type
  consistency rules are unique for each type.
\end{proof}

This proof holds by induction on the form of $[[G |- t1 : A]]$ with a
case split over $<<t1 ~> t2>>$.  We only show the cases for the the
retract rules, and the error rule, because the others are well-known
to hold within any cartesian closed category; see \cite{Lambek:1980}
or \cite{Crole:1994}.  We will routinely use the interpretation given
in the proof of Theorem~\ref{thm:interpretation_of_typing} and
summarized in Figure~\ref{fig:interp-terms} throughout this proof
without mention.


%% % subsection proof_of_interpretation_of_evaluation (end)
% section proofs (end)

%%% Local Variables: ***
%%% mode:latex ***
%%% TeX-master: "main.tex"  ***
%%% End: ***
