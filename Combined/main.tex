\documentclass[sigplan]{acmart}\settopmatter{printfolios=true}

\setcopyright{none}

\usepackage{amssymb,amsmath, amsthm}
\usepackage{thm-restate}
\usepackage{fullpage}
\usepackage{hyperref}
\usepackage{mathpartir}
\usepackage[barr]{xy}
\usepackage{mdframed}
\usepackage{supertabular}
\usepackage{todonotes}
\usepackage{enumitem}
\usepackage{listings}
\usepackage{color}

%% From: https://hal.inria.fr/file/index/docid/881085/filename/lsthaskell.sty
\lstdefinelanguage{Haskell}{ 
  %
  % Anything betweeen $ becomes LaTeX math mode
  mathescape=true,
  %
  % Comments may or not include Latex commands
  texcl=false, 
  %
  morekeywords=[1]{class, instance},
  %
  morekeywords=[2]{where},
  %
  morekeywords=[3]{Maybe},
  %
  morekeywords=[4]{main},
  %
  morekeywords=[6]{do, last, first, try, idtac, repeat},
  %
  % Comments delimiters, we do turn this off for the manual
  morecomment=[s]{(*}{*)},
  %
  % Spaces are not displayed as a special character
  showstringspaces=false,
  %
  % String delimiters
  morestring=[b]",
  morestring=[d]’,
  %
  % Size of tabulations
  tabsize=3,
  %
  % Enables ASCII chars 128 to 255
  extendedchars=false,
  %
  % Case sensitivity
  sensitive=true,
  %
  % Automatic breaking of long lines
  breaklines=false,
  %
  % Default style fors listings
  basicstyle=\small,
  %
  % Position of captions is bottom
  captionpos=b,
  %
  % flexible columns
  columns=[l]flexible,
  %
  % Style for (listings') identifiers
  identifierstyle={\ttfamily\color{black}},
  % Style for declaration keywords
  keywordstyle=[1]{\ttfamily\color{dkviolet}},
  % Style for gallina keywords
  keywordstyle=[2]{\ttfamily\color{dkgreen}},
  % Style for sorts keywords
  keywordstyle=[3]{\ttfamily\color{ltblue}},
  % Style for tactics keywords
  keywordstyle=[4]{\ttfamily\color{dkblue}},
  % Style for terminators keywords
  keywordstyle=[5]{\ttfamily\color{dkred}},
  %Style for iterators
  %keywordstyle=[6]{\ttfamily\color{dkpink}},
  % Style for strings
  stringstyle=\ttfamily,
  % Style for comments
  commentstyle={\ttfamily\color{dkgreen}},
  %
  %moredelim=**[is][\ttfamily\color{red}]{/&}{&/},
  literate=
  {->}{{$\rightarrow\;$}}1
  {=>}{{$\Rightarrow\;$}}1
  {<:}{{$<\hspace{-2px}\colon$}}1
  {++}{{\code{++}}}1
  {\{}{{$\langle$}}1
  {\}}{{$\rangle\;$}}1
  {~}{{\ }}1
  {\\dollar}{{$\$$\;}}1
  %
}[keywords,comments,strings]

\lstnewenvironment{haskell}{\lstset{language=Haskell}}{}

% pour inliner dans le texte
\def\hasqel{\lstinline[language=Haskell, basicstyle=\small]}
% pour inliner dans les tableaux / displaymath...
\def\haskels{\lstinline[language=Haskell, basicstyle=\scriptsize]}

\newenvironment{enumR}{\begin{enumerate}[label=\roman*.,align=left]}{\end{enumerate}}
\newenvironment{enumA}{\begin{enumerate}[label=\alph*.]}{\end{enumerate}}

\newcommand{\cL}{{\cal L}}

\let\mto\to                     % Used for arrows
\let\to\relax                   % Used for implication
\newcommand{\to}{\rightarrow}
\newcommand{\id}{\mathsf{id}}
\newcommand{\redto}{\rightsquigarrow}
\newcommand{\cat}[1]{\mathcal{#1}}
\newcommand{\catop}[1]{\mathcal{#1}^{\mathsf{op}}}
\newcommand{\Case}[0]{\mathsf{case}}

\let\split\relax
\let\S\relax

\newcommand{\split}[0]{\mathsf{split}}
\newcommand{\squash}[0]{\mathsf{squash}}
\newcommand{\bx}[0]{\mathsf{box}}
\newcommand{\error}[0]{\mathsf{error}}
\newcommand{\err}[0]{\mathsf{err}}
\newcommand{\unbox}[0]{\mathsf{unbox}}
\newcommand{\T}[0]{\mathsf{T}}
\newcommand{\S}[0]{\mathsf{S}}
\newcommand{\U}[0]{\mathsf{U}}
\newcommand{\C}[0]{\mathsf{C}}
\newcommand{\z}[0]{\mathsf{z}}
\newcommand{\app}[0]{\mathsf{app}}
\newcommand{\curry}[1]{\mathsf{curry}(#1)}
\newcommand{\interp}[1]{[\negthinspace[#1]\negthinspace]}
\newcommand{\Hom}[3]{\mathsf{Hom}_{\cat{#1}}(#2,#3)}

%% \newtheorem{theorem}{Theorem}
%% \newtheorem{lemma}[theorem]{Lemma}
%% \newtheorem{corollary}[theorem]{Corollary}
%% \newtheorem{definition}[theorem]{Definition}
%% \newtheorem{proposition}[theorem]{Proposition}
%% \newtheorem{example}[theorem]{Example}

%% OTT Includes:
\input{surface-grady-inc}
\renewcommand{\SGradydrule}[4][]{{\displaystyle\frac{\begin{array}{l}#2\end{array}}{#3}\,\SGradydrulename{#4}}}
\renewcommand{\SGradydrulename}[1]{#1}

\renewcommand{\SGradydruleCXXReflName}{\text{refl}}
\renewcommand{\SGradydruleCXXBoxName}{\mathsf{box}}
\renewcommand{\SGradydruleCXXUnboxName}{\mathsf{unbox}}
\renewcommand{\SGradydruleCXXListName}{\mathsf{List}}
\renewcommand{\SGradydruleCXXArrowName}{\to}
\renewcommand{\SGradydruleCXXProdName}{\times}
\renewcommand{\SGradydruleCXXForallName}{\forall}

\renewcommand{\SGradydruleTXXboxName}{\mathsf{box}}
\renewcommand{\SGradydruleTXXunboxName}{\mathsf{unbox}}
\renewcommand{\SGradydruleTXXsplitName}{\mathsf{split}}
\renewcommand{\SGradydruleTXXsquashName}{\mathsf{squash}}
\renewcommand{\SGradydruleTXXvarPName}{\text{var}}
\renewcommand{\SGradydruleTXXunitPName}{\mathsf{Unit}}
\renewcommand{\SGradydruleTXXzeroPName}{\mathsf{zero}}
\renewcommand{\SGradydruleTXXsuccName}{\mathsf{succ}}
\renewcommand{\SGradydruleTXXpairName}{\times_i}
\renewcommand{\SGradydruleTXXfstName}{\times_{e_1}}
\renewcommand{\SGradydruleTXXsndName}{\times_{e_2}}
\renewcommand{\SGradydruleTXXlamName}{\to_i}
\renewcommand{\SGradydruleTXXappName}{\to_e}

\renewcommand{\SGradydruleciXXvarName}{}
\renewcommand{\SGradydruleciXXzeroName}{}
\renewcommand{\SGradydruleciXXtrivName}{}
\renewcommand{\SGradydruleciXXsuccUName}{}
\renewcommand{\SGradydruleciXXsuccName}{}
\renewcommand{\SGradydruleciXXncaseUName}{}
\renewcommand{\SGradydruleciXXncaseName}{}
\renewcommand{\SGradydruleciXXpairName}{}
\renewcommand{\SGradydruleciXXfstUName}{}
\renewcommand{\SGradydruleciXXfstName}{}
\renewcommand{\SGradydruleciXXsndUName}{}
\renewcommand{\SGradydruleciXXsndName}{}
\renewcommand{\SGradydruleciXXlamName}{}
\renewcommand{\SGradydruleciXXappUName}{}
\renewcommand{\SGradydruleciXXappName}{}

\renewcommand{\SGradydrulePXXUName}{?}
\renewcommand{\SGradydrulePXXreflName}{\text{refl}}
\renewcommand{\SGradydrulePXXarrowName}{\to}
\renewcommand{\SGradydrulePXXprodName}{\times}
\renewcommand{\SGradydrulePXXlistName}{\mathsf{List}}
\renewcommand{\SGradydrulePXXforallName}{\forall}

\renewcommand{\SGradydruleTPXXreflName}{\text{refl}}
\renewcommand{\SGradydruleTPXXsuccName}{\mathsf{succ}}
\renewcommand{\SGradydruleTPXXNateName}{\mathsf{Nat}}
\renewcommand{\SGradydruleTPXXpairName}{\times_i}
\renewcommand{\SGradydruleTPXXfstName}{\times_{e_1}}
\renewcommand{\SGradydruleTPXXsndName}{\times_{e_2}}
\renewcommand{\SGradydruleTPXXconsName}{\mathsf{List}_i}
\renewcommand{\SGradydruleTPXXListeName}{\mathsf{List}_e}
\renewcommand{\SGradydruleTPXXFunName}{\to_i}
\renewcommand{\SGradydruleTPXXappName}{\to_2}
\renewcommand{\SGradydruleTPXXtfunName}{\forall_i}
\renewcommand{\SGradydruleTPXXtappName}{\forall_e}

\input{core-grady-inc}
\renewcommand{\CGradydruleTXXBoxPName}{\mathsf{box}}
\renewcommand{\CGradydruleTXXUnboxPName}{\mathsf{unbox}}
\renewcommand{\CGradydruleSXXReflName}{\text{refl}}
\renewcommand{\CGradydruleSXXTopName}{\text{top}}
\renewcommand{\CGradydruleSXXVarName}{\text{var}}
\renewcommand{\CGradydruleSXXTopSLName}{\top_{\mathcal{S}}}
\renewcommand{\CGradydruleSXXNatSLName}{\mathsf{Nat}_{\mathcal{S}}}
\renewcommand{\CGradydruleSXXUnitSLName}{\mathsf{Unit}_{\mathcal{S}}}
\renewcommand{\CGradydruleSXXListSLName}{\mathsf{List}_{\mathcal{S}}}
\renewcommand{\CGradydruleSXXArrowSLName}{\to_{\mathcal{S}}}
\renewcommand{\CGradydruleSXXProdSLName}{\times_{\mathcal{S}}}
\renewcommand{\CGradydruleSXXListName}{\mathsf{List}}
\renewcommand{\CGradydruleSXXProdName}{\times}
\renewcommand{\CGradydruleSXXArrowName}{\to}
\renewcommand{\CGradydruleSXXForallName}{\forall}

\renewcommand{\CGradydrule}[4][]{{\displaystyle\frac{\begin{array}{l}#2\end{array}}{#3}\CGradydrulename{#4}}}
\renewcommand{\CGradydrulename}[1]{#1}
\renewcommand{\CGradydruleTXXvarPName}{\text{var}}
\renewcommand{\CGradydruleTXXvarName}{\text{var}}
\renewcommand{\CGradydruleTXXBoxName}{\mathsf{box}}
\renewcommand{\CGradydruleTXXUnboxName}{\mathsf{unbox}}
\renewcommand{\CGradydruleTXXsquashName}{\mathsf{squash}}
\renewcommand{\CGradydruleTXXsplitName}{\mathsf{split}}
\renewcommand{\CGradydruleTXXunitPName}{\mathsf{Unit}}
\renewcommand{\CGradydruleTXXzeroPName}{\text{zero}}
\renewcommand{\CGradydruleTXXsuccName}{\mathsf{succ}}
\renewcommand{\CGradydruleTXXncaseName}{\mathsf{Nat}_e}
\renewcommand{\CGradydruleTXXemptyName}{\text{empty}}
\renewcommand{\CGradydruleTXXconsName}{\mathsf{List}_i}
\renewcommand{\CGradydruleTXXlcaseName}{\mathsf{List}_e}
\renewcommand{\CGradydruleTXXpairName}{\times_i}
\renewcommand{\CGradydruleTXXfstName}{\times_{e_1}}
\renewcommand{\CGradydruleTXXsndName}{\times_{e_2}}
\renewcommand{\CGradydruleTXXlamName}{\to_i}
\renewcommand{\CGradydruleTXXappName}{\to_e}
\renewcommand{\CGradydruleTXXLamName}{\forall_i}
\renewcommand{\CGradydruleTXXtypeAppName}{\forall_e}
\renewcommand{\CGradydruleTXXSubName}{\text{sub}}
\renewcommand{\CGradydruleTXXerrorName}{\text{error}}

\renewcommand{\CGradydrulerdXXretracTName}{\text{retract}}
\renewcommand{\CGradydrulerdXXretracTEName}{\text{raise}}
\renewcommand{\CGradydrulerdXXretractUName}{\text{retract}_2}
\renewcommand{\CGradydrulerdXXncaseZeroName}{\mathsf{Nat}_{e_1}}
\renewcommand{\CGradydrulerdXXncaseSuccName}{\mathsf{Nat}_{e_2}}
\renewcommand{\CGradydrulerdXXlcaseEmptyName}{\mathsf{List}_{e_1}}
\renewcommand{\CGradydrulerdXXlcaseConsName}{\mathsf{List}_{e_2}}
\renewcommand{\CGradydrulerdXXbetaName}{\beta}
\renewcommand{\CGradydrulerdXXprojOneName}{\times_{e_1}}
\renewcommand{\CGradydrulerdXXprojTwoName}{\times_{e_2}}
\renewcommand{\CGradydrulerdXXtypeBetaName}{\text{type}_{\beta}}
\renewcommand{\CGradydrulerdXXtypeAppName}{}
\renewcommand{\CGradydrulerdXXCongName}{\text{cong}}
\renewcommand{\CGradydrulerdXXerrorName}{\text{error}}

\renewcommand{\CGradydruleTPXXunboxingName}{\mathsf{box}}
\renewcommand{\CGradydruleTPXXboxingName}{\mathsf{unbox}}
\renewcommand{\CGradydruleTPXXsplitingName}{\mathsf{split}}
\renewcommand{\CGradydruleTPXXsquashingName}{\mathsf{squash}}
\renewcommand{\CGradydruleTPXXerrorName}{\mathsf{error}}

\renewcommand{\CGradydruleCtxPXXreflName}{\text{refl}}
\renewcommand{\CGradydruleCtxPXXextName}{\text{ext}}

\input{siek15-gradual-inc}
\renewcommand{\GSiekdrule}[4][]{{\displaystyle\frac{\begin{array}{l}#2\end{array}}{#3}\,\GSiekdrulename{#4}}}
\renewcommand{\GSiekdrulename}[1]{#1}
\renewcommand{\GSiekdruleSXXvarName}[0]{\text{var}}
\renewcommand{\GSiekdruleSXXunitName}[0]{\text{unit}}
\renewcommand{\GSiekdruleSXXzeroName}[0]{\text{zero}}
\renewcommand{\GSiekdruleSXXsuccName}[0]{\text{succ}}
\renewcommand{\GSiekdruleSXXpairName}[0]{\times}
\renewcommand{\GSiekdruleSXXlamName}[0]{\to}
\renewcommand{\GSiekdruleSXXsndName}[0]{\times_{e_2}}
\renewcommand{\GSiekdruleSXXfstName}[0]{\times_{e_1}}
\renewcommand{\GSiekdruleSXXappName}[0]{\to_e}

\renewcommand{\GSiekdruleCXXvarName}[0]{\text{var}}
\renewcommand{\GSiekdruleCXXerrorName}[0]{\text{error}}
\renewcommand{\GSiekdruleCXXunitName}[0]{\text{unit}}
\renewcommand{\GSiekdruleCXXzeroName}[0]{\text{zero}}
\renewcommand{\GSiekdruleCXXsuccName}[0]{\text{succ}}
\renewcommand{\GSiekdruleCXXpairName}[0]{\times}
\renewcommand{\GSiekdruleCXXlamName}[0]{\to}
\renewcommand{\GSiekdruleCXXsndName}[0]{\times_{e_2}}
\renewcommand{\GSiekdruleCXXfstName}[0]{\times_{e_1}}
\renewcommand{\GSiekdruleCXXappName}[0]{\to_e}
\renewcommand{\GSiekdruleCXXcastName}[0]{\text{cast}}
\renewcommand{\GSiekdrulerdAXXcastIdName}{\text{id-atom}}
\renewcommand{\GSiekdrulerdAXXcastUName}{\text{id-U}}
\renewcommand{\GSiekdrulerdAXXsucceedName}{\text{succeed}}
\renewcommand{\GSiekdrulerdAXXfailName}{\text{fail}}
\renewcommand{\GSiekdrulerdAXXcastArrowName}{\to_\Rightarrow}
\renewcommand{\GSiekdrulerdAXXcastGroundName}{\text{expand}_1}
\renewcommand{\GSiekdrulerdAXXcastExpandName}{\text{expand}_2}
\renewcommand{\GSiekdrulereflName}[0]{\text{refl}}
\renewcommand{\GSiekdruleboxName}[0]{\text{box}}
\renewcommand{\GSiekdruleunboxName}[0]{\text{unbox}}
\renewcommand{\GSiekdrulearrowName}[0]{\to}
\renewcommand{\GSiekdruleprodName}[0]{\times}

\newenvironment{typeProofCase}
               {\setlength{\tabcolsep}{1px}\begin{tabular}{cc}\begin{tabular}{l}Case:\\\\\end{tabular}&\begin{math}}
               {\end{math}\end{tabular}}

\newenvironment{ack}{\textbf{Acknowledgments.}}{}

\newcommand{\CGSTLC}{\lambda^{\Rightarrow}_\to} 

\begin{document}

\title{The Combination of Dynamic and Static Typing from a Categorical Perspective}
\author{Harley Eades III}
\affiliation{
  \department{Computer Science}              %% \department is recommended
  \institution{Augusta University}            %% \institution is required
  \city{Augusta}
  \state{GA}
  \country{USA}
}
\email{harley.eades@gmail.com}          %% \email is recommended

\author{Michael Townsend}
\affiliation{
  \department{Computer Science}              %% \department is recommended
  \institution{Augusta University}            %% \institution is required
  \city{Augusta}
  \state{GA}
  \country{USA}
}
\email{mitownsend@augusta.edu}          %% \email is recommended

\thanks{Both authors where supported by the National Science
  Foundation CRII CISE Research Initiation grant, ``CRII:SHF: A New
  Foundation for Attack Trees Based on Monoidal Categories``, under
  Grant No. 1565557.}

%% 2012 ACM Computing Classification System (CSS) concepts
%% Generate at 'http://dl.acm.org/ccs/ccs.cfm'.
\begin{CCSXML}
<ccs2012>
<concept>
<concept_id>10003752.10010124.10010131.10010133</concept_id>
<concept_desc>Theory of computation~Denotational semantics</concept_desc>
<concept_significance>500</concept_significance>
</concept>
<concept>
<concept_id>10003752.10010124.10010131.10010137</concept_id>
<concept_desc>Theory of computation~Categorical semantics</concept_desc>
<concept_significance>500</concept_significance>
</concept>
<concept>
<concept_id>10003752.10003790.10011740</concept_id>
<concept_desc>Theory of computation~Type theory</concept_desc>
<concept_significance>300</concept_significance>
</concept>
<concept>
<concept_id>10003752.10010124.10010125.10010127</concept_id>
<concept_desc>Theory of computation~Functional constructs</concept_desc>
<concept_significance>300</concept_significance>
</concept>
<concept>
<concept_id>10003752.10010124.10010125.10010130</concept_id>
<concept_desc>Theory of computation~Type structures</concept_desc>
<concept_significance>300</concept_significance>
</concept>
</ccs2012>
\end{CCSXML}

\ccsdesc[500]{Theory of computation~Denotational semantics}
\ccsdesc[500]{Theory of computation~Categorical semantics}
\ccsdesc[300]{Theory of computation~Type theory}
\ccsdesc[300]{Theory of computation~Functional constructs}
\ccsdesc[300]{Theory of computation~Type structures}
%% End of generated code

\keywords{  
    static typing, dynamic typing, gradual typing, categorical
    semantics, retract,typed lambda-calculus, untyped lambda-calculus,
    functional programming}

\begin{abstract} 
  In this paper we introduce a new categorical model based on retracts
  that combines static and dynamic typing.  In addition, this model
  formally connects gradual typing to the seminal work of Scott and
  Lambek who showed that the untyped $\lambda$-calculus can be
  considered as typed using retracts, and that the type
  $\lambda$-calculus can be modeled in a cartesian closed category
  respectively.  Following this we extract from our model a new simple
  type system which combines static and dynamic typing called Core
  Grady.  Then we develop a gradually typed surface language for Core
  Grady, and show that it can be translated into the core such that
  the gradual guarantee holds. In addition, to show that the wider
  area of gradual type systems can benefit from our model we show that
  Siek and Taha's gradual simply typed $\lambda$-calculus can be
  modeled by the proposed semantics.  Finally, while a gradual type
  system allows for type casts to be left implicit we show that
  explicit casts can be derived in the gradually typed surface
  language, and using the explicit casts we show that more programs
  can be typed.  For example, we define a typed fixpoint operator that
  can only be defined due to the explicit casts in the gradually typed
  surface language.
\end{abstract}

\maketitle

\section{Introduction}
\label{sec:introduction}
\input{introduction-ott}
% section introduction (end)

\section{The Categorical Model}
\label{subsec:the_categorical_model}
\input{categorical-model-ott}
% section the_interpretation (end)

\section{Core Grady}
\label{sec:core_grady}
\input{core-grady-ott}
% section core_grady (end)

\section{Surface Grady}
\label{sec:surface_grady}
\input{surface-grady-ott}
% section surface_grady (end)

\section{Modeling Siek and Taha's Gradual $\lambda$-Calculus}
\label{sec:modeling_siek_and_taha's_gradual_lambda-calculus}
\input{siek15-ott}
% section modeling_siek_and_taha's_gradual_lambda-calculus (end)

\section{The Gradual Guarantee}
\label{sec:the_gradual_guarantee}
\input{gradual-guarantee-ott}
% section the_gradual_guarantee (end)

\section{Explicit Casts in Gradual Type Systems}
\label{sec:explicit_casts_in_gradual_type_systems}
\input{explicit-casts-ott}
% section explicit_casts_in_gradual_type_systems (end)

\section{Related Work}
\label{sec:related_work}
\input{related-work-ott}
% section related_work (end)


\section{Future Work}
\label{sec:conclusion}
The categorical model and corresponding type system presented here
sets the stage for a number of interesting lines of research.  Linear
typing has a number of applications in functional programming
\cite{Wadler:1990}. For example, allowing values of linear type to
change the world.  Thus, it is worthwhile combining gradual typing
with linear typing.  Benton \cite{Benton:1995} showed that the
statically typed $\lambda$-calculus can be mixed with the statically
typed linear $\lambda$-calculus by taking a symmetric monoidal
adjunction between a cartesian closed category and a symmetric
monoidal closed category called a mixed linear/non-linear model or LNL
model.  Gradual $\lambda$-models defined here make up half of a LNL
model mixing gradual typing with linear typing.  We are actively
working on this new combination.

Gradual $\lambda$-models correspond to the core language of a gradual
type system.  In the future we will investigate the categorical model
for the surface language and a functorial relationship between this
new model and gradual $\lambda$-models.  The most interesting aspect
of this model will be how to handle implicit casting.  Then we will be
able to state and prove a model theoretic version of the gradual
guarantee.  Another interesting aspect will be in how to deal with
type and term precision in the model.
% section conclusion (end)

\section{Acknowledgments}
\label{sec:acknowledgments}
The authors thank Jeremy Siek for his feedback on previous drafts of
this paper.  The first author thanks Ronald Garcia for his wonderful
invited talk at Trends in Functional Programming 2016 which introduced
the first author to gradual typing and its open problems.  This paper
was typeset with the help of the amazing Ott tool \cite{Sewell:2010}.
% section acknowledgments (end)


\bibliographystyle{ACM-Reference-Format}
\bibliography{references}

\appendix

\input{aux-results-ott}

\input{proofs-ott}

% section proofs (end)
\end{document}
