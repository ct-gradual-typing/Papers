One strength and main motivation for giving a categorical model to a
programming language is that it can expose the fundamental structure
of the language.  This arises because a lot of the language features
that often cloud the picture go away, for example, syntactic notions
like variables disappear.  This can often simplify things and expose
the underlying structure.  For example, when giving the simply typed
$\lambda$-calculus a categorical model we see that it is a cartesian
closed category, but we also know that intuitionistic logic has the
same model due to Lambek~\cite{Lambek:1980}; on the syntactic side
these two theories are equivalent as well due to
Howard~\cite{Howard:1980}; this is known as the Curry-Howard-Lambek
correspondence.

The previous point highlights one of the most powerful features of
category theory: its ability to relate seemingly unrelated theories.
It is quite surprising that the typed $\lambda$-calculus and
intuitionistic logic share the same model.  Thus, defining a
categorical model for a particular programming language may reveal new
and interesting relationships with existing work.  In fact, one of the
contributions of this paper is the new connection between Scott and
Lambek's work to the new area of gradual typing and combing static and
dynamic typing.

However, that motivation places defining a categorical model as an
after thought.  The Grady languages developed here were designed from
the other way around.  We started with the question, how do we combine
static and dynamic typing categorically? Then after developing the
model we use it to push us toward the correct language design.
Reynolds \cite{Brookes:2013} was a big advocate for the use of
category theory in programming language research for this reason.  We
think the following quote -- reported by Brookes et
al.~\cite{Brookes:2013} -- makes this point nicely:
\begin{center}
  \parbox{170pt}{
    \emph{Programming language semanticists should be the obstetricians
    of programming languages, not their coroners.\\
    {\rightline{--- John C. Reynolds}}}}
\end{center}

Categorical semantics provides a powerful tool to study language
extensions.  For example, purely functional programming in Haskell
would not be where it is without the seminal work of Moggi and Wadler
\cite{Moggi:1989,Wadler:1995} on using monads -- a purely categorical
notion -- to add side effects to to pure functional programming
languages.  Thus, we believe that developing these types of models for
new language designs and features can be hugely beneficial.

The model we develop here builds on the seminal work of
Lambek~\cite{Lambek:1980} and Scott~\cite{Scott:1980}.
Lambek~\cite{Lambek:1980} showed that the typed $\lambda$-calculus can
be modeled by a cartesian closed category.  In the same volume as
Lambek, Scott essentially showed that the untyped $\lambda$-calculus
is actually typed.  That is, typed theories are more fundamental than
untyped ones.  He accomplished this by adding a single type, $[[?]]$,
and two functions $[[squash : (? -> ?) -> ?]]$ and $[[split : ? -> (?
    -> ?)]]$, such that, $[[squash]];[[split]] = \id : [[(? -> ?) ->
    (?  -> ?)]]$, to a cartesian closed category. At this point he was
able to translate the untyped $\lambda$-calculus into this unityped
one.

Categorically, Scott modeled $[[split]]$ and $[[squash]]$ as the
morphisms in a retract within a cartesian closed category -- the same
model as typed $\lambda$-calculus.
\begin{definition}
  \label{def:retract}
  Suppose $\cat{C}$ is a category.  Then an object $A$ is a
  \textbf{retract} of an object $B$ if there are morphisms $i : A \mto
  B$ and $r : B \mto A$ such that $i;r = \id_A$.%% the following diagram commutes:
  %% \[
  %% \bfig
  %% \qtriangle/->`=`->/[A`B`A;i``r]
  %% \efig
  %% \]
\end{definition}
\noindent
Thus, $[[? -> ?]]$ is a retract of $[[?]]$, but we also require that
$[[? x ?]]$ be a retract of $[[?]]$; this is not new, see Lambek and
Scott~\cite{Lambek:1988}.  Putting this together we obtain Scott's
model of the untyped $\lambda$-calculus.
\begin{definition}
  \label{def:model-untyped}
  An \textbf{untyped $\lambda$-model}, $(\cat{C}, ?, \split,
  \squash)$, is a cartesian closed category $\cat{C}$ with a
  distinguished object $?$ and morphisms $\squash : S \mto ?$ and
  $\split : [[?]] \mto S$ making the object $S$ a retract of $?$, where
  $S$ is either $[[? -> ?]]$ or $[[? x ?]]$.
\end{definition}

\begin{theorem}[Scott~\cite{Scott:1980}]
  \label{thm:untyped-lambda-model-sound-complete}
  An untyped $\lambda$-model is a sound and complete model of the untyped $\lambda$-calculus.
\end{theorem}

So far we know how to model static types (typed $\lambda$-calculus)
and unknown types (the untyped $\lambda$-calculus).  The to make the
Grady languages a bit more interesting we add natural numbers, but we
will need a way to model these in a cartesian closed category.

We model the natural numbers with their (non-recursive) eliminator
using what we call a non-recursive natural number object.  This is a
simplification of the traditional natural number object; see Lambek
and Scott~\cite{Lambek:1988}.
\begin{definition}
  \label{def:SNNO}
  Suppose $\cat{C}$ is a cartesian closed category.  A
  \textbf{non-recursive natural number object (NRNO)} is an object
  $[[Nat]]$ of $\cat{C}$ and morphisms $\mathsf{z} : 1 \mto [[Nat]]$
  and $[[succ]] : [[Nat]] \mto [[Nat]]$ of $\cat{C}$, such that, for
  any morphisms $f : [[Y]] \mto X$ and $g : [[Y x Nat]] \mto X$ of
  $\cat{C}$ there is an unique morphism $\Case_{Y,X} \langle f,g \rangle : [[Y x Nat]] \mto X$
  such that the following hold:
  \[\small
  \setlength{\arraycolsep}{0pt}
  \begin{array}{cccccc}    
    \langle \id_Y, \diamond_Y;\mathsf{z} \rangle;\Case_{Y,X} \langle f,g \rangle = f & \quad &
    \langle \id_Y \times \mathsf{succ} \rangle;\Case_{Y,X} \langle f,g \rangle = g\\
  \end{array}
  \]  
  \noindent
       Informally, the two equations essentially assert that we can
       define $\Case_{Y,X}$ as follows:
       \begin{center}
         \begin{math}
           \setlength{\arraycolsep}{3pt}
           \begin{array}{ccccccccccc}
             {\Case_{Y,X} \langle f,g \rangle\,y\,0 = f\,y}
             & \quad & 
             {\Case_{Y,X} \langle f,g \rangle\,y\,[[(succ n)]] =  g\,y\,n}
           \end{array}
         \end{math}
       \end{center}
\end{definition}

At this point we can model both static and unknown types with natural
numbers in a cartesian closed category, but we do not have any way of
moving typed data into the untyped part and vice versa to obtain
dynamic typing.  To accomplish this we add two new morphisms $[[box C
    : C --> ?]]$ and $[[unbox C : ?  --> C]]$ such that each atomic
type, $[[C]]$, is a retract of $[[?]]$.  This enforces that the only
time we can cast $[[?]]$ to another type is if it were boxed up in the
first place.  Combining all of these insights we obtain the complete
categorical model.
\begin{definition}
  \label{def:gradual-lambda-model}
  A \textbf{gradual $\lambda$-model}, $(\cat{T}, \cat{C}, ?,
  \T,\split,$\\ $ \squash, \bx, \unbox, \error)$, where $\cat{T}$ is a
  discrete category with at least two objects $[[Nat]]$ and
  $[[Unit]]$, $\cat{C}$ is a cartesian closed category with a NRNO,
  $(\cat{C},?,\split,$ $\squash)$ is an untyped $\lambda$-model, $\T :
  \cat{T} \mto \cat{C}$ is an embedding -- a full and faithful functor
  that is injective on objects -- and for every object $A$ of
  $\cat{T}$ there are morphisms $\bx_A : TA \mto ?$ and $\unbox_A : [[?]]
  \mto TA$ making $TA$ a retract of $?$.  Furthermore, to model
  dynamic type errors, there is a morphism, $\err_A : [[Unit --> A]]$
  of $\cat{C}$, such that, the following equations hold w.r.t.
  $\error_{A,B} = [[A]] \mto^{[[triv]]_{A}} [[Unit]] \mto^{\err_{B}} [[B]]$:
  %% diagrams commute for any  morphisms $f : [[A --> B]]$ and $g : [[B --> C]]$:
  \begin{center} 
    \begin{math} \footnotesize
      \setlength{\arraycolsep}{2pt}
      \begin{array}{rll}
        \bx_{T A};\unbox_{T B} & = & \error_{T A,T B}, \text{ where } [[A]] \neq [[B]]\\
        \squash_{S_1};\split_{S_2} & = & \error_{S_1,S_2}, \text{ where } [[S1]] \neq [[S2]]\\
        f;\error_{B,C} & = & \error_{A,C}, \text{ where } f : A \mto B\\
        \error_{A,B};f & = & \error_{A,C}, \text{ where } f : B \mto C\\
        \langle \error_{A,B}, f\rangle & = & \error_{A,[[B x C]]}, \text{ where } f : A \mto C\\
        \langle f , \error_{A,C}\rangle & = & \error_{A,[[B x C]]}, \text{ where } f : A \mto B\\
        \curry{\error_{[[A x B]],C}} & = & \error_{A,[[B -> C]]}\\
      \end{array}
    \end{math}
  \end{center}  
\end{definition}
\noindent
We call the category $\cat{T}$ the category of atomic types.  We call
an object, $[[A]]$, \textbf{atomic} iff there is some object $[[A']]$
in $\cat{T}$ such that $[[A]] = \T[[A']]$. Note that we do not
consider $[[?]]$ an atomic type.

Triggering dynamic type errors is a fundamental property of the
criteria for gradually typed languages, and thus, the model must
capture this.  The new morphism $\err_A : [[Unit --> A]]$ is combined
with the terminal morphism, $[[triv]]_A : [[A --> Unit]]$, which is a
unique morphism guaranteed to exist because $\cat{C}$ is cartesian
closed, to define the morphism $\error_{A,B} : [[A --> B]]$ that
signifies that one tried to unbox or split at the wrong type resulting
in a dynamic type error; this is captured by the first and second
equations in the definition.  If we view morphisms as programs, then
the other equations are congruence rules that trigger a dynamic type
error for the whole program when one of its subparts trigger a dynamic
type error.  The following extends the error equations to the functors
$- \times -$ and $- \to -$:
\begin{lemma}[Extended Errors]
  \label{lemma:extended_errors}
  Suppose $(\cat{T}, \cat{C}, ?, \T, \split,$\\$\squash, \bx, \unbox,\error)$
  is a gradual $\lambda$-model.  Then the following equations hold:
  \[\footnotesize
  \setlength{\arraycolsep}{2pt}
  \begin{array}{rll}
    f \times \error_{B,C} & = & \error_{[[A x B]],[[C x D]]}, \text{ where } f : A \mto C\\
    \error_{A,C} \times f & = & \error_{[[A x B]],[[C x D]]}, \text{ where } f : B \mto D\\
    f \to \error_{B,C}    & = & \error_{[[A -> B]],[[C -> D]]}, \text{ where } f : C \mto A\\
    \error_{C,A} \to f & = & \error_{[[A -> B]],[[C -> D]]}, \text{ where } f : B \mto D\\
  \end{array}
  \]
\end{lemma}
\begin{proof}
  The following define the morphism part of the two functors $f \times
  g : [[(A x B) --> (C x D)]]$ and $f \to g : [[(A -> B) --> (C ->
      D)]]$:
  \[ \footnotesize
  \setlength{\arraycolsep}{2pt}
  \begin{array}{rll}
    f \times g & = & \langle [[fst]];f , [[snd]];g \rangle, \\
    & & \,\,\,\,\text{ where } f : A \mto C\text{ and }g : B \mto D\\
    \\
    f \to g & = & \curry{(\id_{[[A -> B]]} \times f);\app_{A,B};g}, \\
    & & \,\,\,\,\text{ where } f : C \mto A\text{ and }g : B \mto D\\    
  \end{array}
  \]
  First, note that $[[fst]] : [[(A x B) --> A]]$, $[[snd]] : [[(A x B) -->
      B]]$, and $\app_{A,B} : [[((A -> B) x A) --> B]]$ all exist by the
  definition of a cartesian closed category.
  
  It is now quite obvious that if either $f$ or $g$ is $\error$ in the
  previous two definitions, then by using the equations from the
  definition of a gradual $\lambda$-model
  (Definition~\ref{def:gradual-lambda-model}) the application of
  either of the functors will result in $\error$.
\end{proof}

As the model is defined it is unclear if we can cast any type to
$[[?]]$, and vice versa, but we must be able to do this in order to
model full dynamic typing.  In the remainder of this section we show
that we can build up such casts in terms of the basic features of our
model.  To cast any type $[[A]]$ to $[[?]]$ we will build casting
morphisms that first take the object $[[A]]$ to its skeleton, and then
takes the skeleton to $[[?]]$.
\begin{definition}
  \label{def:casting-mor}
  Suppose $(\cat{T}, \cat{C}, ?, \T, \split,\squash, \bx,$\\ $\unbox, \error)$
  is a gradual $\lambda$-model.  Then we call any morphism defined
  completely in terms of $\id$, the functors $- \times -$ and $- \to
  -$, $[[split]]$ and $[[squash]]$, and $[[box]]$ and $[[unbox]]$ a
  \textbf{casting morphism}.
\end{definition}

\begin{definition}
  \label{def:skeleton}
  Suppose $(\cat{T}, \cat{C}, ?, \T, \split,\squash, \bx,$\\ $\unbox, \error)$
  is a gradual $\lambda$-model.  The \textbf{skeleton} of an object
  $[[A]]$ of $\cat{C}$ is an object $[[S]]$ that is constructed by
  replacing each atomic type in $[[A]]$ with $[[?]]$. Given an object
  $[[A]]$ we denote its skeleton by $[[skeleton A]]$.
\end{definition}
One should think of the skeleton of an object as the supporting type
structure of the object, but we do not know what kind of data is
actually in the structure. For example, the skeleton of the object
$[[Nat]]$ is $[[?]]$, and the skeleton of $[[(Nat x Unit) -> Nat ->
    Nat]]$ is $[[(?  x ?) -> ? -> ?]]$.

The next definition defines a means of constructing a casting morphism
that casts a type $[[A]]$ to its skeleton and vice versa.  This
definition is by mutual recursion on the input type.
\begin{definition}
  \label{def:boxing-unboxing}
  Suppose $(\cat{T}, \cat{C}, ?, \T, \split,\squash, \bx,$\\ $\unbox, \error)$
  is a gradual $\lambda$-model.  Then for any object $[[A]]$ whose
  skeleton is $[[S]]$ we define the morphisms $[[lbox A]] : [[A -->
      S]]$ and $[[lunbox A]] : [[S --> A]]$ by mutual recursion on
  $[[A]]$ as follows:
  \[\footnotesize
  \setlength{\arraycolsep}{1pt}
  \begin{array}{ll|ll}
    \begin{array}{lll}
      [[lbox A]] = [[box A]]\\
    \,\,\,\,\text{when } [[A]]\text{ is atomic}\\
    [[lbox ?]] = \id_{[[?]]}\\
    [[lbox (A1 -> A2)]] = [[lunbox A1]] \to [[lbox A2]]\\
    [[lbox (A1 x A2)]] = [[lbox A1]] \times [[lbox A2]]\\        
    \end{array}
    & & \quad & 
    \begin{array}{lll}
    [[lunbox A]] = [[unbox A]]\\
    \,\,\,\,\text{when } [[A]]\text{ is atomic}\\
    [[lunbox ?]] = \id_{[[?]]}\\
    [[lunbox (A1 -> A2)]] = [[lbox A1]] \to [[lunbox A2]]\\
    [[lunbox (A1 x A2)]] = [[lunbox A1]] \times [[lunbox A2]]\\        
    \end{array}
  \end{array}
  \]
\end{definition}
\noindent
The definition of both $\widehat{[[box]]}$ or $\widehat{[[unbox]]}$
uses the functor $- \to - : \catop{C} \times \cat{C} \mto \cat{C}$
which is contravariant in its first argument, and thus, in that
contravariant position we must make a recursive call to the opposite
function, and hence, they must be mutually defined. Every call to
either $\widehat{[[box]]}$ or $\widehat{[[unbox]]}$ in the previous
definition is on a smaller object than the input object.  Thus, their
definitions are well founded.  Furthermore, $\widehat{[[box]]}$ and
$\widehat{[[unbox]]}$ form a retract between $[[A]]$ and $[[S]]$.
\begin{lemma}[Boxing and Unboxing Lifted Retract]
  \label{lemma:lifted_retract}
  Suppose $(\cat{T}, \cat{C}, ?, \T, \split,$ $\squash, \bx,$ $\unbox,
  \error)$ is a gradual $\lambda$-model.  Then for any object $[[A]]$,
  $[[lbox A]];[[lunbox A]] = \id_A : [[A --> A]]$.  Furthermore, for
  any objects $[[A]]$ and $[[B]]$ such that $[[A]] \neq [[B]]$,
  $[[lbox A]];[[lunbox B]] = \error_{A,B}$.
\end{lemma}
\begin{proof}  
  This proof holds by induction on the form $[[A]]$.  Please see
  Appendix~\ref{subsec:proof_of_lifted_retract} for the complete
  proof.
\end{proof}
\noindent
As an example, suppose we wanted to cast the type $[[(Nat x ?) ->
    Nat]]$ to its skeleton $[[(? x ?) -> ?]]$.  Then we can obtain a
casting morphisms that will do this as follows:
\[
\begin{array}{lll}
  [[lbox ((Nat x ?) -> Nat)]] = ([[unbox Nat]] \times \id_{[[?]]}) \to [[box Nat]]\\
\end{array}
\]

We can also cast a morphism $[[A]] \mto^f [[B]]$ to a morphism
$[[lunbox A]];f;$$[[lbox A]] : [[S1]] \mto [[S2]]$
where $[[S1]] = [[skeleton A]]$ and $[[S2]] = [[skeleton B]]$.  Now if
we have a second
$[[lunbox B]];g;[[lbox C]] : [[S2]] \mto [[S3]]$
then their composition reduces to composition at the typed level:
\begin{center}
  \begin{math}
    \bfig
\square|amma|/->`->`->`<-/<500,500>[
  [[S1]]`
  [[A]]`
  [[S3]]`
  [[C]];
  [[lunbox A]]``
  f;g`
  [[lbox C]]
]

\square(500,0)|amma|/->`->`=`<-/<500,500>[
  [[A]]`
  [[B]]`
  [[C]]`
  [[B]];
  f`
  f;g``
  g]

\square(1000,0)|amma|/->`=`=`<-/<500,500>[
  [[B]]`
  [[S2]]`
  [[B]]`
  [[S2]];
  [[lbox B]]```
  [[lunbox B]]
]
\efig
  \end{math}
\end{center} 
The right most diagram commutes because $[[B]]$ is a retract of
$[[S2]]$, and the left unannotated arrow is the composition $[[lunbox A]];f;$ $g;[[lbox C]]$.  This tells us that we have a functor $\S
: \cat{C} \mto \cat{S}$:
\[
\begin{array}{lll}
  \S[[A]] = [[skeleton A]]\\
  \S (f : [[A --> B]]) = [[lunbox A]];f;[[lbox A]]
\end{array}
\]
where $\cat{S}$ is the full subcategory of $\cat{C}$ consisting of the
skeletons and morphisms between them, that is, $\cat{S}$ is a
cartesian closed category with one basic object $[[?]]$ such that
$(\cat{S},[[?]],[[split]],[[squash]])$ is an untyped $\lambda$-model.
The following turns out to be true.
\begin{lemma}[$\S$ is faithful]
  \label{lemma:S_is_faithful}
  Suppose $(\cat{T}, \cat{C}, ?, \T, \split,\squash,$\\ $\bx, \unbox, \error)$
  is a gradual $\lambda$-model, and
  $(\cat{S},[[?]],[[split]],[[squash]])$ is the category of skeletons.
  Then the functor $\S : \cat{C} \mto \cat{S}$ is faithful.
\end{lemma}
\begin{proof}
  This proof follows from the definition of $\S$ and Lemma \ref{lemma:lifted_retract}.  For the full proof see
  Appendix~\ref{subsec:proof_of_S_is_faithful}.
\end{proof}
\noindent
Thus, we can think of the functor $\S$ as an injection of the typed
world into the untyped one.

Now that we can cast any type into its skeleton we must show that
every skeleton can be cast to $[[?]]$.  We do this similarly to the
above and lift $[[split]]$ and $[[squash]]$ to arbitrary skeletons.
\begin{definition}
  \label{def:lifted-split-squash}
  Suppose $(\cat{S}, [[?]], [[split]],[[squash]])$ is the category of
  skeletons.  Then for any skeleton $[[S]]$ we define the morphisms
  $[[lsquash S]] : [[S --> ?]]$ and $[[lsplit S]] : [[? --> S]]$ by
  mutual recursion on $[[S]]$ as follows:
  \[ \footnotesize
    \begin{array}{l}
      \begin{array}{lll}
        [[lsquash ?]] = \id_?\\
        [[lsquash (S1 -> S2)]] = ([[lsplit S1]] \to [[lsquash S2]]);[[squash ? -> ?]]\\
        [[lsquash (S1 x S2)]] = ([[lsquash S1]] \times [[lsquash S2]]);[[squash ? x ?]]\\
      \end{array}
      \vspace{2px}\\[7px]
      \hline 
      \begin{array}{lll}
        \\ [-6px]
        [[lsplit ?]] = \id_?\\
        [[lsplit (S1 -> S2)]] = [[split ? -> ?]];([[lsquash S1]] \to [[lsplit S2]])\\
        [[lsplit (S1 x S2)]] = [[split ? x ?]];([[lsplit S1]] \times [[lsplit S2]])\\      
      \end{array}
    \end{array}  
  \]
\end{definition}
\noindent
As an example we will construct the casting morphism that casts the
skeleton $[[(? x ?) -> ?]]$ to $[[?]]$:
\[ [[lsquash (? x ?) -> ?]] = ([[split ? x ?]] \to \id_?);[[squash ? -> ?]]. \]

Just as we saw above, splitting and squashing forms a retract.
\begin{lemma}[Splitting and Squashing Lifted Retract]
  \label{lemma:lifted_retract_for_?}
  Suppose $(\cat{S}, [[?]], [[split]],[[squash]])$ is the category of
  skeletons.  Then for any skeleton $[[S]]$, $[[lsquash S]];[[lsplit S]] = \id_S : [[S --> S]]$.
  Furthermore, for
  any skeletons $[[S1]]$ and $[[S2]]$ such that $[[S1]] \neq [[S2]]$,
  $[[lsquash S1]];[[lsplit S2]] = \error_{S1,S2}$.
\end{lemma}
\begin{proof}
  The proof is similar to the proof of the boxing and unboxing lifted
  retract (Lemma~\ref{lemma:lifted_retract}).
\end{proof}
\noindent
There is also a faithful functor from $\cat{S}$ to $\cat{U}$ where
$\cat{U}$ is the full subcategory of $\cat{S}$ that consists of the
single object $[[?]]$ and all its morphisms between it:
\[
\begin{array}{lll}
  \U S = ?\\
  \U (f : [[S1 --> S2]]) = [[lsplit S1]];f;[[lsquash S2]]
\end{array}
\]
This finally implies that there is a functor $\C : \cat{C} \mto
\cat{U}$ that injects all of $\cat{C}$ into the object $[[?]]$.
\begin{lemma}[Casting to $[[?]]$]
  \label{lemma:casting_to_?}
  Suppose $(\cat{T}, \cat{C}, ?, \T, \split,\squash,$\\ $\bx, \unbox, \error)$
  is a gradual $\lambda$-model, $(\cat{S},[[?]],[[split]],[[squash]])$
  is the full subcategory of skeletons, and $(\cat{U},[[?]])$ is the
  full subcategory containing only $[[?]]$ and its morphisms.  Then
  there is a faithful functor $\C = \cat{C} \mto^{\S} \cat{S} \mto^{\U} \cat{U}$.
\end{lemma}
\noindent
In a way we can think of $\C : \cat{C} \mto \cat{U}$ as a forgetful
functor.  It forgets the type information.

Getting back the typed information is harder.  There is no nice
functor from $\cat{U}$ to $\cat{C}$, because we need more information.
However, given a type $[[A]]$ we can always obtain a casting morphism
from $[[?]]$ to $[[A]]$ by $[[(lsplit (skeleton A))]];([[lunbox A]]) :
[[? --> A]]$.  Finally, we have the following result.
\begin{lemma}[Casting Morphisms to $[[?]]$]
  \label{lemma:casting_morphisms}
  Suppose $(\cat{T}, \cat{C}, ?, \T, $ $\split, \squash, \bx, \unbox, \error)$ is
  a gradual $\lambda$-model, and $[[A]]$ is an object of $\cat{C}$.
  Then there exists casting morphisms from $[[A]]$ to $[[?]]$ and vice
  versa that make $[[A]]$ a retract of $[[?]]$.
\end{lemma}
\begin{proof}
  The two morphisms are as follows:
  \[
  \begin{array}{lll}
    [[Box A]] := [[lbox A]];[[lsquash (skeleton A)]] : [[A --> ?]]\\
    [[Unbox A]] := [[lsplit (skeleton A)]];[[lunbox A]] : [[? --> A]]
  \end{array}
  \]
  \noindent
  The fact the these form a retract between $[[A]]$ and $[[?]]$, and
  raise dynamic type errors holds by Lemma~\ref{lemma:lifted_retract}
  and Lemma~\ref{lemma:lifted_retract_for_?}.
\end{proof}
The previous result has a number of implications.  It completely
brings together the static and dynamic fragments of the gradual
$\lambda$-model, and thus, fully relating the combination of dynamic
and static typing to the past work of Lambek and Scott
\cite{Scott:1980,Lambek:1980}.  It will allow for the definition of
casting morphisms between arbitrary objects; see the next result.
Finally, from a practical perspective it will simplify our
corresponding type systems derived from this model, because $[[Box S]]
= [[lsquash S]]$ and $[[Unbox S]] = [[lsplit S]]$ when $[[S]]$ is a
skeleton, and hence, we will only need a single retract in the
corresponding type systems.

\begin{corollary}[Arbitrary Casting Morphisms]
  \label{corollary:arbitrary_casting_morphisms}
  Suppose $(\cat{T}, $ $ \cat{C}, ?, \T, \split, \squash, \bx, \unbox,
  \error)$ is a gradual $\lambda$-model, and $[[A]]$ and $[[B]]$ are
  objects of $\cat{C}$.  Then there exists casting morphisms from
  $[[c1]] : [[A --> B]]$ and $[[c2]] : [[B --> A]]$.
\end{corollary}
\begin{proof}
  The two morphisms are as follows:
  \[
  \begin{array}{lll}
    [[c1]] := [[Box A]];[[Unbox B]] : [[A --> B]]\\
    [[c2]] := [[Box B]];[[Unbox A]] : [[B --> A]]\\
  \end{array}
  \]    
\end{proof}
\noindent
Note that $[[c1]]$ and $[[c2]]$ do not form an isomorphism, nor
retracts, but when combined with other casting morphisms will evaluate
away using the retracts in the model. This result will be used in
Section~\ref{sec:surface_grady} and
Section~\ref{sec:interpreting_surface_grady_in_the_model} when
inserting explicit casts during the cast insertion algorithm for
gradual typing.
% subsection the_categorical_model (end)

%%% Local Variables: ***
%%% mode:latex ***
%%% TeX-master: "main.tex"  ***
%%% End: ***
