In this section we show that Siek and Taha's gradual
$\lambda$-calculus \cite{Siek:2006,Siek:2015} can be modeled in a
gradual $\lambda$-model. Thus showing that other gradual type systems
can benefit from our semantics.

We only consider Siek and Taha's casting calculus, called $\CGSTLC$,
because their surface language is essentially Surface Grady.
\begin{figure}
  \begin{mdframed} \footnotesize
    \textbf{Syntax:}
    \[ \small
    \begin{array}{cccccc}
      \begin{array}{c@{\hspace{5pt}}r@{}@{\hspace{5pt}}r@{}@{\hspace{2pt}}l@{}llllllllllll}
        \text{(Atomic Types)}  & [[T]] & ::= & [[Unit]] \mid [[Nat]]\\
        \text{(Ground Types)}  & [[R]] & ::= & [[T]] \mid [[?]] \to [[?]]\\    
        \text{(values)}        & [[v]] & ::= & [[\x:A.t]]\\
        \text{(terms)}         & [[t]] & ::= & \cdots \mid [[t : {A} => {B}]]\\
      \end{array}
    \end{array}
    \]

    \textbf{Typing Rules:}
    \begin{mathpar}
      \cdots \and
      \GSiekdruleCXXcast{}
    \end{mathpar}  

    \textbf{Reduction Relation:}
    \begin{mathpar}
      \cdots                       \and
      \GSiekdrulerdAXXcastId{}     \and
      \GSiekdrulerdAXXcastU{}      \and
      \GSiekdrulerdAXXsucceed{}    \and
      \GSiekdrulerdAXXfail{}       \and
      \GSiekdrulerdAXXcastArrow{}  \and
      \GSiekdrulerdAXXcastGround{} \and
      \GSiekdrulerdAXXcastExpand{}       
    \end{mathpar}
  \end{mdframed}
  \caption{The core casting calculus: $\CGSTLC$}
  \label{fig:CGSTLC}
\end{figure}
The complete language specification is summarized in
Figure~\ref{fig:CGSTLC}.  The casting calculus $\CGSTLC$ is Core Grady
where $\mathsf{box}\,[[A]]$ and $\mathsf{unbox}\,[[A]]$ have been
replaced with the explicit cast $[[t : {A} => {B}]]$.  In addition,
the typing rules for $\mathsf{box}\,[[A]]$ and $\mathsf{unbox}\,[[A]]$
have been replaced with the \text{cast} typing rule. The syntax of
types are the same as for Core Grady; see
Figure~\ref{fig:syntax-core-grady}. We do not consider cartesian
products in $\CGSTLC$, but they can be added to $\CGSTLC$ in the same
way that they are defined for Core Grady.  One interesting aspect of
$\CGSTLC$ is that it depends on type consistency, used in the cast
rule, but it is defined in the same way as in
Figure~\ref{fig:typing-surface-grady}, however without cartesian
products.

The explicit cast, $[[t : {A} => {B}]]$, should be understand as
casting the term $[[t]]$ whose type is $[[A]]$ to the type $[[B]]$.
Thus, boxing a term, $[[t]]$, of type $[[A]]$ is defined by $[[t : {A}
    => {?}]]$, and unboxing is defined by $[[t : {?} =>
    {A}]]$. Semantically, type consistency corresponds to casting
morphisms, and because their morphisms they compose, but type
consistency is not transitive. However, using the explicit cast we can
compose type consistency.  Suppose $[[A ~ B]]$, $[[B ~ C]]$, and $[[G
    |-C t : A]]$, then using the rule, $\GSiekdruleCXXcastName{}$, of
the casting calculus we may type $[[G |-C t : h({A} => {B}) => {C} :
    C]]$.  This composition is the reason why we interpret type
consistency as casting morphisms in the model.

The reduction rules for $\CGSTLC$ includes all of the rules from Core
Grady except for the retract rules and the rules for cartesian
products in addition to the new casting rules given in
Figure~\ref{fig:CGSTLC}.  It is easy to see that the new casting
rules, succeed and fail, correspond to the Core Grady reduction rules,
retract and raise, from Figure~\ref{fig:reduction-core-grady}.  The
other new casting rules are congruence rules to prevent stuck terms
and push casting towards the succeed and fail rules.

We now prove the two main properties for modeling type systems in a
categorical model just as we did in the previous section.  The
interpretation of types and contexts remain the same as in
Definition~\ref{def:interpretation-of-gradual-types}.

\begin{theorem}[Interpretation of Typing for $\CGSTLC$]
  \label{thm:CGSTLC-interpretation_of_typing}
  Suppose $(\cat{T},$\\$\cat{C}, ?, \T,$ $\split, \squash, \bx, \unbox, \error)$
  is a gradual $\lambda$-model. If $[[G |-C t : A]]$, then
  there is a morphism $[[ [| t |] ]] : [[ [| G |] --> A ]]$ in $\cat{C}$.  
\end{theorem}
\begin{proof}
  This is a proof by induction on $[[G |-C t : A]]$.  We only show the
  case for the explicit cast, because all others are equivalent to the
  interpretation of Core Grady.
  \begin{description}
  \item[]
  \item
    \begin{typeProofCase}
      \GSiekdruleCXXcast{}
    \end{typeProofCase}

    \ \\
    \noindent
    By the induction hypothesis there is a morphism $[[ [| t |] ]] : [[ [| G |] --> A]]$,
    and by type consistency in the model \\
    (Lemma~\ref{lemma:type_consistency_in_the_model})
    there is a casting morphism $c_1 : [[A --> B]]$.  So take
    $ [[ [| t : {A} => {B} |] ]] = [[ [| t |] ]];c_1 : [[ [| G |] --> B ]]$.
  \end{description}
\end{proof}

\begin{theorem}[Interpretation of Evaluation for $\CGSTLC$]
  \label{thm:CGSTLC-interpretation_of_evaluation}
  Suppose $(\cat{T}, \cat{C}, ?,\T, \split, \squash, \bx, \unbox, \error)$
  is a gradual $\lambda$-\\model.  If $[[G |-C t1 : A]]$, $[[G |-C t2 : A]]$, and $[[t1 ~> t2]]$, then
  $[[ [| t1 |] ]] = [[ [| t2 |] ]] : [[ [| G |] --> A]]$.
\end{theorem}
\begin{proof}
  This proof holds by induction on the form of $[[t1 ~> t2]]$, and
  uses Theorem~\ref{thm:CGSTLC-interpretation_of_typing},
  Lemma~\ref{lemma:type_consistency_in_the_model},
  Corollary~\ref{corollary:type_consist_coro}, and inversion for
  typing whose definition we omit in the interest of space.

  We show only two of the most interesting cases here, but please see
  Appendix~\ref{subsec:CGSTLC-proof_of_interpretation_of_evaluation} for the
  complete proof.
  \begin{description}
  \item[]
  \item 
  \begin{typeProofCase}
    \GSiekdrulerdAXXsucceed{}
  \end{typeProofCase}

  \ \\
  \noindent
  By inversion for typing the typing derivation for $[[v : h({R} =>
      {?}) => {R}]]$ is as follows:
  \begin{center}
    \begin{math}
      $$\mprset{flushleft}
      \inferrule* [right=] {
        $$\mprset{flushleft}
        \inferrule* [right=] {
          [[G |-C v : R && R ~ ?]]
        }{[[G |-C v : {R} => {?} : ?]]} \quad [[? ~ R]]
      }{[[G |-C v : h({R} => {?}) => {R} : R]]}
    \end{math}
  \end{center}
  By the induction hypothesis we have the morphism $[[ [| v |] ]] : [[ [| G |] --> R]]$.  
  As we can see we will first use $[[ Box R ]]$ and then $[[ Unbox R ]]$ based on
  Corollary~\ref{corollary:type_consist_coro}.  Thus,
  $[[ [| v : h({R} => {?}) => {R} |] ]] = [[ [| v |] ]];[[Box R]];[[Unbox R]] = [[ [| v |] ]]$,
  because $[[ Box R ]]$ and $[[ Unbox R ]]$ form a retract (Lemma~\ref{lemma:casting_morphisms}).

\item[] 
\item
  \begin{typeProofCase}    
    \inferrule* [flushleft,right=$\to_\Rightarrow$] {
      [[A]] = [[(A1 -> B1)]] \\ [[B]] = [[(A2 -> B2)]]
    }{[[H((v1 : {A} => {B}) v2) ~> H(H(v1 (v2 : {A2} => {A1})) : {B1} => {B2})]]}
  \end{typeProofCase}

  \ \\
  \noindent
  First, by inversion for typing we know $[[G |-C v1 : A1 -> B1]]$, $[[G |-C v2 : A2]]$, and $[[(A1 -> B1) ~ (A2 -> B2)]]$.
  Then by the induction hypothesis and Corollary~\ref{corollary:type_consist_coro}
  we have the following morphisms:
  \[ \small
  \begin{array}{lll}
    [[ [| v1 |] ]] : [[ [| G |] --> (A1 -> B1)]]\\
    [[ [| v2 |] ]] : [[ [| G |] --> A2]]\\
    c_1 : [[ A2 --> A1 ]]\\
    c_2 : [[ B1 --> B2 ]]\\
  \end{array}
  \]
  \noindent
  We must show the following:
  \[ \small
  \begin{array}{llllll}
    [[ [| H((v1 : {(A1 -> B1)} => {(A2 -> B2)}) v2) |] ]]\\
    = \langle [[ [| v1 |] ]];(c_1 \rightarrow c_2),[[ [| v_2 |] ]] \rangle;\app_{[[A2]],[[B2]]} \\
    = \langle [[ [| v1 |] ]],[[ [| v_2 |] ]];c_1 \rangle;\app_{[[A1]],[[B1]]};c_2
  \end{array}
  \]
  \noindent
  The previous equation holds as follows where we give properties in
  between the equations for the reason why they hold:
  \[ \small
  \begin{array}{lllllllll}
    \langle [[ [| v1 |] ]],[[ [| v_2 |] ]];c_1 \rangle;\app_{[[A1]],[[B1]]};c_2 \\
    \text{(Cartesian Products)}\\
    = \langle [[ [| v1 |] ]],[[ [| v_2 |] ]] \rangle;(\id_{[[A1 -> B1]]} \times c_1);\app_{[[A1]],[[B1]]};c_2 \\
    \text{(Naturality)}\\
    = \langle [[ [| v1 |] ]],[[ [| v_2 |] ]] \rangle;(\id_{[[A1 -> B1]]} \times c_1);\\
      \,\,\,\,\,\,\,\,\,\,((\id_{[[A1]]} \to c_2) \times \id_{[[A1]]});\app_{[[A1]],[[B2]]}\\
    \text{(Closure)}\\
    = \langle [[ [| v1 |] ]],[[ [| v_2 |] ]] \rangle;(\curry{(\id_{[[A1 -> B1]]} \times c_1);\\      
      \,\,\,\,\,\,\,\,\,\,((\id_{[[A1]]} \to c_2) \times \id_{[[A1]]});\app_{[[A1]],[[B2]]}} \times \id_{[[A2]]});\app_{[[A2]],[[B2]]}\\
    \text{(Closure)}\\
    = \langle [[ [| v1 |] ]],[[ [| v_2 |] ]] \rangle;((c_1 \to c_2) \times \id_{[[A2]]});\app_{[[A2]],[[B2]]}\\
    \text{(Cartesian Products)}\\
    = \langle [[ [| v1 |] ]];(c_1 \to c_2),[[ [| v_2 |] ]];\id_{[[A2]]} \rangle;\app_{[[A2]],[[B2]]} \\
    = \langle [[ [| v1 |] ]];(c_1 \to c_2),[[ [| v_2 |] ]] \rangle;\app_{[[A2]],[[B2]]}\\
  \end{array}
  \]
  \end{description}
\end{proof}
