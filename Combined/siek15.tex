In this section we show that Siek and Taha's gradual
$\lambda$-calculus \cite{Siek:2006,Siek:2015} can be modeled in a
gradual $\lambda$-model. Thus showing that other gradual type systems
can benefit from our semantics.

We only consider Siek and Taha's casting calculus, called $\CGSTLC$,
because their surface language is essentially Surface Grady.
\begin{figure}
  \begin{mdframed} \footnotesize
    \textbf{Syntax:}
    \[ \small
    \begin{array}{cccccc}
      \begin{array}{c@{\hspace{5pt}}r@{}@{\hspace{5pt}}r@{}@{\hspace{2pt}}l@{}llllllllllll}
        \text{(Atomic Types)}  & [[T]] & ::= & [[Unit]] \mid [[Nat]]\\
        \text{(Ground Types)}  & [[R]] & ::= & [[T]] \mid [[?]] \to [[?]]\\    
        \text{(values)}        & [[v]] & ::= & [[\x:A.t]]\\
        \text{(terms)}         & [[t]] & ::= & \ldots \mid [[t : {A} => {B}]]\\
      \end{array}
    \end{array}
    \]

    \textbf{Typing Rules:}
    \begin{mathpar}
      \cdots \and
      \GSiekdruleCXXcast{}
    \end{mathpar}  

    \textbf{Reduction Relation:}
    \begin{mathpar}
      \cdots                       \and
      \GSiekdrulerdAXXcastId{}     \and
      \GSiekdrulerdAXXcastU{}      \and
      \GSiekdrulerdAXXsucceed{}    \and
      \GSiekdrulerdAXXfail{}       \and
      \GSiekdrulerdAXXcastArrow{}  \and
      \GSiekdrulerdAXXcastGround{} \and
      \GSiekdrulerdAXXcastExpand{}       
    \end{mathpar}
  \end{mdframed}
  \caption{The core casting calculus: $\CGSTLC$}
  \label{fig:CGSTLC}
\end{figure}
The complete language specification is summarized in
Figure~\ref{fig:CGSTLC}.  The casting calculus $\CGSTLC$ is Core Grady
where $\mathsf{box}\,[[A]]$ and $\mathsf{unbox}\,[[A]]$ have been
replaced with the explicit cast $[[t : {A} => {B}]]$.  In addition,
the typing rules for $\mathsf{box}\,[[A]]$ and $\mathsf{unbox}\,[[A]]$
have been replaced with the \text{cast} typing rule. We do not
consider cartesian products in $\CGSTLC$, but they can be added to
$\CGSTLC$ in the same way that they are defined for Core Grady.  
