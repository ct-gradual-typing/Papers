We have given a new categorical model that combines static and dynamic
typing using the theory of retracts.  Our model is an extension of
Scott's~\cite{Scott:1980} model of the untyped $\lambda$-calculus.
Then we showed that Siek and Taha's gradually typed $\lambda$-calculus
\cite{Siek:2015} can be soundly interpreted into our model.  Finally,
we define the corresponding typed $\lambda$-calculus called Grady that
corresponds to our model through the Curry-Howard-Lambek
correspondence.

\textbf{Future work.}  Gradual typing reduces the number of explicit
casts into the untyped fragment substantially for the programmer.
However, one open question is how can gradual typing be extended with
polymorphism?  In a follow up paper we show how to extend gradual
typing with bounded quantification.  The bounds can be used to limit
which types are castable to the unknown type.  For example, we will
not allow the programmer to cast a polymorphic type to the unknown
type, because we do not have a good model for this, and we are not yet
sure how this will affect gradual typing.  The core language of the
gradual type system is Grady.  Finally, we show that this system
satisfies the gradual guarantee as laid out by Siek et
al.~\cite{Siek:2015}.  Adding bounded quantification is a non-trivial
extension, and thus, proving the gradual guarantee was quite
laborious. The proofs found in the literature for the gradual
guarantee usually make heavy use of inversion for typing, but for a
system with polymorphism and subtyping this can be difficult.  Instead
of using inversion for typing we show that one can often use inversion
for other judgments and make the proof effort more tractable.

\begin{ack}
  The authors thank Jeremy Siek for his feedback on previous drafts of
  this paper.  The first author thanks Ronald Garcia for his wonderful
  invited talk at Trends in Functional Programming 2016 which
  introduced the first author to gradual typing and its open problems.
  This paper was typeset with the help of the amazing Ott tool
  \cite{Sewell:2010}.
\end{ack}
