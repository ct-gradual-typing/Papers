The aim of the gradual typing is to allow the programmer to program
either statically and catch as many errors at compile time as
possible, or to program in dynamic style and leave some error checking
to run time, but the programmer should not be burdened by having to
explicitly insert casts.  Thus, Surface Grady is gradually typed, and
in this section we give the details of the language.

Surface Grady's syntax is defined in
Fig.~\ref{fig:syntax-surface-grady}.  
\begin{figure}
  \small
  \begin{mdframed}
    \begin{itemize}
    \item[] \textbf{Syntax:}
      \[
      \begin{array}{cl}
        \begin{array}{l}
          \text{(types)}\\\\
        \end{array}     &
        \begin{array}{lcl}
          [[A]],[[B]],[[C]] & ::= & [[X]] \mid [[Top]] \mid [[SL]] \mid [[Unit]] \mid [[Nat]] \mid [[?]] \mid [[List A]] \mid [[A x B]] \mid [[A -> B]] \\ & \mid &[[Forall (X <: A).B]]\\
        \end{array}\\\\
        
        \text{(skeletons)} &
        \begin{array}{lcl}
          [[S]],[[K]],[[U]] & ::= & [[?]] \mid [[List S]] \mid [[S1 x S2]] \mid [[S1 -> S2]]\\
        \end{array}\\\\
        
        \begin{array}{l}
          \text{(terms)}\\\\\\\\
        \end{array}     &
        \begin{array}{lcl}
          [[t]] & ::= & [[x]] \mid [[triv]] \mid [[0]] \mid [[succ t]] \mid [[case t of 0 -> t1, (succ x) -> t2]] \\ & \mid & [[(t1 , t2)]] \mid [[fst t]] \mid [[snd t]] \mid [[ [] ]] \mid [[t1 :: t2]] \\ & \mid & [[case t of [] -> t1, (x :: y) -> t2]] \mid [[\x : A.t]] \\ & \mid & [[t1 t2]] \mid [[Lam X <: A.t]] \mid [[ [ A ] t ]]\\
        \end{array}\\\\
        
        \text{(contexts)}  &
        \begin{array}{lcl}
          [[G]] & ::= & [[.]] \mid [[x : A]] \mid [[G1,G2]]\\
        \end{array}\\
      \end{array}
      \]
    \item[] \textbf{Metafunctions:}
      \[
      \begin{array}{lll}
        \begin{array}{lll}
          [[nat(?) = Nat]]\\
          [[nat(Nat) = Nat]]\\
        \end{array}
        & \quad & 
        \begin{array}{lll}
          [[list(?) = List ?]]\\
          [[list(List A) = List A]]\\
        \end{array}\\\\                       
        \begin{array}{lll}
          [[prod(?) = ? x ?]]\\
          [[prod(A x B) = A x B]]\\
        \end{array}
        & \quad &
        \begin{array}{lll}
          [[fun(?) = ? -> ?]]\\
          [[fun(A -> B) = A -> B]]\\
        \end{array}
      \end{array}
      \]
    \end{itemize}
  \end{mdframed}
  \caption{Syntax and Metafunctions for Surface Grady}
  \label{fig:syntax-surface-grady}
\end{figure}
The types $[[Top]]$ and $[[SL]]$ will be used strictly as upper bounds
with respect to quantification, and so, they will not have any
introduction typing rules. The type of polymorphic functions is
$[[Forall (X <: A).B]]$ where $[[A]]$ is the called the bound on the
type variable $[[X]]$.  This bound will restrict which types are
allowed to replace $[[X]]$ during type application -- also known as
instantiation.  The restriction is that the type one wishes to replace
$[[X]]$ with must be a subtype of the bounds $[[A]]$.  As we mentioned
in the introduction the type $[[?]]$ is the unknown type and should be
thought of as the universe of untyped terms.
%% 
%% Skeletons are the bare bones structure of types.  The skeleton,
%% $[[S]]$, of a type $[[A]]$ is $[[A]]$ where every atomic type as been
%% replaced with $[[?]]$.  
Syntax for terms and typing contexts are fairly standard.  

The rules defining type consistency and consistent subtyping can be
found in Fig.~\ref{fig:subtyping-surface-grady}.  
\begin{figure}
  \small
  \begin{mdframed}
    \begin{itemize}
    \item[] \textbf{Consistent Subtyping:}
      \begin{mathpar}
        \SGradydruleSXXRefl{} \and
        \SGradydruleSXXTop{} \and      
        \SGradydruleSXXVar{} \and
        \SGradydruleSXXBox{} \and    
        \SGradydruleSXXUnbox{} \and
        \SGradydruleSXXSplit{} \and    
        \SGradydruleSXXSquash{} \and                
        \SGradydruleSXXNatSL{} \and
        \SGradydruleSXXUnitSL{} \and    
        \SGradydruleSXXListSL{} \and
        \SGradydruleSXXProdSL{} \and
        \SGradydruleSXXArrowSL{} \and
        \SGradydruleSXXList{} \and
        \SGradydruleSXXProd{} \and
        \SGradydruleSXXArrow{} \and
        \SGradydruleSXXForall{}
      \end{mathpar}
      
    \item[] \textbf{Type Consistency:}
      \begin{mathpar}
      \SGradydruleCXXRefl{} \and
      \SGradydruleCXXBox{} \and
      \SGradydruleCXXUnbox{} \and
      \SGradydruleCXXSplit{} \and
      \SGradydruleCXXSquash{} \and      
      \SGradydruleCXXList{} \and
      \SGradydruleCXXArrow{} \and
      \SGradydruleCXXProd{} \and
      \SGradydruleCXXForall{}      
    \end{mathpar}
    \end{itemize}
  \end{mdframed}
  \caption{Subtyping and Type Consistency for Surface Grady}
  \label{fig:subtyping-surface-grady}
\end{figure}
In a gradual type system we use a reflexive and symmetric, but
non-transitive, relation on types to determine when two types may be
cast between each other \cite{Siek:2006}.  This relation is called
type consistency, and is denoted by $[[G |- A ~ B]]$.
Non-transitivity prevents the system from being able to cast between
arbitrary types.  The type $[[SL]]$ stands for ``simple types'' and is
a super type whose subtypes are all non-polymorphic types that do not
contain the unknown type.  Thus, by the definition of type consistency
the only types that can be boxed or unboxed are non-polymorphic types.
Note that the type consistency rules $\SGradydruleCXXBoxName{}$ and
$\SGradydruleCXXUnboxName{}$ embody boxing and unboxing, and splitting
and squashing.  The remainder of the rules simply are congruence
rules.  We will use this intuition when translating Surface Grady to
Core Grady.  Note that the type consistency, subtyping, and typing
judgments for both Surface Grady and Core Grady all depend on kinding,
denoted $[[G |- A : *]]$, and well-formed contexts, denoted $[[G
    Ok]]$, but these are standard, and in the interest of saving space
we do not define them here.  They simply insure that all type
variables in types in and out of contexts are accounted for.

Consistent subtyping, denoted $[[G |- A <~ B]]$, was proposed by Siek
and Taha~\cite{Siek:2007} in their work extending gradual type systems
to object oriented programming.  It embodies both standard subtyping,
denoted $[[G |- A <: B]]$, and type consistency.  Thus, consistent
subtyping is also non-transitive.  One major difference between this
definition of consistent subtyping and others found in the literature,
for example in \cite{Siek:2007} and \cite{Garcia:2016}, is the rule
for type variables.  Naturally, we must have a rule for type
variables, because we are dealing with polymorphism, but the proof of
the gradual guarantee -- see Section~\ref{sec:results} -- required
that this rule be relaxed and allow the bounds provided by the
programmer to be consistent with the subtype in question.  

Typing for Surface Grady is given in
Fig.~\ref{fig:typing-surface-grady}.
\begin{figure}
  \small
  \begin{mdframed}
    \begin{mathpar}
      \SGradydruleTXXvarP{} \and
      \SGradydruleTXXunitP{} \and
      \SGradydruleTXXzeroP{} \and
      \SGradydruleTXXsucc{} \and
      \SGradydruleTXXncase{} \and
      \SGradydruleTXXempty{} \and
      \SGradydruleTXXcons{} \and
      \SGradydruleTXXlcase{} \and      
      \SGradydruleTXXpair{} \and
      \SGradydruleTXXfst{} \and
      \SGradydruleTXXsnd{} \and      
      \SGradydruleTXXlam{} \and
      \SGradydruleTXXapp{} \and      
      \SGradydruleTXXLam{} \and
      \SGradydruleTXXtypeApp{} \and
      \SGradydruleTXXSub{}
    \end{mathpar}
  \end{mdframed}
  \caption{Typing rules for Surface Grady}
  \label{fig:typing-surface-grady}
\end{figure}
It follows the formulation of the Gradual Simply Typed
$\lambda$-calculus given by Siek et al.~\cite{Siek:2015} pretty
closely.  The most interesting rules are the elimination rules,
because this is where type consistency -- and hence casting -- comes
into play.  Consider the elimination for lists:
\[\small
\SGradydruleTXXlcase{}
\]
The type $[[C]]$ can be either $[[?]]$ or $[[List A]]$.  If it is the
former, then $[[C]]$ will be split into $[[List ?]]$.  In addition, we
allow the type of the branches to be cast to other types as well, just
as long as, they are consistent with $[[B]]$.  For example, if
$[[t1]]$ was a boolean and $[[t2]]$ was a natural number, then type
checking will fail, because the types are not consistent, and hence,
we cannot cast between them.  The other rules are setup similarly.

One non-obvious feature of gradual typing is the ability to use
explicit casts in the surface language without having the explicit
casts as primitive features of the language.  This realization
actually increases the expressivity of the language.  Consider the
rule for function application:
\[\small
\SGradydruleTXXapp{}
\]
Notice that this rule does not apply any implicit casts to the result
of the application.  Thus, if one needs to cast the result, then on
first look, it would seem they are out of luck.  However, if we push
the cast into the argument position then this rule will insert the
appropriate cast.  This leads us to the following definitions:
\[
\begin{array}{lll}
  \begin{array}{lll}
    [[box A t]]    & = & [[(\x:?.x) t]]\\
    [[unbox A t]]  & = & [[(\x:A.x) t]]\\
  \end{array}
  & \quad & 
  \begin{array}{lll}
    [[squash S t]] & = & [[(\x:?.x) t]]\\
    [[split S t]]  & = & [[(\x:S.x) t]]\\
  \end{array}
\end{array}
\]
The reader should keep in mind the differences between $[[box]]$ and
$[[squash]]$, and $[[unbox]]$ and $[[split]]$.  When these are
translated into Core Grady using the cast insertion algorithm --
see Sect.~\ref{subsec:cast_insertion} -- the inserted cast will match the
definition.  We now have the following result.
\begin{lemma}[Explicit Casts Typing]
  \label{lemma:explicit_casts_typing}
  The following rules are derivable in Surface Grady:
  \begin{mathpar}\small
    \SGradydruleTXXbox{} \and
    \SGradydruleTXXunbox{} \and
    \SGradydruleTXXsplit{} \and
    \SGradydruleTXXsquash{}
  \end{mathpar}
\end{lemma}

As an example consider the following Surface Grady program -- here we
use the concrete syntax from Grady's implementation, but it is very
similar to Haskell and not far from the mathematical syntax:
\begin{lstlisting}[language=Haskell]
  omega : ? -> ?
  omega = \(x : ?) -> (x x);

  ycomb : (? -> ?) -> ?
  ycomb = \(f : ? -> ?) -> omega (\(x:?) -> f (x x));

  fix : forall (X <: Simple).((X -> X) -> X)
  fix = \(X <: Simple) -> \(f:X -> X) -> unbox<X> (ycomb f);
\end{lstlisting}
The previous example defines the Y combinator and a polymorphic fix
point operator.  This example is gradually typed, for example, in the
definition of \ginline{omega} we are applying \ginline{x} to itself
without an explicit cast.  However, there is one explicit cast in the
definition of \ginline{fix} which applies \ginline{unbox} to the
result of the Y combinator.  If we did not have this explicit cast,
then \ginline{fix} would not type check for the reasons outlined
above.  Thus, pointing out the explicit casts in the surface language
increases the number of valid programs.  We will use \ginline{fix} to
develop quite a few interesting examples including a full library of
operations on lists.

Being able to define the typed fix point operator makes Grady very
expressive.  Combing \ginline{fix} with the eliminators for natural
numbers and lists results in typed terminating recursion.  We now give
several examples in Surface Grady that illustrate this\footnote{Please
  see the following example file for the complete list library:
  \url{https://github.com/ct-gradual-typing/Grady/blob/master/Examples/Gradual/List.gry}}:
\begin{lstlisting}[language=Haskell]
  append : forall (A <: Simple).([A] -> [A] -> [A])
  append = \(A <: Simple) ->
           ([ [A] -> [A] -> [A] ]fix)
             (\(r : [A] -> [A] -> [A]) -> \(l1 : [A]) -> \(l2 : [A]) ->
               case l1 of
                 [] -> l2,
                 (a :: as) -> a :: (r as l2));

  length : forall (A <: Simple).([A] -> Nat)
  length = \(A <: Simple) -> ([ [A] -> Nat]fix)
             (\(r : [A] -> Nat) -> \(l : [A]) ->
               case l of
                 [] -> 0,
                 (a :: as) -> succ (r as));

  foldr : forall (A <: Simple).
         (forall (B <: Simple).((A -> B -> B) -> B -> ([A] -> B)))
  foldr = \(A <: Simple) -> \(B <: Simple) -> \(f : A -> B -> B) -> \(b : B) ->
          ([ [A] -> B]fix) (\(r : [A] -> B) -> \(l : [A]) ->
                   case l of
                     [] -> b,
                     (a :: as) -> (f a (r as)));

  zipWith : forall (A <: Simple).(forall (B <: Simple).(forall (C <: Simple).
               ((A -> B -> C) -> ([A] -> [B] -> [C]))))
  zipWith = \(A <: Simple) ->
             \(B <: Simple) -> \(C <: Simple) -> \(f : A -> B -> C) ->
            ([ [A] -> [B] -> [C] ]fix)
              (\(r : [A] -> [B] -> [C]) -> \(l1 : [A]) -> \(l2 : [B]) ->
                 case l1 of
                   [] -> [C][],
                   (a :: as) -> case l2 of
                                  [] -> [C][],
                                  (b :: bs) -> (f a b) :: (r as bs));
\end{lstlisting}
All of the previous examples are staticly typed, but eventually the
static types are boxed and moved to the dynamic fragment when running
\ginline{fix}.
