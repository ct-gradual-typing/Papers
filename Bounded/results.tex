We now turn to Grady's metatheory.  Specifically, we show that the
gradual guarantee -- see Theorem~\ref{thm:gradual_guarantee} -- holds
for Grady.  This is the defining property of every gradual type
system.  In fact, Siek et al.~\cite{Siek:2015} argue that the gradual
guarantee is what separates type systems that simply combine dynamic
and static typing from systems that not only combine dynamic and
static typing, but also allow the programmer to program in dynamic
style without the need to insert explicit casts.  Intuitively, the
gradual guarantee states that a gradually typed program should
preserve its type and behavior when explicit casts are either inserted
or removed.

First, we have some basic facts about Grady.  Notice that if we remove
the unknown type, $[[box]]$, and $[[unbox]]$ from Grady, then we are
left with Bounded System F.  Suppose $[[G |-F t : A]]$ holds if
$[[t]]$ and $[[A]]$ are a Bounded System F -- for its definition
please see Pierce\cite{Pierce:2002:TPL:509043} -- term and type
respectively.  We call a type $[[A]]$, static, if it does not contain
the unknown type.
\begin{lemma}[Inclusion of Bounded System F]
  \label{lemma:F-inclusion}
  Suppose $t$ is fully annotated and does not contain any applications
  of $[[box]]$ or $[[unbox]]$, and $[[A]]$ is static.  Then
  \begin{itemize}
  \item[i.] $[[G |-F t : A]]$ if and only if $\,[[G |- t : A]]$, and 
  \item[ii.] $[[t F~>* t']]$ if and only if $[[t ~>* t']]$.
  \end{itemize}
\end{lemma}
\begin{proof}
  We give proof sketches for both parts.  The interesting cases are
  the right-to-left directions of each part.  If we simply remove all
  rules mentioning the unknown type $[[?]]$ and the type consistency
  relation, and then remove $[[box]]$, $[[unbox]]$, and $[[?]]$ from
  the syntax of Surface Grady, then what we are left with is bounded
  system F.  Since $[[t]]$ is fully annotated and $[[A]]$ is static,
  then $[[G |- t : A]]$ will hold within this fragment.

  Moving on to part two, first, we know that $[[t]]$ does not contain
  any occurrence of $[[box]]$ or $[[unbox]]$ and is fully annotated.
  This implies that $[[t]]$ lives within the bounded system F fragment
  of Surface Grady. Thus, before evaluation of $[[t]]$ Surface Grady
  will apply the cast insertion algorithm which will at most insert
  applications of the identity function into $[[t]]$ producing a term
  $\widehat{[[t]]}$, but then after potentially more than one step of
  evaluation within Core Grady, those applications of the identity
  function will be $\beta$-reduced away resulting in $\widehat{[[t]]}
  \rightsquigarrow^* [[t]] \rightsquigarrow^* [[t']]$.  In addition,
  since $[[t]]$ in Surface Grady is the exact same program as $[[t]]$
  in bounded system F, then we know $[[t F~>* t']]$ will hold.
\end{proof}
The call-by-name untyped $\lambda$-calculus is a fragment of Grady.
We now show that one can strictly program in this fragment.  In fact,
the untyped fragment of Grady is the call-by-name unitype
$\lambda$-calculus.  The translation of the untyped $\lambda$-calculus
into Grady is as follows:
\[
\begin{array}{lll}
  [[ | x | ]]     & = & [[x]]\\
  [[ | \x.t | ]]  & = & [[\x:?.|t|]]\\
  [[ | t1 t2 | ]] & = & [[ (split (? -> ?) |t1|) |t2| ]]\\
\end{array}
\]
In the last case of the previous definition the annotation we use
$[[split]]$ to cast the time of $[[ |t1 | ]]$ to $[[? -> ?]]$, and
thus, after the final application we will have a term of type $[[?]]$.
We now have the following result.
\begin{lemma}[Inclusion of the Untyped $\lambda$-Calculus]
  \label{lemma:inclusion_of_dtlc}
  Suppose $[[t]]$ is a closed term of the untyped
  $\lambda$-calculus. Then
  \begin{itemize}
  \item[i.] $[[. |- |t| : ?]]$, and
  \item[ii.] $[[t D~>* t']]$ if and only if $[[|t| ~>* |t'|]]$.
  \end{itemize}
\end{lemma}
\begin{proof}
  This proof is the same result proven by \cite{Siek:2006} and
  \cite{Siek:2015}.
\end{proof}
Another basic fact is that type preservation holds for Core Grady.
\begin{lemma}[Type Preservation]
  \label{lemma:type_preservation}
  If $<<G |- t1 : A>>$ and $<<t1 ~> t2>>$, then $<<G |- t2 : A>>$.
\end{lemma}
\begin{proof}
  This proof holds by induction on $<<G |- t1 : A>>$ with further case
  analysis on the structure the derivation $<<t1 ~> t2>>$.
\end{proof}
This result is needed in the proof of the gradual guarantee,
specifically, in the proof of simulation of more precise programs
(Lemma~\ref{lemma:simulation_of_more_precise_programs}).

We claimed that consistent subtyping is the combination of both
standard subtyping and type consistency. The following results makes
this precise.
\begin{restatable}[Left-to-Right Consistent Subtyping]{lemma}{conSubOne}
  \label{lemma:consistent-subtyping-1}
  Suppose $[[G |- A <~ B]]$.
  \begin{enumR}
    \item $[[G |- A ~ A']]$ and $<<G |- A' <: B>>$ for some $[[A']]$.
    \item $[[G |- B' ~ B]]$ and $<<G |- A <: B'>>$ for some $[[B']]$.
  \end{enumR}   
\end{restatable}
\begin{proof}
  This is a proof by induction on $[[G |- A <~ B]]$.  See
  Appendix~\ref{subsec:proof_of_left-to-right_consistent_subtyping_lemma:consistent-subtyping-1}
  for the complete proof.
\end{proof}
\begin{corollary}[Consistent Subtyping]
  \label{corollary:consistent_subtyping}
  \begin{enumR}
  \item $[[G |- A <~ B]]$ if and only if $[[G |- A ~ A']]$ and $<<G |- A' <: B>>$ for some $[[A']]$.
  \item $[[G |- A <~ B]]$ if and only if $[[G |- B' ~ B]]$ and $<<G |- A <: B'>>$ for some $[[B']]$.
  \end{enumR}
\end{corollary}
\begin{proof}
  The left-to-right direction of both cases easily follows from
  Lemma~\ref{lemma:consistent-subtyping-1}, and the right-to-left
  direction of both cases follows from induction on the subtyping
  derivation and Lemma~\ref{lemma:simply_typed_consistent_types_are_subtypes}.
\end{proof}
The previous two results are similar to those proved by Siek and
Taha~\cite{Siek:2007} and Garcia~\cite{Garcia:2016}.

We now embark on the proof of the gradual guarantee.  Our proof
follows the scheme adopted by Siek et al.~\cite{Siek:2015} in their
proof of the gradual guarantee for the gradual simply typed
$\lambda$-calculus.  That is, we prove the exact same results as they
do.

Cast insertion preserves the type up to type consistency.  This is
fairly straightforward to show.  The only interesting case is the one
for type application.
\begin{restatable}[Type Preservation for Cast Insertion]{lemma}{typePresCastIns}
  \label{lemma:type_preservation_for_cast_insertion}
  If $[[G |- t1 : A]]$ and $[[G |- t1 => t2 : B]]$, then $<<G |- t2 : B>>$ and $[[G |- A ~ B]]$.
\end{restatable}
\begin{proof}
  The cast insertion algorithm is type directed and with respect to
  every term $[[t1]]$ it will produce a term $[[t2]]$ of the core
  language with the type $[[A]]$ -- this is straightforward to show by
  induction on the form of $[[G |- t1 : A]]$ making use of typing for
  casting morphisms Lemma~\ref{lemma:typing_casting_morphisms} --
  except in the case of type application.  Please see
  Appendix~\ref{subsec:proof_of_type_preservation_for_cast_insertion}
  for the complete proof.
\end{proof}

The proof of the gradual guarantee will be phrased in terms of
precision for types and terms.  A type $[[B]]$ is less precise than a
type $[[A]]$, denoted $[[A <= B]]$, if $[[B]]$ replaces some
subexpression(s) of $[[A]]$ with the unknown type.  Type precision is
defined by the rules in Fig.~\ref{fig:type-pre}.
\begin{figure}
  \begin{mdframed}
    \begin{mathpar}
      \SGradydrulePXXU{} \and
      \SGradydrulePXXrefl{} \and
      \SGradydrulePXXarrow{} \and
      \SGradydrulePXXprod{} \and
      \SGradydrulePXXlist{} \and
      \SGradydrulePXXforall{}      
    \end{mathpar}
  \end{mdframed}
  \caption{Type Precision}
  \label{fig:type-pre}
\end{figure}
Term precision, which is defined for both the surface language and the
core language, is similar to type precision, but, and this is where
our definition differs from others, we must factor in the explicit
casts.  Term precision is defined in Fig.~\ref{fig:term-precision}.
\begin{figure}
  \begin{mdframed}
    \begin{itemize}
    \item[] \textbf{Term Precision for Surface Grady:}
      \begin{mathpar}
        \SGradydruleTPXXrefl{} \and
        \SGradydruleTPXXsucc{} \and
        \SGradydruleTPXXNate{} \and
        \SGradydruleTPXXpair{} \and
        \SGradydruleTPXXfst{} \and
        \SGradydruleTPXXsnd{} \and
        \SGradydruleTPXXcons{} \and
        \SGradydruleTPXXListe{} \and
        \SGradydruleTPXXFun{} \and
        \SGradydruleTPXXapp{} \and
        \SGradydruleTPXXtfun{} \and
        \SGradydruleTPXXtapp{}
      \end{mathpar}

    \item[] \textbf{Term Precision for Core Grady:}
      \begin{mathpar}
        \CGradydruleTPXXunboxing{} \and
        \CGradydruleTPXXboxing{} \and
        \CGradydruleTPXXspliting{} \and
        \CGradydruleTPXXsquashing{} \and
        \CGradydruleTPXXerror{}        
      \end{mathpar}
    \end{itemize}
  \end{mdframed}
  \caption{Term Precision}
  \label{fig:term-precision}
\end{figure}
Term precision for Core Grady is an extension of term precision for
Surface Grady.  Thus, we only show the new rules for casting and
error.  As we travel up a term precision chain the terms type tend
toward the unknown type, as opposed to, when we travel down the chain
the terms type tend toward some static type.  The terms in this chain
only differ by the insertion or removal of casts.  This characterizes
our desired property, which is, that a gradual program can transition
towards dynamic or static typing without compromising it behavior.

The gradual guarantee is defined as follows.
\begin{theorem}[Gradual Guarantee]
  \label{thm:gradual_guarantee} 
  \begin{enumR}
  \item If $[[. |- t : A]]$ and $[[t <= t']]$, then $[[. |- t' : B]]$ and $[[A <= B]]$.
  \item Suppose $<<. |- t : A>>$ and $<<. |- t <= t'>>$. Then
    \begin{enumA}
    \item if $<<t ~>* v>>$, then $<<t' ~>* v'>>$ and $<<. |- v <= v'>>$,
    \item if $<<t ^>>$, then $<<t' ^>>$,
    \item if $<<t' ~>* v'>>$, then $<<t ~>* v>>$ where $<<. |- v <= v'>>$, or $<<t ~>* error A>>$, and
    \item if $<<t' ^>>$, then $<<t ^>>$ or $<<t ~>* error A>>$.
    \end{enumA}
  \end{enumR}
\end{theorem}
\begin{proof}
  This result follows from the same proof as \cite{Siek:2015}, and so,
  we only give a brief summary.  Part i. holds by
  Lemma~\ref{lemma:gradual_guarantee_part_one}, and Part ii. follows
  from simulation of more precise programs
  (Lemma~\ref{lemma:simulation_of_more_precise_programs}).
\end{proof}
Part one states that one may insert casts into a closed gradual
program, $[[t]]$, yielding a less precise program, $[[t']]$, and the
program will remain typable, but at a less precise type.  This part
follows from the following generalization:
\begin{restatable}[Gradual Guarantee Part One]{lemma}{gradGuarOne}
  \label{lemma:gradual_guarantee_part_one}
  If $[[G |- t : A]]$, $[[t <= t']]$, and $[[G <= G']]$ then $[[G' |- t' : B]]$ and $[[A <= B]]$.
\end{restatable}
\begin{proof}
  This is a proof by induction on $[[G |- t : A]]$; see
  Appendix~\ref{subsec:proof_of_gradual_guarantee_part_one_lemma:gradual_guarantee_part_one}
  for the complete proof.
\end{proof}
The previous result depends on context precision, denoted $[[G <=
    G']]$, but we omit its definition, because it is the
straightforward extension of type precision.

The remaining parts of the gradual guarantee follow from the next
result.
\begin{restatable}[Simulation of More Precise Programs]{lemma}{simMorePrecPro}
  \label{lemma:simulation_of_more_precise_programs}
  Suppose $<<G |- t1 : A>>$, $<<G |- t1 <= t1'>>$, $<<G |- t'1 : A'>>$, and $<<t1 ~> t2>>$.
  Then $<<t1' ~>* t2'>>$ and $<<G |- t2 <= t2'>>$ for some $<<t2'>>$.
\end{restatable}
\begin{proof}
  This proof holds by induction on $<<G |- t1 : A1>>$.  See
  Appendix~\ref{subsec:proof_of_simulation_of_more_precise_programs_lemma:simulation_of_more_precise_programs}
  for the complete proof.
\end{proof}
This result simply states that programs may become less precise by
adding or removing casts, but they will behave in an expected manner.

The results given here are the main results in the proof of the
gradual guarantee, but they depend on a number of auxiliary results.
Unfortunately, they are too numerous to list here, but they are all
given in Appedix~\ref{sec:auxiliary_results_with_proofs}.  Previous
proofs of the gradual guarantee, e.g. \cite{Siek:2015} and
\cite{Garcia:2016}, make heavy use of inversion for typing.  However,
as type systems become more complex inversion for typing is harder to
obtain.  Instead of using inversion for typing we proved inversion
principles for the other judgments.  This turned out to be a lot
easier, because judgments like type and term precision, consistent
subtyping, etc, have less complex definitions.  All of the inversion
principles we used can be found in
Appedix~\ref{sec:auxiliary_results_with_proofs}.

%%% Local Variables: ***
%%% mode:latex ***
%%% TeX-master: "main.tex"  ***
%%% End: ***
