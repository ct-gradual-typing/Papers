We now give a brief summary of related work. Each of the articles
discussed below can be consulted for further references.
%% - Combing static and dynamic typing
%% - Gradual Typing
%%    - Polymorphism
\begin{itemize}
\item Abadi et al.~\cite{Abadi:1989} combine dynamic and static typing
  by adding a new type called $\mathsf{Dynamic}$ along with a new case
  construct for pattern matching on types.  We do not add such a case
  construct, and as a result, show that we can obtain a surprising
  amount of expressivity without it.  They also provide denotational
  models.
\item[]
\item Henglein~\cite{Henglein:1994} defines the dynamic
  $\lambda$-calculus by adding a new type $\mathsf{Dyn}$ to the simply
  typed $\lambda$-calculus and then adding primitive casting
  operations called tagging and check-and-untag.  These new operations
  tag type constructors with their types.  Then untagging checks to
  make sure the target tag matches the source tag, and if not, returns
  a dynamic type error.  These operations can be used to build casting
  coercions which are very similar to our casting morphisms. We can
  also define $<<split>>$, $<<squash>>$, $<<box>>$, and $<<unbox>>$ in
  terms of Henglein's casting coercions.  We consider our previous
  paper \cite{Eades:2017} as a clarification of Henglein's system.
  His core casting calculus can be interpreted into our setting where
  we require retracts instead of full isomorphisms.  This paper can be
  seen as a further extension of this type of work to include bounded
  quantification.
\item[]
\item There is a long history of polymorphism in both static and
  dynamic typing, and in systems that combine static and dynamic
  typing. Henglein and Rehof~\cite{Henglein:1995,Rehof:1995} show how
  to extend Henglein's previous work on combining static and dynamic
  typing discussed above.  This work improves on their work by
  considering bounded quantification and adding a gradually typed
  surface language.

  \ \\
  Matthews and Ahmed~\cite{Matthews:2008:PPT:1792878.1792881} extend
  Girard/Reynolds' System F with static and dynamic typing where the
  unknown type is allowed to be cast to a type variable.  They call
  this type of cast ``consealing''.  This was extended by Ahmed et
  al.~\cite{Ahmed:2011:BLA:1926385.1926409} into a casting calculus
  with blame that included the ability to cast a polymorphic type to
  the unknown type and vice versa.  Grady also includes the ability
  to box and unbox a type variable, but we have chosen not to allow
  one to cast a polymorphic type to the unknown type or vice
  versa. Currently, we are unsure how this will affect gradual typing,
  and we do not have a denotational model that allows this.  We hope
  to include this feature in future work.

  \ \\
  Ahmed et al.~\cite{Ahmed:2011:BLA:1926385.1926409} has some
  similarities to this paper.  Their ``compatibility'' relation is
  similar to type consistency, and they also discuss
  subtyping. However, they do not give a gradually type surface
  language.  In addition, our system can be seen as a further
  clarification of the underlying structure of the casting fragment of
  their system.  We do not have full explicit casts of the form $t : A
  \Rightarrow B$, but instead only have $<<box>>$, $<<unbox>>$,
  $<<split>>$, and $<<squash>>$.  
\item[]
\item As we mentioned in the introduction Siek and
  Taha~\cite{Siek:2006} were the first to define gradual typing
  especially for functional languages, but only for simple types.
  Since their original paper introducing gradual types lots of
  languages have adopted it, but the term ``gradual typing'' started
  to become a catch all phrase for any language combining dynamic and
  static typing.  As a result of this Siek et al.~\cite{Siek:2015}
  later refine what it means for a language to support gradual typing
  by specifying the necessary metatheoretic properties a gradual type
  system must satisfy called the gradual guarantee.  We prove the
  gradual guarantee for Grady in Section~\ref{sec:results}.

  \ \\
  Subtyping for gradual type systems was introduced by Siek and
  Taha~\cite{Siek:2007}, and then further extended by
  Garcia~\cite{Garcia:2016}.  However, neither consider polymorphism.
  The subtyping system for Grady is based on Garcia's work.  He does
  indeed prove the gradual guarantee, but again, his work does not
  consider polymorphism.

\end{itemize}
