In a previous paper the authors \cite{Eades:2017} have shown that
static and dynamic typing can be combined in a very simple and
intuitive way by combining the work of Scott~\cite{Scott:1980} and
Lambek~\cite{Lambek:1980} on categorical models of the untyped and
typed $\lambda$-calculus, respectively.  First, add a new type $<<?>>$
read ``the type of untyped programs'' -- also sometimes called the
unknown type -- and then add four new programs
$<<split>> : <<? -> (? -> ?)>>$, $ <<squash>> : <<(? -> ?) -> ?>> $,
$<<box C>> : <<C -> ?>>$, and $<<unbox C>> : <<? -> C>>$, such that,
$<<squash>>;<<split>> = \id_{[[? -> ?]]}$ and
$<<box C>>;<<unbox C>> = \id_{[[C]]}$.
Categorically, $<<split>>$ and $<<squash>>$, and $<<box>>$ and
$<<unbox>>$ form two retracts.  Then extending the simply typed
$\lambda$-calculus with these two retracts results in a new core
casting calculus, called Simply Typed Grady, for Siek and Taha's
gradual functional type system \cite{Siek:2015}.  Furthermore, the
authors show that Siek and Taha's system can be given a categorical
model in cartesian closed categories with the two retracts.

In this paper we extend Grady with bounded quantification and lists.
We chose bounded quantification so that the bounds can be used to
control which types are castable and which should not be.  Currently,
we will not allow polymorphic types to be cast to the unknown type,
because we do not have a good model nor are we sure how this would
affect gradual typing.  We do this by adding a new bounds, $[[SL]]$,
whose subtypes are all non-polymorphic types -- referred to hence
forth as simple types.  Then we give $[[box]]$ and $[[unbox]]$ the
following types:
\[\small
\begin{array}{lll}
  \CGradydruleTXXBox{} & \text{ and } & \CGradydruleTXXUnbox{}
\end{array}
\]
This differs from our previous work where we limited $<<box>>$ and
$<<unbox>>$ to only atomic types, but then we showed that they could
be extended to any type by combining $<<box>>$ and $<<unbox>>$ with
$<<split>>$ and $<<squash>>$.  In this paper we take these extended
versions as primitive.

Grady now consists of two languages: a surface language -- called
Surface Grady -- and a core language -- called Core Grady. The
difference between the surface and the core is that the former is
gradually typed while the latter is statically typed. Gradual typing
is the combination of static and dynamic typing in such a way that one
can program in dynamic style.  That is, the programmer should not have
to introduce explicit casts. The first functional gradually typed
language is due to Siek and Taha \cite{Siek:2006} where they extend
the typed $\lambda$-calculus with the unknown type $[[?]]$ and a new
relation on types, called the type consistency relation, that indicated
when types should be considered as being castable or not.  For
example, consider the function application rule of Surface Grady:
\[\small
\SGradydruleTXXapp{}
\]
This rule depends on the relation $[[G |- A2 ~ A1]]$ which is the type
consistency relation.  It is reflexive and symmetric, but not
transitive, or one could prove that any type is consistent with any
other type.  Thus, type consistency indicates exactly where explicit
casts need to be inserted.  This rule also depends on the partial
function $\mathsf{fun}$ which is defined in
Fig.~\ref{fig:syntax-surface-grady}.
%% \[\small
%% \begin{array}{lll}
%%   [[fun(?) = ? -> ?]]\\
%%   [[fun(A1 -> B1) = A1 -> B1]]
%% \end{array}
%% \]

As we will see below $[[G |- ? ~ A]]$ holds for any type $[[A]]$.  As an
example, suppose $[[G |- t1 : ?]]$ and $[[G |- t2 : Nat]]$.  In addition, we know
$[[G |- ? ~ Nat]]$, and $[[fun(?) = ? -> ?]]$.
Then
based on the rule above $[[G |- t1 t2 : ?]]$ is typable.  Notice
there are no explicit casts.  Using $<<split>>$ and $<<box Nat>>$ we
can translate this application into Core Grady by adding the casts:
$<<G |- (split (? -> ?) t1) (box Nat t2) : ?>>$.

Subtyping in Core Grady is standard subtyping for bounded system F
extended with the new bounds for simple types.  One important point is
that in Core Grady the unknown type is not a top type, and in fact, is
only related to itself and $[[SL]]$.  However, subtyping in the
surface language is substantially different.

In Surface Grady subtyping is the combination of subtyping and type
consistency called consistent subtyping due to
Siek and Taha~\cite{Siek:2007}.  We denote consistent subtyping by $[[G |-
    A <~ B]]$.  Unlike Core Grady we have $[[G |- ? <~ A]]$ and $[[G
    |- A <~ ?]]$ for any type $[[A]]$.  This gives us some flexibility
when instantiating polymorphic functions.  For example, suppose $[[G
    |- t : Forall (X <: Nat).(X -> X)]]$.  Then, $[[G |- [?]t : ? ->
    ?]]$ is typable, as well as, $[[G |- [Nat]t : Nat -> Nat]]$ by
subsumption.  Similarly, if $[[G |- t : Forall (X <: ?).(X -> X)]]$,
then we can instantiate $[[t]]$ with any type at all.  This seems very
flexible, but it turns out that it does nothing more than what Core
Grady allows when adding explicit casts.

\paragraph{Contributions.} This paper offers the following contributions:
\begin{itemize}
\item The first gradual type system with bounded quantification.  The
  core casting calculus is based on Simply Typed Grady \cite{Eades:17}
  and Bounded System F \cite{Pierce:2002:TPL:509043}.  We show that
  the bounds on types can be used to restrict which types can be cast
  to the unknown type and vice versa.

\item We prove the gradual guarantee for our gradual type system.

\item We show that explicit casting in the surface language of gradual
  type systems are derivable, and making use of these explicit casts
  increases the expressiveness of the language.
\end{itemize}

%%% Local Variables: ***
%%% mode:latex ***
%%% TeX-master: "main.tex"  ***
%%% End: ***
