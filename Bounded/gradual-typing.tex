\newcommand{\GSTLC}{\lambda^?_\to}
\newcommand{\CGSTLC}{\lambda^{\Rightarrow}_\to}
We begin by introducing a slight variation of \cite{Siek:2006}'s
gradually typed functional language given by Siek et
al.~\cite{Siek:2015}.  The syntax of the gradual type system $\GSTLC$
is defined in the following definition.

This definition is the base syntax for every language in this paper.
The typing rules are defined in Figure~\ref{fig:gradual-typing} and
the type consistency relation is defined in
Figure~\ref{fig:type-consistency}.  The main changes of the version of
$\GSTLC$ defined here from the original due to Siek et
al.~\cite{Siek:2015} is that products and natural numbers have been
added.  The definition of products follows how casting is done for
functions. So it allows casting projections of products, for example,
it is reasonable for terms like $[[\x:(? x ?).(succ (fst x))]]$ to
type check.
\renewcommand{\GSiekdrulereflName}[0]{\text{refl}}
\renewcommand{\GSiekdruleboxName}[0]{\text{box}}
\renewcommand{\GSiekdruleunboxName}[0]{\text{unbox}}
\renewcommand{\GSiekdrulearrowName}[0]{\to}
\renewcommand{\GSiekdruleprodName}[0]{\times}
\begin{figure}
  \begin{mdframed}
    \small
    \begin{itemize}
  \item[] \textbf{Syntax:}
    \[ 
    \begin{array}{c@{\hspace{5pt}}r@{}@{\hspace{5pt}}r@{}@{\hspace{2pt}}l@{}llllllllllll}
      \text{(types)} & [[A]],[[B]],[[C]],[[D]] & ::=  & [[Unit]] \mid [[Nat]] \mid [[?]] \mid [[A x B]] \mid [[A1 -> A2]]\\
      \text{(terms)} & [[t]] & ::=  & [[x]] \mid [[triv]] \mid [[0]] \mid [[succ t]] \mid [[\x:A.t]]  \mid [[t1 t2]]
      \mid [[(t1,t2)]] \mid [[fst t]] \mid [[snd t]] \\
      \text{(contexts)} & [[G]] & ::= & [[.]] \mid [[x : A]] \mid [[G1,G2]]\\
    \end{array}
    \]

  \item[] \textbf{Typing Rules:}
    {    \small
    \begin{mathpar}
      \GSiekdruleSXXvar{} \and
      \GSiekdruleSXXunit{} \and
      \GSiekdruleSXXzero{} \and
      \GSiekdruleSXXsucc{} \and
      \GSiekdruleSXXpair{} \and
      \GSiekdruleSXXfst{} \and
      \GSiekdruleSXXsnd{} \and
      \GSiekdruleSXXlam{} \and
      \GSiekdruleSXXapp{}     
    \end{mathpar}
    }

  \item[] \textbf{Type Consistency:}
    \begin{mathpar}
      \GSiekdrulerefl{} \and
      \GSiekdrulebox{} \and
      \GSiekdruleunbox{} \and
      \GSiekdrulearrow{} \and
      \GSiekdruleprod{}    
  \end{mathpar}
    \end{itemize}
  \end{mdframed}
  \caption{The gradually simply typed $\lambda$-calculus: $\GSTLC$}
  \label{fig:GSTLC}
\end{figure}

We can view gradual typing as a surface language feature much like
type inference, and we give it a semantics by translating it into an
annotated core. The language $\GSTLC$ is given an operational
semantics by translating it to a fully annotated core language called
$\CGSTLC$.  Its syntax is an extension of the syntax of $\GSTLC$
(Definition~\ref{def:gradual-syntax}) where terms are the only
syntactic class that differs, and so we do not repeat the syntax of
types or contexts.
\renewcommand{\GSiekdruleCXXvarName}[0]{\text{var}}
\renewcommand{\GSiekdruleCXXunitName}[0]{\text{unit}}
\renewcommand{\GSiekdruleCXXzeroName}[0]{\text{zero}}
\renewcommand{\GSiekdruleCXXsuccName}[0]{\text{succ}}
\renewcommand{\GSiekdruleCXXpairName}[0]{\times}
\renewcommand{\GSiekdruleCXXlamName}[0]{\to}
\renewcommand{\GSiekdruleCXXsndName}[0]{\times_{e_2}}
\renewcommand{\GSiekdruleCXXfstName}[0]{\times_{e_1}}
\renewcommand{\GSiekdruleCXXappName}[0]{\to_e}
\renewcommand{\GSiekdruleCXXcastName}[0]{\text{cast}}
\renewcommand{\GSiekdrulerdAXXvaluesName}{\text{values}}
\renewcommand{\GSiekdrulerdAXXcastIdName}{\text{id-atom}}
\renewcommand{\GSiekdrulerdAXXcastUName}{\text{id-U}}
\renewcommand{\GSiekdrulerdAXXsucceedName}{\text{succeed}}
\renewcommand{\GSiekdrulerdAXXcastArrowName}{\text{arrow}}
\renewcommand{\GSiekdrulerdAXXcastGroundName}{\text{expand}_1}
\renewcommand{\GSiekdrulerdAXXcastExpandName}{\text{expand}_2}
\renewcommand{\GSiekdrulerdAXXbetaName}{}
\renewcommand{\GSiekdrulerdAXXappOneName}{}
\renewcommand{\GSiekdrulerdAXXappTwoName}{}
\renewcommand{\GSiekdrulerdAXXfstName}{}
\renewcommand{\GSiekdrulerdAXXsndName}{}
\renewcommand{\GSiekdrulerdAXXpairOneName}{}
\renewcommand{\GSiekdrulerdAXXpairTwoName}{}
\begin{figure}
  \begin{mdframed}
    \small
    \begin{itemize}
    \item[] \textbf{Syntax:}
      \[ \small
  \begin{array}{cccccc}
    \begin{array}{c@{\hspace{5pt}}r@{}@{\hspace{5pt}}r@{}@{\hspace{2pt}}l@{}llllllllllll}
    \text{(Atomic Types)}  & [[T]] & ::= & [[Unit]] \mid [[Nat]]\\
    \text{(Ground Types)}  & [[R]] & ::= & [[T]] \mid [[?]] \to [[?]]\\    
    \end{array}
    & \quad & 
  \begin{array}{c@{\hspace{5pt}}r@{}@{\hspace{5pt}}r@{}@{\hspace{2pt}}l@{}llllllllllll}
    \text{(values)}        & [[v]] & ::= & [[\x:A.t]]\\
    \text{(terms)}         & [[t]] & ::=  & \ldots \mid [[t : {A} => {B}]]\\
  \end{array}
  \end{array}
  \]

  \item[] \textbf{Typing Rules:}
    {    \small
      \begin{mathpar}
        \GSiekdruleCXXvar{} \and
      \GSiekdruleCXXunit{} \and
      \GSiekdruleCXXzero{} \and
      \GSiekdruleCXXsucc{} \and
      \GSiekdruleCXXpair{} \and
      \GSiekdruleCXXfst{} \and
      \GSiekdruleCXXsnd{} \and
      \GSiekdruleCXXlam{} \and
      \GSiekdruleCXXapp{} \and      
      \GSiekdruleCXXcast{}    
    \end{mathpar}
    }

  \item[] \textbf{Reduction Relation:}
    \begin{mathpar}
      \GSiekdrulerdAXXvalues{} \and
      \GSiekdrulerdAXXcastId{} \and
      \GSiekdrulerdAXXcastU{} \and
      \GSiekdrulerdAXXsucceed{} \and
      \GSiekdrulerdAXXcastArrow{} \and
      \GSiekdrulerdAXXcastGround{} \and
      \GSiekdrulerdAXXcastExpand{} 
      %% \GSiekdrulerdAXXbeta{} \and
      %% \GSiekdrulerdAXXappOne{} \and
      %% \GSiekdrulerdAXXappTwo{} \and
      %% \GSiekdrulerdAXXfst{} \and
      %% \GSiekdrulerdAXXsnd{} \and
      %% \GSiekdrulerdAXXpairOne{} \and
      %% \GSiekdrulerdAXXpairTwo{}  
  \end{mathpar}
    \end{itemize}
  \end{mdframed}
  \caption{The core casting calculus: $\CGSTLC$}
  \label{fig:GSTLC}
\end{figure}
The typing rules for $\CGSTLC$ can be found in
Figure~\ref{fig:annotated-typing}, and the reduction rules in
Figure~\ref{fig:annotated-reduction}. The reduction relation is the
usual definition of call-by-value for the simply typed
$\lambda$-calculus, and so we do not give all of the rules here, but
rather only the casting rules.  In addition, to save space we do not
define the cast insertion algorithm for $\GSTLC$.  Please see Siek et
al.~\cite{Siek:2015} for its definition.  To make it easier for the
reader to connect the reduction rules to their interpretations we give
the reduction rules for $\CGSTLC$ using the terms-in-context
formulation; for an introduction to this style please see
Crole~\cite{Crole:1994}.

%%% Local Variables: ***
%%% mode:latex ***
%%% TeX-master: "main.tex"  ***
%%% End: ***
