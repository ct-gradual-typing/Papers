Core Grady is a non-gradual type system that combines both static and
dynamic typing.  It is an extension of the authors previous work
\cite{Eades:2017}, adding bounded quantification and lists, on combing
the work of Scott \cite{Scott:1980} and Lambek \cite{Lambek:1988}.
Furthermore, Core Grady is a simple extension of Bounded System F.
The syntax for Core Grady can be found in
Fig.~\ref{fig:syntax-core-grady}.
\begin{figure}
    \small
  \begin{mdframed}
    %% \item[] \textbf{Syntax:}
      \[
      \begin{array}{cl}
        %% \begin{array}{l}
        %%   \text{(skeletons)}
        %% \end{array}     &
        %% \begin{array}{lcl}
        %%   [[S]], [[K]], [[U]] & ::= & [[?]] \mid [[List S]] \mid [[S1 x S2]] \mid [[S1 -> S2]]\\
        %% \end{array}\\\\
                
        \begin{array}{l}
          \text{(terms)}
        \end{array}     &
        \begin{array}{lcl}
          [[t]] & ::= & \cdots \mid [[squash S]] \mid [[split S]] \mid [[error A]]\\
        \end{array}\\\\        
      \end{array}
      \]    
  \end{mdframed}
  \caption{Syntax for Core Grady}
  \label{fig:syntax-core-grady}
\end{figure}
The syntax of types and typing context for Core Grady are exactly the
same as Surface Grady, and so we do not repeat them here.  The syntax
of terms is an extension of the syntax for Surface Grady, and so we
only show the additions.  The term $[[error A]]$ will be used to
trigger a type error during run time.

Subtyping for Core Grady is as one might expect for Bounded System F.
The rules for subtyping are given in
Fig.~\ref{fig:subtyping-core-grady}.
\begin{figure}
  \begin{mdframed}\small
    \begin{mathpar}
      \CGradydruleSXXRefl{} \and
      \CGradydruleSXXTop{} \and
      \CGradydruleSXXVar{} \and
      \CGradydruleSXXNatSL{} \and
      \CGradydruleSXXUnitSL{} \and
      \CGradydruleSXXListSL{} \and
      \CGradydruleSXXArrowSL{} \and
      \CGradydruleSXXProdSL{} \and
      \CGradydruleSXXList{} \and
      \CGradydruleSXXProd{} \and
      \CGradydruleSXXArrow{} \and
      \CGradydruleSXXForall{}      
    \end{mathpar}
  \end{mdframed}
  \caption{Subtyping for Core Grady}
  \label{fig:subtyping-core-grady}
\end{figure}
Note that Core Grady does not depend on type consistency, this is
purely a surface language feature.  In addition, subtyping in the core
is also non-transitive just like subtyping in Surface Grady.  We
axiomatize the super type $[[SL]]$ in the same way as Surface Grady,
but the unknown type is no longer a simple type.  In fact, the unknown
type is only a subtype of itself.

Similarly to subtyping, typing for Core Grady is a simple extension of
typing for Bounded System F.
\begin{figure}
  \begin{mdframed}
    \begin{mathpar}
      \CGradydruleTXXvarP{} \and
      \CGradydruleTXXBox{} \and
      \CGradydruleTXXUnbox{} \and
      \CGradydruleTXXsquash{} \and
      \CGradydruleTXXsplit{} \and
      \CGradydruleTXXunitP{} \and
      \CGradydruleTXXzeroP{} \and
      \CGradydruleTXXsucc{} \and
      \CGradydruleTXXncase{} \and
      \CGradydruleTXXempty{} \and
      \CGradydruleTXXcons{} \and
      \CGradydruleTXXlcase{} \and
      \CGradydruleTXXpair{} \and
      \CGradydruleTXXfst{} \and
      \CGradydruleTXXsnd{} \and
      \CGradydruleTXXlam{} \and
      \CGradydruleTXXapp{} \and
      \CGradydruleTXXLam{} \and
      \CGradydruleTXXtypeApp{} \and
      \CGradydruleTXXSub{} \and
      \CGradydruleTXXerror{} 
    \end{mathpar}
  \end{mdframed}
  \caption{Typing rules for Core Grady}
  \label{fig:typing-core-grady}
\end{figure}
The most interesting rules here are the rules for splitting and
squashing. Core Grady has the following types of casts:
\[
\begin{array}{lll}
  \begin{array}{lll}
    \text{(boxing)} & [[box A]] : [[A -> ?]]\\
  \text{(unboxing)} & [[unbox A]] : [[? -> A]]\\
  \end{array}
  & \quad & 
  \begin{array}{lll}
    \text{(splitting)} & [[split S]] : [[? -> S]]\\
  \text{(squashing)} & [[squash S]] : [[S -> ?]]\\
  \end{array}
\end{array}
\]
These casts are enough to do everything the surface language can
do. In addition, the general casting rule used in the casting calculi
found in the gradual typing literature, e.g.
\cite{Siek:2007,Siek:2006,Ahmed:2011:BLA:1926385.1926409,Siek:2015},
denoted $[[t]] : [[A]] \Rightarrow [[B]]$, can be modeled by these four
operations \cite{Eades:2017}.

Unlike Surface Grady, Core Grady has a reduction relation.  The
surface language will then be translated into the core language where
evaluation will take place.  The reduction relation is defined in
Fig.~\ref{fig:reduction-core-grady}.
\begin{figure}
  \begin{mdframed}
    \begin{mathpar}
      \CGradydrulerdXXretracT{} \and
      \CGradydrulerdXXretracTE{} \and
      \CGradydrulerdXXretractU{} \and
      \CGradydrulerdXXretractUE{} \and
      \CGradydrulerdXXncaseZero{} \and
      \CGradydrulerdXXncaseSucc{} \and
      \CGradydrulerdXXlcaseEmpty{} \and
      \CGradydrulerdXXlcaseCons{} \and      
      \CGradydrulerdXXprojOne{} \and
      \CGradydrulerdXXprojTwo{} \and
      \CGradydrulerdXXbeta{} \and      
      \CGradydrulerdXXtypeBeta{} \and
    \end{mathpar}
  \end{mdframed}
  \caption{Reduction rules for Core Grady}
  \label{fig:reduction-core-grady}
\end{figure}
Reduction is an extended version of call-by-name.  We omit congruence
rules for brevity.  Reduction will not reduce under
$\lambda$-abstractions or arguments to $[[box]]$ or $[[squash]]$.
However, it will reduce arguments to $[[unbox]]$ and $[[split]]$ in
order to insure the retract rules apply as much as possible.

\subsection{Cast Insertion}
\label{subsec:cast_insertion}

Surface Grady is translated into Core Grady by the cast insertion
algorithm.  Anywhere type consistency or one of the metafunctions
defined in Fig.~\ref{fig:syntax-surface-grady} are used a cast must be
inserted.

We call a Core Grady expression that is built from $[[box]]$,
$[[unbox]]$, $[[split]]$, $[[squash]]$, the functors
$\mathsf{List}\,-$, $- \times -$, and $- \to -$, and the identity
function a casting morphism.  Each of the functors are definable as
metafunctions:
\begin{center}\small
  \begin{itemize}
  \item[] (List functor)  \\
    $\mathsf{List}\,[[t]] :=
    [[fix (\r:H(List A) -> List B.H(\x:H(List A).H(case x : List A of [] -> [], (y::ys) -> H((t y) :: (r ys))))) ]]$
    given $[[G |- t : A -> B]]$
  \item[]    
  \item[] (Product functor) \\ 
    $[[t1]] \times [[t2]] := [[\ x : A x B.(t1 (fst x), t2 (snd x))]]$
    given $[[G |- t1 : A -> D]]$ and $[[G |- t2 : B -> E]]$
  \item[]    
  \item[] (Internal Hom Functor) \\  
    $[[t1]] \to [[t2]] := [[\ f : A -> B.\y : D.H(t2 (f (t1 y)))]]$ 
    given $[[G |- t1 : D -> A]]$ and $[[G |- t2 : B -> E]]$
  \end{itemize}
\end{center}
The definition of the list functor is the usual definition of the map
function for lists which requires the use of the fix point operator
given in the introduction.

In the authors previous work \cite{Eades:2017} they showed a similar
result to the following.  In fact, their proofs are nearly identical,
and so we do not give the proof here.
\begin{lemma}[Casting Morphisms]
  \label{lemma:casting_morphisms}
  If $<<G |- A ~ B>>$, then there are casting morphisms $[[G |- c1 : A -> B]]$
  and $[[G |- c2 : B -> A]]$.
\end{lemma}
Using this result we can define the cast insertion algorithm which
takes in a Surface Grady term and then returns a Core Grady term by
inserting casting morphisms where type consistency is used.  Thus,
this algorithm is type directed.  Its definition is given in
Fig.~\ref{fig:cast-insert}.
\begin{figure}
  \begin{mdframed}[innertopmargin=60px]
    \rotatebox{90}{
    \begin{mathpar}
    \SGradydruleciXXsuccU{} \and
    \SGradydruleciXXsucc{} \and      
    \SGradydruleciXXfstU{} \and
    \SGradydruleciXXfst{} \and
    \SGradydruleciXXsndU{} \and
    \SGradydruleciXXsnd{} \and
    \SGradydruleciXXlcaseU{} \and
    \SGradydruleciXXlcase{} \and
    \SGradydruleciXXappU{} \and
    \SGradydruleciXXapp{} 
  \end{mathpar}}
  \end{mdframed}
  \caption{Cast Insertion Algorithm}
  \label{fig:cast-insert}
\end{figure}
We only give the most interesting cases, because the others are either
trivial identity cases or are similar to the ones given.  We denote
constructing the casting morphism for $<<G |- A ~ B>>$ by
$<<caster(A,B) = c>>$.

As an example the following is the result of applying the cast
insertion algorithm -- after some simplifications -- to the fix point
operator given in the introduction:
\begin{lstlisting}[language=Haskell]
  omega : (? -> ?) -> ?
  omega = \(x : ? -> ?) -> (x (squash (? -> ?) x));

  ycomb : (? -> ?) -> ?
  ycomb = \(f : ? -> ?) -> omega (\(x:?) -> f ((split (? -> ?) x) x));

  fix : forall (X <: Simple).((X -> X) -> X)
  fix = \(X <: Simple) -> \(f:X -> X) ->
           unbox<X> (ycomb (\(y:?)->box<X> (f (unbox<X> y))));
\end{lstlisting}

% section cast_insertion (end)
