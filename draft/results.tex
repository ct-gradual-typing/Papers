\begin{lemma}[Inclusion of Bounded System F]
  \label{lemma:F-inclusion}
  Suppose $t$ is fully annotated and does not contain any applications
  of $[[box]]$ or $[[unbox]]$, and $[[A]]$ is static.  Then
  \begin{itemize}
  \item[i.] $[[G |-F t : A]]$ if and only if $\,[[G |- t : A]]$, and 
  \item[ii.] $[[t F~>* t']]$ if and only if $[[t ~>* t']]$.
  \end{itemize}
\end{lemma}
\begin{proof}
  We give proof sketches for both parts.  The interesting cases are
  the right-to-left directions of each part.  If we simply remove all
  rules mentioning the unknown type $[[?]]$ and the type consistency
  relation, and then remove $[[box]]$, $[[unbox]]$, and $[[?]]$ from
  the syntax of Surface Grady, then what we are left with is bounded
  system F.  Since $[[t]]$ is fully annotated and $[[A]]$ is static,
  then $[[G |- t : A]]$ will hold within this fragment.

  Moving on to part two, first, we know that $[[t]]$ does not contain
  any occurrence of $[[box]]$ or $[[unbox]]$ and is fully annotated.
  This implies that $[[t]]$ lives within the bounded system F fragment
  of Surface Grady. Thus, before evaluation of $[[t]]$ Surface Grady
  will apply the cast insertion algorithm which will at most insert
  applications of the identity function into $[[t]]$ producing a term
  $\widehat{[[t]]}$, but then after potentially more than one step of
  evaluation within Core Grady, those applications of the identity
  function will be $\beta$-reduced away resulting in $\widehat{[[t]]}
  \rightsquigarrow^* [[t]] \rightsquigarrow^* [[t']]$.  In addition,
  since $[[t]]$ in Surface Grady is the exact same program as $[[t]]$
  in bounded system F, then we know $[[t F~>* t']]$ will hold.
\end{proof}

\begin{lemma}[Inclusion of DTLC]
  \label{lemma:inclusion_of_dtlc}
  Suppose $[[t]]$ is a closed term of DTLC. Then
  \begin{itemize}
  \item[i.] $[[. |- |t| : ?]]$, and
  \item[ii.] $[[t D~>* t']]$ if and only if $[[|t| ~>* |t'|]]$.
  \end{itemize}
\end{lemma}
\begin{proof}
  In this case DTLC is embedded into the simply typed fragment of
  Grady, and hence, this proof is the same result proven by
  \cite{Siek:2006}, and \cite{Siek:2015}.
\end{proof}


\renewcommand{\SGradydrulePXXUName}{[[?]]}
\renewcommand{\SGradydrulePXXreflName}{\text{refl}}
\renewcommand{\SGradydrulePXXarrowName}{\to}
\renewcommand{\SGradydrulePXXprodName}{\times}
\renewcommand{\SGradydrulePXXlistName}{\mathsf{List}}
\renewcommand{\SGradydrulePXXforallName}{\forall}
\begin{figure}
  \begin{mdframed}
    \begin{mathpar}
      \SGradydrulePXXU{} \and
      \SGradydrulePXXrefl{} \and
      \SGradydrulePXXarrow{} \and
      \SGradydrulePXXprod{} \and
      \SGradydrulePXXlist{} \and
      \SGradydrulePXXforall{}      
    \end{mathpar}
  \end{mdframed}
  \caption{Type Precision}
  \label{fig:type-pre}
\end{figure}

\begin{lemma}[Left-to-Right Consistent Subtyping]
  \label{lemma:consistent-subtyping-1}
  Suppose $[[G |- A <~ B]]$.
  \begin{enumR}
    \item $[[G |- A ~ A']]$ and $<<G |- A' <: B>>$ for some $[[A']]$.
    \item $[[G |- B' ~ B]]$ and $<<G |- A <: B'>>$ for some $[[B']]$.
  \end{enumR}   
\end{lemma}
\begin{proof}
  This is a proof by induction on $[[G |- A <~ B]]$.  See
  Appendix~\ref{subsec:proof_of_left-to-right_consistent_subtyping_lemma:consistent-subtyping-1}
  for the complete proof.
\end{proof}

\begin{corollary}[Consistent Subtyping]
  \label{corollary:consistent_subtyping}
  \begin{enumR}
  \item[]
  \item $[[G |- A <~ B]]$ if and only if $[[G |- A ~ A']]$ and $<<G |- A' <: B>>$ for some $[[A']]$.
  \item $[[G |- A <~ B]]$ if and only if $[[G |- B' ~ B]]$ and $<<G |- A <: B'>>$ for some $[[B']]$.
  \end{enumR}
\end{corollary}
\begin{proof}
  The left-to-right direction of both cases easily follows from
  Lemma~\ref{lemma:consistent-subtyping-1}, and the right-to-left
  direction of both cases follows from induction on the subtyping
  derivation and Lemma~\ref{lemma:simply_typed_consistent_types_are_subtypes}.
\end{proof}


\begin{lemma}[Gradual Guarantee Part One]
  \label{lemma:gradual_guarantee_part_one}
  If $[[G |- t : A]]$, $[[t <= t']]$, and $[[G <= G']]$ then $[[G' |- t' : B]]$ and $[[A <= B]]$.
\end{lemma}
\begin{proof}
  This is a proof by induction on $[[G |- t : A]]$; see
  Appendix~\ref{subsec:proof_of_gradual_guarantee_part_one_lemma:gradual_guarantee_part_one}
  for the complete proof.
\end{proof}

\begin{lemma}[Type Preservation for Cast Insertion]
  \label{lemma:type_preservation_for_cast_insertion}
  If $[[G |- t1 : A]]$ and $[[G |- t1 => t2 : B]]$, then $<<G |- t2 : B>>$ and $[[G |- A ~ B]]$.
\end{lemma}
\begin{proof}
  The cast insertion algorithm is type directed and with respect to
  every term $[[t1]]$ it will produce a term $[[t2]]$ of the core
  language with the type $[[A]]$ -- this is straightforward to show by
  induction on the form of $[[G |- t1 : A]]$ making use of typing for
  casting morphisms Lemma~\ref{lemma:typing_casting_morphisms} --
  except in the case of type application.  Please see
  Appendix~\ref{subsec:proof_of_type_preservation_for_cast_insertion}
  for the complete proof.
\end{proof}

\begin{lemma}[Type Preservation]
  \label{lemma:type_preservation}
  If $<<G |- t1 : A>>$ and $<<t1 ~> t2>>$, then $<<G |- t2 : A>>$.
\end{lemma}
\begin{proof}
  This proof holds by induction on $<<G |- t1 : A>>$ with further case
  analysis on the structure the derivation $<<t1 ~> t2>>$.
\end{proof}

\begin{lemma}[Simulation of More Precise Programs]
  \label{lemma:simulation_of_more_precise_programs}
  Suppose $<<G |- t1 : A>>$, $<<G |- t1 <= t1'>>$, $<<G |- t'1 : A'>>$, and $<<t1 ~> t2>>$.
  Then $<<t1' ~>* t2'>>$ and $<<G |- t2 <= t2'>>$ for some $<<t2'>>$.
\end{lemma}
\begin{proof}
  This proof holds by induction on $<<G |- t1 : A1>>$.  See
  Appendix~\ref{subsec:proof_of_simulation_of_more_precise_programs_lemma:simulation_of_more_precise_programs}
  for the complete proof.
\end{proof}

\begin{theorem}[Gradual Guarantee]
  \label{thm:gradual_guarantee} 
  \begin{enumR}
  \item[] 
  \item If $[[. |- t : A]]$ and $[[t <= t']]$, then $[[. |- t' : B]]$ and $[[A <= B]]$.
  \item Suppose $<<. |- t : A>>$ and $<<. |- t <= t'>>$. Then
    \begin{enumA}
    \item if $<<t ~>* v>>$, then $<<t' ~>* v'>>$ and $<<. |- v <= v'>>$,
    \item if $<<t ^>>$, then $<<t' ^>>$,
    \item if $<<t' ~>* v'>>$, then $<<t ~>* v>>$ where $<<. |- v <= v'>>$, or $<<t ~>* error A>>$, and
    \item if $<<t' ^>>$, then $<<t ^>>$ or $<<t ~>* error A>>$.
    \end{enumA}
  \end{enumR}
\end{theorem}
\begin{proof}
  Part i. holds by Lemma~\ref{lemma:gradual_guarantee_part_one}.

 \ \\
 \noindent
 Part i.a. follows from simulation for more precise programs
 (Lemma~\ref{lemma:simulation_of_more_precise_programs}), because this
 result shows that the evaluation of $<<t>>$ and $<<t'>>$ are in lock
 step, and hence, if $<<t>>$ hits a value, then we know that $<<t'>>$
 terminates, but it will also terminate at a value instead of
 $<<error>>$ by the definition of term precision.
 
 \ \\
 \noindent
 Part i.b. follows from simulation for more precise programs
 (Lemma~\ref{lemma:simulation_of_more_precise_programs}), because
 this result shows that the evaluation of $<<t>>$ and $<<t'>>$ are in
 lock step, and hence, if $<<t>>$ diverges, then so does $<<t'>>$.
\end{proof}

%%% Local Variables: ***
%%% mode:latex ***
%%% TeX-master: "main.tex"  ***
%%% End: ***
