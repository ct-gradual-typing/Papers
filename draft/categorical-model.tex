The strength and main motivation for giving a categorical model to a
programming language is that it can expose the fundamental structure
of the language.  This arises because a lot of the language features
that often cloud the picture go away, for example, syntactic notions
like variables disappear.  This can often simplify things and expose
the underlying structure.  Reynolds \cite{?} was a big advocate for
the use of category theory in programming language research for these
reasons.  For example, when giving the simply typed $\lambda$-calculus
a categorical model we see that it is a cartesian closed category, but
we also know that intuitionistic logic has the same model due to
\cite{Lambek:1980}; on the syntactic side these two theories are
equivalent as well due to \cite{Howard:1980}. Thus, the fundamental
structure of the simply typed $\lambda$-calculus is intuitionistic
logic.  This also shows a relationship between seemingly unrelated
theories.  It is quite surprising that these two theories are related.
Another more recent example of this can be found in the connection
between dependent type theory and homotopy theory \cite{??}.

Another major benefit of studying the categorical model of programming
languages is that it gives us a powerful tool to study language
extensions.  For example, purely functional programming in Haskell
would not be where it is without the seminal work of Moggi and Wadler
\cite{??}  on using monads -- a purely categorical notion -- to add
side effects to Haskell.  Thus, we believe that developing these types
of models for new language designs and features can be hugely
beneficial.

Interpreting a programming language into a categorical model requires
three steps.  First, the types are interpreted as objects.  Then
programs are interpreted as morphisms in the category, but this is a
simplification.  Every morphism, $f$, in a category has a source
object and a target object, we usually denote this by $f : A \mto B$.
Thus, in order to interpret programs as morphisms the program must
have a source and target.  So instead of interpreting raw terms as
morphisms we interpret terms in their typing context.  That is, we
must show how to interpret every $[[G |- t : A]]$ as a morphism $[[t]]
: \interp{[[G]]} \mto \interp{[[A]]}$.  The third step is to show that
whenever one program reduces to another their interpretations are
isomorphic in the model. This means that whenever $[[G |- t1 ~> t2 :
    A]]$, then $[[ [|t1|] ]] \cong [[ [|t2|] ]] : [[ [| G |] --> [| A
      |] ]]$.  This is the reason why we defined our reduction in a
typed fashion to aid us in understanding how it relates to the model.
For a more thorough introduction see \cite{Crole:1994}.

We now give a categorical model for $\lambda^?_\to$ and
$\lambda^{\langle [[A]] \rangle}_\to$.  Then we complete the three
steps summarized above. We will show how to interpret the typing of
the former into the model, and then show how to do the same for the
latter, furthermore, we show that reduction can be interpreted into
the model as well, thus concluding soundness for $\lambda^{\langle
  [[A]] \rangle}_\to$ with respect to our model.

The model we develop here builds on the seminal work of
\cite{Lambek:1980} and \cite{Scott:1980}.  \cite{Lambek:1980} showed
that the typed $\lambda$-calculus can be modeled by a cartesian closed
category.  In the same volume as Lambek, Scott essentially showed that
the untyped $\lambda$-calculus is actually typed.  That is, typed
theories are more fundamental than untyped ones.  He accomplished this
by adding a single type, $[[?]]$, and two functions $[[squash : (? ->
    ?) -> ?]]$ and $[[split : ? -> (?  -> ?)]]$, such that,
$[[squash]];[[split]] = \id : [[(? -> ?) -> (?  -> ?)]]$. At this
point he was able to translate the untyped $\lambda$-calculus into
this unityped one.  Categorically, he modeled $[[split]]$ and
$[[squash]]$ as the morphisms in a retract within a cartesian closed
category -- the same model as typed $\lambda$-calculus.

\begin{definition}
  \label{def:retract}
  Suppose $\cat{C}$ is a category.  Then an object $A$ is a
  \textbf{retract} of an object $B$ if there are morphisms $i : A \mto
  B$ and $r : B \mto A$ such that the following diagram commutes:
  \[
  \bfig
  \qtriangle/->`=`->/[A`B`A;i``r]
  \efig
  \]
\end{definition}
Thus, $[[? -> ?]]$ is a retract of $[[?]]$, but we extend this
slightly to include $[[? x ?]]$ being a retract of $[[?]]$.  This is
only a slight extension of Scott's model, because our languages will
have products where he did not consider products, because he was
considering the traditional definition of the untyped
$\lambda$-calculus.

We can now define our categorical model of the untyped
$\lambda$-calculus with products.
\begin{definition}
  \label{def:model-untyped}
  An \textbf{untyped $\lambda$-model}, $(\cat{C}, ?, \split,
  \squash)$, is a cartesian closed category $\cat{C}$ with a
  distinguished object $?$ and morphisms $\squash : S \mto ?$ and
  $\split : ? \mto S$ making the object $S$ a retract of $?$, where
  $S$ is either $[[? -> ?]]$ or $[[? x ?]]$.
\end{definition}

\begin{theorem}[\citet{Scott:1980}]
  \label{thm:untyped-lambda-model-sound-complete}
  An untyped $\lambda$-model is a sound and complete model of the untyped $\lambda$-calculus.
\end{theorem}

Since all of the languages we are studying here contain the natural
numbers we must be able to interpret them into our model.  We give a
novel approach to modeling the natural numbers with their
(non-recursive) eliminator using what we call a Scott natural number
object.  Now the natural number eliminator is not part of
$\lambda^?_\to$ or $\lambda^{\langle [[A]] \rangle}_\to$, but we want
Grady to contain it, and Grady will directly correspond to the model.
\begin{definition}
  \label{def:SNNO}
  Suppose $\cat{C}$ is a cartesian closed category.  A
  \textbf{Scott natural number object (SNNO)} is an object $[[Nat]]$ of $\cat{C}$ and
  morphisms $\mathsf{z} : 1 \mto [[Nat]]$ and $[[succ]] : [[Nat]]  \mto [[Nat]]$ of $\cat{C}$,
  such that, for any morphisms $f : [[Y]] \mto X$ and $g : [[Y x Nat]] \mto X$ of $\cat{C}$ there is a unique morphism
  $\case_X : [[Y x Nat]] \mto X$ making the following diagrams commute:
       \[
       \begin{array}{lll}
         \bfig
         \qtriangle|amm|<1000,500>[
           [[Y x Nat]]`
           [[Y x Nat]]`
           X;
           \id_Y \times \mathsf{z}`
           \pi_1;f`
           \case_X]

         \ptriangle(1000,0)|amm|/<-`->`->/<1000,500>[
           [[Y x Nat]]`
           [[Y x Nat]]`
           X;
           \id_Y \times [[succ]]`
           \case_X`
           g ]
         \efig                               
       \end{array}
       \]
       Informally, the two diagrams essentially assert that we can
       define $\case_X$ as follows:
       \begin{center}
         \begin{math}
           \begin{array}{rll}
             \case_X\,y\,0 & = & f\,y\\
             \case_X\,y\,[[(succ x)]] & = &  g\,y\,x
           \end{array}
         \end{math}
       \end{center}
\end{definition}
\noindent This formalization of natural numbers is inspired by the
definition of Scott Numerals \cite{??} where the notion of a case
distinction is built into the encoding. We can think of $[[Y]]$ in the
source object of $\case$ as the type of additional inputs that will be
passed to both $f$ and $g$, but we can think of $[[Nat]]$ in the
source object of $\case$ as the type of the scrutiny.  Thus, since in
the base case there is no predecessor, $f$, will not require the
scrutiny, and so it is ignored.

One major difference between SNNOs and the more traditional natural
number objects is that in the definition of the latter $g$ is defined
by well-founded recursion.  However, SNNOs do not allow this, but in
the presence of fixpoints we are able to regain this feature without
having to bake it into the definition of natural number objects.
However, to allow this we have found that when combining fixpoints and
case analysis to define terminating functions on the natural numbers
it is necessary to uniformly construct the input to both $f$ and $g$
due to the reduction rule of the Y combinator.  Thus, we extend the
type of $f$ to $[[Y x Nat]]$, but then ignore the second projection
when reaching the base case.

So far we can model the untyped and the typed $\lambda$-calculi within
a cartesian closed category, but we do not have any way of moving
typed data into the untyped part and vice versa.  To accomplish this
we add two new morphisms $[[box C : C --> ?]]$ and $[[unbox C : ? -->
    C]]$ such that $[[box C]];[[unbox C]] = \id : C \mto C$ for every
atomic type $[[C]]$.  Thus, each atomic type is a retract of $[[?]]$.
This enforces that the only time we can really consider something as
typed is if it were boxed up in the first place.  We can look at this
from another perspective as well.  If the programmer tries to unbox
something that is truly untyped, then their program may actually type
check, but they will obtain a dynamic type error at runtime, because
the unbox will never have been matched up with the correct boxed data.
For example, we can cast $3$ to type $\mathsf{Bool}$ by
$[[unbox]]_{\mathsf{Bool}} ([[box Nat]]\, 3)$, but if this program is
every run, then we will obtain a dynamic type error.  Note that we can
type the previous program in $\lambda^{\langle [[A]] \rangle}_\to$ as
well, but if we run the program it will result in a dynamic type error
too.

Now we combine everything we have discussed so far to obtain the
categorical model.
\begin{definition}
  \label{def:gradual-lambda-model}
  A \textbf{gradual $\lambda$-model}, $(\cat{T}, \cat{C}, ?, \T,
  \split,\squash, \bx,$ $\unbox)$, where $\cat{T}$ is a discrete
  category with at least two objects $[[Nat]]$ and $[[Unit]]$,
  $\cat{C}$ is a cartesian closed category with a SNNO,
  $(\cat{C},?,\split,$ $\squash)$ is an untyped $\lambda$-model, $\T :
  \cat{T} \mto \cat{C}$ is an embedding -- a full and faithful functor
  that is injective on objects -- and for every object $A$ of
  $\cat{T}$ there are morphisms $\bx_A : TA \mto ?$ and $\unbox_A : ?
  \mto TA$ making $TA$ a retract of $?$.
\end{definition}
\noindent
We call the category $\cat{T}$ the category of atomic types.  We call
an object, $[[A]]$, \textbf{atomic} iff there is some object $[[A']]$
in $\cat{T}$ such that $[[A]] = \T[[A']]$. Note that we do not
consider $[[?]]$ an atomic type.  The model really is the cartesian
closed category $\cat{C}$, but it is extended with the structure of
both the typed and the untyped $\lambda$-calculus with the ability to
cast data.

Interpreting the typing rules for $\lambda^?_\to$ will require the
interpretation of type consistency.  Thus, we must be able to cast any
type $[[A]]$ to $[[?]]$, but as stated the model only allows atomic
types to be casted.  It turns out that this can be lifted to any type.

To cast any type $[[A]]$ to $[[?]]$ we will build casting morphisms
that first take the object $[[A]]$ to its skeleton, and then takes the
skeleton to $[[?]]$.
\begin{definition}
  \label{def:skeleton}
  Suppose $(\cat{T}, \cat{C}, ?, \T, \split,\squash, \bx,$ $\unbox)$
  is a gradual $\lambda$-model.  Then the \textbf{skeleton} of an
  object $[[A]]$ of $\cat{C}$ is an object $[[S]]$ that is constructed
  by replacing each atomic type in $[[A]]$ with $[[?]]$. Given an
  object $[[A]]$ we denote its skeleton by $[[skeleton A]]$.
\end{definition}
One should think of the skeleton of an object as the supporting type
structure of the object, but we do not know what kind of data is
actually in the structure. For example, the skeleton of the object
$[[Nat]]$ is $[[?]]$, and the skeleton of $[[(Nat x Unit) -> Nat ->
    Nat]]$ is $[[(?  x ?) -> ? -> ?]]$.

The next definition defines a means of constructing a casting morphism
that casts a type $[[A]]$ to its skeleton and vice versa.  This
definition is by mutual recursion on the input type.
\begin{definition}
  \label{def:boxing-unboxing}
  Suppose $(\cat{T}, \cat{C}, ?, \T, \split,\squash, \bx,$ $\unbox)$
  is a gradual $\lambda$-model.  Then for any object $[[A]]$ whose
  skeleton is $[[S]]$ we define the functions
  $\widehat{\mathsf{box}}\,[[A]]$ and
  $\widehat{\mathsf{unbox}}\,[[S]]\,[[A]]$ by mutual recursion on
  $[[A]]$ as follows:
  \[
  \begin{array}{lll}
    \widehat{[[box]]}\,?\,[[A]] = [[box A]]\\
    \,\,\,\,\text{when } [[A]]\text{ is atomic}\\
    \\
    \widehat{[[box]]}\,[[?]]\,[[?]] = \id_{[[?]]}\\
    \\
    \widehat{[[box]]}\,([[A1 -> A2]]) = (\widehat{[[unbox]]}\,([[skeleton A1]])\,[[A1]]) \to (\widehat{[[box]]}\,[[S2]]\,[[A2]])\\
    \\
    \widehat{[[box]]}\,([[A1 x A2]]) = (\widehat{[[box]]}\,[[A1]]) \times (\widehat{[[box]]}\,[[A2]])\\
    \\
    \widehat{[[unbox]]}\,?\,[[A]] = [[unbox A]]\\
    \,\,\,\,\text{when } [[A]]\text{ is atomic}\\
    \\
    \widehat{[[unbox]]}\,[[?]]\,[[?]] = \id_{[[?]]}\\
    \\
    \widehat{[[unbox]]}\,([[S1 -> S2]])\,([[A1 -> A2]]) = (\widehat{[[box]]}\,[[A1]]) \to (\widehat{[[unbox]]}\,[[S2]]\,[[A2]])\\
    \\
    \widehat{[[unbox]]}\,([[S1 x S2]])\,([[A1 x A2]]) = (\widehat{[[unbox]]}\,[[S1]]\,[[A1]]) \times (\widehat{[[unbox]]}\,[[S2]]\,[[A2]])\\        
  \end{array}
  \]
\end{definition}

