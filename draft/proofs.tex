\input{aux-results-proofs-ott}

\section{Proofs}
\label{sec:proofs}

\subsection{Proof of Lifted Retract (Lemma~\ref{lemma:lifted_retract})}
\label{subsec:proof_of_lifted_retract}
This is a proof by induction on the form of $[[A]]$.

\begin{itemize}
\item[] Case. Suppose $[[A]]$ is atomic.  Then:
  \[
    [[lbox A]];[[lunbox A]] = [[box A]];[[unbox A]] = \id_{[[A]]}
    \]
    
  \item[] Case. Suppose $[[A]]$ is $[[?]]$.  Then:
    \[
    \begin{array}{lll}
      [[lbox A]];[[lunbox A]] & = & [[lbox ?]];[[lunbox ?]]\\
      & = & \id_{[[?]]};\id_{[[?]]}\\
      & = & \id_{[[?]]}\\
      & = & \id_{[[A]]}
    \end{array}
    \]

  \item[] Case. Suppose $[[A]] = [[A1 -> A2]]$.  Then:
    \[\small
    \begin{array}{lll}
      [[lbox A]];[[lunbox A]]
      & = & [[lbox (A1 -> A2)]];[[lunbox (A1 -> A2)]]\\
      & = & ([[lunbox A1]] \to [[lbox A2]]);([[lbox A1]] \to [[lbox A2]])\\
      & = & ([[lbox A1]];[[lunbox A1]]) \to ([[lbox A2]];[[lunbox A2]])\\
    \end{array}
    \]
    By two applications of the induction hypothesis we know the
    following:
    \[
    \begin{array}{lll}
      [[lbox A1]];[[lunbox A1]] = \id_{[[A1]]} & \text{ and } & [[lbox A2]];[[lunbox A2]] = \id_{[[A2]]}
    \end{array}
    \]
    Thus, we know the following:
    \[
    \begin{array}{lll}
      ([[lbox A1]];[[lunbox A1]]) \to ([[lbox A2]];[[lunbox A2]])
      & = & \id_{[[A1]]} \to \id_{[[A2]]}\\
      & = & \id_{[[A1 -> A2]]}\\
      & = & \id_{[[A]]}
    \end{array}
    \]

  \item[] Case. Suppose $[[A]] = [[A1 x A2]]$.  Then:
    \[\small
    \begin{array}{lll}
      [[lbox A]];[[lunbox A]]
      & = & [[lbox (A1 x A2)]];[[lunbox (A1 x A2)]]\\
      & = & ([[lbox A1]] \times [[lbox A2]]);([[lunbox A1]] \times [[lbox A2]])\\
      & = & ([[lbox A1]];[[lunbox A1]]) \times ([[lbox A2]];[[lunbox A2]])\\
    \end{array}
    \]
    By two applications of the induction hypothesis we know the
    following:
    \[
    \begin{array}{lll}
      [[lbox A1]];[[lunbox A1]] = \id_{[[A1]]} & \text{ and } & [[lbox A2]];[[lunbox A2]] = \id_{[[A2]]}
    \end{array}
    \]
    Thus, we know the following:
    \[
    \begin{array}{lll}
      ([[lbox A1]];[[lunbox A1]]) \times ([[lbox A2]];[[lunbox A2]])
      & = & \id_{[[A1]]} \times \id_{[[A2]]}\\
      & = & \id_{[[A1 x A2]]}\\
      & = & \id_{[[A]]}
    \end{array}
    \]
\end{itemize}
% subsection proof_of_lifted_retract (end)

\subsection{Proof of Lemma~\ref{lemma:S_is_faithful}}
\label{subsec:proof_of_S_is_faithful}
We must show that the function
\[ \S_{A,B} : \Hom{C}{A}{B} \mto \Hom{S}{\S A}{\S B} \]
is injective.

So suppose $f \in \Hom{C}{A}{B}$ and $g \in \Hom{C}{A}{B}$ such that
$\S f = \S g : \S A \mto \S B$.  Then we can easily see that:
\[
\begin{array}{lll}
  \S f & = & [[lunbox A]];f;[[lbox B]] \\
  & = & [[lunbox A]];g;[[lbox B]]\\
  & = & \S g\\
\end{array}
\]
But, we have the following equalities:
\[
\begin{array}{rll}
  [[lunbox A]];f;[[lbox B]] & = & [[lunbox A]];g;[[lbox B]]\\
  [[lbox A]];[[lunbox A]];f;[[lbox B]];[[lunbox B]] & = & [[lbox A]];[[lunbox A]];g;[[lbox B]];[[lunbox B]]\\
  \id_A;f;[[lbox B]];[[lunbox B]] & = & \id_A;g;[[lbox B]];[[lunbox B]]\\
  \id_A;f;\id_B & = & \id_A;g;\id_B\\
  f & = & g\\
\end{array}
\]
The previous equalities hold due to
Lemma~\ref{lemma:lifted_retract}.
% subsection proof_of_S_is_faithful (end)

\subsection{Proof of Type Consistency in the Model (Lemma~\ref{lemma:type_consistency_in_the_model})}
\label{subsec:proof_of_type_consistency_in_the_model}
This is a proof by induction on the form of $<<A ~ B>>$.
\begin{itemize}
\item[] Case.
  \[
  \GSiekdrulerefl{}
  \]
  Choose $c_1 = c_2 = \id_A : [[A --> A]]$.

\item[] Case.
  \[
  \GSiekdrulebox{}
  \]
  Choose $c_1 = [[Box A]] : [[A --> ?]]$ and $c_2 = [[Unbox A]] : [[? -> A]]$.

\item[] Case.
  \[
  \GSiekdruleunbox{}
  \]
  Choose $c_1 = [[Unbox A]] : [[? --> A]]$ and $c_2 = [[Box A]] : [[A -> ?]]$.
  
\item[] Case.
  \[
  \GSiekdrulearrow{}
  \]
  By the induction hypothesis there exists four casting morphisms
  $c'_1 : [[A1 --> A2]]$, $c'_2 : [[A2 --> A1]]$, $c'_3 : [[B1 --> B2]]$,
  and $c'_4 : [[B2 --> B1]]$.  Choose
  $c_1 = c'_2 \to c'_3 : [[(A1 -> B1) --> (A2 -> B2)]]$
  and
  $c_2 = c'_1 \to c'_4 : [[(A2 -> B2) --> (A1 -> B1)]]$.

\item[] Case.
  \[
  \GSiekdruleprod{}
  \]
  By the induction hypothesis there exists four casting morphisms
  $c'_1 : [[A1 --> A2]]$, $c'_2 : [[A2 --> A1]]$, $c'_3 : [[B1 --> B2]]$,
  and $c'_4 : [[B2 --> B1]]$.
  Choose
  $c_1 = c'_1 \times c'_3 : [[H(A1 x B1) --> H(A2 x B2)]]$
  and
  $c_2 = c'_2 \times c'_4 : [[H(A2 x B2) --> H(A1 x B1)]]$.
\end{itemize}
% subsection proof_of_type_consistency_in_the_model (end)

\subsection{Proof of Interpretation of Types Theorem~\ref{thm:interpretation_of_typing}}
\label{subsec:proof_of_interpretation_of_types}
  First, we show how to interpret the rules of $\GSTLC$, and then $\CGSTLC$.
  This is a proof by induction on $<<G |-S t : A>>$.

  \begin{itemize} 
  \item[] Case.
    \[
    \GSiekdruleSXXvar{}
    \]
    Suppose with out loss of generality that $[[ [| G |] ]] = [[A1]]
    \times \cdots \times [[Ai]] \times \cdots \times [[Aj]]$ where
    $[[Ai]] = [[A]]$.  We know that $j > 0$ or the assumed typing
    derivation would not hold.  Then take
    $[[ [| x |] ]] = \pi_i : [[ [| G |] --> A]]$.

  \item[] Case.
    \[
    \GSiekdruleSXXunit{}
    \]

    Take $[[ [| triv |] ]] = \diamond_{[[ [| G |] ]]} : [[ [| G |] -->
    1 ]]$ where $\diamond_{[[ [| G |] ]]}$ is the unique terminal
    arrow that exists because $\cat{C}$ is cartesian closed.
    
  \item[] Case.
    \[
    \GSiekdruleSXXzero{}
    \]

    Take $[[ [| 0 |] ]] = \diamond_{[[ [| G |] ]]};\z : [[ [| G |] --> Nat]]$
    where $\z : [[1 --> Nat]]$ exists because $\cat{C}$
    has a SNNO.

  \item[] Case.
    \[
    \GSiekdruleSXXsucc{}
    \]
    First, by the induction hypothesis there is a morphism $[[ [| t |] ]] : [[ [| G |] --> A]]$.
    Now we have two cases to consider, one when $<<A>> = <<?>>$ and one when $<<A>> = <<Nat>>$.
    Consider the former.  Then interpret
    $<< [| succ t |] >> = << [| t |] >>;[[ unbox Nat ]];[[succ]] : [[ [| G |] --> Nat]]$ where
    $[[succ]] : [[Nat --> Nat]]$ exists because $\cat{C}$ has a SNNO.  Similarly,
    when $<<A>> = <<Nat>>$, 
    $<< [| succ t |] >> = << [| t |] >>;[[succ]] : [[ [| G |] --> Nat]]$.
    
  \item[] Case.
    \[
    \GSiekdruleSXXpair{}
    \]
    By two applications of the induction hypothesis there are two morphisms
    $[[ [| t1 |] ]] : [[ [| G |] --> A]]$ and $[[ [| t2 |] ]] : [[ [| G |] --> B]]$.
    Then using the fact that $\cat{C}$ is cartesian we take
    $[[ [| (t1 , t2) |] ]] = \langle [[ [| t1 |] ]] , [[ [| t2 |] ]] \rangle : [[ [| G |] --> H(A x B)]]$.

  \item[] Case.
    \[
    \GSiekdruleSXXfst{}
    \]
    First, by the induction hypothesis there is a morphism $[[ [| t |] ]] : [[ [| G |] --> B]]$.
    Now we have two cases to consider, one when $<<B>> = <<?>>$ and one when $<<B>> = <<A1 x A2>>$
    for some types $<<A1>>$ and $<<A2>>$.  Consider the former.  We then know that it must
    be the case that $<<A1 x A2>> = <<? x ?>>$.  Thus, we obtain the following interpretation
    $[[ [| fst t |] ]] = [[ [| t |] ]];[[split (? x ?)]];\pi_1 : [[ [| G |] --> ?]]$.  Similarly,
    when $<<B>> = <<A1 x A2>>$, then
    $[[ [| fst t |] ]] = [[ [| t |] ]];\pi_1 : [[ [| G |] --> A1]]$.

  \item[] Case.
    \[
    \GSiekdruleSXXsnd{}
    \]
    First, by the induction hypothesis there is a morphism $[[ [| t |] ]] : [[ [| G |] --> B]]$.
    Now we have two cases to consider, one when $<<B>> = <<?>>$ and one when $<<B>> = <<A1 x A2>>$
    for some types $<<A1>>$ and $<<A2>>$.  Consider the former.  We then know that it must
    be the case that $<<A1 x A2>> = <<? x ?>>$.  Thus, we obtain the following interpretation
    $[[ [| snd t |] ]] = [[ [| t |] ]];[[split (? x ?)]];\pi_2 : [[ [| G |] --> ?]]$.  Similarly,
    when $<<B>> = <<A1 x A2>>$, then
    $[[ [| snd t |] ]] = [[ [| t |] ]];\pi_2 : [[ [| G |] --> A2]]$.

  \item[] Case.
    \[
    \GSiekdruleSXXlam{}
    \] 
    By the induction hypothesis there is a morphism $[[ [| t |] ]] :
    [[ H([| G |] x A) --> B]]$.  Then take $[[ [| \x : A.t |] ]] =
    \curry {[[ [| t |] ]]} : [[ [| G |] --> (A -> B)]]$, where
    $\mathsf{curry} : \Hom{C}{X \times Y}{Z} \mto \Hom{C}{X}{Y \to Z}$
    exists because $\cat{C}$ is closed.

  \item[] Case.
    \[
    \GSiekdruleSXXapp{}
    \]
    By the induction hypothesis there are two morphisms
    $[[ [| t1 |] ]] : [[ [| G |] --> C ]]$ and
    $[[ [| t2 |] ]] : [[ [| G |] --> A2 ]]$.  In addition, by assumption we know that
    $<<A2 ~ A1>>$, and hence, by type consistency in the model (Lemma~\ref{lemma:type_consistency_in_the_model})
     there are casting morphisms $c_1 : [[A2 --> A1]]$ and $c_2 : [[A1 --> A2]]$.  We have two cases to consider,
    one when $[[C]] = [[?]]$ and one when $[[C]] = [[A1 -> B1]]$.  Consider
    the former. Then we have the interpretation
    \[ [[ [| t1 t2 |] ]] = \langle [[ [| t1 |] ]];[[split (? -> ?)]], [[ [| t2 |] ]];c_1 \rangle : [[ [| G |] --> B1 ]]. \]
    Similarly, for the case when $[[C]] = [[A1 -> B1]]$ we have the interpretation
    \[ [[ [| t1 t2 |] ]] = \langle [[ [| t1 |] ]], [[ [| t2 |] ]];c_1 \rangle : [[ [| G |] --> B1 ]]. \]
    
  \end{itemize}

  Next we turn to $\CGSTLC$, but we do
not show every rule, because it corresponds to the simply typed
$\lambda$-calculus whose interpretation is similar to what we have
already shown above except without casting morphism, and so we only
show the case for the cast rule.
\begin{itemize}
\item[] Case.
  \[
  \GSiekdruleCXXcast{}
  \]
  By the induction hypothesis there is a morphism $[[ [| t |] ]] : [[ [| G |] --> A]]$,
  and by type consistency in the model (Lemma~\ref{lemma:type_consistency_in_the_model})
  there is a casting morphism $c_1 : [[A --> B]]$.  So take
  $ << [| t : {A} => {B} |] >> = [[ [| t |] ]];c_1 : [[ [| G |] --> B ]]$.
\end{itemize}
% subsection proof_of_interpretation_of_types (end)

\subsection{Proof of Interpretation of Evaluation (Theorem~\ref{thm:interpretation_of_evaluation})}
\label{subsec:proof_of_interpretation_of_evaluation}
This proof holds by induction on the form of $[[G |- t1 ~> t2 : A]]$.
We only show the cases for the casting rules, because the others are
well-known to hold within any cartesian closed category; see
\cite{Lambek:1980} or \cite{Crole:1994}.  We will routinely use
Theorem~\ref{thm:interpretation_of_typing} throughout this proof
without mention.

% subsection proof_of_interpretation_of_evaluation (end)

\subsection{Proof of Lemma~\ref{lemma:syntactic_box_and_unbox}}
\label{subsec:proof_of_lemma_syntactic_box_and_unbox}
First, we define the identify meta-function:
  \[
  \id_A := [[\x : A . x]]
  \]
  Then composition.  Suppose $[[G |- t1 : A -> B]]$ and $[[G |- t2 : B -> D]]$
  are two terms, then we define:
  \[
  [[t1 ; t2]] := [[\x : A . H(t2 (t1 x))]]
  \]
  It is easy to see that the following rule is admissible:
  \begin{center}
    \begin{math}
      $$\mprset{flushleft}
      \inferrule* [right=\text{comp}] {
        [[G |- t1 : A -> B]]
        \\
          [[G |- t2 : B -> D]]
      }{[[G |- t1;t2 : A -> D]]}
    \end{math}
  \end{center}
  The functor $- \times -$ requires two morphisms $[[G |- t1 : A -> D]]$ and
  $[[G |- t2 : B -> E]]$, and is defined as follows:
  \[
  [[t1]] \times [[t2]] := [[\ x : A x B.(t1 (fst x), t2 (snd x))]]
  \]
  The following rule is admissible:
  \begin{center}
    \begin{math}
      $$\mprset{flushleft}
      \inferrule* [right=\text{prod}] {
        [[G |- t1 : A -> D]]
        \\
        [[G |- t2 : B -> E]]
      }{[[G |- t1 XX t2 : H(A x B) -> H(D x E)]]}
    \end{math}
  \end{center}
  The functor $- \to -$ requires two morphisms $[[G |- t1 : D -> A]]$ and
  $[[G |- t2 : B -> E]]$, and is defined as follows:
  \[
  [[t1 -> t2]] := [[\ f : A -> B.\y : D.H(t2 (f (t1 y)))]]
  \]
  The following rule is admissible:
  \begin{center}
    \begin{math}
      $$\mprset{flushleft}
      \inferrule* [right=\text{prod}] {
        [[G |- t1 : D -> A]]
        \\
        [[G |- t2 : B -> E]]
      }{[[G |- t1 -> t2 : (A -> B) -> (D -> E)]]}
    \end{math}
  \end{center}
  At this point it is straightforward to carry out the definition of
  $[[Box A]]$ and $[[Unbox A]]$ using the definitions from the model.
  Showing the admissibility of the typing and reduction rules follows
  by induction on $[[A]]$.
  % subsection proof_of_lemma_syntactic_box_and_unbox (end)
% section proofs (end)

\input{proofs-gradual-guarantee-part-one-ott}

%%% Local Variables: ***
%%% mode:latex ***
%%% TeX-master: "main.tex"  ***
%%% End: ***
