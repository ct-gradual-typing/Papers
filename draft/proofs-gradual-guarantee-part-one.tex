\subsection{Proof of Left-to-Right Consistent Subtyping (Lemma~\ref{lemma:consistent-subtyping-1})}
\label{subsec:proof_of_left-to-right_consistent_subtyping_lemma:consistent-subtyping-1}
This is a proof by induction on $[[G |- A <~ B]]$.  We only show a
few of the most interesting cases.
\begin{itemize}
\item[] Case.\ \\ 
  \begin{center}
    \begin{math}
      $$\mprset{flushleft}
      \inferrule* [right=$\SGradydruleSXXBoxName{}$] {
        [[G |- A <~ SL]]
      }{[[G |- A <~ ?]]}
    \end{math}
  \end{center}
  In this case $[[B]] = [[?]]$.

  \noindent
  \textbf{Part i.} Choose $[[A']] = [[?]]$.

  \noindent
  \textbf{Part ii.} Choose $[[B']] = [[A]]$.

\item[] Case.\ \\ 
  \begin{center}
    \begin{math}
      $$\mprset{flushleft}
      \inferrule* [right=$\SGradydruleSXXUnboxName{}$] {
        [[G |- B <~ SL]]
      }{[[G |- ? <~ B]]}
    \end{math}
  \end{center}
  In this case $[[A]] = [[?]]$.

  \noindent
  \textbf{Part i.} Choose $[[A']] = [[B]]$.

  \noindent
  \textbf{Part ii.} Choose $[[B']] = [[?]]$.

\item[] Case.\ \\ 
  \begin{center}
    \begin{math}
      $$\mprset{flushleft}
      \inferrule* [right=$\SGradydruleSXXArrowName{}$] {
        [[G |- A2 <~ A1 && G |- B1 <~ B2]]
      }{[[G |- (A1 -> B1) <~ (A2 -> B2)]]}
    \end{math}
  \end{center}

  In this case $[[A]] = [[A1 -> B1]]$ and $[[B]] = [[A2 -> B2]]$.

  \noindent
  \textbf{Part i.} By part two of the induction hypothesis we know
  that $[[G |- A1' ~ A1]]$ and $<<G |- A2 <: A1'>>$, and by part one of the induction hypothesis
  $[[G |- B1 ~ B1']]$ and $<<G |- B1' <: B2>>$.  By symmetry of type consistency
  we may conclude that $[[G |- A1 ~ A1']]$ which along with $[[G |- B1 ~ B1']]$
  implies that $[[G |- (A1 -> B1) ~ (A1' -> B1')]]$, and by reapplying the rule
  we may conclude that $<<G |- (A1' -> B1') <: (A2 -> B2)>>$.

  \noindent
  \textbf{Part ii.} Similar to part one, except that we first
  applying part one of the induction hypothesis to the first
  premise, and then the second part to the second premise.
  
\end{itemize}

% subsection proof_of_left-to-right_consistent_subtyping_lemma:consistent-subtyping-1 (end)

\subsection{Proof of Congruence of Type Consistency Along Type Precision (Lemma~\ref{lemma:congruence_of_type_consistency_along_type_precision})}
\label{subsec:proof_of_congruence_of_type_consistency_along_type_precision_lemma:congruence_of_type_consistency_along_type_precision}
The proofs of both parts are similar, and so we only show a few
cases of the first part, but the omitted cases follow similarly.

\noindent
\textbf{Proof of part one.} This is a proof by induction on the form
of $[[A1 <= A1']]$.
\begin{itemize}
\item[] Case.\ \\ 
  \begin{center}
    \begin{math}
      $$\mprset{flushleft}
      \inferrule* [right=$\SGradydrulePXXUName{}$] {
        [[G |- A1 <~ SL]]
      }{[[A1 <= ?]]}
    \end{math}
  \end{center}
  In this case $[[A1']] = [[?]]$.  Suppose $[[G |- A1 ~ A2]]$.  Then
  it suffices to show that $[[G |- ? ~ A2]]$, and hence, we must show
  that $[[G |- A2 <~ SL]]$, but this follows by Lemma~\ref{lemma:simply_typed_consistent_types_are_subtypes}.

\item[] Case.\ \\ 
  \begin{center}
    \begin{math}
      $$\mprset{flushleft}
      \inferrule* [right=$\SGradydrulePXXarrowName{}$] {
        [[A <= C && B <= D]]
      }{[[(A -> B) <= (C -> D)]]}
    \end{math}
  \end{center}
  In this case $[[A1]] = [[A -> B]]$ and $[[A1']] = [[C -> D]]$.  Suppose
  $[[G |- A1 ~ A2]]$.  Then by inversion for type consistency it must
  be the case that either $[[A2]] = [[?]]$ and $[[G |- A1 <~ SL]]$, or
  $[[A2]] = [[A' -> B']]$, $[[G |- A ~ A']]$, and $[[G |- B ~ B']]$.
  
  Consider the former.  Then it suffices to show that $[[G |- A1' ~ ?]]$,
  and hence we must show that $[[G |- A1' <~ SL]]$, but this follows
  from Lemma~\ref{lemma:type_precision_preserves_SL}.

  Consider the case when $[[A2]] = [[A' -> B']]$, $[[G |- A ~ A']]$, and $[[G |- B ~ B']]$.
  It suffices to show that $[[G |- (C -> D) ~ (A' -> B')]]$ which follows from
  $[[G |- A' ~ C]]$ and $[[G |- D ~ B']]$.  Thus, it suffices to show that latter.
  By assumption we know the following:
  \begin{center}
    \begin{tabular}{lll}
      $[[A <= C]]$ and $[[G |- A ~ A']]$\\
      $[[B <= D]]$ and $[[G |- B ~ B']]$
    \end{tabular}
  \end{center}
  Now by two applications of the induction hypothesis we obtain $[[G |- C ~ A']]$
  and $[[G |- D ~ B']]$. By symmetry the former implies $[[G |- A ~ C]]$ and
  we obtain our result.
\end{itemize}  
% subsection proof_of_congruence_of_type_consistency_along_type_precision_lemma:congruence_of_type_consistency_along_type_precision (end)

\subsection{Proof of Congruence of Subtyping Along Type Precision (Lemma~\ref{lemma:congruence_of_subtyping_along_type_precision})}
\label{subsec:proof_of_congruence_of_subtyping_along_type_precision_lemma:congruence_of_subtyping_along_type_precision}
This is a proof by induction on the form of $[[A <= B]]$.  The proof
of part two follows similarly to part one.  We only give the most interesting cases.  All others follow similarly.

\noindent
\textbf{Proof of part one.}  We only show the most interesting case,
because all others are similar.
\begin{itemize}    

\item[] Case.\ \\ 
  \begin{center}
    \begin{math}
      $$\mprset{flushleft}
      \inferrule* [right=$\SGradydrulePXXarrowName{}$] {
        [[A1 <= A2 && B1 <= B2]]
      }{[[(A1 -> B1) <= (A2 -> B2)]]}
    \end{math}
  \end{center}
  In this case $[[A]] = [[A1 -> B1]]$ and $[[B]] = [[A2 -> B2]]$.
  Suppose $[[G |- A <~ C]]$.  Thus, by inversion for consistency subtyping
  it must be the case that $[[C]] = [[Top]]$ and $[[G |- A : *]]$, $[[C]] = [[?]]$ and $[[G |- A <~ SL]]$, or
  $[[C]] = [[A1' -> B1']]$, $[[G |- A1' <~ A1]]$, and $[[G |- B1 <~ B1']]$.  The case when $[[C]] = [[Top]]$
  is trivial, and the case when $[[C]] = [[?]]$ is similarly to the proof of
  Lemma~\ref{lemma:congruence_of_type_consistency_along_type_precision}.

  Consider the case when $[[C]] = [[A1' -> B1']]$, $[[G |- A1' <~ A1]]$, and $[[G |- B1 <~ B1']]$.
  By assumption we know the following:
  \begin{center}
    \begin{tabular}{lll}
      $[[A1 <= A2]]$ and $[[G |- A1' <~ A1]]$\\
      $[[B1 <= B2]]$ and $[[G |- B1 <~ B1']]$
    \end{tabular}
  \end{center}
  So by part two and one, respectively, of the induction hypothesis we know
  that $[[G |- A1' <~ A2]]$ and $[[G |- B2 <~ B1']]$.  Thus, by reapplying the rule above
  we may now conclude that $[[G |- (A2 -> B2) <~ (A1' -> B2')]]$ to obtain our result.
\end{itemize}
% subsection proof_of_congruence_of_subtyping_along_type_precision_lemma:congruence_of_subtyping_along_type_precision (end)  

\subsection{Proof of Gradual Guarantee Part One (Lemma~\ref{lemma:gradual_guarantee_part_one})}
\label{subsec:proof_of_gradual_guarantee_part_one_lemma:gradual_guarantee_part_one}
This is a proof by induction on $[[G |- t : A]]$.  We only show the
most interesting cases, because the others follow similarly.

\begin{itemize}
\item[] Case.\ \\ 
  \begin{center}
    \begin{math}
      $$\mprset{flushleft}
      \inferrule* [right=$\SGradydruleTXXvarPName{}$] {
        [[x : A elem G && G Ok]]
      }{[[G |- x : A]]}
    \end{math}
  \end{center}
  In this case $[[t]] = [[x]]$.  Suppose $[[t <= t']]$.  Then
  it must be the case that $[[t']] = [[x]]$.  If $[[x : A elem G]]$,
  then there is a type $[[A']]$ such that $[[x : A' elem G']]$ and
  $[[A <= A']]$.  Thus, choose $[[B]] = [[A']]$ and the result follows.

\item[] Case.\ \\ 
  \begin{center}
    \begin{math}
      $$\mprset{flushleft}
      \inferrule* [right=$\SGradydruleTXXsuccName{}$] {
        [[G |- t1 : A' && nat(A') = Nat]]
      }{[[G |- succ t1 : Nat]]}
    \end{math}
  \end{center}
  In this case $[[A]] = [[Nat]]$ and $[[t]] = [[succ t1]]$.  Suppose $[[t <= t']]$ and $[[G <= G']]$.
  Then by definition it must be the case that $[[t']] = [[succ t2]]$ where $[[t1 <= t2]]$.
  By the induction hypothesis $[[G' |- t2 : B']]$ where $[[A' <= B']]$.  Since $[[nat(A') = Nat]]$
  and $[[A' <= B']]$, then it must be the case that $[[nat(B') = Nat]]$ by Lemma~\ref{lemma:fun_type_pre}.
  At this point we obtain our result by choosing $[[B]] = [[Nat]]$, and reapplying the rule above.

\item[] Case.\ \\ 
  \begin{center}
    \begin{math}
      $$\mprset{flushleft}
      \inferrule* [right=$\SGradydruleTXXncaseName{}$] {
        [[(G |- t1 : C  && nat(C) = Nat) && G |- A1 ~ A]]
        \\\\
            [[(G |- t2 : A1 && G, x : Nat |- t3 : A2) && G |- A2 ~ A]]
      }{[[G |- case t1 of 0 -> t2, (succ x) -> t3 : A]]}
    \end{math}
  \end{center}
  In this case $[[t]] = [[case t1 of 0 -> t2, (succ x) -> t3]]$.  Suppose $[[t <= t']]$ and $[[G <= G']]$.  This
  implies that $[[t']] = [[case t1' of 0 -> t2', (succ x) -> t3']]$ such that
  $[[t1 <= t1']]$, $[[t2 <= t2']]$, and $[[t3 <= t3']]$.  Since $[[G <= G']]$ then $[[(G,x:Nat) <= (G',x:Nat)]]$.
  By the induction hypothesis we know the following:
  \begin{center}
    \begin{tabular}{lll}
      $[[G' |- t1' : C']]$ for $[[C <= C']]$\\
      $[[G' |- t2 : A1']]$ for $[[A1 <= A1']]$\\
      $[[G', x : Nat |- t3 : A2']]$ for $[[A2 <= A2']]$
    \end{tabular}
  \end{center}
  By assumption we know that $[[G |- A1 ~ A]]$, $[[G |- A2 ~ A]]$, and $[[G <= G']]$,
  hence, by Lemma~\ref{lemma:type_cons_ctx_pre} we know $[[G' |- A1 ~ A]]$ and $[[G' |- A2 ~ A]]$.  
  By the induction hypothesis we know that $[[A1 <= A1']]$ and $[[A2 <= A2']]$, so
  by using Lemma~\ref{lemma:type_cons_type_pre_2} we may obtain that
  $[[G' |- A1' ~ A]]$ and $[[G' |- A2' ~ A]]$.  At this point choose $[[B]] = [[A]]$
  and we obtain our result by reapplying the rule.
  
\item[] Case.\ \\ 
  \begin{center}
    \begin{math}
      $$\mprset{flushleft}
      \inferrule* [right=$\SGradydruleTXXconsName{}$] {
        [[((G |- t1 : A1 && G |- t2 : A2) && list(A2) = List A3) && G |- A1 ~ A3]]
      }{[[G |- t1 :: t2 : List A3]]}
    \end{math}
  \end{center}
  In this case $[[A]] = [[List A3]]$ and $[[t]] = [[t1 :: t2]]$.  Suppose $[[t <= t']]$ and $[[G <= G']]$.
  Then it must be the case that $[[t']] = [[t1' :: t2']]$ where $[[t1 <= t1']]$ and
  $[[t2 <= t2']]$.  Then by the induction hypothesis we know the following:
  \begin{center}
    \begin{tabular}{lll}
      $[[G' |- t1' : A1']]$ where $[[A1 <= A1']]$\\
      $[[G' |- t2' : A2']]$ where $[[A2 <= A2']]$\\
    \end{tabular}
  \end{center}
  By Lemma~\ref{lemma:fun_type_pre} $[[list(A2') = List A3']]$ where $[[A3 <= A3']]$.
  Now by Lemma~\ref{lemma:type_cons_ctx_pre} and Lemma~\ref{lemma:type_cons_type_pre_2} we know that
  $[[G' |- A1' ~ A3]]$, and by using the same lemma again, $[[G' |- A1' ~ A3']]$
  because $[[G' |- A3 ~ A1']]$ holds by symmetry.  Choose $[[B]] = [[List A3']]$
  and the result follows.

\item[] Case.\ \\ 
  \begin{center}
    \begin{math}
      $$\mprset{flushleft}
      \inferrule* [right=$\SGradydruleTXXpairName{}$] {
        [[G |- t1 : A1 && G |- t2 : A2]]
      }{[[G |- (t1,t2) : A1 x A2]]}
    \end{math}
  \end{center}
  In this case $[[A]] = [[A1 x A2]]$ and $[[t]] = [[(t1,t2)]]$. Suppose
  $[[t <= t']]$ and $[[G <= G']]$.  This implies that $[[t']] = [[(t1',t2')]]$ where
  $[[t1 <= t1']]$ and $[[t2 <= t2']]$.
  
  By the induction hypothesis we know:
  \begin{center}
    \begin{tabular}{lll}
      $[[G' |- t1' : A1']]$ and $[[A1 <= A1']]$\\
      $[[G' |- t2' : A2']]$ and $[[A2 <= A2']]$\\
    \end{tabular}
  \end{center}
  Then choose $[[B]] = [[A1' x A2']]$ and the result follows by reapplying
  the rule above and the fact that $[[(A1 x A2) <= (A1' x A2')]]$.  

\item[] Case.\ \\ 
  \begin{center}
    \begin{math}
      $$\mprset{flushleft}
      \inferrule* [right=$\SGradydruleTXXlamName{}$] {
        [[G, x : A1 |- t1 : B1]]
      }{[[G |- \x:A1.t1 : A1 -> B1]]}
    \end{math}
  \end{center}
  In this case $[[A1 -> B2]]$ and $[[t]] = [[\x:A1.t1]]$.  Suppose $[[t <= t']]$ and $[[G <= G']]$.
  Then it must be the case that $[[t']] = [[\x:A2.t2]]$, $[[t1 <= t2]]$, and $[[A1 <= A2]]$.
  Since $[[G <= G']]$ and $[[A1 <= A2]]$, then $[[(G, x : A1) <= (G', x : A2)]]$ by definition.
  Thus, by the induction hypothesis we know the following:
  \begin{center}
    \begin{tabular}{lll}
      $[[G', x : A2 |- t1' : B2]]$ and $[[B1 <= B2]]$
    \end{tabular}
  \end{center} 
  Choose $[[B]] = [[A2 -> B2]]$ and the result follows by reapplying the rule above
  and the fact that $[[(A1 -> B1) <= (A2 -> B2)]]$.

\item[] Case.\ \\ 
  \begin{center}
    \begin{math}
      $$\mprset{flushleft}
      \inferrule* [right=$\SGradydruleTXXtypeAppName{}$] {
        [[G |- t1 : Forall (X<:C0).C2 && G |- C1 <~ C0]]
      }{[[G |- [C1]t1 : [C1/X]C2]]}
    \end{math}
  \end{center}
  In this case $[[t]] = [[ [C1]t1]]$.  Suppose $[[t <= t']]$ and $[[G <= G']]$.
  Then it must be the case that $[[t']] = [[ [C1']t2]]$ such that $[[t1 <= t2]]$
  and $[[C1 <= C1']]$.  By the induction hypothesis:
  \begin{center}
    \begin{tabular}{lll}
      $[[G' |- t2 : C]]$ where $[[Forall (X<:C0).C2 <= C]]$
    \end{tabular}
  \end{center}
  Thus, it must be the case that $[[C]] = [[Forall (X <: C0).C2']]$ such that $[[C2 <= C2']]$.
  By assumption we know that $[[G |- C1 <~ C0]]$ and $[[C1 <= C1']]$, and thus,
  by Corollary~\ref{corollary:congruence_of_subtyping_along_type_precision} and Lemma~\ref{lemma:subtyping_context_precision}
  we know $[[G' |- C1' <~ C0]]$.  Thus, choose $[[B]] = [[C]]$, and the result follows by reapplying
  the rule above, and the fact that $[[A <= C]]$, because $[[C2 <= C2']]$.

\item[] Case.\ \\ 
  \begin{center}
    \begin{math}
      $$\mprset{flushleft}
      \inferrule* [right=$\SGradydruleTXXSubName{}$] {
        [[G |- t : A' && G |- A' <~ A]]
      }{[[G |- t : A]]}
    \end{math}
  \end{center}
  Suppose $[[t <= t']]$ and $[[G <= G']]$.
  By the induction hypothesis we know that $[[G' |- t' : A'']]$ for $[[A' <= A'']]$.
  We know $[[A'' <= A]]$ or $[[A <= A'']]$, because we know that $[[G |- A' <~ A]]$
  and $[[A' <= A'']]$.   Suppose $[[A'' <= A]]$, then by Corollary~\ref{corollary:type_precision_and_subtyping}
  $[[G' |- A'' <~ A]]$, and then by subsumption $[[G' |- t' : A]]$, hence, choose $[[B]] = [[A]]$
  and the result follows.  If $[[A <= A'']]$, then choose $[[B]] = [[A'']]$ and the result follows.

\item[] Case.\ \\ 
  \begin{center}
    \begin{math}
      $$\mprset{flushleft}
      \inferrule* [right=$\SGradydruleTXXappName{}$] {
        [[G |- t1 : C && fun(C) = A1 -> B1]]
        \\\\
            [[G |-t2 : A2 && G |- A2 ~ A1]]
      }{[[G |- t1 t2 : B1]]}
    \end{math}
  \end{center}
  In this case $[[A]] = [[B1]]$ and $[[t]] = [[t1 t2]]$.  Suppose $[[t <= t']]$
  and $[[G <= G']]$.  The former implies that $[[t']] = [[t1' t2']]$ such that
  $[[t1 <= t1']]$ and $[[t2 <= t2']]$.  By the induction hypothesis we know the
  following:
  \begin{center}
    \begin{tabular}{lll}
      $[[G' |- t1' : C']]$ for $[[C <= C']]$\\
      $[[G' |- t2' : A2']]$ for $[[A2 <= A2']]$\\
    \end{tabular}
  \end{center}
  We know by assumption that $[[G |- A2 ~ A1]]$ and hence $[[G' |- A2 ~ A1]]$
  because bounds on type variables are left unchanged by context precision.
  Since $[[C <= C']]$ and $[[fun(C) = A1 -> B1]]$, then $[[fun(C') = A1' -> B1']]$
  where $[[A1 <= A1']]$ and $[[B1 <= B1']]$ by Lemma~\ref{lemma:fun_type_pre}.
  Furthermore, we know $[[G' |- A2 ~ A1]]$ and $[[A2 <= A2']]$ and $[[A1 <= A1']]$, then
  we know $[[G' |- A2' ~ A1']]$ by Corollary~\ref{corollary:congruence_of_type_consistency_along_type_precision}.
  So choose $[[B]] = [[B1']]$. Then reapply the rule above and the result follows, because
  $[[B1 <= B1']]$.
\end{itemize}
% subsection proof_of_gradual_guarantee_part_one_lemma:gradual_guarantee_part_one (end)

\subsection{Proof of Type Preservation for Cast Insertion (Lemma~\ref{lemma:type_preservation_for_cast_insertion})}
\label{subsec:proof_of_type_preservation_for_cast_insertion}
The cast insertion algorithm is type directed and with respect to every term $[[t1]]$
it will produce a term $[[t2]]$ of the core language with the type $[[A]]$ --
this is straightforward to show by induction on the form of $[[G |- t1 : A]]$ making
use of typing for casting morphisms Lemma~\ref{lemma:typing_casting_morphisms} -- except in
the case of type application.  We only consider this case here.

This is a proof by induction on the form of $[[G |- t1 : A]]$.
Suppose the form of $[[G |- t1 : A]]$ is as follows:
\begin{center}
  \begin{math}
    $$\mprset{flushleft}
    \inferrule* [right=$\SGradydruleTXXtypeAppName{}$] {
      [[G |- t1' : Forall (X<:B1).B2 && G |- A1 <~ B1]]
    }{[[G |- [A1]t1' : [A1/X]B2]]}
  \end{math}
\end{center}
In this case $[[t1]] = [[ [A1]t1']]$ and $[[A]] = [[ [A1/X]B2]]$.
Cast insertion is syntax directed, and hence, inversion for it holds
trivially.  Thus, it must be the case that the form of $[[G |- t1 => t2 : B]]$
is as follows:
\begin{center}
  \begin{math}
    $$\mprset{flushleft}
    \inferrule* [right=$\SGradydruleciXXtypeAppName{}$] {
      [[(G |- t1' => t2' : Forall (X <: B1).B2' && G |- A1 ~ A2) && G |- A2 <: B1]]
    }{[[G |- ([A1]t1') => ([A2]t2') : [A2/X]B2']]}
  \end{math}
\end{center}
So $[[t2]] = [[ [A2]t2']]$ and $[[B]] = [[ [A2/X]B2']]$.  Since we know
$[[G |- t1' : Forall (X<:B1).B2]]$ and $[[G |- t1' => t2' : Forall (X <: B1).B2']]$ we can apply the induction hypothesis
to obtain $<<G |- t2' : Forall (X <: B1).B2'>>$ and $[[G |- (Forall (X <: B1).B2) ~ (Forall (X <: B1).B2')]]$, and thus,
$[[G, X <: B1 |- B2 ~ B2']]$ by inversion for type consistency.  If $[[G, X <: B1 |- B2 ~ B2']]$ holds, then
$[[G |- [A1/X]B2 ~ [A2/X]B2']]$ when $[[G |- A1 ~ A2]]$ by substitution for type consistency (Lemma~\ref{lemma:substitution_for_type_consistency}).
Since we know $<<G |- t2' : Forall (X <: B1).B2'>>$ by the induction hypothesis and $<<G |- A2 <: B1>>$ by assumption,
then we know $<<G |- [A2]t2' : [A2/X]B2'>>$ by applying the Core Grady typing rule $\CGradydruleTXXtypeAppName{}$.
% subsection proof_of_type_preservation_for_cast_insertion (end)

\subsection{Proof of Simulation of More Precise Programs (Lemma~\ref{lemma:simulation_of_more_precise_programs})}
\label{subsec:proof_of_simulation_of_more_precise_programs_lemma:simulation_of_more_precise_programs}
This is a proof by induction on $<<G |- t1 : A1>>$.  We only give the most interesting cases.  All others follow similarly.

\begin{itemize}
\item[] Case.\ \\ 
  \begin{center}
    \begin{math}
      $$\mprset{flushleft}
      \inferrule* [right=$\CGradydruleTXXsuccName{}$] {
        <<G |- t : Nat>>
      }{<<G |- succ t : Nat>>}
    \end{math}
  \end{center}
  In this case $<<t1>> = <<succ t>>$ and $<<A>> = <<Nat>>$.  Thus, $<<t1'>> = <<succ t'>>$ where $<<G |- t <= t'>>$
  by inversion of term precision.

  Suppose $<<t1 ~> t2>>$. Then $<<t2>> = <<succ t''>>$ and $<<t ~> t''>>$.  Then by the induction hypothesis
  we know that there is some $<<t'''>>$ such that $<<t' ~> t'''>>$ and $<<G |- t'' <= t'''>>$.  Choose
  $<<t2'>> = <<succ t'''>>$ and the result follows.    

\item[] Case.\ \\ 
  \begin{center}
    \begin{math}
      $$\mprset{flushleft}
      \inferrule* [right=$\CGradydruleTXXncaseName{}$] {
        <<G |- t : Nat>>
        \\\\
        <<G |- t3 : A && G, x : Nat |- t4 : A>>
      }{<<G |- case t : Nat of 0 -> t3, (succ x) -> t4 : A>>}
    \end{math}
  \end{center}
  In this case $<<t1>> = <<case t : Nat of 0 -> t3, (succ x) -> t4>>$.  This implies
  that $<<t1'>> = <<case t' : Nat of 0 -> t3', (succ x) -> t4'>>$.  Furthermore,
  suppose $<<G |- t1 <= t1'>>$.  Then by inversion of term precision we know that
  $<<G |- t <= t'>>$, $<<G |- t3 <= t3'>>$, and $<<G, x : Nat |- t4 <= t4'>>$.  

  We case split over $<<t1 ~> t2>>$.
  \begin{itemize}
  \item[] Case.  Suppose $<<t>> = <<0>>$ and $<<t2>> = <<t3>>$.  Since $<<G |- t1 <= t1'>>$ we know that
    it must be the case that $<<t'>> = <<0>>$ and $<<t1' ~> t3'>>$.  Thus, choose $<<t2'>> = <<t3'>>$ and the result follows.
    
  \item[] Case.  Suppose $<<t>> = <<succ t''>>$ and $<<t2>> = << [t''/x]t4>>$.  Since $<<G |- t1 <= t1'>>$
    we know that $<<t'>> = <<succ t'''>>$ and $<<G |- t'' <= t'''>>$ by inversion of term precision. In addition,
    $<<t'1 ~> [t'''/x]t'4>>$. Choose $<<t2>> = << [t'''/x]t'4>>$.  Then it suffices to show that
    $<<G |- [t''/x]t4 <= [t'''/x]t'4>>$ by substitution for term precision (Lemma~\ref{lemma:substitution_for_term_precision}).            
    
  \item[] Case.  Suppose a congruence rule was used.  Then $<<t2>> = <<case t'' : Nat of 0 -> t3'', (succ x) -> t4''>>$.
    This case will follow straightforwardly by induction and a case split over which congruence rule was used.
    
  \end{itemize}   

\item[] Case.\ \\ 
  \begin{center}
    \begin{math}
      $$\mprset{flushleft}
      \inferrule* [right=$\CGradydruleTXXfstName{}$] {
        <<G |- t : A x B>>
      }{<<G |- fst t : A>>}
    \end{math}
  \end{center}
  In this case $<<t1>> = <<fst t>>$.  Suppose $<<G |- t1 <= t1'>>$ and $<<G |- t1' : A'>>$.  Then
  it must be the case that $<<t'1>> = <<fst t'>>$, $<<G |- t <= t'>>$ by inversion for term
  precision.

  We case split over $<<t1 ~> t2>>$.
  \begin{itemize}
  \item[] Case. Suppose $<<t>> = <<(t'3,t''3)>>$ and $<<t2>> = <<t'3>>$.  The former implies
    that $<<t'>> = <<(t'4,t''4)>>$ because $<<G |- t1 <= t'1>>$ by assumption.  In addition,
    by inversion for term precision we know $<<G |- t'3 <= t'4>>$ and $<<G |- t''3 <= t''4>>$.
    Thus, $<<t'1 ~> t'4>>$. Thus, choose$<<t2'>> = <<t'4>>$ and the result follows.

  \item[] Case. Suppose a congruence rule was used.  Then $<<t2>> = <<fst t''>>$.
    This case will follow straightforwardly by induction and a case split over which congruence rule was used.
  \end{itemize}
  
\item[] Case.\ \\ 
  \begin{center}
    \begin{math}
      $$\mprset{flushleft}
      \inferrule* [right=$\CGradydruleTXXlamName{}$] {
        <<G, x : A1 |- t : A2>>
      }{<<G |- \x:A1.t : A1 -> A2>>}
    \end{math}
  \end{center}
  In this case $<<t1>> = <<\x:A1.t>>$. Suppose $<<G |- t1 <= t1'>>$ and $<<G |- t1' : A'>>$.
  This implies that $<<t'1>> = <<\x:A'1.t'>>$.  The reduction relation does not reduce under
  $\lambda$-expressions.  Hence, $<<t2>> = <<t1>>$, and thus, choose $<<t'2>> = <<t'1>>$, and
  the case trivially follows.
  
  
\item[] Case.\ \\ 
  \begin{center}
    \begin{math}
      $$\mprset{flushleft}
      \inferrule* [right=$\CGradydruleTXXappName{}$] {
        <<G |- t3 : A1 -> A2 && G |- t4 : A1>>
      }{<<G |- t3 t4 : A2>>}
    \end{math}
  \end{center}
  In this case $<<t1>> = <<t3 t4>>$.  Suppose $<<G |- t1 <= t1'>>$ and $<<G |- t1' : A'>>$.
  This implies that $<<t'1>> = <<t'3 t'4>>$, $<<G |- t3 <= t'3>>$, and $<<G |- t4 <= t'4>>$
  by inversion for term precision.

  We case split on the from of $<<t1 ~> t2>>$.
  \begin{itemize}
  \item[] Case.  Suppose $<<t3>> = <<\x:A1.t5>>$ and $<<t2>> = << [t4/x]t5>>$.
    Then $<<t'3>> = <<\x:A1'.t'5>>$, because $<<G |- t3 <= t'3>>$. Choose $<<t'2>> = << [t'4/x]t'5>>$
    and it is easy to see that $<<t'1 ~> [t'4/x]t'4>>$.
    We know that $<<G, x : A2' |- t5 <= t'5>>$ (by inversion for term precision) and $<<G |- t4 <= t'4>>$, and hence,
    by Lemma~\ref{lemma:substitution_for_term_precision} we know that
    $<<G |- [t4/x]t5 <= [t'4/x]t'5>>$, and we obtain our result.

  \item[] Case.  Suppose $<<t3>> = <<unbox A>>$, $<<t4>> = <<box A t5>>$, and $<<t2>> = <<t5>>$.
    Then $<<t'3>> = <<unbox A>>$ and $<<t'4>> = <<box A t'5>>$.  Choose $<<t'2>> = <<t'5>>$ and
    it is easy to see that $<<t'1 ~> t'5>>$.
    By inversion for term precision we know that $<<G |- t5 <= t'5>>$ because we know
    that $<<G |- t4 <= t'4>>$, and we obtain our result.

  \item[] Case.  Suppose $<<t3>> = <<unbox A>>$, $<<t4>> = <<box B t5>>$, $<<A != B>>$,
    and $<<t2>> = <<error B>>$.  Then $<<t'3>> = <<unbox A>>$ and $<<t'4>> = <<box B t'5>>$.
    Choose $<<t'2>> = <<error B>>$ and it is easy to see that $<<t'1 ~> t'5>>$.  Finally,
    we can see that $<<G |- t2 <= t'2>>$ by reflexivity.
    
  \item[] Case.  Suppose $<<t3>> = <<split U>>$, $<<t4>> = <<squash U t5>>$, and $<<t2>> = <<t5>>$.
    Similar to the case for boxing and unboxing.
    
  \item[] Case.  Suppose $<<t3>> = <<split U1>>$, $<<t4>> = <<squash U2 t5>>$, $<<U1 != U2>>$, and $<<t2>> = <<t5>>$.
    Similar to the case for boxing and unboxing.

  \item[] Case. Suppose a congruence rule was used.  Then $<<t2>> = <<t'5 t'6>>$.
    This case will follow straightforwardly by induction and a case split over which congruence rule was used.
  \end{itemize}


\item[] Case.\ \\ 
  \begin{center}
    \begin{math}
      $$\mprset{flushleft}
      \inferrule* [right=$\CGradydruleTXXtypeAppName{}$] {
        <<G |- t : Forall (X<:A2).A3 && G |- A1 <: A2>>
      }{<<G |- [A1]t : [A1/X]A3>>}
    \end{math}
  \end{center}
  In this case $<<t1>> = << [A1]t>>$ and $<<A>> = << [A1/X]A3>>$.  Suppose $<<G |- t1 <= t1'>>$ and $<<G |- t1' : A'>>$.
  This implies that $<<t'1>> = << [A1']t'>>$, $<<G |- t <= t'>>$, and $<<A1 <= A1'>>$
  by inversion for term precision.  

  We case split over the form of $<<t1 ~> t2>>$.
  \begin{itemize}
  \item[] Case. Suppose $<<t>> = <<Lam X <: A2.t3>>$ and $<<t2>> = << [A1/X]t3>>$.  Then we know
    that $<<t'>> = <<Lam X <: A2.t'3>>$ and by inversion for term precision that
    $<<G, X <: A2 |- t3 <= t3'>>$, because we know $<<G |- t <= t'>>$, and by substitution for term precision
    we know that $<<G |- [A1/X]t3 <= [A1'/X]t3'>>$ by substitution for term precision
    (Lemma~\ref{lemma:substitution_for_term_precision}), because we know that $<<A1 <= A1'>>$.  Choose
    $<<t'2>> = << [A1'/X]t3'>>$ and the result follows, because $<<t'1 ~> t'2>>$.

  \item[] Case. Suppose a congruence rule was used.  Then $<<t2>> = << [A1]t''>>$.
    This case will follow straightforwardly by induction and a case split over which congruence rule was used.
  \end{itemize}

\item[] Case.\ \\ 
  \begin{center}
    \begin{math}
      $$\mprset{flushleft}
      \inferrule* [right=$\CGradydruleTXXSubName{}$] {
        <<G |- t : A1 && G |- A1 <: A2>>
      }{<<G |- t : A2>>}
    \end{math}
  \end{center}
  In this case $<<t1>> = <<t>>$ and $<<A>> = <<A2>>$.  Suppose $<<G |- t1 <= t1'>>$ and $<<G |- t1' : A'>>$.
  Assume $<<t1 ~> t2>>$.  Then by the induction hypothesis there is a $<<t2'>>$ such that
  $<<t1' ~>* t2'>>$ and $<<G |- t2 <= t2'>>$, thus, we obtain our result.

\end{itemize}
% subsection proof_of_simulation_of_more_precise_programs_lemma:simulation_of_more_precise_programs (end)
