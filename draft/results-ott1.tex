\begin{lemma}[Inclusion of Bounded System F]
  \label{lemma:F-inclusion}
  Suppose $t$ is fully annotated and does not contain any applications
  of $ \mathsf{box} $ or $ \mathsf{unbox} $, and $\SGradynt{A}$ is static.  Then
  \begin{itemize}
  \item[i.] $ \Gamma  \vdash_F  \SGradynt{t}  :  \SGradynt{A} $ if and only if $\, \Gamma  \vdash  \SGradynt{t}  :  \SGradynt{A} $, and 
  \item[ii.] $ \SGradynt{t}  \rightsquigarrow^*_F  \SGradynt{t'} $ if and only if $ \SGradynt{t}  \rightsquigarrow^*  \SGradynt{t'} $.
  \end{itemize}
\end{lemma}
\begin{proof}
  We give proof sketches for both parts.  The interesting cases are
  the right-to-left directions of each part.  If we simply remove all
  rules mentioning the unknown type $\SGradysym{\mbox{?}}$ and the type consistency
  relation, and then remove $ \mathsf{box} $, $ \mathsf{unbox} $, and $\SGradysym{\mbox{?}}$ from
  the syntax of Surface Grady, then what we are left with is bounded
  system F.  Since $\SGradynt{t}$ is fully annotated and $\SGradynt{A}$ is static,
  then $ \Gamma  \vdash  \SGradynt{t}  :  \SGradynt{A} $ will hold within this fragment.

  Moving on to part two, first, we know that $\SGradynt{t}$ does not contain
  any occurrence of $ \mathsf{box} $ or $ \mathsf{unbox} $ and is fully annotated.
  This implies that $\SGradynt{t}$ lives within the bounded system F fragment
  of Surface Grady. Thus, before evaluation of $\SGradynt{t}$ Surface Grady
  will apply the cast insertion algorithm which will at most insert
  applications of the identity function into $\SGradynt{t}$ producing a term
  $\widehat{\SGradynt{t}}$, but then after potentially more than one step of
  evaluation within Core Grady, those applications of the identity
  function will be $\beta$-reduced away resulting in $\widehat{\SGradynt{t}}
  \rightsquigarrow^* \SGradynt{t} \rightsquigarrow^* \SGradynt{t'}$.  In addition,
  since $\SGradynt{t}$ in Surface Grady is the exact same program as $\SGradynt{t}$
  in bounded system F, then we know $ \SGradynt{t}  \rightsquigarrow^*_F  \SGradynt{t'} $ will hold.
\end{proof}

\begin{lemma}[Inclusion of DTLC]
  \label{lemma:inclusion_of_dtlc}
  Suppose $\SGradynt{t}$ is a closed term of DTLC. Then
  \begin{itemize}
  \item[i.] $  \cdot   \vdash   \lceil  \SGradynt{t}  \rceil   :  \SGradysym{\mbox{?}} $, and
  \item[ii.] $ \SGradynt{t}  \rightsquigarrow^*_{DTLC}  \SGradynt{t'} $ if and only if $  \lceil  \SGradynt{t}  \rceil   \rightsquigarrow^*   \lceil  \SGradynt{t'}  \rceil  $.
  \end{itemize}
\end{lemma}
\begin{proof}
  In this case DTLC is embedded into the simply typed fragment of
  Grady, and hence, this proof is the same result proven by
  \cite{Siek:2006}, and \cite{Siek:2015}.
\end{proof}

\renewcommand{\SGradydrulePXXUName}{\SGradysym{\mbox{?}}}
\renewcommand{\SGradydrulePXXreflName}{\text{refl}}
\renewcommand{\SGradydrulePXXarrowName}{\to}
\renewcommand{\SGradydrulePXXprodName}{\times}
\renewcommand{\SGradydrulePXXlistName}{\mathsf{List}}
\renewcommand{\SGradydrulePXXforallName}{\forall}
\begin{figure}
  \begin{mdframed}
    \begin{mathpar}
      \SGradydrulePXXU{} \and
      \SGradydrulePXXrefl{} \and
      \SGradydrulePXXarrow{} \and
      \SGradydrulePXXprod{} \and
      \SGradydrulePXXlist{} \and
      \SGradydrulePXXforall{}      
    \end{mathpar}
  \end{mdframed}
  \caption{Type Precision}
  \label{fig:type-pre}
\end{figure}

\begin{lemma}[Subtyping and Precision]
  \label{lemma:subtyping_and_precision}
  Suppose $ \Gamma  \vdash  \SGradynt{A}  <\hspace{-2px}\colon  \SGradynt{B} $. Then
  \begin{itemize}
  \item[i.] if $ \SGradynt{A}  \sqsubseteq  \SGradynt{A'} $, then $ \Gamma  \vdash  \SGradynt{A'}  <\hspace{-2px}\colon  \SGradynt{B'} $ for $ \SGradynt{B}  \sqsubseteq  \SGradynt{B'} $, and
  \item[i.] if $ \SGradynt{B}  \sqsubseteq  \SGradynt{B'} $, then $ \Gamma  \vdash  \SGradynt{A'}  <\hspace{-2px}\colon  \SGradynt{B'} $ for $ \SGradynt{A}  \sqsubseteq  \SGradynt{A'} $.
  \end{itemize}
\end{lemma}
\begin{proof}
  This proof holds by induction on $ \Gamma  \vdash  \SGradynt{A}  <\hspace{-2px}\colon  \SGradynt{B} $.
  \begin{itemize}
  \item[] Case.\ \\ 
    \begin{center}
      \begin{math}
        $$\mprset{flushleft}
        \inferrule* [right=$\SGradydruleSXXReflName{}$] {
           \Gamma \,\text{Ok} 
        }{ \Gamma  \vdash  \SGradynt{A}  <\hspace{-2px}\colon  \SGradynt{A} }
      \end{math}
    \end{center}        
    \textbf{Proof of part i.} Choose $\SGradynt{B'} = \SGradynt{A'}$.
    
    \noindent
    \textbf{Proof of Part ii.} Choose $\SGradynt{A'} = \SGradynt{B'}$.

  \item[] Case.\ \\ 
    \begin{center}
      \begin{math}
        $$\mprset{flushleft}
        \inferrule* [right=$\SGradydruleSXXTransName{}$] {
            \Gamma  \vdash  \SGradynt{A}  <\hspace{-2px}\colon  \SGradynt{B}   \quad   \Gamma  \vdash  \SGradynt{B}  <\hspace{-2px}\colon  \SGradynt{C}  
        }{ \Gamma  \vdash  \SGradynt{A}  <\hspace{-2px}\colon  \SGradynt{C} }
      \end{math}
    \end{center}
    \textbf{Proof of part i.} Suppose $ \SGradynt{A}  \sqsubseteq  \SGradynt{A'} $.  Then by part one of applying
    the induction hypothesis to $ \Gamma  \vdash  \SGradynt{A}  <\hspace{-2px}\colon  \SGradynt{B} $ we know that $ \Gamma  \vdash  \SGradynt{A'}  <\hspace{-2px}\colon  \SGradynt{B'} $ for
    $ \SGradynt{B}  \sqsubseteq  \SGradynt{B'} $, and by part one of applying the induction hypothesis to $ \Gamma  \vdash  \SGradynt{B}  <\hspace{-2px}\colon  \SGradynt{C} $ and $ \SGradynt{B}  \sqsubseteq  \SGradynt{B'} $
    we know $ \Gamma  \vdash  \SGradynt{B'}  <\hspace{-2px}\colon  \SGradynt{C'} $ for $ \SGradynt{C}  \sqsubseteq  \SGradynt{C'} $.  Thus, by transitivity of subtyping 
    we know $ \Gamma  \vdash  \SGradynt{A'}  <\hspace{-2px}\colon  \SGradynt{C'} $ and $ \SGradynt{C}  \sqsubseteq  \SGradynt{C'} $.
    \noindent
    \textbf{Proof of Part ii.}  Suppose $ \SGradynt{B}  \sqsubseteq  \SGradynt{B'} $.  Then by applying part two of 
    the induction hypothesis to $ \Gamma  \vdash  \SGradynt{A}  <\hspace{-2px}\colon  \SGradynt{B} $ we know that $ \Gamma  \vdash  \SGradynt{A'}  <\hspace{-2px}\colon  \SGradynt{B'} $ for
    $ \SGradynt{A}  \sqsubseteq  \SGradynt{A'} $, and by applying part one of the induction hypothesis to $ \Gamma  \vdash  \SGradynt{B}  <\hspace{-2px}\colon  \SGradynt{C} $ and $ \SGradynt{B}  \sqsubseteq  \SGradynt{B'} $
    we know $ \Gamma  \vdash  \SGradynt{B'}  <\hspace{-2px}\colon  \SGradynt{C'} $ for $ \SGradynt{C}  \sqsubseteq  \SGradynt{C'} $.  Thus, by transitivity of subtyping 
    we know $ \Gamma  \vdash  \SGradynt{A'}  <\hspace{-2px}\colon  \SGradynt{C'} $ and $ \SGradynt{C}  \sqsubseteq  \SGradynt{C'} $.

  \item[] Case.\ \\ 
    \begin{center}
      \begin{math}
        $$\mprset{flushleft}
        \inferrule* [right=$\SGradydruleSXXVarName{}$] {
            \SGradymv{X}  <\hspace{-2px}\colon  \SGradynt{B}  \in  \Gamma   \quad   \Gamma \,\text{Ok}  
        }{ \Gamma  \vdash  \SGradymv{X}  <\hspace{-2px}\colon  \SGradynt{B} }
      \end{math}
    \end{center}
    \textbf{Proof of part i.}  We know that $\SGradynt{A} = \SGradymv{X}$ and $ \SGradynt{A}  \sqsubseteq  \SGradynt{A'} $.  But, this implies that
    $\SGradynt{A'} = \SGradymv{X}$ or $\SGradynt{A'} = \SGradysym{\mbox{?}}$.  Choose $\SGradynt{B'} = \SGradynt{B}$ in the former case, and $\SGradynt{B'} = \SGradysym{\mbox{?}}$ in the latter.

    \noindent
    \textbf{Proof of Part ii.}

  \item[] Case.\ \\ 
    \begin{center}
      \begin{math}
        $$\mprset{flushleft}
        \inferrule* [right=$\SGradydruleSXXUName{}$] {
           \Gamma \,\text{Ok} 
        }{ \Gamma  \vdash  \SGradynt{A}  <\hspace{-2px}\colon  \SGradysym{\mbox{?}} }
      \end{math}
    \end{center}    
    \textbf{Proof of part i.}
    
    \noindent
    \textbf{Proof of Part ii.}

  \item[] Case.\ \\ 
    \begin{center}
      \begin{math}
        $$\mprset{flushleft}
        \inferrule* [right=$\SGradydruleSXXNatSName{}$] {
           \Gamma \,\text{Ok} 
        }{ \Gamma  \vdash   \mathsf{Nat}   <\hspace{-2px}\colon   \mathsf{SL}  }
      \end{math}
    \end{center}
    \textbf{Proof of part i.}  

    \noindent
    \textbf{Proof of Part ii.}

  \item[] Case.\ \\ 
    \begin{center}
      \begin{math}
        $$\mprset{flushleft}
        \inferrule* [right=$\SGradydruleSXXUnitSName{}$] {
           \Gamma \,\text{Ok} 
        }{ \Gamma  \vdash   \mathsf{Unit}   <\hspace{-2px}\colon   \mathsf{SL}  }
      \end{math}
    \end{center}
    \textbf{Proof of part i.}  

    \noindent
    \textbf{Proof of Part ii.}

  \item[] Case.\ \\ 
    \begin{center}
      \begin{math}
        $$\mprset{flushleft}
        \inferrule* [right=$\SGradydruleSXXListSName{}$] {
           \Gamma  \vdash  \SGradynt{A}  <\hspace{-2px}\colon   \mathsf{SL}  
        }{ \Gamma  \vdash   \mathsf{List}\, \SGradynt{A}   <\hspace{-2px}\colon   \mathsf{SL}  }
      \end{math}
    \end{center}
    \textbf{Proof of part i.}  

    \noindent
    \textbf{Proof of Part ii.}

  \item[] Case.\ \\ 
    \begin{center}
      \begin{math}
        $$\mprset{flushleft}
        \inferrule* [right=$\SGradydruleSXXArrowSName{}$] {
            \Gamma  \vdash  \SGradynt{A}  <\hspace{-2px}\colon   \mathsf{SL}    \quad   \Gamma  \vdash  \SGradynt{B}  <\hspace{-2px}\colon   \mathsf{SL}   
        }{ \Gamma  \vdash  \SGradysym{(}  \SGradynt{A}  \rightarrow  \SGradynt{B}  \SGradysym{)}  <\hspace{-2px}\colon   \mathsf{SL}  }
      \end{math}
    \end{center}
    \textbf{Proof of part i.}  

    \noindent
    \textbf{Proof of Part ii.}

  \item[] Case.\ \\ 
    \begin{center}
      \begin{math}
        $$\mprset{flushleft}
        \inferrule* [right=$\SGradydruleSXXProdSName{}$] {
            \Gamma  \vdash  \SGradynt{A}  <\hspace{-2px}\colon   \mathsf{SL}    \quad   \Gamma  \vdash  \SGradynt{B}  <\hspace{-2px}\colon   \mathsf{SL}   
        }{ \Gamma  \vdash  \SGradysym{(}   \SGradynt{A}  \times  \SGradynt{B}   \SGradysym{)}  <\hspace{-2px}\colon   \mathsf{SL}  }
      \end{math}
    \end{center}
    \textbf{Proof of part i.}  

    \noindent
    \textbf{Proof of Part ii.}

  \item[] Case.\ \\ 
    \begin{center}
      \begin{math}
        $$\mprset{flushleft}
        \inferrule* [right=$\SGradydruleSXXListName{}$] {
           \Gamma  \vdash  \SGradynt{A}  <\hspace{-2px}\colon  \SGradynt{B} 
        }{ \Gamma  \vdash  \SGradysym{(}   \mathsf{List}\, \SGradynt{A}   \SGradysym{)}  <\hspace{-2px}\colon  \SGradysym{(}   \mathsf{List}\, \SGradynt{B}   \SGradysym{)} }
      \end{math}
    \end{center}
    \textbf{Proof of part i.}  

    \noindent
    \textbf{Proof of Part ii.}

  \item[] Case.\ \\ 
    \begin{center}
      \begin{math}
        $$\mprset{flushleft}
        \inferrule* [right=$\SGradydruleSXXProdName{}$] {
            \Gamma  \vdash  \SGradynt{A_{{\mathrm{1}}}}  <\hspace{-2px}\colon  \SGradynt{A_{{\mathrm{2}}}}   \quad   \Gamma  \vdash  \SGradynt{B_{{\mathrm{1}}}}  <\hspace{-2px}\colon  \SGradynt{B_{{\mathrm{2}}}}  
        }{ \Gamma  \vdash  \SGradysym{(}   \SGradynt{A_{{\mathrm{1}}}}  \times  \SGradynt{B_{{\mathrm{1}}}}   \SGradysym{)}  <\hspace{-2px}\colon  \SGradysym{(}   \SGradynt{A_{{\mathrm{2}}}}  \times  \SGradynt{B_{{\mathrm{2}}}}   \SGradysym{)} }
      \end{math}
    \end{center}
    \textbf{Proof of part i.}  

    \noindent
    \textbf{Proof of Part ii.}

  \item[] Case.\ \\ 
    \begin{center}
      \begin{math}
        $$\mprset{flushleft}
        \inferrule* [right=$\SGradydruleSXXArrowName{}$] {
            \Gamma  \vdash  \SGradynt{A_{{\mathrm{2}}}}  <\hspace{-2px}\colon  \SGradynt{A_{{\mathrm{1}}}}   \quad   \Gamma  \vdash  \SGradynt{B_{{\mathrm{1}}}}  <\hspace{-2px}\colon  \SGradynt{B_{{\mathrm{2}}}}  
        }{ \Gamma  \vdash  \SGradysym{(}  \SGradynt{A_{{\mathrm{1}}}}  \rightarrow  \SGradynt{B_{{\mathrm{1}}}}  \SGradysym{)}  <\hspace{-2px}\colon  \SGradysym{(}  \SGradynt{A_{{\mathrm{2}}}}  \rightarrow  \SGradynt{B_{{\mathrm{2}}}}  \SGradysym{)} }
      \end{math}
    \end{center}
    \textbf{Proof of part i.}  

    \noindent
    \textbf{Proof of Part ii.}

  \item[] Case.\ \\ 
    \begin{center}
      \begin{math}
        $$\mprset{flushleft}
        \inferrule* [right=$\SGradydruleSXXForallName{}$] {
            \Gamma , \SGradymv{X}  <\hspace{-2px}\colon  \SGradynt{A}   \vdash  \SGradynt{B_{{\mathrm{1}}}}  <\hspace{-2px}\colon  \SGradynt{B_{{\mathrm{2}}}} 
        }{ \Gamma  \vdash  \SGradysym{(}   \forall ( \SGradymv{X}  <\hspace{-2px}\colon  \SGradynt{A} ).  \SGradynt{B_{{\mathrm{1}}}}   \SGradysym{)}  <\hspace{-2px}\colon  \SGradysym{(}   \forall ( \SGradymv{X}  <\hspace{-2px}\colon  \SGradynt{A} ).  \SGradynt{B_{{\mathrm{2}}}}   \SGradysym{)} }
      \end{math}
    \end{center}
    \textbf{Proof of part i.}  

    \noindent
    \textbf{Proof of Part ii.}

  \end{itemize}
\end{proof}

\begin{theorem}[Gradual Guarantee]
  \label{thm:gradual_guarantee}
  Suppose $  \cdot   \vdash  \SGradynt{t}  :  \SGradynt{A} $ and $ \SGradynt{t}  \sqsubseteq  \SGradynt{t'} $.  Then
  \begin{itemize}
  \item[i.] $  \cdot   \vdash  \SGradynt{t'}  :  \SGradynt{B} $ and $ \SGradynt{A}  \sqsubseteq  \SGradynt{B} $,
  \item[ii.] if $ \SGradynt{t}  \rightsquigarrow^*  \SGradynt{v} $, then $ \SGradynt{t'}  \rightsquigarrow^*  \SGradynt{v'} $ and $ \SGradynt{v}  \sqsubseteq  \SGradynt{v'} $,
  \item[iii.] if $ \SGradynt{t}  \uparrow $, then $ \SGradynt{t'}  \uparrow $,
  \item[iv.] if $ \SGradynt{t'}  \rightsquigarrow^*  \SGradynt{v'} $, then $ \SGradynt{t}  \rightsquigarrow^*  \SGradynt{v} $ where $ \SGradynt{v}  \sqsubseteq  \SGradynt{v'} $, or $[[t ~>* error]]$, and
  \item [v.] if $ \SGradynt{t'}  \uparrow $, then $ \SGradynt{t}  \uparrow $ or $[[t ~>* error]]$.
  \end{itemize}
\end{theorem}
\begin{proof}
  \textbf{Part i}. This case holds by induction on $  \cdot   \vdash  \SGradynt{t}  :  \SGradynt{A} $,
  and is fairly straightforward.  We only consider the two non-trivial
  cases.
  \begin{itemize}
  \item[] Case.
    \[
    \SGradydruleTXXapp{}
    \]
    
  \item[] Case.
    \[
    \SGradydruleTXXSub{}
    \]
    
  \end{itemize}

\end{proof}
