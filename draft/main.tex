%% Submission instructions:
%%
%% Full paper of at most 12 pages (6 pages for an Experience Report),
%% in standard SIGPLAN conference format, including figures but
%% ***excluding bibliography***.
%% 
%% The deadlines will be strictly enforced and papers exceeding the page
%% limits will be summarily rejected.
%% 
%% ***ICFP 2017 will employ a lightweight double-blind reviewing
%% process.*** To facilitate this, submitted papers must adhere to two
%% rules:
%% 
%%  1. ***author names and institutions must be omitted***, and
%% 
%%  2. ***references to authors' own related work should be in the third
%%     person*** (e.g., not "We build on our previous work ..." but
%%     rather "We build on the work of ...").
\documentclass[preprint]{sigplanconf}
\usepackage[utf8]{inputenc}

\usepackage{amssymb,amsmath,amsthm}
\usepackage{natbib}
\usepackage{graphicx}
\usepackage{hyperref}
\usepackage{mathpartir}
\usepackage[barr]{xy}
\usepackage{mdframed}
\usepackage{supertabular}
\usepackage{todonotes}

\newcommand{\cL}{{\cal L}}

\let\mto\to                     % Used for arrows
\let\to\relax                   % Used for implication
\newcommand{\to}{\rightarrow}
\newcommand{\id}{\mathsf{id}}
\newcommand{\redto}{\rightsquigarrow}
\newcommand{\cat}[1]{\mathcal{#1}}
\newcommand{\case}[0]{\mathsf{case}}

\let\split\relax

\newcommand{\split}[0]{\mathsf{split}}
\newcommand{\squash}[0]{\mathsf{squash}}
\newcommand{\bx}[0]{\mathsf{box}}
\newcommand{\unbox}[0]{\mathsf{unbox}}
\newcommand{\T}[0]{\mathsf{T}}

\newtheorem{theorem}{Theorem}
\newtheorem{lemma}[theorem]{Lemma}
\newtheorem{corollary}[theorem]{Corollary}
\newtheorem{definition}[theorem]{Definition}
\newtheorem{proposition}[theorem]{Proposition}
\newtheorem{example}[theorem]{Example}

%% Ott includes:
\input{grady-inc}
\renewcommand{\Gradydrule}[4][]{{\displaystyle\frac{\begin{array}{l}#2\end{array}}{#3}\,\Gradydrulename{#4}}}
\renewcommand{\Gradydrulename}[1]{#1}
\renewcommand{\GradydrulesndUName}[0]{\Gradydrulename{}}
\renewcommand{\GradydrulefstUName}[0]{\Gradydrulename{}}
\renewcommand{\GradydrulesuccUName}[0]{\Gradydrulename{}}
\renewcommand{\GradydrulecastName}[0]{\Gradydrulename{}}
\renewcommand{\GradydruleappCName}[0]{\Gradydrulename{}}
\renewcommand{\GradydruleappUName}[0]{\Gradydrulename{}}
\renewcommand{\GradydrulereflName}[0]{\Gradydrulename{}}
\renewcommand{\GradydruleboxName}[0]{\Gradydrulename{}}
\renewcommand{\GradydruleunboxName}[0]{\Gradydrulename{}}
\renewcommand{\GradydrulearrowName}[0]{\Gradydrulename{}}
\renewcommand{\GradydruleprodName}[0]{\Gradydrulename{}}

\begin{document}

\conferenceinfo{ICFP ’17}{September 3--9, 2017, Oxford, United Kingdom}
\copyrightyear{2017}
\copyrightdata{???}
\copyrightdoi{????}

\titlebanner{}
\preprintfooter{}

\title{The Combination of Dynamic and Static Typing from a Categorical Perspective}

\authorinfo{Harley Eades III and Michael Townsend}
           {Augusta University}
           {\{heades,mitownsend\}@augusta.edu}

\maketitle

\category{D.1.1}{Programming Technique}[Applicative (Functional) Programming]
\category{D.3.1}{Formal Definitions and Theory}[Semantics]
\category{F.3.3}{Studies of Program Constructs}[Type structure]
\category{F.4.1}{Mathematical Logic and Formal Languages}[Lambda calculus and related systems]

\terms typed lambda-calculus, untyped lambda-calculus, gradual typing,
static typing, dynamic typing, categorical model, functional
programming

\keywords static typing, dynamic typing, gradual typing, categorical
semantics, retract

\begin{abstract}
  Gradual typing was first proposed by Siek and Taha in 2006 as a way
  for a programming language to combine the strengths of both static
  and dynamic typing.  However one question we must ask is, what is
  gradual typing?  This paper contributes to answering this question
  by providing the first categorical model of gradual typing using the
  seminal work of Scott and Lambek on the categorical models of the
  untyped and typed $\lambda$-calculus.  We then extract a functional
  programming language, called Grady, from the categorical model using
  the Curry-Howard-Lambek correspondence that combines both static and
  dynamic typing, but Grady is an annotated language and not a gradual
  type system.  Finally, we show that Siek and Taha's gradual type
  system can be translated into Grady, and that their original
  annotated language is equivalent in expressive power to Grady.
\end{abstract}

\section{Introduction}
\label{sec:introduction}
\input{introduction-ott}
% section introduction (end)

\section{Gradual Typing}
\label{sec:gradual_typing}
\input{gradual-typing-ott}
% section gradual_typing (end)

\section{The Categorical Model}
\label{sec:categorical_model}
\input{categorical-model-ott}
% section categorical_model (end)

\section{Grady}
\label{sec:grady}

% section grady (end)

\section{Grady and Gradual Typing}
\label{sec:grady_and_gradual_typing}

% section grady_and_gradual_typing (end)


\nocite{*}
\bibliographystyle{plainnat}
\bibliography{references}

\appendix

\section{The Complete Spec of Grady}
\label{sec:the_complete_spec_of_grady}
\Gradyall{}
% section the_complete_spec_of_grady (end)

\end{document}
