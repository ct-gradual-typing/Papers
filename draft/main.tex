%% Submission instructions:
%%
%% Full paper of at most 24 pages ***excluding bibliography***.
%% 
%% The deadlines will be strictly enforced and papers exceeding the page
%% limits will be summarily rejected.
%% 
%% ***ICFP 2017 will employ a lightweight double-blind reviewing
%% process.*** To facilitate this, submitted papers must adhere to two
%% rules:
%% 
%%  1. ***author names and institutions must be omitted***, and
%% 
%%  2. ***references to authors' own related work should be in the third
%%     person*** (e.g., not "We build on our previous work ..." but
%%     rather "We build on the work of ...").
\documentclass[pacmpl,review,anonymous]{acmart}\settopmatter{printfolios=true}
\usepackage[utf8]{inputenc}

\usepackage{amssymb,amsmath,amsthm}
\usepackage{natbib}
\usepackage{graphicx}
\usepackage{hyperref}
\usepackage{mathpartir}
\usepackage[barr]{xy}
\usepackage{mdframed}
\usepackage{supertabular}
\usepackage{todonotes}
\usepackage{enumerate}

\newcommand{\cL}{{\cal L}}

\let\mto\to                     % Used for arrows
\let\to\relax                   % Used for implication
\newcommand{\to}{\rightarrow}
\newcommand{\id}{\mathsf{id}}
\newcommand{\redto}{\rightsquigarrow}
\newcommand{\cat}[1]{\mathcal{#1}}
\newcommand{\catop}[1]{\mathcal{#1}^{\mathsf{op}}}
\newcommand{\case}[0]{\mathsf{case}}

\let\split\relax
\let\S\relax

\newcommand{\split}[0]{\mathsf{split}}
\newcommand{\squash}[0]{\mathsf{squash}}
\newcommand{\bx}[0]{\mathsf{box}}
\newcommand{\unbox}[0]{\mathsf{unbox}}
\newcommand{\T}[0]{\mathsf{T}}
\newcommand{\S}[0]{\mathsf{S}}
\newcommand{\U}[0]{\mathsf{U}}
\newcommand{\C}[0]{\mathsf{C}}
\newcommand{\z}[0]{\mathsf{z}}
\newcommand{\app}[0]{\mathsf{app}}
\newcommand{\curry}[1]{\mathsf{curry}(#1)}
\newcommand{\interp}[1]{[\negthinspace[#1]\negthinspace]}
\newcommand{\Hom}[3]{\mathsf{Hom}_{\cat{#1}}(#2,#3)}

%% \newtheorem{theorem}{Theorem}
%% \newtheorem{lemma}[theorem]{Lemma}
%% \newtheorem{corollary}[theorem]{Corollary}
%% \newtheorem{definition}[theorem]{Definition}
%% \newtheorem{proposition}[theorem]{Proposition}
%% \newtheorem{example}[theorem]{Example}

%% Ott includes:
\input{sl-grady-inc}
\renewcommand{\RightTirName}{\text}
\renewcommand{\SLGradydrule}[4][]{{\displaystyle\frac{\begin{array}{l}#2\end{array}}{#3}\,\SLGradydrulename{#4}}}
\renewcommand{\SLGradydrulename}[1]{#1}
\renewcommand{\SLGradydrulesndUName}[0]{\SLGradydrulename{}}
\renewcommand{\SLGradydrulefstUName}[0]{\SLGradydrulename{}}
\renewcommand{\SLGradydrulesuccUName}[0]{\SLGradydrulename{}}
\renewcommand{\SLGradydrulecastName}[0]{\SLGradydrulename{}}
\renewcommand{\SLGradydruleappCName}[0]{\SLGradydrulename{}}
\renewcommand{\SLGradydruleappUName}[0]{\SLGradydrulename{}}
\renewcommand{\SLGradydrulereflName}[0]{\SLGradydrulename{}}
\renewcommand{\SLGradydruleboxName}[0]{\SLGradydrulename{}}
\renewcommand{\SLGradydruleunboxName}[0]{\SLGradydrulename{}}
\renewcommand{\SLGradydrulearrowName}[0]{\SLGradydrulename{}}
\renewcommand{\SLGradydruleprodName}[0]{\SLGradydrulename{}}

\input{surface-grady-inc}
\renewcommand{\RightTirName}{\text}
\renewcommand{\SGradydrule}[4][]{{\displaystyle\frac{\begin{array}{l}#2\end{array}}{#3}\,\SGradydrulename{#4}}}
\renewcommand{\SGradydrulename}[1]{#1}

\input{core-grady-inc}
\renewcommand{\RightTirName}{\text}
\renewcommand{\CGradydrule}[4][]{{\displaystyle\frac{\begin{array}{l}#2\end{array}}{#3}\,\CGradydrulename{#4}}}
\renewcommand{\CGradydrulename}[1]{#1}

\makeatletter\if@ACM@journal\makeatother
%% Journal information (used by PACMPL format)
%% Supplied to authors by publisher for camera-ready submission
\acmJournal{PACMPL}
\acmVolume{1}
\acmNumber{1}
\acmArticle{1}
\acmYear{2017}
\acmMonth{1}
\acmDOI{10.1145/nnnnnnn.nnnnnnn}
\startPage{1}
\else\makeatother

\acmConference[ICFP'17]{ACM SIGPLAN International Conference on Functional Programming}{September 03--09, 2017}{Oxford, United Kingdom}
\acmYear{2017}
\acmISBN{978-x-xxxx-xxxx-x/YY/MM}
\acmDOI{10.1145/nnnnnnn.nnnnnnn}
\startPage{1}
\fi

%% Copyright information
%% Supplied to authors (based on authors' rights management selection;
%% see authors.acm.org) by publisher for camera-ready submission
\setcopyright{none}             %% For review submission
%\setcopyright{acmcopyright}
%\setcopyright{acmlicensed}
%\setcopyright{rightsretained}
%\copyrightyear{2017}           %% If different from \acmYear

%% Bibliography style
\bibliographystyle{ACM-Reference-Format}
%% Citation style
\citestyle{acmauthoryear}  %% For author/year citations

\begin{document}

\title[Combining Dynamic and Static Typing: A Categorical Perspective]{The Combination of Dynamic and Static Typing from a Categorical Perspective}

%% \authorinfo{Harley Eades III and Michael Townsend}
%%            {Augusta University}
%%            {\{heades,mitownsend\}@augusta.edu}

\author{Harley Eades III}
\affiliation{
  \department{Computer and Information Sciences}              %% \department is recommended
  \institution{Augusta University}            %% \institution is required
  \streetaddress{1120 15th Street}
  \city{Augusta}
  \state{Georgia}
  \postcode{30912}
  \country{USA}
}
\email{heades@augusta.edu}          %% \email is recommended

\author{Michael Townsend}
\affiliation{
  \department{Computer and Information Sciences}              %% \department is recommended
  \institution{Augusta University}            %% \institution is required
  \streetaddress{1120 15th Street}
  \city{Augusta}
  \state{Georgia}
  \postcode{30912}
  \country{USA}
}
\email{mitownsend@augusta.edu}          %% \email is recommended

\begin{abstract}
  Gradual typing was first proposed by Siek and Taha in 2006 as a way
  for a programming language to combine the strengths of both static
  and dynamic typing.  However one question we must ask is, what is
  gradual typing?  This paper contributes to answering this question
  by providing the first categorical model of gradual typing using the
  seminal work of Scott and Lambek on the categorical models of the
  untyped and typed $\lambda$-calculus.  We then extract a functional
  programming language, called Grady, from the categorical model using
  the Curry-Howard-Lambek correspondence that combines both static and
  dynamic typing, but Grady is an annotated language and not a gradual
  type system.  Finally, we show that Siek and Taha's gradual type
  system can be translated into Grady, and that their original
  annotated language is equivalent in expressive power to Grady.
\end{abstract}

%% 2012 ACM Computing Classification System (CSS) concepts
%% Generate at 'http://dl.acm.org/ccs/ccs.cfm'.
\begin{CCSXML}
<ccs2012>
<concept>
<concept_id>10003752.10010124.10010131.10010133</concept_id>
<concept_desc>Theory of computation~Denotational semantics</concept_desc>
<concept_significance>500</concept_significance>
</concept>
<concept>
<concept_id>10003752.10010124.10010131.10010137</concept_id>
<concept_desc>Theory of computation~Categorical semantics</concept_desc>
<concept_significance>500</concept_significance>
</concept>
<concept>
<concept_id>10003752.10003790.10011740</concept_id>
<concept_desc>Theory of computation~Type theory</concept_desc>
<concept_significance>300</concept_significance>
</concept>
<concept>
<concept_id>10003752.10010124.10010125.10010127</concept_id>
<concept_desc>Theory of computation~Functional constructs</concept_desc>
<concept_significance>300</concept_significance>
</concept>
<concept>
<concept_id>10003752.10010124.10010125.10010130</concept_id>
<concept_desc>Theory of computation~Type structures</concept_desc>
<concept_significance>300</concept_significance>
</concept>
</ccs2012>
\end{CCSXML}

\ccsdesc[500]{Theory of computation~Denotational semantics}
\ccsdesc[500]{Theory of computation~Categorical semantics}
\ccsdesc[300]{Theory of computation~Type theory}
\ccsdesc[300]{Theory of computation~Functional constructs}
\ccsdesc[300]{Theory of computation~Type structures}

\keywords{static typing, dynamic typing, gradual typing, categorical
  semantics, retract,typed lambda-calculus, untyped lambda-calculus,
  gradual typing, static typing, dynamic typing, categorical model,
  functional programming}

\maketitle

\section{Introduction}
\label{sec:introduction}
\input{introduction-ott}
% section introduction (end)

\section{Gradual Typing}
\label{sec:gradual_typing}
\input{gradual-typing-ott}
% section gradual_typing (end)

\section{The Categorical Perspective}
\label{sec:categorical_perspective}
\input{categorical-model-ott}
% section categorical_perspective (end)

\section{Simply Typed Core Grady}
\label{sec:sl-grady}
\input{sl-grady-ott}
% section grady (end)

\section{Grady: A Categorically Inspired Gradual Type System}
\label{sec:grady:_a_categorical_inspired_gradual_type_system}

\subsection{Surface Grady: A Gradual Type System}
\label{subsec:surface_grady:_a_gradual_type_system}
\input{surface-grady-ott}
% subsection surface_grady:_a_gradual_type_system (end)

\subsection{Core Grady: The Casting Calculus}
\label{subsec:core_grady:_the_casting_calculus}
\input{core-grady-ott}
% subsection core_grady:_the_casting_calculus (end)

\subsection{Analyzing Grady}
\label{subsec:results}
\input{results-ott}
% subsection results (end)

% section grady:_a_categorically_inspired_gradual_type_system (end)

\nocite{*}
\bibliography{references}

\appendix

\section{Proofs}
\label{sec:proofs}
\input{proofs-ott}
% section proofs (end)

%% \section{The Complete Spec of Grady}
%% \label{sec:the_complete_spec_of_grady}
%% \Gradyall{}
%% % section the_complete_spec_of_grady (end)
\end{document}
