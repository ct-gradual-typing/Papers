%% Submission instructions:
%%
%% 15 pages max
\documentclass{entcs}

\usepackage{preentcsmacro}
\usepackage{graphicx}

\usepackage{amssymb,amsmath}
\usepackage{hyperref}
\usepackage{mathpartir}
\usepackage[barr]{xy}
\usepackage{mdframed}
\usepackage{supertabular}
\usepackage{todonotes}
\usepackage{enumitem}

\newenvironment{enumR}{\begin{enumerate}[label=\roman*.,align=left]}{\end{enumerate}}
\newenvironment{enumA}{\begin{enumerate}[label=\alph*.]}{\end{enumerate}}

\newcommand{\cL}{{\cal L}}

\let\mto\to                     % Used for arrows
\let\to\relax                   % Used for implication
\newcommand{\to}{\rightarrow}
\newcommand{\id}{\mathsf{id}}
\newcommand{\redto}{\rightsquigarrow}
\newcommand{\cat}[1]{\mathcal{#1}}
\newcommand{\catop}[1]{\mathcal{#1}^{\mathsf{op}}}
\newcommand{\Case}[0]{\mathsf{case}}

\let\split\relax
\let\S\relax

\newcommand{\split}[0]{\mathsf{split}}
\newcommand{\squash}[0]{\mathsf{squash}}
\newcommand{\bx}[0]{\mathsf{box}}
\newcommand{\unbox}[0]{\mathsf{unbox}}
\newcommand{\T}[0]{\mathsf{T}}
\newcommand{\S}[0]{\mathsf{S}}
\newcommand{\U}[0]{\mathsf{U}}
\newcommand{\C}[0]{\mathsf{C}}
\newcommand{\z}[0]{\mathsf{z}}
\newcommand{\app}[0]{\mathsf{app}}
\newcommand{\curry}[1]{\mathsf{curry}(#1)}
\newcommand{\interp}[1]{[\negthinspace[#1]\negthinspace]}
\newcommand{\Hom}[3]{\mathsf{Hom}_{\cat{#1}}(#2,#3)}

%% \newtheorem{theorem}{Theorem}
%% \newtheorem{lemma}[theorem]{Lemma}
%% \newtheorem{corollary}[theorem]{Corollary}
%% \newtheorem{definition}[theorem]{Definition}
%% \newtheorem{proposition}[theorem]{Proposition}
%% \newtheorem{example}[theorem]{Example}

%% Ott includes:
\input{sl-grady-inc}
\renewcommand{\SLGradydrule}[4][]{{\displaystyle\frac{\begin{array}{l}#2\end{array}}{#3}\,\SLGradydrulename{#4}}}
\renewcommand{\SLGradydrulename}[1]{#1}

\input{surface-grady-inc}
\renewcommand{\SGradydrule}[4][]{{\displaystyle\frac{\begin{array}{l}#2\end{array}}{#3}\,\SGradydrulename{#4}}}
\renewcommand{\SGradydrulename}[1]{#1}

\input{core-grady-inc}
\renewcommand{\CGradydrule}[4][]{{\displaystyle\frac{\begin{array}{l}#2\end{array}}{#3}\,\CGradydrulename{#4}}}
\renewcommand{\CGradydrulename}[1]{#1}

\input{siek15-gradual-inc}
\renewcommand{\GSiekdrule}[4][]{{\displaystyle\frac{\begin{array}{l}#2\end{array}}{#3}\,\GSiekdrulename{#4}}}
\renewcommand{\GSiekdrulename}[1]{#1}
\renewcommand{\GSiekdruleSXXvarName}[0]{\text{var}}
\renewcommand{\GSiekdruleSXXunitName}[0]{\text{unit}}
\renewcommand{\GSiekdruleSXXzeroName}[0]{\text{zero}}
\renewcommand{\GSiekdruleSXXsuccName}[0]{\text{succ}}
\renewcommand{\GSiekdruleSXXpairName}[0]{\times}
\renewcommand{\GSiekdruleSXXlamName}[0]{\to}
\renewcommand{\GSiekdruleSXXsndName}[0]{\times_{e_2}}
\renewcommand{\GSiekdruleSXXfstName}[0]{\times_{e_1}}
\renewcommand{\GSiekdruleSXXappName}[0]{\to_e}

\def\lastname{Eades III and Townsend}

\begin{document}

\begin{frontmatter}
  \title{The Combination of Dynamic and Static Typing from a Categorical Perspective}
  \author{Harley Eades III\thanksref{ALL}\thanksref{myemail}}
  \address{Computer and Information Sciences\\ Augusta University\\
    Augusta, USA}
  \author{Michael Townsend\thanksref{coemail}}
  \address{Computer and Information Sciences\\ Augusta University\\
    Augusta, USA}

  \thanks[ALL]{Thanks: TODO}
  \thanks[myemail]{Email:\href{mailto:heades@augusta.edu} {\texttt{\normalshape heades@augusta.edu}}}
  \thanks[coemail]{Email: \href{mailto:mitownsend@augusta.edu} {\texttt{\normalshape mitownsend@augusta.edu}}}
  \begin{abstract} 
    In this paper we introduce a new categorical model based on
    retracts that combines static and dynamic typing.  This model is
    initially based on the seminal work of Scott who showed that the
    untyped $\lambda$-calculus can be considered as typed using
    retracts.  We then show that our model gives rise to a new and
    simple type system which combines static and dynamic typing.
    Finally, we extend this type system with bounded quantification
    and lists, and then develop a gradually typed surface language
    that uses our new type system as a core casting calculus.  This
    paper can be seen as the first of a new project to investigate the
    categorical semantics of gradual type systems.
  \end{abstract}
  \begin{keyword}  
    static typing, dynamic typing, gradual typing, categorical
    semantics, retract,typed lambda-calculus, untyped lambda-calculus,
    gradual typing, static typing, dynamic typing, categorical model,
    functional programming
  \end{keyword}
\end{frontmatter}

\section{Introduction}
\label{sec:introduction}
\input{introduction-ott}
% section introduction (end)

\section{Gradual Typing}
\label{sec:gradual_typing}
\input{gradual-typing-ott}
% section gradual_typing (end)

\section{The Categorical Perspective}
\label{sec:categorical_perspective}
\input{categorical-model-ott}
% section categorical_perspective (end)

\section{Simply Typed Core Grady}
\label{sec:sl-grady}
\input{sl-grady-ott}
% section grady (end)

\section{Grady: A Categorically Inspired Gradual Type System}
\label{sec:grady:_a_categorical_inspired_gradual_type_system}

\subsection{Surface Grady: A Gradual Type System}
\label{subsec:surface_grady:_a_gradual_type_system}
\input{surface-grady-ott}
% subsection surface_grady:_a_gradual_type_system (end)

\subsection{Core Grady: The Casting Calculus}
\label{subsec:core_grady:_the_casting_calculus}
\input{core-grady-ott}
% subsection core_grady:_the_casting_calculus (end)

\subsection{Analyzing Grady}
\label{subsec:results}
\input{results-ott}
% subsection results (end)

% section grady:_a_categorically_inspired_gradual_type_system (end)

\bibliographystyle{entcs}
\bibliography{references}

\appendix

\input{proofs-ott}

% section proofs (end)
\end{document}
