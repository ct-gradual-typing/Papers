%% Submission instructions:
%%
%% 15 pages max
\documentclass{entcs}

\usepackage{preentcsmacro}
\usepackage{graphicx}

\usepackage{amssymb,amsmath}
\usepackage{hyperref}
\usepackage{mathpartir}
\usepackage[barr]{xy}
\usepackage{mdframed}
\usepackage{supertabular}
\usepackage{todonotes}
\usepackage{enumitem}

\newenvironment{enumR}{\begin{enumerate}[label=\roman*.,align=left]}{\end{enumerate}}
\newenvironment{enumA}{\begin{enumerate}[label=\alph*.]}{\end{enumerate}}

\newcommand{\cL}{{\cal L}}

\let\mto\to                     % Used for arrows
\let\to\relax                   % Used for implication
\newcommand{\to}{\rightarrow}
\newcommand{\id}{\mathsf{id}}
\newcommand{\redto}{\rightsquigarrow}
\newcommand{\cat}[1]{\mathcal{#1}}
\newcommand{\catop}[1]{\mathcal{#1}^{\mathsf{op}}}
\newcommand{\Case}[0]{\mathsf{case}}

\let\split\relax
\let\S\relax

\newcommand{\split}[0]{\mathsf{split}}
\newcommand{\squash}[0]{\mathsf{squash}}
\newcommand{\bx}[0]{\mathsf{box}}
\newcommand{\unbox}[0]{\mathsf{unbox}}
\newcommand{\T}[0]{\mathsf{T}}
\newcommand{\S}[0]{\mathsf{S}}
\newcommand{\U}[0]{\mathsf{U}}
\newcommand{\C}[0]{\mathsf{C}}
\newcommand{\z}[0]{\mathsf{z}}
\newcommand{\app}[0]{\mathsf{app}}
\newcommand{\curry}[1]{\mathsf{curry}(#1)}
\newcommand{\interp}[1]{[\negthinspace[#1]\negthinspace]}
\newcommand{\Hom}[3]{\mathsf{Hom}_{\cat{#1}}(#2,#3)}

%% \newtheorem{theorem}{Theorem}
%% \newtheorem{lemma}[theorem]{Lemma}
%% \newtheorem{corollary}[theorem]{Corollary}
%% \newtheorem{definition}[theorem]{Definition}
%% \newtheorem{proposition}[theorem]{Proposition}
%% \newtheorem{example}[theorem]{Example}

%% Ott includes:
\input{sl-grady-inc}
\renewcommand{\SLGradydrule}[4][]{{\displaystyle\frac{\begin{array}{l}#2\end{array}}{#3}\,\SLGradydrulename{#4}}}
\renewcommand{\SLGradydrulename}[1]{#1}

\input{siek15-gradual-inc}
\renewcommand{\GSiekdrule}[4][]{{\displaystyle\frac{\begin{array}{l}#2\end{array}}{#3}\,\GSiekdrulename{#4}}}
\renewcommand{\GSiekdrulename}[1]{#1}
\renewcommand{\GSiekdruleSXXvarName}[0]{\text{var}}
\renewcommand{\GSiekdruleSXXunitName}[0]{\text{unit}}
\renewcommand{\GSiekdruleSXXzeroName}[0]{\text{zero}}
\renewcommand{\GSiekdruleSXXsuccName}[0]{\text{succ}}
\renewcommand{\GSiekdruleSXXpairName}[0]{\times}
\renewcommand{\GSiekdruleSXXlamName}[0]{\to}
\renewcommand{\GSiekdruleSXXsndName}[0]{\times_{e_2}}
\renewcommand{\GSiekdruleSXXfstName}[0]{\times_{e_1}}
\renewcommand{\GSiekdruleSXXappName}[0]{\to_e}

\renewcommand{\GSiekdruleCXXvarName}[0]{\text{var}}
\renewcommand{\GSiekdruleCXXunitName}[0]{\text{unit}}
\renewcommand{\GSiekdruleCXXzeroName}[0]{\text{zero}}
\renewcommand{\GSiekdruleCXXsuccName}[0]{\text{succ}}
\renewcommand{\GSiekdruleCXXpairName}[0]{\times}
\renewcommand{\GSiekdruleCXXlamName}[0]{\to}
\renewcommand{\GSiekdruleCXXsndName}[0]{\times_{e_2}}
\renewcommand{\GSiekdruleCXXfstName}[0]{\times_{e_1}}
\renewcommand{\GSiekdruleCXXappName}[0]{\to_e}
\renewcommand{\GSiekdruleCXXcastName}[0]{\text{cast}}
\renewcommand{\GSiekdrulerdAXXvaluesName}{\text{values}}
\renewcommand{\GSiekdrulerdAXXcastIdName}{\text{id-atom}}
\renewcommand{\GSiekdrulerdAXXcastUName}{\text{id-U}}
\renewcommand{\GSiekdrulerdAXXsucceedName}{\text{succeed}}
\renewcommand{\GSiekdrulerdAXXcastArrowName}{\text{arrow}}
\renewcommand{\GSiekdrulerdAXXcastGroundName}{\text{expand}_1}
\renewcommand{\GSiekdrulerdAXXcastExpandName}{\text{expand}_2}
\renewcommand{\GSiekdrulerdAXXbetaName}{}
\renewcommand{\GSiekdrulerdAXXappOneName}{}
\renewcommand{\GSiekdrulerdAXXappTwoName}{}
\renewcommand{\GSiekdrulerdAXXfstName}{}
\renewcommand{\GSiekdrulerdAXXsndName}{}
\renewcommand{\GSiekdrulerdAXXpairOneName}{}
\renewcommand{\GSiekdrulerdAXXpairTwoName}{}
\renewcommand{\SLGradydruleBoxName}{\text{box}}
\renewcommand{\SLGradydruleUnboxName}{\text{unbox}}
\renewcommand{\SLGradydruleBoxGName}{\text{Box}}
\renewcommand{\SLGradydruleUnboxGName}{\text{Unbox}}
\renewcommand{\SLGradydruleSplitGName}{\text{Split}}
\renewcommand{\SLGradydruleSquashGName}{\text{Squash}}
\renewcommand{\SLGradydrulesquashName}{\text{squash}}
\renewcommand{\SLGradydrulesplitName}{\text{split}}
\renewcommand{\SLGradydrulecaseName}{\mathsf{Nat}_e}
\renewcommand{\SLGradydrulerdXXvarName}{\text{var}}
\renewcommand{\SLGradydrulerdXXunboxName}{\text{unbox}}
\renewcommand{\SLGradydrulerdXXsplitName}{\text{split}}
\renewcommand{\SLGradydrulerdXXretracTName}{\text{retract}_1}
\renewcommand{\SLGradydrulerdXXretracTGName}{\text{retract}_3}
\renewcommand{\SLGradydrulerdXXretracTSGName}{\text{retract}_4} 
\renewcommand{\SLGradydrulerdXXretractUName}{\text{retract}_2} 
\renewcommand{\SLGradydrulerdXXbetaName}{\beta} 
\renewcommand{\SLGradydrulerdXXprojOneName}{\times_{e_1}} 
\renewcommand{\SLGradydrulerdXXprojTwoName}{\times_{e_2}} 
\renewcommand{\SLGradydrulerdXXappOneName}{\to_{e_1}} 
\renewcommand{\SLGradydrulerdXXappTwoName}{\to_{e_2}} 
\renewcommand{\SLGradydrulerdXXfstName}{\mathsf{fst}} 
\renewcommand{\SLGradydrulerdXXsndName}{\mathsf{snd}} 
\renewcommand{\SLGradydrulerdXXpairOneName}{\times_1} 
\renewcommand{\SLGradydrulerdXXpairTwoName}{\times_2}
\renewcommand{\SLGradydrulerdXXsuccName}{\text{succ}}       
\renewcommand{\SLGradydrulerdXXcaseZeroName}{\mathsf{Nat}_{e_0}} 
\renewcommand{\SLGradydrulerdXXcaseSuccName}{\mathsf{Nat}_{e_1}} 
\renewcommand{\SLGradydrulerdXXcaseName}{\text{case}_1} 
\newcommand{\GSTLC}{\lambda^?_\to}
\newcommand{\CGSTLC}{\lambda^{\Rightarrow}_\to} 

\def\lastname{Eades III and Townsend}

\begin{document}

\begin{frontmatter}
  \title{The Combination of Dynamic and Static Typing from a Categorical Perspective}
  \author{Harley Eades III\thanksref{ALL}\thanksref{myemail}}
  \address{Computer and Information Sciences\\ Augusta University\\
    Augusta, USA}
  \author{Michael Townsend\thanksref{coemail}}
  \address{Computer and Information Sciences\\ Augusta University\\
    Augusta, USA}

  \thanks[ALL]{Thanks: TODO}
  \thanks[myemail]{Email:\href{mailto:heades@augusta.edu} {\texttt{\normalshape heades@augusta.edu}}}
  \thanks[coemail]{Email: \href{mailto:mitownsend@augusta.edu} {\texttt{\normalshape mitownsend@augusta.edu}}}
  \begin{abstract} 
    This paper is part one of a two part series.  In this paper we
    introduce a new categorical model based on retracts that combines
    static and dynamic typing.  This model is initially based on the
    seminal work of Scott who showed that the untyped
    $\lambda$-calculus can be considered as typed using retracts.  We
    then show that our model gives rise to a new and simple type
    system which combines static and dynamic typing.  We then show
    that Siek and Taha's gradually typed $\lambda$-calculus can be
    interpreted in our new model.  Finally, we set the stage for a
    followup paper that greatly extends this type system with bounded
    quantification and lists, and then develops a gradually typed
    surface language that uses our new type system as a core casting
    calculus.
  \end{abstract}
  \begin{keyword}  
    static typing, dynamic typing, gradual typing, categorical
    semantics, retract,typed lambda-calculus, untyped lambda-calculus,
    gradual typing, static typing, dynamic typing, categorical model,
    functional programming
  \end{keyword}
\end{frontmatter}

\section{Introduction}
\label{sec:introduction}
\input{introduction-ott}
% section introduction (end)

\section{The Categorical Model}
\label{subsec:the_categorical_model}
\input{categorical-model-ott}
% section the_interpretation (end)

\section{Gradual Typing}
\label{sec:gradual_typing}

\input{gradual-typing-ott}

\subsection{The Interpretation}
\label{subsec:the_interpretation}
\input{categorical-interp-ott}
% subsection the_interpretation (end)
% section gradual_typing (end)

\section{Simply Typed Grady}
\label{sec:sl-grady}
\input{sl-grady-ott}
% section grady (end)

\bibliographystyle{entcs}
\bibliography{references}

\appendix

\input{proofs-ott}

% section proofs (end)
\end{document}
