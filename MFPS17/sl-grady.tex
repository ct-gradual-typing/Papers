Just as the simply typed $\lambda$-calculus corresponds to cartesian
closed categories our categorical model has a corresponding type
theory we call Simply Typed Grady (Grady).  It consists of all of the
structure found in the model.  Its syntax is an extension of the
syntax for $\CGSTLC$.

\begin{definition}
  \label{def:grady-syntax}
  Syntax for Grady:
  \[
  \begin{array}{c@{\hspace{5pt}}r@{}@{\hspace{5pt}}r@{}@{\hspace{2pt}}l@{}llllllllllll}
    \text{(basic skeletons)} & [[U]] & ::= & [[? -> ?]] \mid [[? x ?]]\\
    \text{(skeletons)}       & [[S]] & ::= & [[?]] \mid [[S1 x S2]] \mid [[S1 -> S2]]\\
    \text{(atomic types)}    & [[C]] & ::= & [[Unit]] \mid [[Nat]]\\
    \text{(terms)}           & [[t]] & ::= & \ldots \mid [[split U]] \mid [[squash U]]
    \mid [[box T]] \mid [[unbox T]]\\ & & \mid & [[case t of 0 -> t1, (succ x) -> t2]]\\
    \text{(values)}          & [[v]] & ::= & \ldots \mid [[squash U]] \mid [[box T]]\\
  \end{array}
  \]
\end{definition}
\noindent
The typing rules for Grady can be found in
Figure~\ref{fig:grady-typing} and its reduction rules can be found in
Figure~\ref{fig:grady-reduction}.
\begin{figure}
  \small
  \begin{mdframed}
    \begin{mathpar}
      \SLGradydrulevar{} \and
      \SLGradydruleBox{} \and
      \SLGradydruleUnbox{} \and
      \SLGradydrulesquash{} \and
      \SLGradydrulesplit{} \and
      \SLGradydruleunit{} \and
      \SLGradydrulezero{} \and
      \SLGradydrulesucc{} \and
      \SLGradydrulecase{} \and
      \SLGradydrulepair{} \and
      \SLGradydrulefst{} \and
      \SLGradydrulesnd{} \and
      \SLGradydrulelam{} \and
      \SLGradydruleapp{}
    \end{mathpar}
  \end{mdframed}
  \caption{Typing rules for Grady}
  \label{fig:grady-typing}
\end{figure}
\begin{figure}
  \small
  \begin{mdframed}
    \begin{mathpar}
      \SLGradydrulerdXXvar{} \and
      \SLGradydrulerdXXretracT{} \and
      \SLGradydrulerdXXunbox{} \and
      \SLGradydrulerdXXretractU{} \and
      \SLGradydrulerdXXsplit{} \and
      \SLGradydrulerdXXbeta{} \and
      \SLGradydrulerdXXprojOne{} \and
      \SLGradydrulerdXXprojTwo{} \and
      \SLGradydrulerdXXsucc{} \and      
      \SLGradydrulerdXXcaseZero{} \and
      \SLGradydrulerdXXcaseSucc{} \and
      \SLGradydrulerdXXcase{} \and
      \SLGradydrulerdXXappOne{} \and
      \SLGradydrulerdXXappTwo{} \and
      \SLGradydrulerdXXfst{} \and
      \SLGradydrulerdXXsnd{} \and
      \SLGradydrulerdXXpairOne{} \and
      \SLGradydrulerdXXpairTwo{}    
    \end{mathpar}
  \end{mdframed}
  \caption{Reduction rules for Grady}
  \label{fig:grady-reduction}
\end{figure}
Reduction for Grady differs from $\CGSTLC$ in that it is an extended
formulation of call-by-name.  We only allow reduction under
$[[unbox]]$ and $[[split]]$, and we do not allow reduction under the
branches of case.

Just as we did for the categorical model
(Lemma~\ref{lemma:casting_morphisms}) we can lift $[[box C]]$ and
$[[unbox C]]$ to arbitrary type.
\begin{lemma}[Syntactic $[[Box A]]$ and $[[Unbox A]]$]
  \label{lemma:syntactic_box_and_unbox}
  Given any type $[[A]]$ there are functions $[[Box A]]$ and $[[Unbox
      A]]$ such that the following typing rules are admissible:
  \[
  \begin{array}{lll}
    \SLGradydruleBoxG{} & \quad & \SLGradydruleUnboxG{}
  \end{array}
  \]
  Furthermore, the following reduction rule is admissible:
  \[
  \begin{array}{lll}
    \SLGradydrulerdXXretracTG{} 
  \end{array}
  \]
\end{lemma}
\begin{proof}
  The functions $[[Box A]]$ and $[[Unbox A]]$ can be defined using the
  construction from the categorical model,
  e.g. Definition~\ref{def:boxing-unboxing},
  Definition~\ref{def:lifted-split-squash}, and
  Lemma~\ref{lemma:casting_morphisms}.  However, the categorical
  notions of composition, identity, and the functors $- \to -$ and $-
  \times -$ must be defined as meta-functions first, but after they
  are, then the same constructions apply.  Please see
  Appendix~\ref{subsec:proof_of_lemma_syntactic_box_and_unbox} for the
  constructions.
\end{proof}
\noindent
Similarly, we can lift $[[split]]$ and $[[squash]]$ to arbitrary
skeletons.
\begin{lemma}[Syntactic $[[Split S]]$ and $[[Squash S]]$]
  \label{lemma:syntactic_box_and_unbox}
  Given any type $[[A]]$ there are functions $[[Split A]]$ and $[[Squash S]]$
  such that the following typing rules are admissible:
  \[
  \begin{array}{lll}
    \SLGradydruleSplitG{} & \quad & \SLGradydruleSquashG{}
  \end{array}
  \]
  Furthermore, the following reduction rule is admissible:
  \[
  \begin{array}{lll}
    \SLGradydrulerdXXretracTSG{} 
  \end{array}
  \]
\end{lemma}
\noindent
Perhaps unsurprisingly, due to our results with respect to the
categorical model, we can use the previous results to construct a
type-directed translation of both $\GSTLC$ and $\CGSTLC$ into Grady.
\begin{lemma}[Translations]
  \label{lemma:translations}
  \begin{center}
    \begin{itemize}
    \item[] 
    \item[i.] If $[[G |- t : A]]$ hold in either $\GSTLC$ or
      $\CGSTLC$, then there exists a term
      $[[t']]$ such that $[[G |- t' : A]]$ holds in Grady.
    \item[ii.] If $[[G |- t1 ~> t2 : A]]$ holds in $\CGSTLC$, then
      $[[G |- t'1 ~>* t'2 : A]]$ holds in Grady, where $[[G |- t'1 :
        A]]$ and $[[G |- t'2 : A]]$ are both the corresponding Grady
      terms.
    \end{itemize}
  \end{center}
\end{lemma}
\begin{proof}
  The proof of this result is similar to the proof that both
  $\GSTLC$ and $\CGSTLC$ can be
  interpreted into the categorical model,
  Theorem~\ref{thm:interpretation_of_typing} and
  Theorem~\ref{thm:interpretation_of_evaluation}, and thus, we do not
  give the full proof.  The proof of part one holds by induction on
  $[[G |- t : A]]$, and using the realization that if $<<A ~ B>>$ for
  some types $[[A]]$ and $[[B]]$ then there are casting terms $[[. |-
      c1 : A -> B]]$ and $[[. |- c2 : B -> A]]$ following the proof of
  Lemma~\ref{lemma:type_consistency_in_the_model}. The proof of part
  two holds by induction on $[[G |- t1 ~> t2 : A]]$ making use of part
  one; it is similar to the proof of
  Theorem~\ref{thm:interpretation_of_evaluation}.
\end{proof}

One important point of this system is it is straightforward to extend
with new features, because it does not depend on type consistency.
For example, in the sequel paper\footnote{TODO cite draft.} we show
that extending this system with bounded quantification is
straightforward, but quite difficult for the gradually typed surface
language.  Furthermore, the different types of casts in Grady are
tracked within the term which may lead to finer grain analysis of this
system.  Lastly, this system has a corresponding categorical model.

%% Having the untyped $\lambda$-calculus along side the typed
%% $\lambda$-calculus can be a lot of fun.  This section can be seen from
%% two perspectives: it gives a number of examples in Grady, and shows
%% several ways the typed and untyped fragments can be mixed.
%% \todo[inline]{Michael is writing this section.}
%% \begin{itemize}
%% \item Church Encoded Data
%% \item Y-combinator and the natural number eliminator, e.g. terminating recursion on natural numbers
%% \item Scott Encoded data, this is not available in terminating type theories
%% \item Parigot Encoded Data, better efficiency 
%% \end{itemize}
Just as Abadi et al.~\cite{Abadi:1989} argue it is quite useful to
have access to the untyped $\lambda$-calculus.  In the remainder of
this section we give some example Grady programs utilizing this
powerful feature.  We have a full implementation of Grady with several
extensions available\footnote{Please see TODO for access to the
  implementation as well as full documentation on how to install and
  use it.}.  All examples in this section can be typed and ran in the
implementation, and thus, we make use of Grady's concrete syntax which
is very similar to Haskell's and does not venture too far from the
mathematical syntax given above.

Grady does not have a primitive notion of recursion, but it is
well-known that we can define the Y combinator in the untyped
$\lambda$-calculus.  Its definition in Grady is as follows:
\begin{lstlisting}[language=Haskell]
  omega : (? -> ?) -> ?
  omega = \(x : ? -> ?) -> (x (squash (? -> ?) x));
  
  fix : (? -> ?) -> ?
  fix = \(f : ? -> ?) -> omega (\(x:?) -> f ((split (? -> ?) x) x));
\end{lstlisting}
Using $\lstinline{fix}$ we can define the usual arithmetic operations
in Grady, but we use a typed version of $\lstinline{fix}$.
\begin{lstlisting}[language=Haskell]
  fixNat : ((Nat -> Nat) -> (Nat -> Nat)) -> (Nat -> Nat)
  fixNat = \(f:(Nat -> Nat) -> (Nat -> Nat))    ->
           unbox{Nat -> Nat} (fix (\(y:?)->box{Nat -> Nat} (f (unbox{Nat -> Nat} y))));
  
  pred : Nat -> Nat
  pred = \(n:Nat) -> case n of 0 -> 0, (succ n') -> n';

  add : Nat -> Nat -> Nat
  add = \(m:Nat) -> fixNat
       (\(r:Nat -> Nat) ->
        \(n:Nat) -> case n of 0 -> m, (succ n') -> succ (r n'));
  
  mult : Nat -> Nat -> Nat
  mult = \(m:Nat) -> fixNat
       (\(r:Nat -> Nat) ->
        \(n:Nat) -> case n of 0 -> 0, (succ n') -> add m (r n'));

  exp : Nat -> Nat -> Nat
  exp = \(m:Nat) -> fixNat
       (\(r:Nat -> Nat) ->
        \(n:Nat) -> case n of 0 -> 1, (succ n') -> mult m (r n'));
\end{lstlisting}
The body of each of the examples above are fully typed and the only
dynamic typing that occurs is in the use of $\lstinline{fix}$.
Extending Grady with polymorphism would allow for the definition of
$\lstinline{fixNat}$ to be abstracted, and then we could do statically
typed recursion at any type.  In fact, the implementation of Grady
already supports this:
\begin{lstlisting}[language=Haskell]
fixp : forall (X <: Simple).((X -> X) -> X)
fixp = \(X <: Simple) ->
       \(f:X -> X)    ->
           unbox{X} (fix (\(y:?)->box{X} (f (unbox{X} y))));
\end{lstlisting}
The type bound $\lstinline{Simple}$ forces $\lstinline{X}$ to be a
simple type.  We do not allow casting of polymorphic types.

%%% Local Variables: ***
%%% mode:latex ***
%%% TeX-master: "main.tex"  ***
%%% End: ***
