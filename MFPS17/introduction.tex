Scott \cite{Scott:1980} showed how to model the untyped
$\lambda$-calculus within a cartesian closed category, $\cat{C}$, with
a distinguished object we will call $[[?]]$ -- read as the type of
untyped terms -- such that the object\footnote{We will use the terms
  ``object'' and ``type'' interchangeably.} $[[? ->?]]$ is a retract
of $?$.  That is, there are morphisms $[[squash : (? -> ?)  --> ?]]$
and $[[split : ?  --> (? -> ?)]]$ where $[[squash]];[[split]] = \id :
[[(? -> ?)  --> (? -> ?)]]$\footnote{We denote composition of
  morphisms by $f;g : A \mto C$ given morphisms $f : A \mto B$ and $g
  : B \mto C$.}.  For example, taking these morphisms as terms in the
typed $\lambda$-calculus we can define the prototypical looping term
$(\lambda x.x\,x)(\lambda x.x\,x)$ by $[[(\x:?.H((split x) x))(squash
    (\x:?.H((split x) x)))]]$.

In the same volume as Scott, Lambek \cite{Lambek:1980} showed that
cartesian closed categories also model the typed
$\lambda$-calculus. Suppose we want to model the typed
$\lambda$-calculus with pairs and natural numbers.  That is, given two
types $[[A1]]$ and $[[A2]]$ there is a type $[[A1 x A2]]$, and there
is a type $[[Nat]]$.  Furthermore, we have first and second
projections, and zero and successor functions. This situation can
easily be modeled by a cartesian closed category $\cat{C}$ -- see
Section~\ref{sec:categorical_model} for the details.  Now combine both
of Scott and Lambek's work by adding to $\cat{C}$ the type of untyped
terms $[[?]]$, $[[squash]]$, and $[[split]]$. At this point $\cat{C}$
is a model of both the typed and the untyped $\lambda$-calculus.
However, the two theories are really just sitting side by side in
$\cat{C}$ and cannot really interact much.

Suppose $A$ is an atomic type.  Then we add to $\cat{C}$ the morphisms
$[[box]] : A \mto [[?]]$ and $[[unbox]] :\, ? \mto A$ such that
$[[box]];[[unbox]] = \id : A \mto A$ making $A$ a retract of $[[?]]$.
This is the bridge allowing the typed world to interact with the
untyped one.  We can think of $[[box]]$ as injecting typed data into
the untyped world, and $[[unbox]]$ as taking it back.  Notice that the
only time we can actually get the typed data back out is if it were
injected into the untyped world initially.  In the model this is
enforced through composition, but in the language this will be
enforced at runtime, and hence, requires the language to contain
dynamic typing.  Thus, what we have just built up is a categorical
model that offers a new perspective of how to combine static and
dynamic typing.

Siek and Taha \cite{Siek:2006} define gradual typing to be the
combination of both static and dynamic typing that allows for the
programmer to program in dynamic style without the need for the them
to explicitly insert casts into their programs.  Siek and Taha's
initial paper laid out the first gradually typed $\lambda$-calculus,
but Siek et al. \cite{Siek:2015} layout a refined criteria for what
metatheortic properties a gradual type system should have called the
gradual guarantee.  In this paper we show that our categorical model
leads to a new core casting calculus for gradual type systems.

Siek and Taha's gradually typed $\lambda$-calculus is defined as the
simply typed $\lambda$-calculus with the type of untyped terms $[[?]]$
and the following new application rule:
\begin{mathpar}
\GSiekdruleSXXapp{}
\end{mathpar}
The premise $<<A2 ~ A1>>$ is read, the type $<<A2>>$ is consistent
with the type $<<A1>>$, and is defined in
Figure~\ref{fig:type-consistency}.  If we squint we can see
$[[split]]$, $[[squash]]$, $[[box]]$, and $[[unbox]]$ hiding in the
definition of the previous rules, but they have been suppressed.  We
will show that when one uses either of the two typing rules then one
is really implicitly using a casting morphism built from $[[split]]$,
$[[squash]]$, $[[box]]$, and $[[unbox]]$.  In fact, the consistency
relation $<<A ~ B>>$ can be interpreted as such a morphism.  Then the
typing above can be read semantically as a saying if a casting
morphism exists, then the type really can be converted into the
necessary type.  The premise $<<fun(C) = A1 -> B1>>$ requires that
$<<C>>$ either be $<<A1 -> B1>>$ or $<<C>> = <<?>>$ and
$\mathsf{fun}(C) = [[? -> ?]]$ where $<<A1>> = <<?>>$ and
$<<B1>> = <<?>>$. Again, we can see $[[split]]$ and $[[box]]$ hiding in
the shadows.  When $<<C>> = <<?>>$, then $<<A1>> = <<?>>$ which implies
that $<<A2 ~ A1>>$ will correspond to $[[box]]$, and the premise
$\mathsf{fun}(C) = [[? -> ?]]$ corresponds to using $[[split]]$.  Thus, in this
case the term $<<t1 t2>>$ can be converted into a term in our calculus by
$[[(split (? -> ?) t1) (box A2 t2)]]$.  An interesting point about this
rule is that dynamically typed programs tend toward statically typed programs.
However, the gradual guarantee shows that any program in the gradual type system
can slide either more towards static typing or more towards dynamic typing by
inserting or removing casts.

\textbf{Contributions.} This paper offers the following contributions:
\begin{itemize}

\item A new categorical model for gradual typing for functional
  languages.  We show how to interpret Siek and Taha's
  \cite{Siek:2015} gradual type system in the categorical model
  outlined above.

\item We then extract a functional programming language called Simple
  Grady from the categorical model via the Curry-Howard-Lambek
  correspondence.  This is not a gradual type system, but can be seen
  as an alternative core casting calculus in which Siek and Taha's
  gradual type system can be translated to.
  
\item We extend Simple Grady into a new language we call Core Grady
  that has natural numbers, lists, and bounded quantification. In
  addition, natural numbers and lists will contain an eliminator that
  when combined with the Y combinator yields terminating recursion
  without the need for Grady to explicitly contain a fixpoint
  operator.

\item Finally, we develop a gradually typed surface language called
  Surface Grady and show that it satisfies the gradual guarantee.

\end{itemize}

\textbf{Related work.}  We now give a brief summary of related
work. Each of the articles discussed below can be consulted for
further references.
%% - Combing static and dynamic typing
%% - Gradual Typing
%%    - Polymorphism
\begin{itemize}
\item Abadi et al.~\cite{Abadi:1989} combine dynamic and static typing
  by adding a new type called $\mathsf{Dynamic}$ along with a new case
  construct for pattern matching on types.  We do not add such a case
  construct, and as a result, show that we can obtain a surprising
  amount of expressivity without it.  They also provide denotational
  models.

\item Henglein~\cite{Henglein:1994} defines the dynamic
  $\lambda$-calculus by adding a new type $\mathsf{Dyn}$ to the simply
  typed $\lambda$-calculus and then adding primitive casting
  operations called tagging and check-and-untag.  These new operations
  tag type constructors with their types.  Then untagging checks to
  make sure the target tag matches the source tag, and if not, returns
  a dynamic type error.  These operations can be used to build casting
  coercions which are very similar to our casting morphisms. We can
  also define $[[split]]$, $[[squash]]$, $[[box]]$, and $[[unbox]]$ in
  terms of Henglein's casting coercions.  However, our setting is a
  bit more strict then his, because his ``boxing'' and ``unboxing''
  operations form an isomorphism, but we show that a retract is only
  needed.

\item As we mentioned in the introduction Siek and
  Taha~\cite{Siek:2006} were the first to define gradual typing
  especially for functional languages.  We show that their language
  can be given a straightforward categorical model in cartesian closed
  categories.  Since their original paper introducing gradual types
  lots of languages have adopted it, but the term ``gradual typing''
  started to become a catch all phrase for any language combining
  dynamic and static typing.  As a result of this Siek et
  al.~\cite{Siek:2015} later refine what it means for a language to
  support gradual typing by specifying the necessary metatheoretic
  properties a gradual type system must satisfy called the gradual
  guarantee.

\item The categorical model we give here is with respect to the core
  casting calculus.  In future work we plan to investigate a
  categorical model for the surface language, and we believe that
  Garcia's~\cite{Garcia:2016} work showing how to extract the gradual
  type system from a fully static type system using abstract
  interpretation will play a key role.  Abstract interpretation uses
  Galois connections which can be studied as adjoints in the category
  of posets.  Garcia shows that one can start with a static type
  system, and then use abstract interpretation to infer the gradually
  typed surface language complete with both statics and dynamics.

\end{itemize}

%%% Local Variables: ***
%%% mode:latex ***
%%% TeX-master: "main.tex"  ***
%%% End: ***
