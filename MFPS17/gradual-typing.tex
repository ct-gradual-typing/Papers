In this section we introduce a slight variation of the gradually typed
functional language given by Siek et al.~\cite{Siek:2015}.  The syntax
and typing rules of the gradual type system $\GSTLC$ are defined in
Figure~\ref{fig:GSTLC}.  The main changes of the version of $\GSTLC$
defined here from the original is that products and natural numbers
have been added.  The definition of products follows how casting is
done for functions. So it allows casting projections of products, for
example, it is reasonable for terms like $[[\x:(? x ?).(succ (fst
    x))]]$ to type check.
\renewcommand{\GSiekdrulereflName}[0]{\text{refl}}
\renewcommand{\GSiekdruleboxName}[0]{\text{box}}
\renewcommand{\GSiekdruleunboxName}[0]{\text{unbox}}
\renewcommand{\GSiekdrulearrowName}[0]{\to}
\renewcommand{\GSiekdruleprodName}[0]{\times}
\begin{figure}
  \begin{mdframed}
    \small
    \begin{itemize}
  \item[] \textbf{Syntax:}
    \[ 
    \begin{array}{c@{\hspace{5pt}}r@{}@{\hspace{5pt}}r@{}@{\hspace{2pt}}l@{}llllllllllll}
      \text{(types)} & [[A]],[[B]],[[C]],[[D]] & ::=  & [[Unit]] \mid [[Nat]] \mid [[?]] \mid [[A x B]] \mid [[A1 -> A2]]\\
      \text{(terms)} & [[t]] & ::=  & [[x]] \mid [[triv]] \mid [[0]] \mid [[succ t]] \mid [[\x:A.t]]  \mid [[t1 t2]]
      \mid [[(t1,t2)]] \mid [[fst t]] \mid [[snd t]] \\
      \text{(contexts)} & [[G]] & ::= & [[.]] \mid [[x : A]] \mid [[G1,G2]]\\
    \end{array}
    \]

  \item[] \textbf{Typing Rules:}
    {    \small
    \begin{mathpar}
      \GSiekdruleSXXvar{} \and
      \GSiekdruleSXXunit{} \and
      \GSiekdruleSXXzero{} \and
      \GSiekdruleSXXsucc{} \and
      \GSiekdruleSXXpair{} \and
      \GSiekdruleSXXfst{} \and
      \GSiekdruleSXXsnd{} \and
      \GSiekdruleSXXlam{} \and
      \GSiekdruleSXXapp{}     
    \end{mathpar}
    }

  \item[] \textbf{Type Consistency:}
    \begin{mathpar}
      \GSiekdrulerefl{} \and
      \GSiekdrulebox{} \and
      \GSiekdruleunbox{} \and
      \GSiekdrulearrow{} \and
      \GSiekdruleprod{}    
  \end{mathpar}
    \end{itemize}
  \end{mdframed}
  \caption{The gradually simply typed $\lambda$-calculus: $\GSTLC$}
  \label{fig:GSTLC}
\end{figure}

The typing rules depend on the following partial functions:
\[
\begin{array}{lll}
  \begin{array}{lll}
  [[prod(?) = ? x ?]]\\
  [[prod(A1 x A2) = A1 x A2]]\\
  \end{array}
  & \quad &
  \begin{array}{lll}
  [[fun(?) = ? -> ?]]\\
  [[fun(A1 -> A2) = A1 -> A2]]\\
\end{array}
\end{array}
\]
As one can see these allow the unknown type to be split, otherwise
they simply return the given product/arrow type.  The typing rules
also depend on type consistency which is a reflexive and symmetric
binary relation.  Intuitively, $[[A ~ B]]$ states that $[[A]]$ can
be cast into $[[B]]$ and vice versa.  As we will see in the
interpretation type consistency amounts to requiring the existence of
a casting morphism.

We can view gradual typing as a surface language feature much like
type inference, and we give $\GSTLC$ its operational semantics by
translating it to a fully annotated core language -- sometimes called
the casting calculus -- called $\CGSTLC$. The syntax, typing rules,
and reduction relation for $\CGSTLC$ can be found in
Figure~\ref{fig:CGSTLC}. Its syntax is an extension of the syntax of
$\GSTLC$ where terms are the only syntactic class that differs, and so
we do not repeat the syntax of types or contexts. The reduction
relation is defined as a simple extension of call-by-value for the
simply typed $\lambda$-calculus, and so we do not give all of the
rules here, but rather only the casting rules.  In addition, to save
space we do not define the cast insertion algorithm for $\GSTLC$.
Please see Siek et al.~\cite{Siek:2015} for its definition.  To make
it easier for the reader to connect the reduction rules to their
interpretations we give the reduction rules for $\CGSTLC$ using the
terms-in-context formulation; for an introduction to this style please
see Crole~\cite{Crole:1994}.

\begin{figure}
  \begin{mdframed}
    \small
    \begin{itemize}
    \item[] \textbf{Syntax:}
      \[ \small
  \begin{array}{cccccc}
    \begin{array}{c@{\hspace{5pt}}r@{}@{\hspace{5pt}}r@{}@{\hspace{2pt}}l@{}llllllllllll}
    \text{(Atomic Types)}  & [[T]] & ::= & [[Unit]] \mid [[Nat]]\\
    \text{(Ground Types)}  & [[R]] & ::= & [[T]] \mid [[?]] \to [[?]]\\    
    \end{array}
    & \quad & 
  \begin{array}{c@{\hspace{5pt}}r@{}@{\hspace{5pt}}r@{}@{\hspace{2pt}}l@{}llllllllllll}
    \text{(values)}        & [[v]] & ::= & [[\x:A.t]]\\
    \text{(terms)}         & [[t]] & ::=  & \ldots \mid [[t : {A} => {B}]]\\
  \end{array}
  \end{array}
  \]

  \item[] \textbf{Typing Rules:}
    {    \small
      \begin{mathpar}
        \GSiekdruleCXXvar{} \and
      \GSiekdruleCXXunit{} \and
      \GSiekdruleCXXzero{} \and
      \GSiekdruleCXXsucc{} \and
      \GSiekdruleCXXpair{} \and
      \GSiekdruleCXXfst{} \and
      \GSiekdruleCXXsnd{} \and
      \GSiekdruleCXXlam{} \and
      \GSiekdruleCXXapp{} \and      
      \GSiekdruleCXXcast{}    
    \end{mathpar}
    }

  \item[] \textbf{Reduction Relation:}
    \begin{mathpar}
      \GSiekdrulerdAXXvalues{} \and
      \GSiekdrulerdAXXcastId{} \and
      \GSiekdrulerdAXXcastU{} \and
      \GSiekdrulerdAXXsucceed{} \and
      \GSiekdrulerdAXXcastArrow{} \and
      \GSiekdrulerdAXXcastGround{} \and
      \GSiekdrulerdAXXcastExpand{} 
      %% \GSiekdrulerdAXXbeta{} \and
      %% \GSiekdrulerdAXXappOne{} \and
      %% \GSiekdrulerdAXXappTwo{} \and
      %% \GSiekdrulerdAXXfst{} \and
      %% \GSiekdrulerdAXXsnd{} \and
      %% \GSiekdrulerdAXXpairOne{} \and
      %% \GSiekdrulerdAXXpairTwo{}  
  \end{mathpar}
    \end{itemize}
  \end{mdframed}
  \caption{The core casting calculus: $\CGSTLC$}
  \label{fig:CGSTLC}
\end{figure}

The casting calculus $\CGSTLC$ also depends on type consistency.  Now
type consistency is not transitive, because in the surface language we
do not want every type to be consistent with every other type.
However, the casting calculus allows one to compose consistency
relations.  For example, suppose $[[A ~ B]]$, $[[B ~ C]]$, and $[[G |-C
    t : A]]$, then using the rule, $\GSiekdruleCXXcastName{}$, of the
casting calculus we may type $[[G |-C t : h({A} => {B}) => {C} : C]]$.  This composition
is the reason why we interpret type consistency as casting morphisms
in the model.

%%% Local Variables: ***
%%% mode:latex ***
%%% TeX-master: "main.tex"  ***
%%% End: ***
