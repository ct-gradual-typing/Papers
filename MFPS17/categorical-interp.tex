In this section we show how to interpret $\GSTLC$ and
$\CGSTLC$ into the categorical model given
in the previous section. We complete the three steps summarized
above. We will show how to interpret the typing of the former into the
model, and then show how to do the same for the latter, furthermore,
we show that reduction can be interpreted into the model as well, thus
concluding soundness for $\CGSTLC$ with
respect to our model.

First, we must give the interpretation of types and contexts, but this
interpretation is obvious, because we have been making sure to match
the names of types and objects throughout this paper.
\begin{definition}
  \label{def:interpretation-of-gradual-types}
  Suppose $(\cat{T}, \cat{C}, ?, \T, \split, \squash, \bx, \unbox)$ is
  a gradual $\lambda$-model.  Then we define the interpretation of
  types into $\cat{C}$ as follows:
  \[
  \begin{array}{lll}
    [[ [| Unit |] ]] & = & 1\\
    [[ [| Nat |] ]] & = & [[Nat]]\\
    [[ [| ? |] ]] & = & [[?]]\\
    [[ [| A1 -> A2 |] ]] & = & [[ [| A1 |] -> [| A2 |] ]]\\
    [[ [| A1 x A2 |] ]] & = & [[ [| A1 |] x [| A2 |] ]]\\
  \end{array}
  \]
  We extend this interpretation to typing contexts as follows:
  \[
    \begin{array}{lll}
      [[ [| . |] ]] & = & 1\\
      [[ [| G,x : A |] ]] & = & [[ [| G |] x [| A |] ]]\\
    \end{array}
  \]
\end{definition}
\noindent Throughout the remainder of this paper we will drop the
interpretation symbols around types.

Before we can interpret the typing rules of $\GSTLC$ and
$\CGSTLC$ we must show how to interpret
the consistency relation from Figure~\ref{fig:type-consistency}.
These will correspond to casting morphisms.
\begin{lemma}[Type Consistency in the Model]
  \label{lemma:type_consistency_in_the_model}
  Suppose $(\cat{T}, \cat{C}, ?, \T,$ $\split, \squash, \bx, \unbox)$ is
  a gradual $\lambda$-model, and $<<A ~ B>>$ for some types $[[A]]$
  and $[[B]]$.  Then there are two casting morphisms $c_1 : [[ A --> B ]]$ and $c_2 : [[ B --> A ]]$.
\end{lemma}
\begin{proof}
This proof holds by induction on the form $<<A ~ B>>$ using the
morphisms $[[Box A]] : [[A --> ?]]$ and $[[Unbox A]] : [[? --> A]]$.
Please see
Appendix~\ref{subsec:proof_of_type_consistency_in_the_model} for the
complete proof.
\end{proof}

\begin{corollary}
  \label{corollary:type_consist_coro}
  Suppose $(\cat{T}, \cat{C}, ?, \T,$ $\split, \squash, \bx, \unbox)$ is
  a gradual $\lambda$-model.  Then we know the following:
  \begin{enumR}
  \item If $<<A ~ A>>$, then $c_1 = c_2 = \id_{[[A]]} : [[A --> A]]$.

  \item If $<<A ~ ?>>$, then there are casting morphisms:
    \[
    \begin{array}{lllll}
      c_1 & = & [[Box A]] : [[A --> ?]]\\
      c_2 & = & [[Unbox A]] : [[? --> A]]
    \end{array}
    \]

    \item If $<<? ~ A>>$, then there are casting morphisms:
    \[
    \begin{array}{lllll}      
      c_1 & = & [[Unbox A]] : [[? --> A]]\\
      c_2 & = & [[Box A]] : [[A --> ?]]
    \end{array}
    \]
    
  \item If $<<A1 -> B1 ~ A2 -> B2>>$, then there are casting morphisms:
    \[
    \begin{array}{lllll}
      c & = & c_1 \to c_2 : [[(A1 -> B1) --> (A2 -> B2)]]\\
      c' & = & c_3 \to c_4 : [[(A2 -> B2) --> (A1 -> B1)]]
    \end{array}
    \]
    where $c_1 : [[A2 --> A1]]$ and $c_2 : [[B1 --> B2]]$, and $c_3 :
    [[A1 --> A2]]$ and $c_4 : [[B2 --> B1]]$.
    
  \item If $<<A1 x B1 ~ A2 x B2>>$, then there are casting
    morphisms:
    \[
    \begin{array}{lll}
       c & = & c_1 \times c_2 : [[(A1 x B1) --> (A2 x B2)]]\\
      c' & = & c_3 \times c_4 : [[(A2 x B2) --> (A1 x B1)]]
    \end{array}
    \]
    where $c_1 : [[A1 --> A2]]$ and $c_2 : [[B1 --> B2]]$, and $c_3 :
    [[A2 --> A1]]$ and $c_4 : [[B2 --> B1]]$.
  \end{enumR}
\end{corollary}
\begin{proof}
  This proof holds by the construction of the casting morphisms from
  the proof of the previous result, and the fact that the type
  consistency rules are unique for each type.
\end{proof}

Showing that both $c_1$ and $c_2$ exist corresponds to the fact that
$<<A ~ B>>$ is symmetric.  But, this interpretation is an over
approximation of type consistency, because type consistency is not
transitive, but function composition is.  Leaving type consistency
implicit in the model just does not make good sense categorically,
because it would break composition.  For example, if we have morphisms
$f : [[A --> ?]]$ and $g : [[B --> C]]$, then if we implicitly allowed
$[[?]]$ to be cast to $[[B]]$, then we could compose these two
morphisms, but this does not fit the definition of a category, because
it requires the target of $f$ to match the source of $g$, but this
just is not the case.  Thus, the explicit cast must be used to obtain
$f;[[lunbox B]];g$.

At this point we have everything we need to show our main result which
is that typing in both $\GSTLC$ and $\CGSTLC$, and evaluation in
$\CGSTLC$ can be interpreted into the categorical model.

\begin{theorem}[Interpretation of Typing]
  \label{thm:interpretation_of_typing}
  Suppose $(\cat{T}, \cat{C}, ?, \T, \split,$ $\squash, \bx, \unbox)$
  is a gradual $\lambda$-model.
  \begin{enumR}
  \item If $<<G |-S t : A>>$, then there is a morphism $[[ [| t |] ]] : [[ [| G |] --> A ]]$ in $\cat{C}$.
  \item If $<<G |-C t : A>>$, then there is a morphism $[[ [| t |] ]] : [[ [| G |] --> A ]]$ in $\cat{C}$.
  \end{enumR}
\end{theorem}
\begin{proof}
  Both parts of the proof hold by induction on the form of the assumed
  typing derivation, and uses most of the results we have developed up
  to this point.  Please see
  Appendix~\ref{subsec:proof_of_interpretation_of_types} for the
  complete proof.
\end{proof}

\begin{theorem}[Interpretation of Evaluation]
  \label{thm:interpretation_of_evaluation}
  Suppose $(\cat{T}, \cat{C}, ?, \T,$ $\split, \squash, \bx, \unbox)$
  is a gradual $\lambda$-model.  If $<<G |- t1 ~> t2 : A>>$, then
  $[[ [| t1 |] ]] = [[ [| t2 |] ]] : [[ [| G |] --> A]]$.
\end{theorem}
\begin{proof}
  This proof holds by induction on the form of $[[G |- t1 ~> t2 :
      A]]$, and uses Theorem~\ref{thm:interpretation_of_typing},
  Lemma~\ref{lemma:type_consistency_in_the_model}, and
  Corollary~\ref{corollary:type_consist_coro}.  Please see
  Appendix~\ref{subsec:proof_of_interpretation_of_evaluation} for the
  complete proof.  
\end{proof}

%%% Local Variables: ***
%%% mode:latex ***
%%% TeX-master: "main.tex"  ***
%%% End: ***
