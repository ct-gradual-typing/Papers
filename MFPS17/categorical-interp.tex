Next we show how to interpret $\GSTLC$ and $\CGSTLC$ into a gradual
$\lambda$-model. We will show how to interpret the typing of the
former into the model, and then show how to do the same for the
latter, furthermore, we show that reduction can be interpreted into
the model as well, thus concluding soundness for $\CGSTLC$ with
respect to our model.

First, we must give the interpretation of types and contexts, but this
interpretation is obvious, because we have been making sure to match
the names of types and objects throughout this paper.
\begin{definition}
  \label{def:interpretation-of-gradual-types}
  Suppose $(\cat{T}, \cat{C}, ?, \T, \split, \squash, \bx, \unbox)$ is
  a gradual $\lambda$-model.  Then we define the interpretation of
  types into $\cat{C}$ as follows:
  \[
  \begin{array}{lll}
    \begin{array}{lll}
      [[ [| Unit |] ]] & = & 1\\
      [[ [| Nat |] ]] & = & [[Nat]]\\
      [[ [| ? |] ]] & = & [[?]]\\    
    \end{array}
    & \quad &
    \begin{array}{lll}
      [[ [| A1 -> A2 |] ]] & = & [[ [| A1 |] -> [| A2 |] ]]\\
      [[ [| A1 x A2 |] ]] & = & [[ [| A1 |] x [| A2 |] ]]\\\\
    \end{array}
  \end{array}
  \]
  We extend this interpretation to typing contexts as follows:
  \[
  \begin{array}{lll}
    \begin{array}{lll}
      [[ [| . |] ]] & = & 1      
    \end{array}
    & \quad &
    \begin{array}{lll}
      [[ [| G,x : A |] ]] & = & [[ [| G |] x [| A |] ]]
    \end{array}
  \end{array}
  \]
\end{definition}
\noindent Throughout the remainder of this paper we will drop the
interpretation symbols around types.

Before we can interpret the typing rules of $\GSTLC$ and
$\CGSTLC$ we must show how to interpret
the consistency relation from Figure~\ref{fig:GSTLC}.
These will correspond to casting morphisms.
\begin{lemma}[Type Consistency in the Model]
  \label{lemma:type_consistency_in_the_model}
  Suppose $(\cat{T}, \cat{C}, ?, \T,$ $\split, $\\ $\squash, \bx, \unbox)$ is
  a gradual $\lambda$-model, and $<<A ~ B>>$ for some types $[[A]]$
  and $[[B]]$.  Then there are two casting morphisms $c_1 : [[ A --> B ]]$ and $c_2 : [[ B --> A ]]$.
\end{lemma}
\begin{proof}
This proof holds by induction on the form $<<A ~ B>>$ using the
morphisms $[[Box A]] : [[A --> ?]]$ and $[[Unbox A]] : [[? --> A]]$.
Please see
Appendix~\ref{subsec:proof_of_type_consistency_in_the_model} for the
complete proof.
\end{proof}
\noindent
Showing that both $c_1$ and $c_2$ exist corresponds to the fact that
$<<A ~ B>>$ is symmetric.  

At this point we have everything we need to show our main result which
is that typing in both $\GSTLC$ and $\CGSTLC$, and evaluation in
$\CGSTLC$ can be interpreted into the categorical model.

\begin{theorem}[Interpretation of Typing]
  \label{thm:interpretation_of_typing}
  Suppose $(\cat{T}, \cat{C}, ?, \T, \split,$ $\squash, \bx,$\\$ \unbox)$
  is a gradual $\lambda$-model. If $<<G |-S t : A>>$ or $<<G |-C t : A>>$, then
  there is a morphism $[[ [| t |] ]] : [[ [| G |] --> A ]]$ in $\cat{C}$.  
\end{theorem}
\begin{proof}
  Both parts of the proof hold by induction on the form of the assumed
  typing derivation, and uses most of the results we have developed up
  to this point.  Please see
  Appendix~\ref{subsec:proof_of_interpretation_of_types} for the
  complete proof.
\end{proof}

\begin{theorem}[Interpretation of Evaluation]
  \label{thm:interpretation_of_evaluation}
  Suppose $(\cat{T}, \cat{C}, ?, \T,$ $\split, \squash,$\\$\bx, \unbox)$
  is a gradual $\lambda$-model.  If $<<G |- t1 ~> t2 : A>>$, then
  $[[ [| t1 |] ]] = [[ [| t2 |] ]] : [[ [| G |] --> A]]$.
\end{theorem}
\begin{proof}
  This proof holds by induction on the form of $[[G |- t1 ~> t2 : A]]$,
  and uses Theorem~\ref{thm:interpretation_of_typing},
  Lemma~\ref{lemma:type_consistency_in_the_model}, and
  Corollary~\ref{corollary:type_consist_coro}.  Please see
  Appendix~\ref{subsec:proof_of_interpretation_of_evaluation} for the
  complete proof.  
\end{proof}

%%% Local Variables: ***
%%% mode:latex ***
%%% TeX-master: "main.tex"  ***
%%% End: ***
