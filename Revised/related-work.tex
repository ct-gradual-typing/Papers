Abadi et al.~\cite{Abadi:1989} combine dynamic and static typing
by adding a new type called $\mathsf{Dynamic}$ along with a new case
construct for pattern matching on types.  We do not add such a case
construct, and as a result, show that we can obtain a surprising
amount of expressivity without it.  They also provide denotational
models.

Henglein~\cite{Henglein:1994} defines the dynamic
$\lambda$-calculus by adding a new type $\mathsf{Dyn}$ to the simply
typed $\lambda$-calculus and then adding primitive casting
operations called tagging and check-and-untag.  These new operations
tag type constructors with their types.  Then untagging checks to
make sure the target tag matches the source tag, and if not, returns
a dynamic type error.  These operations can be used to build casting
coercions which are very similar to our casting morphisms. We can
also define $[[split]]$, $[[squash]]$, $[[box]]$, and $[[unbox]]$ in
terms of Henglein's casting coercions.  We consider this paper as a
clarification of Henglein's system.  His core casting calculus can
be interpreted into our setting where we require retracts instead of
full isomorphisms.
