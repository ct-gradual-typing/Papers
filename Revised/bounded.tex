The design of Core Grady is very simple.  We simply started with a
purely static programming language, in this case the simply typed
$\lambda$-calculus, and then added the unknown type, $[[?]]$, and
$<<box A>>$, $<<unbox A>>$, and $<<error A>>$ to the syntax of types
and terms respectively.  Unlike other casting calculi for gradual type
systems found in the literature \cite{Siek:2006,Siek:2015,Garcia:2016}
Core Grady does not depend on type consistency.  These facts imply
that Core Grady should be more amenable to extension.

We illustrate this by extending Surface and Core Grady with bounded
quantification and lists; we primarily follow
Pierce's~\cite{Pierce:2002:TPL:509043} definition of Bounded System F
when defining the extension of Core Grady.  We show that the bounds on
quantifiers can be used to control which types are castable and which
are not to offer more fine grain control of casting.  For example,
here we will not allow a polymorphic type to be cast to the unknown
type.  We do this as as example, but it also simplifies the
metatheory.

The extension of the syntax, typing rules, and reduction rules for
Core and Surface Grady can be found in Figure~\ref{fig:bounded-grady}
and Figure~\ref{fig:bounded-grady-cont}.  
\begin{figure}
  \begin{mdframed} \footnotesize
    \textbf{Extended Syntax of Core Grady:}\\
    \[
      \setlength{\arraycolsep}{0.6pt}
      \begin{array}{cl}      
        \begin{array}{l}
          \text{(types)}\\
        \end{array}     &
        \begin{array}{lcl}
          <<A>>,<<B>>,<<C>> & ::= & \cdots \mid <<Top>> \mid <<SL>> \mid <<X>> \mid <<List A>> \mid <<Forall (X <: A).B>> \\
        \end{array}\\\\
                
        \begin{array}{l}
          \text{(terms)}\\\\
        \end{array}     &
        \begin{array}{lcl}
          <<t>> & ::= & \cdots \mid <<Lam X <: A .t >> \mid << [A]t >> \mid [] \mid <<t1 :: t2>> \\
                & \mid & <<case t : List A of [] -> t1, (x :: y) -> t2>>\\
        \end{array}
        \\\\
        \begin{array}{l}
          \text{(values)}\\
        \end{array}     &
        \begin{array}{lcl}
          <<v>> & ::=  & \cdots \mid <<Lam X <: A.t>>\\                
        \end{array}
        \\\\
        \begin{array}{lll}
          \text{(evaluation contexts)}\\\\
        \end{array}  &
        \begin{array}{lcl}
          <<EC>> & ::=  & \cdots \mid << [A]HL >> \mid << HL :: t2>> \mid <<t1 :: HL>>\\
                 & \mid & <<case HL : List A of [] -> t1, (x::y) -> t2>>\\
        \end{array}\\\\
        
        \begin{array}{lll}
          \text{(contexts)}\\
        \end{array}  &
        \begin{array}{lcl}
          <<G>> & ::= & <<.>> \mid <<x : A>> \mid <<X <: A>> \mid <<G1,G2>>\\
        \end{array}\\
      \end{array}
      \]    
      \\
      \textbf{Extended Syntax of Surface Grady:}\\
      \[
    \setlength{\arraycolsep}{1pt}
    \begin{array}{cl}      
        \begin{array}{l}
          \text{(terms)}\\\\
        \end{array}     &
        \begin{array}{lcl}
          [[t]] & ::=  & \cdots \mid [[Lam X <: A.t]] \mid [[ [A]t ]] \mid [[ [] ]] \mid [[t1 :: t2]] \\
                & \mid & [[case t of [] -> t1, (x::y) -> t2]]\\
        \end{array}\\\\
        
      \end{array}
      \]    
    \textbf{Extended Metafunctions for Surface Grady:}\\
    \begin{mathpar}
      \cdots
      \and
      \begin{array}{lll}
        [[list(?) = List ?]]\\
        [[list(List A) = List A]]\\
      \end{array}
    \end{mathpar}
    
    \textbf{Extended Typing for Core Grady:}\\
    \begin{mathpar}
      \cdots
      \and
      \CGradydruleTXXBoxP{}
      \and
      \CGradydruleTXXUnboxP{}
      \and
      \CGradydruleTXXempty{}
      \and
      \CGradydruleTXXcons{}
      \and
      \CGradydruleTXXlcase{}
      \and
      \CGradydruleTXXLam{}
      \and
      \CGradydruleTXXtypeApp{}
      \and
      \CGradydruleTXXSub{}
    \end{mathpar}  

  \textbf{Extended Typing for Surface Grady:}\\
  \begin{mathpar}
    \cdots
    \and
    \SGradydruleTXXempty{}
    \and
    \SGradydruleTXXcons{}
    \and
    \inferrule* [flushleft,right=$\mathsf{List}_e$] {
      {
        \begin{array}{lll}
          [[G |- t : C]]            & [[list(C) = List A]]\\
          [[G |- t1 : B1]]          & [[G |- B1 ~ B]]\\
          [[G, x : A, y : List A |- t2 : B2]] & [[G |- B2 ~ B]]
        \end{array}
      }
      }{[[G |- case t of [] -> t1, (x::y) -> t2 : B]]}
    \and
    \SGradydruleTXXLam{}
    \and
    \SGradydruleTXXtypeApp{}
    \and
    \SGradydruleTXXSub{}
  \end{mathpar}  
  \end{mdframed}
  \caption{Bounded Grady}
  \label{fig:bounded-grady}
\end{figure}
\begin{figure}
  \begin{mdframed}\footnotesize
    \textbf{Subtyping for Core Grady:}\\
    \begin{mathpar}
      \CGradydruleSXXRefl{} \and
      \CGradydruleSXXTop{} \and
      \CGradydruleSXXVar{} \and
      \CGradydruleSXXTopSL{} \and
      \CGradydruleSXXNatSL{} \and
      \CGradydruleSXXUnitSL{} \and
      \CGradydruleSXXListSL{} \and
      \CGradydruleSXXArrowSL{} \and
      \CGradydruleSXXProdSL{} \and
      \CGradydruleSXXList{} \and
      \CGradydruleSXXProd{} \and
      \CGradydruleSXXArrow{} \and
      \CGradydruleSXXForall{}      
    \end{mathpar}

    \textbf{Consistent Subtyping for Surface Grady:}\\
    \begin{mathpar}
      \SGradydruleSXXRefl{} \and
      \SGradydruleSXXTop{} \and      
      \SGradydruleSXXVar{} \and
      \SGradydruleSXXBox{} \and    
      \SGradydruleSXXUnbox{} \and
      \SGradydruleSXXUSL{} \and
      \SGradydruleSXXTopSL{} \and                        
      \SGradydruleSXXNatSL{} \and
      \SGradydruleSXXUnitSL{} \and    
      \SGradydruleSXXListSL{} \and
      \SGradydruleSXXProdSL{} \and
      \SGradydruleSXXArrowSL{} \and
      \SGradydruleSXXList{} \and
      \SGradydruleSXXProd{} \and
      \SGradydruleSXXArrow{} \and
      \SGradydruleSXXForall{}
    \end{mathpar}             
    
    \textbf{Type Consistency:}
    \begin{mathpar}
      \cdots
      \and
      \SGradydruleCXXList{} \and
      \SGradydruleCXXForall{}      
    \end{mathpar}
    
    \textbf{Extended Reduction Rules for Core Grady:}\\
    \begin{mathpar}
      \cdots
      \and
      \CGradydrulerdXXlcaseEmpty{}
      \and
      \CGradydrulerdXXlcaseCons{}
      \and
      \CGradydrulerdXXtypeBeta{}
    \end{mathpar}
  \end{mdframed}
  \caption{Bounded Grady Continued}
  \label{fig:bounded-grady-cont}
\end{figure}
We only give the new additions, but both Core and Surface Grady still
contain everything already introduced.  We add two new atomic types,
$[[Top]]$ and $[[SL]]$, that are used primarily for subtyping
purposes, where the former is the top type of which everything is a
subtype, and the latter is the universe of simple types, that is,
every non-polymorphic type will be a subtype of $[[SL]]$.  Then the
type variables, denoted by $[[X]]$, the type of lists, denoted by
$[[List A]]$, and the type of bounded polymorphic programs, denoted by
$[[Forall (X <: A).B]]$, are added to the syntax of types.  In
$[[Forall (X <: A).B]]$ we call $[[A]]$ the bounds of $[[X]]$.  The
syntax of types is the same for both Core and Surface Grady.

We extend the term syntax for Core and Surface Grady with new term
constructors for type abstraction, $[[Lam X <: A.t]]$, type
application, $[[ [A]t ]]$, the empty list, $[[ [] ]]$, cons, $[[ t1 ::
    t2 ]]$, where $[[t1]]$ is the head of the list and $[[t2]]$ is the
tail, and finally, the non-recursive eliminator for lists, $[[case t
    of [] -> t1, (x :: y) -> t2]]$ in Surface Grady, and
$<<case t : List A of [] -> t1, (x :: y) -> t2>>$ in Core Grady.
Values and evaluation contexts are extended in the expected way.

Typing for Core and Surface Grady depend on subtyping which are both
defined in Figure~\ref{fig:bounded-grady-cont}.  Subtyping for Core
Grady is a simple extension of the usual definition for bounded system
F \cite{Pierce:2002:TPL:509043} with the new type $[[SL]]$.  Every
non-polymorphic type is a subtype of $[[SL]]$.  One important thing to
note is that in Core Grady the unknown type is not a top type, and in
fact, the only type that is a subtype of the unknown type is the
unknown type.

Subtyping in Surface Grady differs from Core Grady in that it uses the
notion of consistent subtyping, denoted $[[G |- A <~ B]]$, which was
proposed by Siek and Taha~\cite{Siek:2007} in their work extending
gradual type systems to object oriented programming.  It embodies both
standard subtyping defined above and type consistency.  Thus,
consistent subtyping is also non-transitive.  One major difference
between this definition of consistent subtyping and others found in
the literature, for example in \cite{Siek:2007} and
\cite{Garcia:2016}, is the rule for type variables.  Naturally, we
must have a rule for type variables, because we are dealing with
polymorphism, but the proof of the gradual guarantee required that
this rule be relaxed and allow the bounds provided by the programmer
to be consistent with the subtype in question.  Unlike Core Grady in
consistent subtyping the unknown type is a subtype of every type and
vice versa.

Just as before typing for Surface Grady depends on type consistency,
but it has been extended with rules for the list type as well as the
polymorphic type.  At this point contexts annotating type consistency
are used to track type variables.  We make a simplifying assumption
that all free type variables occurring with in the types of a typing
context are also members of the context.  This assumption simply makes
the presentation of both Core and Surface Grady cleaner, because it
removes all of the additional constraints on the various inference
rules that the typing context always be well formed.

Typing for Core Grady is extended with the rules in
Figure~\ref{fig:bounded-grady}.  Because of the simplicity of Core
Grady's definition the rules are the standard ones, but we generalize
the types of $<<box A>>$ and $<<unbox A>>$ to the types $<<Forall (X
<: SL).(X -> ?)>>$ and $<<Forall (X <: SL).(X -> ?)>>$ respectively.
These types require that the only types that can be cast to or from
the unknown type is if they are non-polymorphic.  To allow more types
to be castable to or from the unknown type one can modify the bounds,
thus, bounded quantification can be used to control casting.
